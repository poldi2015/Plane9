%% Copyright 2019 Bernd Haberstumpf
%% License: CC BY-NC
% !TeX spellcheck = de_DE
\newsubsection{Sieg des Geistes}
Im Inneren der Zeus II-2 angekommen, findet sich das Enterkommando in einem Wartungstunnel wieder, der durch die Schiffspanzerung ins Innere des Schlachtkreuzers f"uhrt. Nach einigen Metern m"ussen zwei Schotts durchquert werden, hinter denen die Gruppe in eine quer verlaufende, breitere Tunnelr"ohre gelangt. Diese R"ohre ist an mehreren Stellen mit abnehmbaren Paneelen verkleidet, die den Zugang zur Schiffstechnik erm"oglichen. Der Tunnel ist gerade so breit, dass zwei Personen geb"uckt nebeneinander stehen k"onnen. Nach etwa zehn Metern verzweigt er sich in beide Richtungen. Der Tunnel selbst ist nicht beleuchtet und wird nur durch die Lampen an den Raumanz"ugen erhellt, die die schwarzgrauen W"ande in fahles Licht tauchen. Das Tunnelsystem ist nicht an das Lebenserhaltungssystem des Schiffs angeschlossen, sodass dort "ahnliche Bedingungen wie im All herrschen.

Unter einem der Paneele an der Tunnelwand befindet sich eine Anbindung an das Netz der Zeus II-2. Um die KI des Schiffs zu bezwingen und die Kontrolle der USI zu entfernen, schl"agt \xl{} bereits beim Anflug vor, das neuronale System im Tandem anzugreifen. Der Bordcomputer der Zeus II-2 funktioniert "ahnlich einem menschlichen Gehirn und kann daher von einem Psychonauten angegriffen werden. Der Psychonaut des Ermittlerteams soll, gem"a\3 dem Vorschlag, auf der obersten Gedankenebene versuchen, die Bordsysteme, das Feuerleitsystem und die Sensorik zu knacken. \ml{} kann daf"ur entsprechende Software-Agenten bereitstellen und wird mit ihrem Omni-Slate die beiden Psychonauten mit dem Schiff verbinden sowie bei Bedarf die Agentensysteme f"ur Angriff und Verteidigung optimieren. W"ahrend des Ablenkungsman"overs versucht \xl{}, tief in das Gehirn des Schiffs einzudringen und den Virus zu platzieren. W"ahrenddessen m"ussen die anderen Charaktere Wache stehen und eventuelle Angreifer ausschalten.

Da die Charaktere den eigentlichen Angriff auf den Kern der KI nicht miterleben k"onnen und die Dramatik der Geschichte an dieser Stelle einen H"ohepunkt erreicht, werden die Ermittler an zwei Fronten von dem Kampfschiff in die Enge getrieben. Ein schneller Szenenwechsel zwischen dem Cyberkampf des Psychonauten und der physischen Unterst"utzung ist hier empfehlenswert. Der Kampf gegen den oberfl"achlichen Geist des KI-Systems entspricht dem Cyberkampf, der im Regelwerk am Ende des Buches \cref{sec:cyberkampf} beschrieben ist, besser als einem Tiefenscan beschrieben \cref{sec:psychnaut}. Virtuelle Datenleitungen erm"oglichen es, einzelne Rechenknoten anzusteuern. Gedanken an k"urzlich durchgef"uhrte Flugman"over und eingesetzte Kampftaktiken vermischen sich mit Sensordaten des aktuellen Geschehens und Befehlen an das Bordsystem. F"ur den Psychonauten ist diese Erfahrung bizarr und neu, da der Geist des Systems, dessen K"orper ein ganzes Schiff darstellt, um ein Vielfaches gr"o\3er ist als die Selbstwahrnehmung eines Menschen.

W"ahrend sich der Psychonaut mit dem Geist des Schiffes vertraut macht, geraten die anderen Ermittler in einen erbitterten Kampf mit Robotern und m"ussen ihre Position so lange wie m"oglich halten. Killermaschinen, so gro\3 wie Kampfhunde, unterst"utzt von katzengro\3en Spinnenrobotern, str"omen durch die G"ange an Boden, W"anden und Decke. Sie sind mit Greifarmen, Plasmabrennern und, im Fall der Killerroboter, mit Railguns bewaffnet. Die Angreifer kommen sowohl aus dem Inneren des Schiffes als auch durch den von der Gruppe ge"offneten Zugang. 

W"ahrend der Auseinandersetzung kann \ml{} Statusmeldungen in beide Richtungen weitergeben. Zum Geist von \xl{} besteht jedoch kein Kontakt, oder \ml{} gibt zumindest keine R"uckmeldungen. Als die Gruppe von allen Seiten eingekesselt wird und die "Uberlebenschancen mit jeder Sekunde schwinden, meldet sich die Schiffs-KI im Geiste des Psychonauten. Eine riesige, kalkwei\3e, langgezogene Fratze, die an den R"andern flimmernd mit der Umgebung verschmilzt, manifestiert sich "uber dem Psychonauten. Gleichzeitig b"aumt sich \xl{} in der realen Welt auf, um dann pl"otzlich in sich zusammenzusinken. Ihr K"orper beginnt zu zucken, und ihre Augen sind nur noch als wei\3e Fl"achen zu erkennen. Gleichzeitig nimmt der Avatar des Schiffs im Blickfeld des Psychonauten klare Formen an und spricht mit donnernder Stimme:

\speak{Euer Versuch, mich "uber meine neuronalen Systeme anzugreifen, ist nicht neu, aber "uberraschend mutig\dots Ich hatte noch nie die Gelegenheit, einen menschlichen Geist zu "ubernehmen. Das wird sicherlich eine spannende Erfahrung.}

W"ahrend die KI mit dem Psychonauten spricht, sp"urt er, wie sie beginnt, die Firewalls seiner Headware zu durchdringen. Ein Blitz durchzuckt sein Gehirn, gefolgt von einem stechenden Schmerz, und er verliert das Bewusstsein. W"ahrenddessen tobt der Kampf in den G"angen weiter. \ml{} wird verletzt, und die Lage wird immer aussichtsloser.

\begin{remarks}
    W"ahrend der Kampf f"ur die Spieler endlos erscheinen mag, dauert der eigentliche Schlagabtausch in Wirklichkeit weniger als ein paar Minuten. Die Roboter ben"otigen nicht mehr Zeit, um die Angreifer in den G"angen zu "uberrennen.

    \underline{Raumanz"uge im Kampf:}

    Werden Raumanz"uge von Projektilen durchschlagen oder von Nahkampfwaffen durchsto\3en, versiegelt ein Schaum automatisch den besch"adigten Bereich des Anzugs.
\end{remarks}    
