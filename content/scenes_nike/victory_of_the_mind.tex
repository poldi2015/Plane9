%% Copyright 2019 Bernd Haberstumpf
%% License: CC BY-NC
% !TeX spellcheck = de_DE
\subsection{Sieg des Geistes}
Im Inneren der Zeus II-2 angekommen findet sich das Enterkommando in einem Wartungstunnel wieder. Der Wartungstunnel führt mit mehreren Abzweigungen weiter nach innen in den Schlachtkreuzer. Der Tunnel ist mit Paneelen verkleidet die abgenommen werden können um auf die Technik des Schiffs zugreifen zu können. Er ist gerade so breit, dass eine Person darin stehen kann. Der Tunnel selbst ist nicht beleuchtet. Als Zugang zum Tunnelsystem finden sich weiter innen im Schiff Schleusen. Das Tunnelsystem ist nicht an das Lebenserhaltungssystem des Schiffs angebunden. Es herrschen Bedingungen wie außerhalb des Kreuzers.

Unter einem der Paneele an der Tunnelwand findet sich eine Anbindung an das System Netz der Zeus II-2. Um die KI des Schiffs zu bezwingen und die USI Kontrolle zu entfernen schlägt \xl{} bereits beim Abflug von der Plane 9 vor zu zweit das neuronale System anzugreifen. Der Bordcomputer der Zeus II-2 funktioniert ähnlich einem menschlichen Gehirn kann also durch einen Psychonauten angegriffen werden. Der Psychonaut des Ermittlerteams muss als Ablenkungsmanöver auf oberster Gedankenebene versuchen die Bordsysteme, das Feuerleitsystem und die Sensorik zu knacken. \ml{} kann dafür entsprechende Software bereitstellen und wird mit ihrem Computersystem die beiden Psychonauten anbinden und wenn benötigt Agentensysteme für Angriff und Verteidigung anpassen. Ein weiterer ComNetz affiner Charakter kann sich ebenfalls in die Kommunikationskette einklinken, um den Psychonauten zu unterstützen. Während des Ablenkungsmanövers versucht \xl{} tief in das Gehirn des Schiffes einzudringen und den Virus zu platzieren. Währenddessen müssen die anderen Charaktere Wache stehen und eventuelle Angreifer ausschalten.

Da die Charaktere den eigentlichen Angriff auf den Kern der KI nicht miterleben können, gleichzeitig die Dramatik der Geschichte an dieser Stelle einen Höhepunkt erreicht werden die Ermittler an zwei Fronten von dem Kampfschiff in die Enge getrieben. Ein schneller Szenenwechsel zwischen dem Matrixkampf des Psychonauten und der Unterstützung auf physischer Seite ist hier empfehlenswert. Der oberflächliche Geist des KI-Systems entspricht dem Cyberspace bekannt aus Cyberpunk Literatur und Spielen. Virtuelle Datenleitungen erlauben es einzelne Rechenknoten anzunavigieren. Gedanken an vergangene Flugmanöver, Schlachten, Weltraumtechnik mischen sich mit Sensordaten des aktuellen Geschehens und Befehlen an das Bordsystem. Für den Psychonauten ist diese Erfahrung erst einmal verwirrend. Der Geist des Systems dessen Körper ein ganzes Schiff darstellt, ist um einen mehrfaches größer als die Selbstwahrnehmung eines Menschen.

Während der Psychonaut sich mit dem System des Schiffes vertraut macht werden die anderen Charaktere angegriffen und müssen so lange wie möglich die Stellung zu halten. Kampfhund große Killerroboter in Spinnenform füllen die Gänge an Boden, Wand und Decke. Sie sind mit Greifarmen und Railguns bewaffnet. Sie greifen von innerhalb des Schiffes als auch über den von der Gruppe geöffneten Zugang an. Während des Angriffs durch die Droiden können \ml{} und ein etwaiger ans System angekoppelter Ermittler Statusmeldungen in beide Richtungen abgeben. Zum Geist von \xl{} hat niemand Kontakt oder \ml{} will zumindest keine Rückmeldungen abgeben. Ist die Gruppe von allen Seiten eingekesselt und die Gewinnchancen schwinden, meldet sich die Schiffs-KI im Geiste des Psychonauten zu Wort. Eine riesige kalkweiße langgezogene Fratze, die an den Rändern wabernd mit der Umgebung verschmilzt, manifestiert sich langsam über dem Psychonauten. Währenddessen bäumt sich \xl{} in der Realwelt auf, um dann plötzlich in sich zusammenzusinken. Ihre Augenlider beginnen zu flattern. Die Augen sind nur noch als weiße Fläche zu erkennen. Sie scheint bewusstlos geworden zu sein. Währenddessen nimmt der Kopf im Blickfeld des Psychonauten klar gezeichnete Formen an und spricht:

\speak{"`Euer Ansatz ein Schlachtschiff über sein neuralen Systeme anzugreifen ist nicht neu aber interessant. Ich hatte noch nie die Gelegenheit einen menschlichen Geist zu besetzen. Das wird sicherlich eine spannende Erfahrung."'}

Während die KI mit dem Psychonauten spricht, spürt er wie sie anfängt seinen Geist zu ertasten. Ein Blitz und ein starker Schmerz durchfährt das Gehirn des Psychonauten. Dann wird er ohnmächtig. Währenddessen geht der Kampf auf den Gängen weiter. \ml{} wird verletzt. Die Lage erscheint aussichtslos.
