%% Copyright 2019 Bernd Haberstumpf
%% License: CC BY-NC
% !TeX spellcheck = de_DE
\newsubsection{Sieg des Geistes}
Im Inneren der Zeus II-2 angekommen, findet sich das Enterkommando in einem Wartungstunnel wieder. Dieser f"uhrt durch die Schiffspanzerung ins Innere des Schlachtkreuzers. Nach einigen Metern m"ussen zwei Schotts durchquert werden. Dahinter gelangt die Gruppe in eine quer verlaufende, breitere Tunnelr"ohre. Diese R"ohre ist an mehreren Stellen mit abnehmbaren Paneelen verkleidet, die den Zugang zur Schiffstechnik erm"oglichen. Der Tunnel ist gerade so breit, dass zwei Personen nebeneinander darin stehen k"onnen. Nach etwa zehn Metern verzweigt er sich in beide Richtungen. Der Tunnel selbst ist nicht beleuchtet und wird nur durch die Lampen an den Raumanz"ugen erhellt, die die schwarzgrauen W"ande in fahles Licht tauchen. Das Tunnelsystem ist nicht an das Lebenserhaltungssystem des Schiffs angebunden, sodass dort "ahnliche Bedingungen wie im All herrschen.

Unter einem der Paneele an der Tunnelwand findet sich eine Anbindung an das Netz der Zeus II-2. Um die KI des Schiffs zu bezwingen und die Kontrolle der USI zu entfernen, schl"agt \xl{} bereits beim Anflug vor, das neuronale System zu zweit anzugreifen. Der Bordcomputer der Zeus II-2 funktioniert "ahnlich einem menschlichen Gehirn und kann daher von einem Psychonauten angegriffen werden. Der Psychonaut des Ermittlerteams soll, gem"a\3 dem Vorschlag, auf der obersten Gedankenebene versuchen, die Bordsysteme, das Feuerleitsystem und die Sensorik zu knacken. \ml{} kann daf"ur entsprechende Software-Agenten beisteuern und wird mit ihrem Omni-Slate die beiden Psychonauten mit dem Schoff verbinden, sowie bei Bedarf die Agentensysteme f"ur Angriff und Verteidigung nach Bedarf optimieren. Ein weiterer ComNetz-affiner Charakter kann sich ebenfalls in die Kommunikationskette einklinken, um die Psychonauten zu unterst"utzen. W"ahrend des Ablenkungsman"overs versucht \xl{} tief in das Gehirn des Schiffes einzudringen und den Virus zu platzieren. W"ahrenddessen m"ussen die anderen Charaktere Wache stehen und eventuelle Angreifer ausschalten.

Da die Charaktere den eigentlichen Angriff auf den Kern der KI nicht miterleben k"onnen und gleichzeitig die Dramatik der Geschichte an dieser Stelle einen H"ohepunkt erreicht, werden die Ermittler an zwei Fronten von dem Kampfschiff in die Enge getrieben. Ein schneller Szenenwechsel zwischen dem Cyberkampf des Psychonauten und der Unterst"utzung auf physischer Ebene ist hier empfehlenswert. Der Kampf gegen den oberfl"achlichen Geist des KI-Systems entspricht dem Cyberkampf, der im Regelwerk am Ende des Buches  \cref{sec:cyberkampf} beschrieben ist. Virtuelle Datenleitungen erm"oglichen es, einzelne Rechenknoten anzusteuern. Gedanken an k"urzlich durchgef"uhrte Flugman"over und eingesetzte Kampftaktiken vermischen sich mit Sensordaten des aktuellen Geschehens und Befehlen an das Bordsystem. F"ur den Psychonauten ist diese Erfahrung zun"achst "uberw"altigend. Der Geist des Systems, dessen K"orper ein ganzes Schiff darstellt, ist um ein Vielfaches gr"o\3er als die Selbstwahrnehmung eines Menschen.

W"ahrend der Psychonaut sich mit dem Geist des Schiffes vertraut macht, werden die anderen Ermittler in einen Kampf verwickelt und m"ussen so lange wie m"oglich die Stellung halten. Kampfhund gro\3e Killerroboter unterst"utzt von Katzengro\3en Spinnenrobotern f"ullen die G"ange an Boden, Wand und Decke. Sie sind mit Greifarmen, Plasmabrennern und im Falle der Killerroboter mit Bolzenpistolen bewaffnet. Sie greifen sowohl von innerhalb des Schiffes als auch durch den von der Gruppe ge"offneten Zugang an. W"ahrend des Angriffs durch die Kampfroboter k"onnen \ml{} und ein etwaiger ans System angekoppelter Ermittler Statusmeldungen in beide Richtungen abgeben. Zum Geist von \xl{} hat niemand Kontakt oder \ml{} gibt zumindest keine R"uckmeldungen. Als die Gruppe von allen Seiten eingekesselt ist und die Gewinnchancen schwinden, meldet sich die Schiffs-KI im Geiste des Psychonauten zu Wort. Eine riesige, kalkwei\3e, langgezogene Fratze, die an den R"andern wabernd mit der Umgebung verschmilzt, manifestiert sich langsam "uber dem Psychonauten. W"ahrenddessen b"aumt sich \xl{} in der realen Welt auf, um dann pl"otzlich in sich zusammenzusinken. Ihr K"orper beginnt zu zucken, und die Augen sind nur noch als wei\3e Fl"achen zu erkennen. Gleichzeitig nimmt der Kopf im Blickfeld des Psychonauten klar gezeichnete Formen an und spricht mit donnernder Stimme:

\speak{Euer Versuch, mich "uber meine neuronalen Systeme anzugreifen, ist nicht neu, aber "uberraschend mutig. Ich hatte noch nie die Gelegenheit, einen menschlichen Geist zu besetzen. Das wird sicherlich eine spannende Erfahrung.}

W"ahrend die KI mit dem Psychonauten spricht, sp"urt er, wie sie beginnt, die Firewalls seiner Headware zu durchdringen. Ein Blitz und ein starker Schmerz durchfahren sein Gehirn, und er wird ohnm"achtig. W"ahrenddessen tobt der Kampf in den G"angen weiter. \ml{} wird verletzt, und die Lage erscheint aussichtslos.

\begin{remarks}
    W"ahrend der Kampf f"ur die Spieler endlos erscheinen mag, handelt es sich in Wirklichkeit nur um einen kurzen Schlagabtausch von weniger als einer Minute. Mehr Zeit ben"otigen die Roboter nicht, um die Angreifer in den G"angen zu bezwingen.

    Werden Raumanz"uge von Projektilen durchschlagen oder von Nahkampfwaffen durchsto\3en, versiegelt ein Schaum automatisch den besch"adigten Teil des Anzugs.
\end{remarks}    
