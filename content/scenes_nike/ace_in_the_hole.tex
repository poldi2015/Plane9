%% Copyright 2019 Bernd Haberstumpf
%% License: CC BY-NC
% !TeX spellcheck = de_DE
\newsubsection{Ein Ass im "Armel}
Urpl"otzlich stoppen die Roboter ihren Angriff. \xl{} erwacht, kurz darauf auch der Psychonaut des Teams. Ihm ist schwindelig, und er kann das Wahrgenommene zun"achst nicht einordnen. Desorientiert schaut sich \xl{} um und erstarrt immer wieder in ihren Bewegungen. Nach fast f"unf Minuten, als der Psychonaut der Gruppe l"angst wieder auf den Beinen ist, erhebt sie sich wackelig. Wer in ihren Helm blickt, wird feststellen, dass weder Pupille noch Iris in ihren Augen zu sehen sind. Zielgerichtet, aber immer noch mit teilweise unkoordinierten Bewegungen, hangelt sie sich ins Innere des Schiffes, ignoriert dabei die Roboter und fordert die Gruppe auf, ihr zu folgen.

\speak{Los gehen wir. Ich bin schon auf die Br"ucke gespannt.}

In den teilweise engen Wartungsr"ohren des Schiffs liegen und stehen immer wieder inaktive Roboter. Die pl"otzliche "`Stille"' nach dem Kampf ist be"angstigend. "Uber eine Schleuse gelangt die Gruppe in einen gr"o\3eren Tunnel, der ebenfalls nicht unter Druck steht und nur sp"arlich beleuchtet ist. \xl{} folgt zielstrebig dem Gang. Bewegungen hinter der Gruppe k"undigen vier menschengro\3e, spinnenartige Kampfroboter an, die sich der Gruppe anschlie\3en. Mehrere harte Kurswechsel lassen oben zu unten und links zu rechts werden, was die Fortbewegung erschwert. Die Gruppe betritt eine weitere Schleuse. Nach einem weiteren Kurswechsel "offnet sich die T"ur zum Kommandodeck.

Auf der kardanisch aufgeh"angten Br"ucke im Zentrum des Kreuzers ist das Lebenserhaltungssystem aktiv, und eine Beleuchtung taucht den Raum in wei\3es Licht. Die Konsolen und die dazugeh"origen Beschleunigungsliegen rund um einen zentralen Kommandostand sind mit Blut besudelt. Blutstropfen treiben durch den Raum. Acht Leichen sind in ihren Liegen angegurtet. Ein Roboter, selbst mit Blut bespritzt, beginnt die Leichen einzusammeln und durch eine weitere Schleuse aus dem Raum zu bugsieren. \xl{} "offnet ihr Helmvisier, hakt sich in das zentral erh"ohte Kommandopult ein und "offnet eine Richtfunkverbindung zur Martell. Lord Commander Steeler erscheint auf dem zentralen Display der Br"ucke. \xl{} l"achelt in die Kamera und wendet sich an den Kommandanten des Flottentr"agers:

\speak{Hier \xl{}, Kommandantin des Schlachtkreuzers Dragon Fist. Ich gr"u\3e dich Lord Commander. Ich fordere einen sofortigen Waffenstillstand. Dieser Kreuzer steht nicht mehr unter der Kontrolle der Feinde des Protektorats.}

Lord Commander Steeler ist aus seiner Liege aufgesprungen, verharrt kurz und kneift die Augen zusammen:

\speak{\xl{}?~\dots{} die Piratin, die Anf"uhrerin der "`Red Dragons"'? Die die wir vor einem halben Jahr festgesetzt haben, nachdem sie von ihren Leuten verraten wurde??}

Sp"atestens jetzt d"urfte einem Charakter, der im G"urtel t"atig war, klar werden, mit wem sie es bei \xl{} zu tun haben.

\xl{} lacht auf, wird aber sofort wieder ernst:

\speak{Lord Commander, ich wiederhole mein Angebot nur ungern ein zweites Mal. Wenn du erlaubst, ziehe ich meine J"ager ab und erwarte das gleiche von mir.}

Steeler z"ogert und gibt dann Anweisungen an seine Crew, bevor er das Wort wieder an \xl{} richtet. Die Verhandlungen dauern noch ein paar Minuten an, dann schlie\3t \xl{} die Verbindung. Danach "offnet sie einen Kanal zur Hyperion und "ubermittelt die neue Situation.

Nach Abschluss der Formalit"aten wendet sie sich an die Ermittler. Inzwischen haben sich auch die vier Roboter, die der Gruppe gefolgt sind, auf der Br"ucke eingefunden. \xl{} blickt die Ermittler grinsend an.

\speak{Das hat doch ganz gut geklappt. Meine lieben Freunde, dieser neue K"orper f"uhlt sich wirklich gut an. Wie gef"allt euch mein neuer Name "`Dragon Fist"'?}
%\vfill\pagebreak

\begin{remarks}
	\begin{center}\huge{}Ende\end{center}

	Die Szene "`Sieg des Geistes"' ist die letzte, bei der die Spieler aktiv eingebunden sind und dient als Cliffhanger zu der Abschlussszene "`Ein Ass im "Armel"'. In dieser Szene erz"ahlt der Spielleiter den Ausgang der Schlacht um die Nike-Station. Durch die "Ubernahme der Zeus II-2, unabh"angig davon, ob das Schiff anschlie\3end selbst am weiteren Kriegsgeschehen teilnimmt oder nicht, erh"alt das Protektorat und Cynarian eine "Ubermacht, der die USI nicht ausreichend Gegenwehr entgegenstellen kann.

	Diese letzte Szene dient somit als Abschluss der Geschichte und l"uftet mit einer weiteren "Uberraschung die letzten Geheimnisse. In dieser Szene erfahren die Charaktere und die Spieler, wenn sie genau zugeh"ort haben, um wen es sich bei ihrer Begleiterin handelt und welchen Plan \xl{} seit dem Auftauchen von \ml{} in die Tat umgesetzt hat. Durch die Verschmelzung der Schiffs-KI mit der eigenen gibt sie dem Schiff in gewisser Weise einen eigenen menschlichen K"orper und wird gleichzeitig ein Teil des Schiffs. Damit erf"ullt sie ihr Versprechen, den Kreuzer aus den Reihen der Feinde zu l"osen, und hat sich selbst eine m"achtige Waffe geschaffen. Ob sich \xl{}s Pers"onlichkeit behaupten konnte oder sie von der Schiffs-KI assimiliert wurde, bleibt offen. Ihr folgendes Schicksal bildet eine gute Basis f"ur zuk"unftige Abenteuer in der Welt von C23.
\end{remarks}
\vfill\pagebreak
