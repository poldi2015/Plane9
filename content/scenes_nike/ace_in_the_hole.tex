%% Copyright 2019 Bernd Haberstumpf
%% License: CC BY-NC
% !TeX spellcheck = de_DE
\subsection{Ein Ass im "Armel}
Urpl"otzlich stoppen die Spinnen ihren Angriff. \xl{} erwacht. Kurz darauf auch der Psychonaut des Teams. Ihm ist schwindelig und er kann das Wahrgenommene zun"achst gar nicht einordnen. Desorientiert, schaut \xl{} sich um, erstarrt immer wieder in ihren Bewegungen. Nach fast 5 Minuten, der Psychonaut der Gruppe ist l"angst wieder auf den Beinen, erhebt sie sich zuerst wackelig. Wer in ihren Helm blickt, wird zun"achst feststellen, dass in ihre Augen keine Pupille und keine Iris mehr zu sehen ist. Nach dem Aufstehen verschwindet der Effekt. Zielgerichtet aber immre noch mit teilweise uncoordinierten Bewegungen hangelt sie sich ins Innere des Schiffes, ignoriert die Spinnendroiden und fordert die Gruppe auf ihr zu folgen.

\speak{Los gehen wir. Ich bin schon auf den Kommandostand gespannt.}

In den engen Wartungsg"angen des Schiffs liegen und stehen inaktive Androiden. Die pl"otzliche Stille nach dem Kampf ist be"angstigend. "Uber eine Schleuse gelangt die Gruppe in einen gr"o\3eren Tunnel, der aber ebenfalls nicht unter Druck steht und nur sp"arlich beleuchtet ist. \xl{} folgt zielstrebig dem Gang. Ein leises Klicken k"undigt vier menschengro\3e spinnenartige Kampfdroiden an die sich der Gruppe anschlie\3en. Mehrere harte Kurswechsel lassen oben zu unten, links zu rechts werden und erschweren die Fortbewegung. 
Die Gruppe betritt eine weitere Schleuse. Nach einem weiteren Kurswechsel "offnet sich die T"ur zum Kommandodeck.

Auf dem kardanisch aufgeh"angten Kommandodeck im Zentrum des Kreuzers ist das Lebenserhaltungssystem aktiv und eine Beleuchtung taucht den Raum in wei\3es Licht. Die Konsolen und die dazugeh"origen Beschleunigungsliegen rund um den Kommandostand sind mit Blut besudelt. Blutstropfen treiben durch den Raum. 8 Leichen sind in ihren Liegen angegurtet. Ein Droide, selbst mit Blut bespritzt beginnt die Leichen einzusammeln und durch eine weitere Schleuse aus dem Raum heraus zu bugsieren. \xl{} "offnet ihr Helmvisier, hakt sich im zentralen Kommandostand ein und "offnet eine Richtfunkverbindung zur Donar. Lord Commander Steeler erscheint auf dem zentralen Display der Br"ucke. \xl{} l"achelt in die Kamera und wendet das Wort an den Kommandanten des Flottentr"agers:

\speak{Hier \xl{} Kommandantin des Schlachtkreuzers Dragon Fist. Ich gr"u\3e dich Lord Commander. Ich fordere einen sofortigen Waffenstillstand. Dieser Kreuzer steht nicht mehr unter der Kontrolle der Feinde des Protektorats.}

Lord Commander Steeler ist aus seiner Liege aufgesprungen, verharrt kurz und kneift die Augen zusammen:

\speak{\xl{}?~\dots{} die Piratin, die Drachen Lady die wir vor einem halben Jahr festgesetzt haben, nachdem sie von ihren Leuten verraten wurde??}

Sp"atestens jetzt d"urfte einem Charakter der im G"urtel t"atig war oder einem Angeh"origen der Protektoratsstreitkr"afte klar werden, mit wem sie es bei \xl{} zu tun haben.

\xl{} muss lachen, wird aber sofort wieder ernst:

\speak{Lord Commander, ich wiederhole mein Angebot nur ungern ein zweites Mal. Wenn du erlaubst, ziehe ich meine J"ager ab und erwarte das gleiche von mir.}

Steeler z"ogert und gibt dann Anweisungen an seine Crew bevor er das Wort wieder an \xl{} wendet. Die Verhandlungen gehen noch ein paar Minuten, dann schlie\3t \xl{} die Verbindung. Danach "offnet sie einen Kanal an die Hyperion und "ubermittelt die neue Situation.

Nach Abschluss der Formalit"aten wendet sie sich den Ermittlern. Inzwischen haben sich auch die vier Droiden die der Gruppe gefolgt sind auf der Br"ucke eingefunden. \xl{} blickt die Gruppe grinsend an.

\speak{Das hat doch ganz gut geklappt. Meine ehrenwerten Freunde, der Geist des Schiffes gepart mit dem K"orper einer Frau ist atemberaubend.}
\vfill\pagebreak

\begin{remarks}
	\begin{center}\huge{}Ende\end{center}

	Die Szene "`Sieg des Geistes"' ist die letzte, bei der die Spieler aktiv eingebunden sind und dient als Cliffhanger zu der Abschlussszene "`Ein Ass im "Armel"'. In dieser Szene erz"ahlt der Spielleiter den Ausgang der Schlacht um die Nike Station.    
    Durch die "Ubernahme der Zeus II-2, unabh"angig davon, ob das Schiff im folgenden selbst am weiteren Kriegsgeschehen teilnimmt oder nicht, gibt dem Protektorat und Cynarian eine "Ubermacht an die Hand der die USI nicht ausreichend Verteidigung entgegenstellen kann. Dadurch, dass die Drohnen im Schlachtkreuzer bereits \xl{} unterstehen haben die Ermittler wenig M"oglichkeit in das Handlungsgeschehen einzugreifen.

	Diese letzte Szene dient somit dem Abschluss der Geschichte und l"uftet mit einer weiteren "Uberraschung die letzten Geheimnisse. In dieser Szene erfahren die Charaktere und die Spieler, wenn sie genau zugeh"ort haben, um wen es sich bei ihrer Begleiterin handelt und welchen Plan \xl{} seit dem Auftauchen von \ml{} in die Tat umgesetzt hat. Durch die Verschmelzung der Schiffs-KI mit der eigenen gibt sie dem Schiff in gewisser Weise einen eigenen menschlichen K"orper und erweitert ihren Geist und ihre ganze Person gleichzeitig um die ehemalige Zeus II-2 KI. Damit erf"ullt sie ihr versprechen den Kreuzer aus den Reihen der Feinde zu l"osen und hat sich selbst eine m"achtige Waffe geschaffen. Wie sie ihren zweiten K"orper in Zukunft einsetzt, bildet eine gute Basis einer zuk"unftigen Geschichte in der Welt von c23.
\end{remarks}
\pagebreak