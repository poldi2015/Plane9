%% Copyright 2019 Bernd Haberstumpf
%% License: CC BY-NC
% !TeX spellcheck = de_DE
\newsection{Die letzte Schlacht}\anchor{sec:lastbattle}

Die ``letzte Schlacht'' in dieser Form findet statt, wenn die Charaktere mit \xl{} zusammen auf der Dragon Blade die Zeus II-2 entern. \xl{} hat hierf"ur ein weiteres Ass im "Armel, um die Zeus KI zu "uberzeugen sich auf ihre Seite zu schlagen. Ein alternatives Ende bei dem sich die Ermittler gegen \xl{} stellen ist in dieser Geschichte nicht ausgearbeitet.

\newsubsection{Das Ende der Plane 9}
Um unerkannt auf die Plane 9 zu kommen haben \xl{} und \ml{} ihr Schiff im Raumanzug und mit Jetpacks ausgestattet verlassen. Seitdem treibt die Dragon Blade mit heruntergefahrenen Systemen nahezu bewegungslos durchs All. Steht die Dawn of Day zur Verf"ugung, kann die Gruppe mithilfe ihres Schiffes zur Dragon Blade "ubersetzen. Ist dies nicht der Fall, kann \xl{} ihr Schiff per Richtstrahlfunk aktivieren und es zu sich holen.

Mit einem kontrollierten Man"over bringt \xl{} die Dragon Blade weiter von der im All treibenden Plane 9 und von der Dawn of Day weg. W"ahrend der Reaktor hochf"ahrt, spricht \xl{} das Unvermeidliche aus:

\speak{Ruhe in Frieden Professor. Was f"ur ein tragisches Ende \dots{} Festhalten!}

Der im All treibende Ring zerbirst in einem Feuerball. Aus der Entfernung l"asst sich nicht feststellen, ob dies durch einen Torpedo der Donar oder durch eine von der Firmenchefin selbst ausgel"oste Sprengung verursacht wurde. Das Shuttle der Zeus II-2 wird durch anfliegenden Tr"ummerteilen zerst"ort. Wenn nicht durch die Charaktere gewarnt, erleidet das Shuttle der Nike Station dasselbe Schicksal. Gl"ucklicherweise ist die fl"uchtende Dragon Blade weit genug entfernt, um den umherfliegenden Teilen auszuweichen. \xl{} gibt Schub und schwenkt in Richtung Zeus II-2 ein. Obwohl sie bereits damit gerechnet hat, ist \ml{} von dem Ereignis ersch"uttert und flucht leise vor sich hin.

\newsubsection{Enterkommando Zeus II-2}
Zeitgleich mit dem Start der Triebwerke authentisiert sich \xl{} mit dem Transponder der Dragon Blade bei den anfliegenden J"agern der Zeus II-2 als Teil ihrer Flotte. Da ihre KI-Pers"onlichkeit eine Kopie der milit"arischen KI der USI darstellt, stehen ihr daf"ur alle M"oglichkeiten zur Verf"ugung. Nach dem Einschwenken auf einen Rendezvouskurs deaktiviert sie den Reaktor, f"ahrt alle nicht notwendigen Systeme herunter und aktiviert damit wieder die Tarnfunktionalit"at des Schiffs. Die Dragon Blade ist damit f"ur die Sensorik der anderen Schiffe im Umkreis von "uber einem Kilometer unsichtbar. Das Cockpit ist auf minimale Beleuchtung heruntergefahren. Die Umgebung ist nur visuell durch Au\3enkameras sichtbar.

Eine bange Viertelstunde vergeht, bevor sich die Dragon Blade der Zeus II-2 so weit gen"ahert hat, dass ein Bremsman"over notwendig wird. \xl{} f"ahrt den Reaktor hoch und startet ihr Haupttriebwerk f"ur ein kurzes, aber hartes Bremsman"over. Die Nahkampfwaffen der Zeus II-2 erwachen zum Leben. Hochgeschwindigkeitsgeschosse rasen auf die Dragon Blade zu. Die Dragon Blade verf"ugt "uber einen Torpedoschacht, acht Torpedos und vier Railgun-Gesch"utze, mit denen ein Bordsch"utze die Zeus unter Beschuss nehmen kann.

Zwei Minute sp"ater schl"agt die Dragon Blade auf der Zeus II-2 auf und krallt sich an der Bordwand fest. "Uber einen Andocktunnel kann das Enterkommando bestehend aus den Ermittlern, \xl{} und \ml{} die Bordwand des Kreuzers betreten. Der Andockpunkt ist nicht optimal was bedeutet, dass das Enterkommando sich einen Zugang zum Schlachtschiff an einer anderen Stelle suchen muss. Die Dragon Blade verf"ugt "uber zwei Exoskelette ausgestattet mit je einer Plasmaschleuder und einem Plasmabrenner. Der Plasmabrenner kann dazu genutzt werden ein Loch in die Au\3enh"ulle der Zeus II-2 zu brennen. Einen g"unstigen Zugangsort bietet eine Wartungsschleuse in etwa 50 Meter Entfernung. Mittels Kletterseile und Magnetstiefeln kann sich die Gruppe "uber die Oberfl"ache des Schiffs bewegen. Auf halbem Weg wird die Gruppe von zwei spinnenartigen Kampfrobotern von zwei Seiten aus angegriffen. Die Kampfroboter sind \cref{sec:guardian} beschrieben. An der Wartungsschleuse kann die Gruppe entweder versuchen die Verriegelung zu knacken oder die Schleuse aus der Bordwand mittels des Plasmabrenners zu schneiden.

\begin{remarks}
	Die Zeus II-2 ist w"ahrend des Enterns durch die Gruppe nach wie vor im Gefecht mit der Donar. Sie passt also den Flugvector an das Kampfgeschehen an. Dabei muss die Gruppe damit k"ampfen durch Flugman"over durch die Gegend geworfen zu werden.

	Der Einstieg "uber eine Schleuse und der Angriff durch die Kampfroboter sind nur Vorschl"age und k"onnen je nach dem Verhalten der Spieler angepasst werden.
\end{remarks}
