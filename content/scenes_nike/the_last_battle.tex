%% Copyright 2019 Bernd Haberstumpf
%% License: CC BY-NC
% !TeX spellcheck = de_DE
\newsection{Die letzte Schlacht}\anchor{sec:lastbattle}

Die "`letzte Schlacht"' in dieser Form findet statt, wenn die Charaktere mit \xl{} zusammen auf der Dragon Blade die Zeus II-2 entern. \xl{} hat hierfür ein weiteres Ass im Ärmel, um die Zeus KI zu überzeugen sich auf ihre Seite zu schlagen. Ein alternatives Ende bei dem sich die Ermittler gegen \xl{} stellen ist in dieser Geschichte nicht ausgearbeitet.

\subsection{Das Ende der Plane 9}
Um unerkannt auf die Plane 9 zu kommen haben \xl{} und \ml{} ihr Schiff im Raumanzug und mit Jetpacks ausgestattet verlassen. Seitdem treibt die Dragon Blade mit heruntergefahrenen Systemen nahezu bewegungslos durchs All. Steht die Dawn of Day zur Verfügung, kann die Gruppe mithilfe ihres Schiffes zur Dragon Blade übersetzen. Ist dies nicht der Fall, kann \xl{} ihr Schiff per Richtstrahlfunk aktivieren und es zu sich holen.

Mit einem kontrollierten Manöver bringt \xl{} die Dragon Blade weiter von der im All treibenden Plane 9 und von der Dawn of Day weg. Während der Reaktor hochfährt, spricht \xl{} das Unvermeidliche aus:

\speak{Ruhe in Frieden Professor. Was für ein tragisches Ende \dots{} Festhalten!}

Der im All treibende Ring zerbirst in einem Feuerball. Aus der Entfernung lässt sich nicht feststellen, ob dies durch einen Torpedo der Donar oder durch eine von der Firmenchefin selbst ausgelöste Sprengung verursacht wurde. Das Shuttle der Zeus II-2 wird durch anfliegenden Trümmerteilen zerstört. Wenn nicht durch die Charaktere gewarnt, erleidet das Shuttle der Nike Station dasselbe Schicksal. Glücklicherweise ist die flüchtende Dragon Blade weit genug entfernt, um den umherfliegenden Teilen auszuweichen. \xl{} gibt Schub und schwenkt in Richtung Zeus II-2 ein. Obwohl sie bereits damit gerechnet hat, ist \ml{} von dem Ereignis erschüttert und flucht leise vor sich hin.

\subsection{Enterkommando Zeus II-2}
Zeitgleich mit dem Start der Triebwerke authentisiert sich \xl{} mit dem Transponder der Dragon Blade bei den anfliegenden Jägern der Zeus II-2 als Teil ihrer Flotte. Da ihre KI-Persönlichkeit eine Kopie der militärischen KI der USI darstellt, stehen ihr dafür alle Möglichkeiten zur Verfügung. Nach dem Einschwenken auf einen Rendezvouskurs deaktiviert sie den Reaktor, fährt alle nicht notwendigen Systeme herunter und aktiviert damit wieder die Tarnfunktionalität des Schiffs. Die Dragon Blade ist damit für die Sensorik der anderen Schiffe im Umkreis von über einem Kilometer unsichtbar. Das Cockpit ist auf minimale Beleuchtung heruntergefahren. Die Umgebung ist nur visuell durch Außenkameras sichtbar.

Eine bange Viertelstunde vergeht, bevor sich die Dragon Blade der Zeus II-2 so weit genähert hat, dass ein Bremsmanöver notwendig wird. \xl{} fährt den Reaktor hoch und startet ihr Haupttriebwerk für ein kurzes, aber hartes Bremsmanöver. Die Nahkampfwaffen der Zeus II-2 erwachen zum Leben. Hochgeschwindigkeitsgeschosse rasen auf die Dragon Blade zu. Die Dragon Blade verfügt über einen Torpedoschacht, acht Torpedos und vier Railgun-Geschütze, mit denen ein Bordschütze die Zeus unter Beschuss nehmen kann.

Zwei Minute später schlägt die Dragon Blade auf der Zeus II-2 auf und krallt sich an der Bordwand fest. Über einen Andocktunnel kann das Enterkommando bestehend aus den Ermittlern, \xl{} und \ml{} die Bordwand des Kreuzers betreten. Der Andockpunkt ist nicht optimal was bedeutet, dass das Enterkommando sich einen Zugang zum Schlachtschiff an einer anderen Stelle suchen muss. Die Dragon Blade verfügt über zwei Exoskelette ausgestattet mit je einer Plasmaschleuder und einem Plasmabrenner. Der Plasmabrenner kann dazu genutzt werden ein Loch in die Außenhülle der Zeus II-2 zu brennen. Einen günstigen Zugangsort bietet eine Wartungsschleuse in etwa 50 Meter Entfernung. Mittels Kletterseile und Magnetstiefeln kann sich die Gruppe über die Oberfläche des Schiffs bewegen. Auf halbem Weg wird die Gruppe von zwei spinnenartigen KI Droiden von zwei Seiten aus angegriffen. Die Droiden sind im Kapitel "`Personenverzeichnis"' unter Guardian Klasse Schlachtkreuzer beschrieben. An der Wartungsschleuse kann die Gruppe entweder versuchen die Verriegelung zu knacken oder die Schleuse aus der Bordwand mittels des Plasmabrenners zu schneiden.

\begin{remarks}
	Die Zeus II-2 ist während des Enterns durch die Gruppe nach wie vor im Gefecht mit der Donar. Sie passt also den Flugvector an das Kampfgeschehen an. Dabei muss die Gruppe damit kämpfen durch Flugmanöver durch die Gegend geworfen zu werden.

	Der Einstieg über eine Schleuse und der Angriff durch die Droiden sind nur Vorschläge und können je nach dem Verhalten der Spieler angepasst werden.
\end{remarks}
