%% Copyright 2019 Bernd Haberstumpf
%% License: CC BY-NC
% !TeX spellcheck = de_DE
\pageimage{images/plane9.jpg}
\newsection{Die letzte Schlacht}\anchor{sec:lastbattle}

Die ``letzte Schlacht'' in dieser Form findet statt, wenn die Charaktere mit \xl{} zusammen auf der Dragon Blade die Zeus II-2 entern. \xl{} hat hierf"ur ein weiteres Ass im "Armel, um die Zeus KI zu "uberzeugen sich auf ihre Seite zu schlagen. Ein alternatives Ende bei dem sich die Ermittler gegen \xl{} stellen ist in dieser Geschichte nicht ausgearbeitet.

\newsubsection{Das Ende der Plane 9}
Um unerkannt auf die Plane 9 zu kommen haben \xl{} und \ml{} ihr Schiff im Raumanzug und mit Jetpacks ausgestattet verlassen. Seitdem treibt die Dragon Blade mit heruntergefahrenen Systemen nahezu bewegungslos durchs All. Steht die Dawn of Day zur Verf"ugung, kann die Gruppe mithilfe ihres Schiffes zur Dragon Blade "ubersetzen. Ist dies nicht der Fall, kann \xl{} ihr Schiff per Richtstrahlfunk aktivieren und es zu sich holen.

Mit einem kontrollierten Man"over bringt \xl{} die Dragon Blade weiter von der im All treibenden Plane 9 und von der Dawn of Day weg. W"ahrend der Reaktor hochf"ahrt, spricht \xl{} das Unvermeidliche aus:

\speak{Ruhe in Frieden Professor. Was f"ur ein tragisches Ende \dots{} Festhalten!}

Der im All treibende Ring zerbirst in einem Feuerball. Aus der Entfernung l"asst sich nicht feststellen, ob dies durch einen Torpedo der Donar oder durch eine von der Firmenchefin selbst ausgel"oste Sprengung verursacht wurde. Das Shuttle der Zeus II-2 wird durch anfliegenden Tr"ummerteilen zerst"ort. Wenn nicht durch die Charaktere gewarnt, erleidet das Shuttle der Nike Station dasselbe Schicksal. Gl"ucklicherweise ist die fl"uchtende Dragon Blade weit genug entfernt, um den umherfliegenden Tr"ummern auszuweichen. \xl{} gibt Schub und schwenkt in Richtung Zeus II-2 ein. Obwohl sie bereits damit gerechnet hat, ist \ml{} von dem Ereignis ersch"uttert und flucht mit Tr"anen im Gesicht leise vor sich hin.

\newsubsection{Enterkommando Zeus II-2}
Zeitgleich mit dem Start der Triebwerke authentifiziert sich \xl{} "uber den Transponder der Dragon Blade bei den anfliegenden J"agern der Zeus II-2 als Teil ihrer Flotte. Da ihre KI-Pers"onlichkeit eine Kopie der milit"arischen KI der USI ist, stehen ihr daf"ur die entsprechenden M"oglichkeiten zur Verf"ugung. Nach dem Einschwenken auf den Rendezvouskurs deaktiviert sie den Reaktor, f"ahrt alle nicht notwendigen Systeme herunter und reaktiviert die Tarnvorrichtungen des Schiffs. Dadurch wird die Dragon Blade f"ur die Sensoren der umliegenden Schiffe im Umkreis von "uber einem Kilometer unsichtbar. Das Cockpit wird auf minimale Beleuchtung reduziert, und die Umgebung ist nur noch visuell "uber Au\3enkameras sichtbar.

Eine bange Viertelstunde vergeht, bis die Dragon Blade der Zeus II-2 so nahe gekommen ist, dass ein Bremsman"over erforderlich wird. \xl{} f"ahrt den Reaktor hoch und aktiviert das Haupttriebwerk f"ur ein kurzes, aber intensives Bremsman"over. Die Nahkampfwaffen der Zeus II-2 erwachen zum Leben, und Hochgeschwindigkeitsgeschosse rasen auf die Dragon Blade zu. Die Dragon Blade ist mit einem Torpedoschacht, acht Torpedos und vier Railgun-Gesch"utzen ausgestattet, mit denen ein Bordsch"utze die Zeus unter Beschuss nehmen kann.

Zwei Minuten sp"ater dockt die Dragon Blade an der Bordwand der Zeus II-2 an und verankert sich dort. "Uber einen Andocktunnel kann das Enterkommando, bestehend aus den Ermittlern, \xl{} und \ml{}, das Schlachtschiff betreten. Da die Dragon Blade nicht optimal positioniert ist, was bedeutet, nicht direkt "uber einer Schleuse gelandet ist, muss das Enterkommando zun"achst einen Zugang zum Inneren des Schlachtschiffs finden. Die Dragon Blade verf"ugt "uber zwei Exoskelette, die jeweils mit einer Plasmaschleuder und einem Plasmabrenner ausger"ustet sind. Der Plasmabrenner kann dazu verwendet werden, ein Loch in die Au\3enh"ulle der Zeus II-2 zu schneiden. Ein g"unstiger Zugangspunkt ist eine Wartungsschleuse in etwa 50 Metern Entfernung. Mittels Kletterseilen und Magnetstiefeln kann sich die Gruppe "uber die Oberfl"ache des Schiffs bewegen. Auf halbem Weg wird die Gruppe von zwei spinnenartigen Kampfrobotern aus zwei Richtungen angegriffen. Die Kampfroboter sind \cref{sec:guardian} beschrieben. An der Wartungsschleuse angekommen kann die Gruppe entweder versuchen, die Verriegelung zu knacken, oder die Schleuse mit den Plasmabrennern aus der Bordwand zu schneiden.

\begin{remarks}
	Die Zeus II-2 befindet sich w"ahrend des Enterman"overs der Gruppe weiterhin im Gefecht mit der Donar und passt ihren Flugvektor entsprechend dem Kampfgeschehen an. Dadurch wird die Gruppe durch die Flugman"over immer wieder herumgeworfen.

	Der Einstieg "uber eine Schleuse und der Angriff durch die Kampfroboter sind lediglich Vorschl"age und k"onnen je nach dem Verhalten der Spieler angepasst werden.
\end{remarks}
