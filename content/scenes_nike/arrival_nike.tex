%% Copyright 2019 Bernd Haberstumpf
%% License: CC BY-NC
% !TeX spellcheck = de_DE
\newsection{Neuro Intelligence auf Nike}

Die Nike Station ist der Verwaltungssitz der Cynarian Coopertation im Jovianischen System. Neben vielen anderen Einrichtungen beherbergt sie auch den Stammsitz von Neuro Intelligence und damit den Ursprung der Operation P9. Wie die Charaktere bei der Befragung von \ml{} erfahren haben ist Neuro Intelligence der Wirkungsbereich ihrer Chefin Prof.~Dr.~Naratova, der Leiterin des Instituts. Bei Neuro Intelligence auf der Nike Station wurden die KI Nanobots hergestellt, um sie auf Kallisto zu implantieren. Alle noch vorhandenen Fertigungsdaten, Apparaturen und Bestandteile des KI-Systems liegen auf Nike. Neuro Intelligence ist auch der Arbeitsplatz von \ml{} an dem sie die KI auf den Einsatz im menschlichen Gehirn angepasst und sp"ater die USI Bindung entfernt hat.

Nike ist eine Kombination aus Zylinder- und Ringhabitat. Um eine zentrale Nabe sind 9 Ringe, \emph{Planes} genannt angeordnet. Jeder Ring ist jeweils zwei Stockwerke mit je zwei Ebenen hoch. Die zentrale Nabe beherbergt am unteren Ende das Raumdock der Station.  Die einzelnen Planes k"onnen nur durch Aufz"uge und R"ohren in der Nabe erreicht werden. In der Nabe herrscht weitgehend Schwerelosigkeit. Innerhalb der Nabe unterhalten mehrere Forschungseinrichtungen Zero-Gravitiy Labore. Die untersten drei Planes werden von der Verwaltung der Cynarian Dependance im Jovianischen System belegt. Dar"uber befinden sich Einrichtungen von Cynarian und anderen Unternehmen.

Im obersten Stock der Plane 9 ist Neuro Intelligence untergebracht. Der Ring der Plane 9 kann durch die Aufz"uge in den vier Speichen des Rings erreicht werden. Die Aufz"uge laufen innerhalb des Rings in einem Schacht bis zum "`Boden"' des Ringes und enden in einem den Ring umlaufenden 15 Meter breiten Korridor. Zu beiden Seiten des Korridors k"onnen weitere R"aume betreten werden. Die R"aume in der zweiten Ebene erreicht man "uber Treppen zu einer Galerie. Der Korridor ist mit Pflanzenk"ubeln dekoriert. Auf der Station zugewandten Seite befinden sich im Erdgeschoss Produktionsst"atten und in der ersten Ebene Labore. Auf der dem Weltall zugewandten Seite befinden sich Wohnr"aume und B"uros. F"ur die Evakuierung der Station sind am Ring Notfallkapseln f"ur alle Mitarbeiter angedockt, die "uber den Mittelgang bestiegen werden k"onnen. Aus Sicherheitsgr"unden k"onnen die Nabe der Planes einzeln vom Rest der Station abgesprengt werden. Die so abgekoppelten Planes treiben dann eigenst"andig im All weg vom Rest der Station. Die Nabe der Plane 9 endet an meinem Weltraumobservatorium. Der Raum hat die Form einer Kugel. Der dem Weltall zugewandte Teil ist dabei komplett verglast. In der Mitte des Raumes ist eine Konstruktion mit Konsolen und Liegen aufgeh"angt. Im ganzen Raum herrscht Schwerelosigkeit. Der Raum selbst kann nur durch zwei Druckschotts von der Nabe aus betreten werden. Die Druckschotts befinden sich etwas versteckt im hinteren Bereich von zwei Laboren und sind mit Magschl"ossern gesichert. Das Observatorium ist zur Zeit von Neuro Intelligence belegt. Alle von Neuro Intelligence belegten R"aume haben in den letzten Wochen eine Modifikation erhalten. Sie k"onnen durch Sprengladungen vollst"andig zerst"ort werden.


\newsection{Eintreffen auf Nike}

Nach zwei Tagen in etwa zeitgleich mit der Zeus II-2 und dem zweiten Gro\3kampfschiff des Protektorats der Donnar erreicht die Gruppe die Nike Station mit einer Geschwindigkeit von rund 5000 km/h und einem Abstand vom 1000 km. Der leichten Kreuzer Hyperion ist im Orbit der Station zur Sicherung der Anlage abgestellt. Zum Zeitpunkt der Ankunft ist der Kampf um Valhalla in vollem Gange und das Protektorat befindet sich im Krieg. Parallel zum Anflug beginnt ein Gefecht zwischen der Donnar und der Zeus II-2. Raumj"ager treffen auf Raumj"ager. Die Station wird dabei nicht in das Kampfgeschehen mit einbezogen. Die Hyperion verh"alt sich neutral.

\subsection{Die Dragon Blade dockt an}
W"ahrend des Anflugs auf Nike fliegt die Dragon Blade unerkannt wegen ihrer Tarnummantelung im "`Windschatten"' der Dawn of Day. Startet die Dawn of Day ihr Bremsman"over startet die Dragon Blade ebenfalls ihr Haupttriebwerk und taucht damit auf dem Display der Dawn of Day in n"achster N"ahe auf. Der Ann"aherungssensor der Dawn of Day meldet pl"arrend einen Kollisionsalarm. Kurz darauf wird das Schiff durchgesch"uttelt und kommt ins Trudeln bevor sie sich mit Unterst"utzung durch die Dragon Blade wieder auf den Kurs auspendelt. Die Dragon Blade ist jetzt nicht mehr auf den Sensoranzeige erkennbar. Sie hat sich an der Dawn of Day angeklammert. Auf den Au\3enbord Kameras ist nur eine schwarze Fl"ache sehen. Die Dragon Blade is n"amlich deutlich gr"o\3er als die Dawn of Day.

\xl{} meldet sich:

\speak{Ich hoffe, ihr k"onnt einen guten Grund vorweisen um an der Station anzudocken.}

Offensichtlich will sie die Dawn of Day als Tarnung f"ur ihren Anflug nutzen.

In Sensorreichweite der Kampfschiffe und der Station wird die Dawn of Day erfasst und von der Flugkontrolle der Station kontaktiert. Wurde der Transfer zur Nike Station im Vorfeld angek"undigt und genehmigt bekommen die Ermittler einen Landeslot auf der Station. Wurde das Schiff nicht angek"undigt, wird eine Landung ohne triftigen Grund verweigert. Wird das Schiff von der Zeus II-2 als Dawn of Day erkannt, werden sie von einem J"ager des Kreuzers angegriffen. Die Hyperion nimmt in diesem Fall den J"ager unter Beschuss, wenn er in Schussreichweite ihrer Bordgesch"utze kommt. Ein entsprechender Funkverkehr l"asst sich beim Ausspielen des Raumkampfes einflechten. Die Dawn of Day unter dieser Kennung ist auch Lord Commander Steeler dem Kommandanten der Donar bekannt. Er kontaktiert die Dawn of Day und fragt, was sie ausgerechnet zur Nike Station f"uhrt. Er fordert dabei mit dem Soldaten an Bord alleine sprechen zu k"onnen.

Gelinkt es den Charakteren Nike anzufliegen k"onnen sie am Raumdock oder auch direkt an der Plan 9 andocken. Die Dragon Blade wird sich in n"achster N"ahe der Plane 9 ausklinken, und am Ring der Plane 9 festzumachen.

\begin{remarks}
	Beim Anflug auf Nike kann es zu einem Kampf mit den J"agern der Zeus II-2 kommen. Ein solcher Kampf sollte sich, wenn m"oglich auf ein kurzes Intermezzo beschr"anken. Die J"ager der Donnar und die Hyperion werden im Zweifel zur Unterst"utzung mit eingreifen. 
	
	Die KI Raumj"ager der Zeus II-2 sind mit Railgun Gesch"utzen ausgestattet und deutlich wendiger als die Dawn of Day und die Dragon Blade. Das Schiff der Ermittler h"atte bei einem l"angeren Gefecht nur geringe Chancen. Der Kampf kann insoweit narrativ gespielt werden als, dass sich der Spielleiter die Flugman"over, Schub auf den Steuerd"usen, Einsatz des Haupttriebwerkes beschreiben l"asst. Das Haupttriebwerk schleudert einen mehrere Kilometer langen Plasmastrahl in den Weltraum. Die Bordgesch"utze k"onnen daf"ur aber nicht nach hinten abgefeuert werden. Einschlagende Geschosse durchschlagen die Bordwand auf der einen Seite und treten auf der anderen wieder aus, Steuertriebwerke, Gesch"utze oder der Hauptreaktor erleiden Schaden. Die zur Reparatur eingeteilten Charaktere werden durch die schnellen G-Wechsel durch das Schiff geschleudert.

	W"ahrend die Dawn of Day und die Dragon Blade als Tandem fliegen wird \xl{} die Flugsteuerung der beiden Schiffe "ubernehmen.

	Die Charaktere k"onnen als Grund des Besuchs die Sicherung von wichtigen Informationen zu den Attentaten nennen. Sie sollten aber nicht erw"ahnen einen oder mehrere an den Attentaten beteiligte Personen nach Nike zu bringen, um eine zwangsl"aufige "Ubergabe der Gefangenen auf Nike zu vermeiden.
\end{remarks}
\newpage
