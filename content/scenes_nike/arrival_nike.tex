%% Copyright 2019 Bernd Haberstumpf
%% License: CC BY-NC
% !TeX spellcheck = de_DE
\newsection{Neuro Intelligence auf Nike}\anchor{sec:nike}

Die Nike Station ist der Verwaltungssitz der Cynarian Corporation im jovianischen System. Neben vielen anderen Einrichtungen beherbergt sie auch den Stammsitz von Neuro Intelligence und damit den Ursprung der Operation P9. Wie die Charaktere bei der Befragung von \ml{} erfahren haben, ist Neuro Intelligence der Wirkungsbereich ihrer Chefin Prof.~Dr.~Naratova, der Leiterin des Instituts. Bei Neuro Intelligence auf der Nike Station wurden die KI-Nanobots hergestellt, die auf Kallisto implementiert wurden. Alle Fertigungsdaten, Apparaturen und Bestandteile des KI-Systems befinden sich auf Nike. Neuro Intelligence ist auch der Arbeitsplatz von \ml{}, an dem sie die KI auf den Einsatz im menschlichen Gehirn angepasst und sp"ater die USI-Bindung entfernt hat.

Nike ist eine Kombination aus Zylinder- und Ringhabitat mit einem Durchmesser von 2 Kilometern und einer L"ange von 300 Metern. Die Station beherbergt rund 40'000 Einwohner. Um eine zentrale Nabe sind 9 Ringe, \emph{Planes} genannt, angeordnet. Jeder Ring ist vier Stockwerke hoch, wobei jeder Stock zwei Ebenen umfasst. 

Die unterste Plane beherbergt am unteren Ende das zentrale Raumdock der Station. Die einzelnen Ringe k"onnen durch die Nabe und durch R"ohren zwischen den Ringen erreicht werden. In der Nabe herrscht weitgehend Schwerelosigkeit. Mehrere Forschungseinrichtungen innerhalb der Nabe unterhalten Zero-Gravity-Labore. Die untersten drei Ringe werden von der Cynarian-Dependance im jovianischen System belegt. Dar"uber befinden sich weitere Einrichtungen von Cynarian und anderen Unternehmen.

Die Ringe k"onnen von der Nabe aus durch die Aufz"uge in den vier Speichen des Rings erreicht werden. Die Aufz"uge verlaufen innerhalb des Rings in einem Schacht bis zur Au\3enseite des Rings und enden in einem umlaufenden 8 Meter breiten Korridor in jedem Stockwerk. Zu beiden Seiten des Korridors k"onnen weitere R"aume betreten werden. Die R"aume in der zweiten Ebene erreicht man "uber Treppen zu einer Galerie. Die W"ande sind mit Pflanzen dekoriert, um ein besseres Raumklima zu schaffen. Auf den der Nabe zugewandten oberen Stockwerken befinden sich neben Wohnr"aumen Produktionsst"atten und Labore. Auf der dem Weltall zugewandten Seite befinden sich haupts"achlich Wohnr"aume und B"uros. Neben dem Raumdock hat jede der Planes eigene kleine Anlegepunkte f"ur Raumfahrzeuge, die im Wesentlichen f"ur Lieferungen von Material bestimmt sind. Von den Anlegepunkten aus kann durch Schleusen der zentrale Korridor des Rings auf dem untersten, dem Weltall zugewandten Stockwerk betreten werden.

F"ur die Evakuierung der Station sind am Ring Notfallkapseln angedockt, die "uber den Mittelgang bestiegen werden k"onnen. Aus Sicherheitsgr"unden k"onnen die Planes einzeln vom Rest der Station abgesprengt werden. Die so abgekoppelten Planes treiben dann eigenst"andig im All, weg vom Rest der Station.

Die Nabe der Plane 9 endet an einem Observatorium. Der Raum hat die Form einer Kugel. Der dem Weltall zugewandte Teil ist komplett verglast. In der Mitte des Raumes ist eine Konstruktion mit Konsolen und Liegen aufgeh"angt. Im gesamten Raum herrscht Schwerelosigkeit. Der Raum selbst ist durch Druckschotts von der Nabe getrennt, die mit Magnetschl"ossern gesichert sind.

Im obersten Stockwerk der Plane 9 ist Neuro Intelligence untergebracht. Neuro Intelligence belegt auch derzeit das Observatorium. Alle von Neuro Intelligence belegten R"aume haben in den letzten Wochen eine Modifikation erhalten. Dort installierte Sprengladungen k"onnen die Plane 9 vollst"andig zerst"oren.

\newsection{Eintreffen auf Nike}

Nach zwei Tagen, ungef"ahr zeitgleich mit der Zeus II-2 und dem zweiten Gro\3kampfschiff des Protektorats, der Donar, erreicht die Gruppe die Nike Station mit einer Geschwindigkeit von etwa 5000 km/h und einem Abstand von 10'000 Kilometern. Der leichte Kreuzer Hyperion der Cynarian Corporation ist im Orbit der Station stationiert, um die Anlage zu sichern. Der Kampf um Valhalla ist zum gegenw"artigen Zeitpunkt in vollem Gange, und das Protektorat ist bereits im Krieg. Parallel zum Anflug beginnt ein Gefecht zwischen der Donar und der Zeus II-2. Raumj"ager begegnen sich im Gefecht, wobei die Station selbst nicht in das Kampfgeschehen einbezogen wird. Die Hyperion verh"alt sich neutral und positioniert sich zwischen der Station und den beiden k"ampfenden Gro\3kampfschiffen, um anfliegende Wrackteile so gut wie m"oglich abzufangen.

\newsubsection{Die Dragon Blade dockt an}
W"ahrend des Anflugs auf Nike fliegt die Dragon Blade unerkannt durch ihre Tarnummantelung im ``Windschatten'' der Dawn of Day. Als die Dawn of Day ihr Bremsman"over startet, z"undet auch die Dragon Blade kurz ihr Haupttriebwerk und taucht damit auf dem Display der Dawn of Day in unmittelbarer N"ahe auf. Der Ann"aherungssensor der Dawn of Day schl"agt Alarm wegen einer drohenden Kollision. Kurz darauf wird das Schiff durchgesch"uttelt und ger"at ins Trudeln. Die Dragon Blade hat sich am Rumpf der Dawn of Day festgekrallt. Das Kampfschiff hat bereits seinen Reaktor heruntergefahren, bevor der Pilot der Dawn of Day das Schiff wieder unter Kontrolle gebracht hat. Mit Unterst"utzung durch die Man"ovrierd"usen der Dragon Blade pendelt sich die Dawn of Day wieder auf ihren urspr"unglichen Kurs ein. Die Dragon Blade ist nun auf den Sensoranzeigen nicht mehr erkennbar. Auf den Au\3enbordkameras ist nur eine schwarze Fl"ache zu sehen.

\xl{} meldet sich "uber Funk:

\speak{Entschuldigt. \pinyin{Biing1 bu4 yan4 zha4}. \pinyin{Bei4 hou4 tong3 dao1 zi}. Ich hoffe, ihr habt mein Schiff nicht angek"undigt und f"ur euer Schiff einen guten Grund mitgebracht, um an der Station anzudocken.}

Offensichtlich will sie die Dawn of Day als Tarnung f"ur ihren Anflug nutzen. Hat die Gruppe den ersten Schrecken "uberwunden, meldet sich \xl{} erneut:

\speak{Ich werde euch jetzt einen Anflugvektor "ubertragen. Der Kurs f"uhrt uns l"angs zur Station, an der Plane 9 vorbei. Nahe der Station werde ich die Dragon Blade abkoppeln und Neuro Intelligence einen Besuch abstatten. Wenn ihr mitkommen wollt, dann bitte ohne Begleitschutz.}

Kurz darauf werden auf der Dragon Blade alle Systeme au\3er den passiven Sensoren und der Lebenserhaltung St"uck f"ur St"uck heruntergefahren.

Einem aufmerksamen Zuh"orer wird vermutlich aufgefallen sein, dass \xl{} \ml{} nicht erw"ahnt hat. Fordern die Ermittler ein Lebenszeichen von \ml{}, meldet sie sich direkt zu Wort:

\speak{Mir geht's gut. Wir gehen zur Station, holen mein Omni-Slate und sind dann wieder weg. Sch"uttelt die Kerle von Cynarian ab und kommt zu uns, dann k"onnt ihr alles Weitere erfahren. Haltet euer Schiff bereit.}

Geben die Charaktere zu erkennen, dass sie nichts Verwertbares haben, um die Cynarian zu besch"aftigen, bietet \ml{} an, ihnen eine abgewandelte Version des KI-Codes zu "ubermitteln, der zwar als USI-Code erkennbar, aber letztendlich unbrauchbar ist.

In Sensorreichweite der Nike Station wird die Dawn of Day von der Sensorphalanx der Station erfasst und von der Flugkontrolle kontaktiert. Wurde der Transfer zur Nike Station im Vorfeld angek"undigt und genehmigt, erh"alt die Crew von Fluglotse \emph{Hitesh Devi} einen Landeplatz am Raumdock zugewiesen. Wurde das Schiff nicht angek"undigt, wird eine Landung ohne triftigen Grund verweigert. Auf der Station wurde bereits der Notstand ausgerufen. Gelingt es den Charakteren, Hitesh Devi von der Dringlichkeit einer Landung zu "uberzeugen, wird ihnen ein Landeplatz am Raumhafen der Station zugewiesen. Mit einer "uberzeugenden Begr"undung k"onnte Hitesh auch einen Anflug zu einem der Andockpunkte der Plane 9 genehmigen.

Die sich im Kampf befindlichen Schlachtschiffe nehmen von dem anfliegenden Tandem aus Dawn of Day und Dragon Blade keine Notiz.

Die Dragon Blade klinkt sich in unmittelbarer N"ahe der Plane 9 aus. Nach einem kurzen Schubman"over mit den Steuerd"usen entfernt sie sich und ist kurz darauf nicht mehr zu sehen. Ohne die Dragon Blade kann die Dawn of Day direkt an der Plane 9 andocken oder zum Raumdock weiterfliegen.
\vfill

\begin{remarks}
	Die Ermittler k"onnen im Folgenden entweder direkt an Plane 9 festmachen oder weiter zum Raumdock fliegen. Beide Optionen werden in den folgenden Kapiteln beschrieben.

	\underline{Notstand:}

	W"ahrend des Anflugs der Ermittler wird auf der Nike Station aufgrund eines drohenden Angriffs der Notstand ausgerufen. Man bereitet sich auf eine Evakuierung vor und versucht, alle anfliegenden Schiffe abzuweisen. Die Charaktere k"onnen als Grund ihres Besuchs die Sicherung wichtiger Informationen zu den Attentaten angeben. Mithilfe von \ml{} h"atten sie sogar brauchbares Material in der Hand.

	\underline{Andocken an Plane 9:}
	
	F"ur ein Andocken direkt an Plane 9 ist die Kreativit"at der Spieler gefragt. Hier sollte der Spielleiter nicht einfach eine Freigabe erteilen.

	\underline{Gegenseitige Abh"angigkeit:}

	Unter den gegebenen Umst"anden ist \xl{} auf die Unterst"utzung durch die Dawn of Day angewiesen und hofft, auch bei ihrem Coup zur "Ubernahme der Zeus II-2 auf Unterst"utzung durch die Charaktere z"ahlen zu k"onnen. 
	
	Um auf der anderen Seite die Sicherheitskr"afte auf der Station nicht aufzuschrecken, muss die Gruppe den Anflugplan von \xl{} decken.

	\underline{Chinesische Redewendungen:}

	Die im ersten Dialog verwendeten chinesischen Redewendungen:
	
	\begin{itemize}
		\item \say{\pinyin{Biing1 bu4 yan4 zha4}} bedeutet \say{Im Krieg ist jede List erlaubt.}
		\item \say{\pinyin{Bei4 hou4 tong3 dao1 zi}} bedeutet \say{Ein Messer in den R"ucken rammen.}
	\end{itemize}
\end{remarks}
\vfill
\pagebreak