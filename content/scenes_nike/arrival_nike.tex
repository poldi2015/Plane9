%% Copyright 2019 Bernd Haberstumpf
%% License: CC BY-NC
% !TeX spellcheck = de_DE
\newsection{Neuro Intelligence auf Nike}

Die Nike Station ist der Verwaltungssitz der Cynarian Coopertation im Jovianischen System. Neben vielen anderen Einrichtungen, beherbergt sie auch den Stammsitz von Neuro Intelligence und damit den Ursprung der Operation P9. Wie die Charaktere, bei der Befragung von \ml{}, erfahren haben ist Neuro Intelligence der Wirkungsbereich ihrer Chefin Prof.~Dr.~Naratova, der Leiterin des Instituts. Bei Neuro Intelligence auf der Nike Station wurden die KI-Nanobots hergestellt, um sie auf Kallisto zu implantieren. Alle Fertigungsdaten, Apparaturen und Bestandteile des KI-Systems liegen auf Nike. Neuro Intelligence ist auch der Arbeitsplatz von \ml{}, an dem sie die KI auf den Einsatz im menschlichen Gehirn angepasst und sp"ater die USI Bindung entfernt hat.

Nike ist eine Kombination aus Zylinder- und Ringhabitat mit einem Durchmesser von 2 Kilometern und einer L"ange "uber alle Ringe von 300 Metern. Nike beherbergt rund 40'000 Einwohnern. Um eine zentrale Nabe sind 9 Ringe, \emph{Planes} genannt, angeordnet. Jeder Ring ist jeweils vier Stockwerke mit je zwei Ebenen hoch. Die unterste Plane beherbergt am unteren Ende das zentrale Raumdock der Station.  Die einzelnen Ringe k"onnen durch R"ohren in der Nabe und durch R"ohren zwischen den Ringen erreicht werden. In der Nabe herrscht weitgehend Schwerelosigkeit. Innerhalb der Nabe unterhalten mehrere Forschungseinrichtungen Zero-Gravitiy Labore. Die untersten drei Ringe werden von der Cynarian Dependance im Jovianischen System belegt. Dar"uber befinden sich weitere Einrichtungen von Cynarian und auch anderen Unternehmen.

Die Ringe k"onnen durch die Aufz"uge in den vier Speichen des Rings erreicht werden. Die Aufz"uge laufen innerhalb des Rings in einem Schacht bis zur Au\3enseite des Ringes und enden in einem in jedem Stockwerk umlaufenden 8 Meter breiten Korridor. Zu beiden Seiten des Korridors k"onnen weitere R"aume betreten werden. Die R"aume in der zweiten Ebene erreicht man "uber Treppen zu einer Galerie. Die W"ande sind mit Pflanzen dekoriert, um ein besseres Raumklima zu schaffen. Auf denm der Nabe zugewandten oberen Stockwerken befinden sich neben Wohnr"aumen im Erdgeschoss, Produktionsst"atten und in der ersten Ebene Labore. Auf der dem Weltall zugewandten Seite befinden sich haupts"achlich Wohnr"aume und B"uros. Neben dem Raumdock hat jede der Planes eigene kleine Anlegepunkte f"ur Raumfahrzeuge die im Wesentlichen f"ur Lieferungen von Material bestimmt sind. Von den Anlegepunkten aus kann, durch Schleusen der zentrale Korridor des Rings auf dem untersten, dem Weltall zugewandten Stockwerk, der Ring betreten werden.

F"ur die Evakuierung der Station sind am Ring Notfallkapseln angedockt, die "uber den Mittelgang bestiegen werden k"onnen. Aus Sicherheitsgr"unden k"onnen die Planes einzeln vom Rest der Station abgesprengt werden. Die so abgekoppelten Planes treiben dann eigenst"andig im All, weg vom Rest der Station. 

Die Nabe der Plane 9 endet an einem Observatorium. Der Raum hat die Form einer Kugel. Der dem Weltall zugewandte Teil ist dabei komplett verglast. In der Mitte des Raumes ist eine Konstruktion mit Konsolen und Liegen aufgeh"angt. Im ganzen Raum herrscht Schwerelosigkeit. Der Raum selbst ist nur durch Druckschotts von der Nabe getrennt. Die Druckschotts sind mit Magschl"ossern gesichert. 

Im obersten Stock der Plane 9 ist Neuro Intelligence untergebracht. Neuro Intelligence belegt auch derzeit das Observatorium. Alle von Neuro Intelligence belegten R"aume haben in den letzten Wochen eine Modifikation erhalten. Sie k"onnen durch Sprengladungen vollst"andig zerst"ort werden.


\newsection{Eintreffen auf Nike}

Nach zwei Tagen, in etwa zeitgleich mit der Zeus II-2 und dem zweiten Gro\3kampfschiff des Protektorats der Donnar, erreicht die Gruppe die Nike Station mit einer Geschwindigkeit von rund 5000 km/h und einem Abstand vom 10'000 km. Der leichte Kreuzer Hyperion der Cynarian Corporation ist im Orbit der Station zur Sicherung der Anlage abgestellt. Zum gegenw"artigen Zeitpunkt ist der Kampf um Valhalla in vollem Gange und das Protektorat befindet sich bereits im Krieg. Parallel zum Anflug beginnt ein Gefecht zwischen der Donnar und der Zeus II-2. Raumj"ager treffen auf Raumj"ager. Die Station wird dabei nicht in das Kampfgeschehen einbezogen. Die Hyperion verh"alt sich neutral. Sie schiebt sich zwischen Station und die beiden im Kampf verwickelten Kampfschiffe, um so gut es geht anfliegende Wrackteile abzufangen.

\newsubsection{Die Dragon Blade dockt an}
W"ahrend des Anflugs auf Nike fliegt die Dragon Blade, unerkannt durch ihrer Tarnummantelung im ``Windschatten'' der Dawn of Day. Startet die Dawn of Day ihr Bremsman"over, startet die Dragon Blade ebenfalls kurz ihr Haupttriebwerk und taucht damit auf dem Display der Dawn of Day in n"achster N"ahe auf. Der Ann"aherungssensor der Dawn of Day meldet einen Kollisionsalarm. Kurz darauf wird das Schiff durchgesch"uttelt und kommt ins Trudeln. Die Dragon Blade hat sich an den Rumpf der Dawn of Day gekrallt. Das Kampfschiff hat schon seinen Reaktor heruntergefahren, bevor der Pilot der Dawn of Day das Schiff wieder unter Kontrolle gebracht hat. Mit Unterst"utzung durch die Man"ovrierd"usen der Dragon Blade, pendelt sich die Dawn of Day wieder auf ihren urspr"unglichen Kurs ein. Die Dragon Blade ist jetzt nicht mehr auf den Sensoranzeigen erkennbar. Auf den Au\3enbord Kameras ist nur eine schwarze Fl"ache zu sehen.

\xl{} meldet sich "uber Funk:

\speak{Entschuldigt. \pinyin{Biing1 bu4 yan4 zha4}. \pinyin{Bei4 hou4 tong3 dao1 zi}. Ich hoffe, ihr habt mein Schiff nicht angek"undigt und habt f"ur euer Schiff einen guten Grund mitgebracht, um an der Station anzudocken.}

Offensichtlich will sie die Dawn of Day als Tarnung f"ur ihren Anflug nutzen. Hat die Gruppe den ersten Schrecken "uberstanden, meldet sich \xl{} erneut:

\speak{Ich werde euch jetzt einen Anflugvektor "ubertragen. Der Kurs bringt uns l"angsseits zur Station, an der Plane 9 vorbei. Nahe der Station werde ich meine Dragon Blade abkoppeln und der Neuro Intelligence einen Besuch abstatten. Wenn ihr dazu kommen wollt, dann bitte ohne Begleitschutz.}

Kurz darauf werden auf der Dragon Blade alle Systeme au\3er den passiven Sensoren und der Lebenserhaltung St"uck f"ur St"uck heruntergefahren. 

Einem aufmerksamen Zuh"orer wird vermutlich aufgefallen sein, dass \xl{} \ml{} nicht erw"ahnt hat. Fordern die Ermittler ein Lebenszeichen von \ml{}, meldet die sich direkt zu Wort:

\speak{Mir geht es gut. Wir gehen auf die Station, holen mein Omni-Slate und sind wieder weg. Sch"uttelt die Kerle von Cynarian ab und kommt zu uns, dann k"onnt ihr alles Weitere erfahren. Haltet euer Schiff bereit.}

Geben die Charaktere zu erkennen, dass sie nichts Verwertbares in der Hand haben, um Cynarian zu besch"aftigen, bietet \ml{} an, ihnen eine abgewandelte Version des KI-Codes zu "ubermitteln, der als USI-Code erkennbar, aber am Ende unbrauchbar ist.

In Sensorreichweite von Nike wird die Dawn of Day durch die Sensorphalanx der Station erfasst und von der Flugkontrolle der Station kontaktiert. Wurde der Transfer zur Nike-Station im Vorfeld angek"undigt und genehmigt, bekommen die Ermittler vom Fluglotsen Hitesh Devi einen Landeplatz am Raumdock zugewiesen. Wurde das Schiff nicht angek"undigt, wird eine Landung ohne triftigen Grund verweigert. Auf der Station wurde bereits der Notstand ausgerufen. Gelingt es den Charakteren, Hitesh Devi von der Notwendigkeit einer Landung zu "uberzeugen, bekommen sie einen Landeplatz am Raumhafen der Station zugewiesen. Mit einer guten Begr"undung w"urde Hitesh auch einen Anflug an einen der Andockpunkte der Plane 9 genehmigen.

Die sich im Kampf befindlichen Schlachtschiffe nehmen von dem anfliegenden Tandem aus Dawn of Day und Dragon Blade keine Notiz.

Die Dragon Blade klinkt sich in n"achster N"ahe der Plane 9 aus. Nach einem kurzen Schubman"over mit den Steuerd"usen entfernt sie sich und ist kurz darauf nicht mehr zu sehen. Ohne die Dragon Blade kann die Dawn of Day direkt an der Plane 9 andocken oder zum Raumdock weiterfliegen.

\begin{remarks}
	W"ahrend des Anflugs der Ermittler wird auf der Nike-Station angesichts eines drohenden Angriffs der Notstand ausgerufen. Man bereitet sich auf eine Evakuierung vor und versucht, alle anfliegenden Schiffe abzuweisen. Die Charaktere k"onnen als Grund ihres Besuchs die Sicherung wichtiger Informationen zu den Attentaten angeben. Mit Hilfe von \ml{} h"atten sie sogar brauchbares Material in der Hand. 
	
	F"ur ein Andocken direkt an der Plane 9 ist die Kreativit"at der Spieler gefragt. Hier sollte der Spielleiter nicht so einfach einwilligen.

	Unter den gegebenen Umst"anden ist \xl{} auf die Unterst"utzung durch die Dawn of Day angewiesen und hofft, auch bei ihrem Coup zur "Ubernahme der Zeus II-2 auf Unterst"utzung durch die Charaktere. Um auf der anderen Seite die Sicherheitskr"afte auf der Station nicht aufzuschrecken, muss die Gruppe den Anflugplan von \xl{} decken.

	Die im ersten Dialog verwendeten chinesischen Redewendungen:
	
	\begin{itemize}
		\item \say{\pinyin{Biing1 bu4 yan4 zha4}} bedeutet \say{Im Krieg ist jede List erlaubt.}
		\item \say{\pinyin{Bei4 hou4 tong3 dao1 zi}} bedeutet \say{Ein Messer in den R"ucken rammen.}
	\end{itemize}

	Die Ermittler k"onnen im Folgenden direkt an Plane 9 festmachen oder weiter zum Raumdock fliegen. Beide Optionen werden in den folgenden Kapiteln beschrieben.
\end{remarks}
