%% Copyright 2019 Bernd Haberstumpf
%% License: CC BY-NC
% !TeX spellcheck = de_DE
\newsection{Antworten auf die letzten Fragen}

Der Zugang zum Observatorium ist \ml{} bekannt. Ist \ml{} nicht bereits anwesend, müssen sich die Ermittler selbst auf die Suche machen. Das Observatorium lässt sich mit der ID von \ml{} oder mit einem Magschloßknacker öffnen. Wenn beides nicht vorhanden ist, muss die Verriegelung überbrückt werden.

Das Observatorium bietet einen atemberaubenden Blick ins All. Dem Eintretenden hängt die majestätische Kugel des Jupiter über dem Kopf. Zwischen dem Planeten und der Station schieben sich die Kampfschiffe des Protektorats, des Konzernrates und der Cynarian Corporation in das Blickfeld. Jäger stürzen aufeinander, Salven von Hochgeschwindigkeitsgeschossen werden zwischen den Schiffen ausgeteilt. Trotz des Gefechtes ist im Observatorium nur der Stationsalarm zu hören. Die Wissenschaftlerin steht in der Mitte des Raumes auf einem erhöhten Bereich, mit dem Blick zum Weltraum gewandt, hinter einer Konsole mit ihren Magnetstiefeln am Boden verankert. Vom Zugang bis zu der Mitte des Raumes sind es rund 20 Meter. Nachdem die Gruppe den Raum betreten hat, dreht sich die Wissenschaftlerin um und wendet sich erstaunt an die Eintretenden.

\speak{Guten Tag\dots wer sind Sie? Was wollen Sie? \dots{} Sollten Sie nicht dem Alarm folgend die Station bereits verlassen haben?}

Der erste Dialog obliegt der Gruppe. Die schlanke Frau mit langen leicht ergrauten, zu einem Zopf gebundenen Haaren wirkt leicht abwesend, niedergeschlagen und ausgelaugt. Der Alarm scheint sie nicht zu erreichen. Sind \xl{} und \ml{} unabhängig von den Charakteren gekommen betreten sie unbemerkt an einer anderen Stelle den Raum. Sind die beiden zusammen mit den Ermittlern unterwegs, treten die beiden Frauen nach den Ermittlern in den Raum. Nach einem ersten Wortaustausch zwischen der Professorin und den Ermittlern treten die beiden zur Seite und machen sich damit sichtbar. \xl{} ist inzwischen mit ihrer kurzläufigen Railgun bewaffnet. Die Professorin wendet sich den beiden Frauen zu. Bisher weitestgehend emotionslos, entgleisen ihr ganz kurz die Gesichtszüge, bevor sie sich wieder unter Kontrolle hat. Zögerlich beginnt sie zu sprechen:

\speak{\ml{}. Mein Experiment war erfolgreich,\dots{} wie erfreulich! \dots{} Es ist schön, dass Ihr zu mir gekommen seid.}

\xl{} ergreift das Wort mit Sarkasmus in der Stimme:

\speak{Ich hoffe Sie hatten nie daran gezweifelt, \pinyin{Lao3} Professorin!}

\ml{} blickt unsicher zwischen den beiden hin und her. Naratova kneift die Augen zusammen. Ihre Stimme hat sich gefestigt.

\speak{Liebe \xl{}, ich würde gerne wissen, ob es zu den geplanten Verschmelzung ihres Gehirns mit meiner KI gekommen ist und \dots{} und welcher Geist die Oberhand gewonnen hat.}

Sie blickt erwartungsvoll zu \xl{}, erhält keine Antwort. \xl{} ist zwei Schritte nach hinten getreten, ihre Waffe schussbereit aber auf keine Person direkt gerichtet. Ein verschlagenes Lächeln huscht über ihr Gesicht.

In diesem Augenblick zerreißt ein Donnern die Stille. Der Boden bebt. Die ganze Plane neigt sich von den Charakteren aus gesehen nach vorne. Der plötzliche Ruck droht alle Personen vom Boden zu reißen. Ein Würfelwurf sollte bestimmen, ob jemand den Kontakt zum Boden verliert. Eine kurze rhetorische Pause kann den Spielern zu diesem Zeitpunkt die Möglichkeit eröffnen die neue Situation zu erfassen und zu bewerten. \xl{} wurde soeben als KI Hybride enttarnt. Neben \ml{} ist sie die einzige, die bereits im Vorfeld die Rollen aller anwesenden Personen kennt und hat sich entsprechend auf das unvermeidbare vorbereitet. Naratova die das Abkoppeln von der Nike Station eingeleitet hat, hat sich neben \xl{} als Erstes wieder aufgerichtet. An die Ermittler gewendet:

\speak{In wessen Auftrag sind Sie zu mir gekommen, nochmal, was wollen Sie. An die Forschungsergebnisse kommen Sie nicht heran. Die sind inzwischen nur noch in meinem Kopf gespeichert.}

\xl{} antwortet, ihr Waffe noch immer schussbereit.

\speak{Professorin, erzählen sie Ihnen was sie getan haben, was sie mit mir gemacht haben.}

Nach kurzem Zögern gibt sie der Gruppe einen kurzen Abriss der Ereignisse bis zur Flucht von \xl{}. Ein genaue Beschreibung von \xl{} und ihrer Geschichte findet sich im zweiten Teil des Buchs "`Personenverzeichnis"'.

\xl{} komplementiert die Erzählung mit einem grimmigen Blick:

\speak{Mein Geschichte kennen inzwischen nur noch die Personen in diesem Raum. Alle Ihre eingeweihten Mitarbeiter einschließlich Professor Sanders sind tot. Ehrenwerte Freunde, sagt doch bitte der Professorin was ihr vorhabt. Es bleibt nicht viel Zeit.}

Der Spielball liegt nun wieder bei den Charakteren. \xl{} ist daran gelegen die Aufmerksamkeit wieder von sich abzuwenden. Naratova hingegen ist nicht ohne weiteres geneigt die Forschungsergebnisse herauszugeben und damit das Risiko einzugehen, dass ihre Technologie nochmals in falschen Hände gerät. Die Professorin muss von den redlichen Absichten der Gruppe überzeugt werden, um ihr Wissen preis zu geben.

Mitten in der Unterhaltung erreicht ein Funkspruch der Nike Station das Observatorium. Der Kontakt ist an die ganze Plane gerichtet und liegt deshalb auf den Stationslautsprechern an.

\speak{Hier Kurt Stromberg an die Crew der Dawn of Day. Vertreter der United Space Industries behaupten auf Nike, Ebene 9 befänden sich gestohlene Forschungsergebnisse, die sie jetzt sichern werden. Wer hat die Plane von der Station abgesprengt? Ein Enterkommando des Feindes werden wir nicht dulden. Wir senden ein Shuttle. Verhalten Sie sich kooperativ. Ende.}

Kurt Stromberg der Stationsleiter der Nike Station wirkt aufgebracht. Die Plane 9 ist zum Glück zu diesem Zeitpunkt bereits evakuiert. Es befinden sich keine Sicherheitskräfte mehr an Bord. Falls sich die Gruppe nicht zurückmeldet wiederholt sich die Nachricht. 

Kommt der Angriff auf die Zeus II-2 nicht zur Sprache mischt sich \xl{} mit drängendem Unterton in das Gespräch:

\speak{Wir werden das feindliche Schiff übernehmen. \pinyin{Jiaoshou4}, geben sie den Code an \ml{}, jetzt sofort.}

\ml{} ergänzt:

\speak{Larissa, ich will auf Basis meines Codes einen Virus entwickeln, um die USI Kontrolle von den KIs zu entfernen. Bitte gib mir Zugriff auf meinen Code.}

Naratova blickt ob dieser Enthüllung entgeistert auf bis ihr \xl{}s eigentlich Plan (Die letzte Schlacht) zur Gänze bewusst wird. Sie ist hin- und hergerissen ob der weiteren Einsatzmöglichkeit der menschlichen KI Fähigkeit. Naratova blickt fragend zu \ml{}. Diese signalisiert wortlos, dass sie in den Plan eingeweiht ist und zustimmt. 

Kurz darauf meldet sich die Donar. Lord Commander Steeler der Kommandant der Donar spricht den Soldaten der Ermittler direkt über sein Kommunikationssystem an.

\speak{Lord Commander Steeler von der Donar. Was verdammt ist hier los. Ein Shuttle der Zeus ist auf dem Weg eure abgesprengte Scheibe zu kapern. Ein Shuttle von der Station ist auch unterwegs. Blackheart hat befohlen alles wegzublasen, auf was die Aasgeier von der USI scharf sein könnten. Wenn ich von euch da unten innerhalb von zwei Minuten keine klaren Antworten bekomme, seid ihr Asche. Ich hoffe ich hab mich klar ausgedrückt. Over and Out.}

Die Situation wird zunehmend brenzlig. Kann Naratova davon überzeugt, dass die Gruppe die Daten nicht an Dritte weiter gibt, ist sie bereit \ml{} die Zugangsdaten für die Computersysteme zu geben. Den Zugang zum Tresor mit den Nanobot Druckvorlagen wird sie nicht bereitstellen.

\speak{\ml{}, du weißt, ich vertraue dir. Wir haben einen großen Fehler gemacht. Weder die USI, noch Cynarian und auch nicht das Protektorat dürfen jemals die Forschungsergebnisse in die Hände bekommen. Ihr seid die letzten die diese Technologie nutzen dürfen.}

Sie überträgt \ml{} den Codes. \xl{} richtet darauf hin ihre Waffe auf die Firmenchefin. Die beiden Frauen blicken sich in die Augen. Dann senkt \xl{} das Gewehr und verlässt mit den anderen wortlos das Observatorium.

In den Räumen der Neuro Intelligence steht \ml{} ein mobiler Computer zur Verfügung, auf dem sie die KI zu einem Virus umprogrammieren kann. Der Gruppe bleibt nicht viel Zeit eine Entscheidung über das weitere Vorgehen zu treffen. \xl{} ist bereit mit den Ermittlern und ihrem Schiff zur Zeus II-2 überzusetzen und es mit der fremden KI aufzunehmen. Sie legt dazu ihre Karten offen auf den Tisch:

\begin{itemize}
	\item Ihr Schiff ist das Einzige, dass Chancen hat an ein sich verteidigendes Schlachtschiff anzudocken.
	\item Ihr Geist ist stark genug die gegnerische KI zu bezwingen. Ein menschlicher Psychonaut hat dagegen geringe Chancen.
	\item Ihr Schiff ist das einzige, dass die Ermittler sicher vor der drohenden Zerstörung der Plane 9 wegbringen kann. Eine zivile Fähre 
        wäre den Angreifern nicht gewachsen.
	\item Sie ist die Einzige die der befreiten KI der Zeus II-2 ein Angebot zur Kooperation machen kann, dass sie nicht ablehnen wird. 
		Welches das ist, wird sie allerdings nicht verraten.
    \item Neben dem was sie ihr persönlich angetan haben eröffnet sie, dass die USI ihre Eltern entehrt und vertrieben hat und sie aus all diesen die USI mit Freuden an einer Stelle Treffen will, die besonders weh tut, egal für welchen Preis.
\end{itemize}

Es obliegt den Ermittlern nun ihre Entscheidung zu treffen wie sie mit dem Virus umgehen wollen. Viel Zeit sollte der Spielleiter ihnen allerdings nicht geben. Zumindest dem Soldaten der Gruppe ist klar, dass Lord Commander Steeler in Kürze seiner Warnung Taten folgen lassen wird.

\vfill
\newpage

\begin{remarks}
	In dieser Szene werden fast alle noch offenen Fragen geklärt und die letzten Entscheidungen getroffen. Prof.~Dr.~Naratova legt ihre Geschichte offen, kann Fragen beantworten und gesteht ihr Fehlverhalten ein. Die Szene ist damit der zweite Höhepunkt der Geschichte, mit der Melancholie eines tragischen Endes. Die ganze Szene wird dominiert durch den Dialog zwischen der Firmenchefin, \xl{}, \ml{} und den Rückmeldungen von der Donar und der Nike Station. Der Dialog muss durch den Spielleiter zügig vorgetragen werden, um den Spannungsbogen aufrecht halten zu können.

	Je nachdem in welchem Kontakt die Charaktere vor dem Eintreffen auf Nike mit der Stationsleitung und der Donar waren, müssen die Funksprüche der beiden Parteien angepasst werden.

	Wenn Prof.~Dr.~Naratova \xl{} enttarnt ist es möglich einem Charakter der vorher im Asteroidengürtel zwischen Mars und Jupiter tätig war zu erkennen, dass \xl{} die berüchtigte Piratin ist die auf Valhalla gefangen genommen wurde und kann daraus schließen, dass sie dort in der Gefangenschaft die KI implantiert bekommen hat.

	Beim Blickaustausch zwischen der Wissenschaftlerin und \xl{} erkennt \xl{}, dass Naratova vorhat ihre Erfindungen zu vernichten und sich selbst zu töten. Sie verzichtet deswegen darauf sie auszuschalten.

	Für das weitere Vorgehen bleiben den Charakteren zwei Optionen. Entweder sie bitten \ml{} einen Virus zu erstellen der die feindliche KIs zerstört. Oder sie entwickelt eine Modifikation der den USI Code entfernt. Die endgültige Entscheidung unterliegt am Ende \ml{}. Niemand außer ihr weiß vor seiner Anwendung was der Virus wirklich tun wird. \ml{} ist  demgegenüber nicht geneigt sich gegen \xl{} zu stellen. Für die Erstellung des Virus bleibt auf der Station keine Zeit. \ml{} muß dfen neunten Ring verlassen bevor sie in Aktion treten kann.
	
	\xl{} will in jedem Fall versuchen die Zeus II-2 zu kapern. Ist die Gruppe nicht bereit \xl{} zu unterstützen müssen sie sie ausschalten. \xl{} wird in diesem Falle versuchen \ml{} in ihre Gewalt zu bekommen und mit der Dragon Blade zur Zeus II-2 überzuwechseln. Gelingt es den Ermittlern nicht \xl{} zu überrumpeln, wird sie ihren Plan in die Tat umsetzen und das folgende Kapitel umsetzen. Die Charaktere müssen dann mit der Dawn of Day oder mit Rettungskapseln fliehen oder die im nächsten Kapitel beschriebene Zerstörung der Plane 9 verhindern.
\end{remarks}
