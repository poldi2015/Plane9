%% Copyright 2019 Bernd Haberstumpf
%% License: CC BY-NC
% !TeX spellcheck = de_DE
\newsection{Antworten auf die letzten Fragen}

Der Zugang zum Observatorium ist \ml{} bekannt. Ist \ml{} nicht bereits anwesend m"ussen sich die Ermittler selbst auf die Suche machen. Das Observatorium l"asst sich mit der ID von \ml{} oder mit einem Magschlo\3knacker "offnen. Wenn beides nicht vorhanden ist, muss die Verriegelung "uberbr"uckt werden.

Das Observatorium bietet einen atemberaubenden Blick ins All. Dem Eintretenden h"angt die majest"atische Kugel des Jupiter "uber dem Kopf. Zwischen dem Planeten und der Station schieben sich die Schiffe des Protektorats, des Konzernrates und der Cynarian Corporation in das Blickfeld. J"ager st"urzen aufeinander, Salven von Hochgeschwindigkeitsgeschossen werden zwischen den Schiffen ausgeteilt. Trotz des Gefechtes ist es im Observatorium totenstill. Die Wissenschaftlerin steht in der Mitte des Raumes mit dem Blick zum Weltraum gewandt hinter einer Konsole mit ihren Magnetstiefeln am Boden verankert. Vom Zugang bis zu der Mitte des Raumes sind es rund 20 Meter. Nachdem die Gruppe den Raum betreten hat, dreht sich die Wissenschaftlerin um und wendet sich an die Eintretenden.

\speak{Guten Tag. Wer sind sie? Solltet Sie nicht dem Alarm folgend die Station bereits verlassen haben?}

Der erste Dialog obliegt nun den Charakteren. Die schlanke Frau mit langen leicht ergrauten zu einem Zopf gebundenen Haaren wirkt niedergeschlagen und ausgelaugt. Der gellende Alarm scheint sie nicht zu erreichen. Sind \xl{} und \ml{} unabh"angig von den Charakteren gekommen betreten sie von der Gruppe unbemerkt oberhalb der Charaktere durch eine weitere Luke den Raum. Ihre Magnetstiefel halten sie im rechten Winkel zu den Charakteren an der Wand zum Rest der Nike Station in Position. Sind die beiden zusammen mit den Ermittlern unterwegs treten die beiden Frauen nach den Ermittlern in den Raum. Nach einem ersten Wortaustausch zwischen der Professorin und den Ermittlern treten die beiden zur Seite und machen sich damit sichtbar. \xl{} ist inzwischen mit ihrer kurz l"aufigen Railgun bewaffnet. Die Professorin wendet sich den beiden Frauen zu. Bisher weitestgehend emotionslos kommt leben in ihren K"orper.

\speak{\ml{} \dots{} \xl{}. Mein Experiment war erfolgreich, wie mir scheint! Ich freue mich, dass ihr zu mir gekommen seid.}

\xl{} ergreift das Wort mit drohender Stimme:

\speak{Ich hoffe Sie hatten nie daran gezweifelt, \pinyin{Lao3} Professorin!}

\ml{} blickt unsicher zwischen den beiden hin und her. Naratova kneift die Augen zusammen:

\speak{Liebe \xl{}, ich w"urde zu gerne wissen, ob es zu den geplanten Verschmelzung ihres Gehirns und der KI gekommen ist und welcher Geist die Oberhand gewonnen hat.}

Sie blickt erwartungsvoll zu \xl{}, erh"alt aber keine Antwort. \xl{} ist zwei Schritte nach hinten getreten ihre Waffe schussbereit aber auf keine Person direkt gerichtet. Ein verschlagenes L"acheln huscht "uber ihr Gesicht.

In diesem Augenblick zerrei\3t ein Donnern die Stille. Der Boden bebt. Die ganze Plane neigt sich von den Charakteren aus gesehen nach vorne. Der pl"otzliche Ruck droht alle Personen von den zu Boden zu sto\3en. Eine kurze rhetorische Pause kann den Spielern zu diesem Zeitpunkt die M"oglichkeit er"offnen die neue Situation zu erfassen und zu bewerten. \xl{} wurde soeben als KI Hybride enttarnt. Neben \ml{} ist sie die einzige die bereits im Vorfeld die Rollen aller anwesenden Personen kennt und hat sich entsprechend auf das unvermeidbare vorbereitet.

Naratova die das Abkoppeln von der Nike Station eingeleitet hat, hat sich neben \xl{} als erstes wieder aufgerichtet. An die Ermittler gewendet:

\speak{In wessen Auftrag sind Sie zu mir gekommen, nochmal, was wollen Sie. An die Forschungsergebnisse kommen Sie nicht heran. Die sind inzwischen nur noch in meinem Kopf gespeichert.}

\xl{} antwortet:

\speak{Ehrenwerte Freunde, sagt doch der Professorin was ihr vorhabt. Es bleibt nicht viel Zeit.}

Der Spielball liegt nun wieder bei den Charakteren. \xl{} ist daran gelegen die Aufmerksamkeit wieder von sich abzuwenden. Naratova hingegen ist nicht ohne weiteres geneigt die Forschungsergebnisse herauszugeben und damit das Risiko einzugehen, dass ihre Technologie nochmals in falschen H"ande ger"at. Die Professorin muss von den redlichen Absichten der Gruppe "uberzeugt werden, um ihr Wissen preis zu geben.

Mitten in der Unterhaltung erreicht ein Funkspruch der Nike Station das Observatorium. Der Kontakt ist an die ganze Plane gerichtet und liegt deshalb bereits auf den Lautsprechern an.

\speak{Hier Kurt Stromberg an die Crew der Dawn of Day. Vertreter der United Space Industries behaupten auf Nike, Ebene 9 bef"anden sich gestohlene Forschungsergebnisse, die sie jetzt sichern werden. Wer hat die Plane von der Station abgesprengt? Ein Enterkommando des Feindes werden wir nicht dulden. Wo sind Sie? Was ist hier los? Wir senden ein Shuttle. Verhalten Sie sich kooperativ. Ende.}

Die Plane 9 ist zum Gl"uck zu diesem Zeitpunkt bereits evakuiert. Es befinden sich keine Sicherheitskr"afte mehr an Bord. Falls sich die Gruppe nicht zur"uckmeldet wiederholt Stromberg die Nachricht. 

Kommt der Angriff auf die Zeus II-2 nicht zur Sprache mischt sich \xl{} mit dr"angendem Unterton in das Gespr"ach:

\speak{Wir werden das feindliche Schiff "ubernehmen. \pinyin{Jiaoshou4}, geben sie den Code an \ml{}, jetzt sofort.}

\ml{} erg"anzt:

\speak{Larissa, ich will auf Basis meines Codes einen Virus entwickeln, um die USI Kontrolle von den KIs zu entfernen. Bitte gib mir Zugriff auf meinen Code.}

Naratova blickt ob dieser Enth"ullung entgeistert auf bis ihr \xl{}s eigentlich Plan (Die letzte Schlacht) zur G"anze bewusst wird. Sie ist hin- und hergerissen ob der weiteren Einsatzm"oglichkeit der menschlichen KI F"ahigkeit. Naratova blickt fragend zu \ml{}. Diese signalisiert wortlos, dass sie dem Plan zustimmt. 

Kurz darauf meldet sich die Donar:

\speak{Lord Commander Steeler von der Donar. Was verdammt ist hier los. Ein Shuttle der Zeus ist auf dem Weg eure abgesprengte Scheibe zu kapern. Ein Shuttle von der Station ist auch unterwegs. Blackheart hat befohlen alles wegzublasen, auf was die Angreifer scharf sein k"onnten. Wenn ich von euch da unten innerhalb von zwei Minuten keine klaren Antworten bekomme, seid ihr Asche. Ich hoffe ich hab mich klar ausgedr"uckt. Over and Out.}

Die Situation wird zunehmend brenzlig. Kann Naratova davon "uberzeugt, dass die Gruppe die Daten nicht an Dritte weiter gibt, ist sie bereit \ml{} die Zugangsdaten f"ur die Computersysteme zu geben. Den Zugang zum Tresor mit den Nanobot Druckvorlagen wird sie aber nicht bereitstellen.

\speak{\ml{}, du wei\3t, ich vertraue dir. Wir haben einen gro\3en Fehler gemacht. Weder die USI, noch Cynarian und auch nicht das Protektorat d"urfen jemals die Forschungsergebnisse in die H"ande bekommen. Ihr seid die letzten die diese Technologie in den H"anden halten werden.}

Sie "ubertr"agt \ml{} den Codes. \xl{} richtet darauf hin ihre Waffe auf die Firmenchefin. Die beiden Frauen blicken sich lange in die Augen. Dann senkt \xl{} das Gewehr und verl"asst mit den anderen wortlos das Observatorium.

In den R"aumen der Neuro Intelligence steht \ml{} ein mobiler Computer zur Verf"ugung, auf dem sie die KI zu einem Virus umprogrammieren kann. Der Gruppe bleibt nicht viel Zeit eine Entscheidung "uber das weitere Vorgehen zu treffen. \xl{} ist bereit mit den Ermittlern und ihrem Schiff zur Zeus II-2 "uberzusetzen und es mit der fremden KI aufzunehmen. Sie legt dazu ihre Karten offen auf den Tisch:

\begin{itemize}
	\item Ihr Schiff ist das Einzige, dass Chancen hat an ein sich verteidigendes Schlachtschiff anzudocken.
	\item Ihr Geist ist stark genug die gegnerische KI zu bezwingen. Ein menschlicher Psychonaut hat dagegen geringe Chancen.
	\item Ihr Schiff ist das einzige, dass die Ermittler sicher vor der drohenden Zerst"orung der Plane 9 wegbringen kann. Eine zivile F"ahre 
        w"are den Angreifern nicht gewachsen und kann jeder Zeit verfolgt werden.
	\item Sie ist die Einzige die der befreiten KI der Zeus II-2 ein Angebot zur Kooperation machen kann, dass sie nicht ablehnen wird. 
		Welches das ist, wird sie allerdings nicht verraten.
    \item Sie er"offnet, dass die USI ihre Eltern entehrt und vertrieben hat und sie deshalb die USI mit Freuden an einer Stelle Treffen 
        will die besonders weh tut.
\end{itemize}

Es obliegt den Ermittlern nun ihre Entscheidung zu treffen wie sie mit dem Virus umgehen wollen. Viel Zeit sollte der Spielleiter ihnen nicht geben. Lord Commander Steeler wird in K"urze seiner Warnung Taten folgen lassen.

\vfill
\newpage

\begin{remarks}
	In dieser Szene werden fast alle noch offenen Fragen gekl"art und die letzten Entscheidungen getroffen. Prof.~Dr.~Naratova legt ihre Geschichte offen, kann Fragen beantworten und gesteht ihr Fehlverhalten ein. Die Szene ist damit der H"ohepunkt der Geschichte mit der Melancholie eines tragischen Endes. Die ganze Szene wird dominiert durch den Dialog zwischen der Firmenchefin, \xl{}, \ml{} und den R"uckmeldungen von der Donar und der Nike Station. Der Dialog muss durch den Spielleiter z"ugig vorgetragen werden, um den Spannungsbogen aufrecht halten zu k"onnen.

	Je nachdem in welchem Kontakt die Charaktere vor dem Eintreffen auf Nike mit der Stationsleitung und der Donar im Austausch waren m"ussen die Funkspr"uche der beiden Parteien angepasst werden.

	Wenn Prof.~Dr.~Naratova \xl{} enttarnt ist es m"oglich einem Charakter der vorher im Asteroideng"urtel zwischen Mars und Jupiter t"atig war zu erkennen, dass \xl{} die ber"uchtigte Piratin ist die auf Valhalla gefangen genommen wurde und kann daraus schlie\3en, dass sie dort in der Gefangenschaft die KI implantiert bekommen hat.

	Beim Blickaustausch zwischen der Wissenschaftlerin und \xl{} erkennt \xl{}, dass Naratova vorhat ihre Erfindungen zu vernichten und sich selbst zu t"oten. Sie verzichtet deswegen darauf sie auszuschalten.

	F"ur das weitere Vorgehen bleiben den Charakteren zwei Optionen. Entweder sie bitten \ml{} einen Virus zu erstellen der die feindliche KIs zerst"ort. Oder sie entwickelt eine Modifikation der den USI Code entfernt. Die endg"ultige Entscheidung unterliegt am Ende \ml{}. Niemand au\3er ihr wei\3 vor seiner Anwendung was der Virus wirklich tun wird. \ml{} ist  demgegen"uber nicht geneigt sich gegen \xl{} zu stellen. F"ur die Erstellung des Virus bleibt auf der Station keine Zeit. \ml{} mu\3 dfen neunten Ring verlassen bevor sie in Aktion treten kann.
	
	\xl{} will in jedem Fall versuchen die Zeus II-2 zu kapern. Ist die Gruppe nicht bereit \xl{} zu unterst"utzen m"ussen sie sie ausschalten. \xl{} wird in diesem Falle versuchen \ml{} in ihre Gewalt zu bekommen und mit der Dragon Blade zur Zeus II-2 "uberzuwechseln. Gelingt es den Ermittlern nicht \xl{} zu "uberrumpeln, wird sie ihren Plan in die Tat umsetzen und das folgende Kapitel umsetzen. Die Charaktere m"ussen dann mit der Dawn of Day oder mit Rettungskapseln fliehen oder die im n"achsten Kapitel beschriebene Zerst"orung der Plane 9 verhindern.
\end{remarks}
