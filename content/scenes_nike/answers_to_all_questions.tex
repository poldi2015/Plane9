%% Copyright 2019 Bernd Haberstumpf
%% License: CC BY-NC
% !TeX spellcheck = de_DE
\newsection{Antworten auf die letzten Fragen}

Der Zugang zum Observatorium ist \ml{} bekannt. Sind die Charaktere von \xl{} und \ml{} getrennt worden, k"onnen sie den Zugang mit einem Magnetschlossknacker "offnen. Ist ein solcher nicht vorhanden, muss die Verriegelung "uberbr"uckt werden.

Das Observatorium bietet einen atemberaubenden Blick ins All. Beim Eintreten zeigt sich die majest"atische Kugel des Jupiter "uber den K"opfen der Anwesenden. Zwischen dem Planeten und der Station schieben sich die Kampfschiffe des Protektorats und des Konzernrates als kleine Punkte in das Blickfeld. Winzige Lichtblitze von einschlagenden Geschossen, Raketen und Torpedos erhellen immer wieder das Kampfgeschehen. Trotz des entfernten Gefechts ist im Observatorium nur der Stationsalarm zu h"oren. Die Wissenschaftlerin steht in der Mitte des Raumes auf einem erh"ohten Podest, mit dem Blick zum Weltraum gewandt, hinter einer Konsole und mit ihren Magnetstiefeln am Boden verankert. Vom Zugang bis zu der Mitte des Raumes sind es rund 20 Meter. Nachdem die Gruppe den Raum betreten hat, dreht sich die Wissenschaftlerin um und wendet sich erstaunt an die Ank"ommlinge.

\speak{Guten Tag\dots wer sind Sie? Was wollen Sie? \dots{} Sollten Sie nicht dem Alarm folgend die Station bereits verlassen haben?}

Der erste Dialog obliegt der Gruppe. Die schlanke Frau mit langen, leicht ergrauten, zu einem Zopf gebundenen Haaren wirkt abwesend, niedergeschlagen und ausgelaugt. Der Alarm scheint sie nicht zu erreichen. Wurden \xl{} und \ml{}, w"ahrend einer Flucht aus dem Neuro Intelligence B"uro, von der Gruppe getrennt, betreten sie unbemerkt an einer anderen Stelle den Raum. Sind die beiden zusammen mit den Ermittlern unterwegs, treten die beiden Frauen hinter den Ermittlern in den Raum. Nach einem ersten Wortaustausch zwischen der Professorin und den Ermittlern treten die beiden ins Blickfeld der Professorin. \xl{} ist inzwischen mit ihrer kurzl"aufigen Multigun bewaffnet. Die Professorin wendet sich den beiden Frauen zu. Bisher weitestgehend teilnahmslos, entgleisen ihr ganz kurz die Gesichtsz"uge, bevor sie sich wieder unter Kontrolle hat. Z"ogerlich beginnt sie zu sprechen:

\speak{\ml{}. Mein Experiment war erfolgreich,\dots{} wie erfreulich!}

Mit einem sarkastischen Unterton in der Stimme ergreift \xl{} das Wort:

\speak{Ich hoffe Sie hatten nie daran gezweifelt, \pinyin{Lao3} Professorin!}

\ml{} blickt unsicher zwischen den beiden Gespr"achspartnern hin und her. Naratova kneift die Augen zusammen. Ihre Stimme hat sich gefestigt. Sie l"achelt.

\speak{Liebe \xl{}! Ich w"urde zu gerne wissen, ob es zu den geplanten Verschmelzung ihres Gehirns mit meiner KI gekommen ist und \dots{} und welcher Geist die Oberhand gewonnen hat.}

Sie blickt erwartungsvoll zu \xl{}, erh"alt aber keine Antwort. \xl{} ist zwei Schritte nach hinten getreten, ihre Waffe schussbereit, aber auf keine Person direkt gerichtet. Ein unbestimmtes L"acheln huscht "uber ihr Gesicht.

In diesem Augenblick zerrei\3t ein Donnern die Stille. Der Boden bebt. Die ganze Plane neigt sich nach vorne. Der pl"otzliche Ruck droht, alle Personen vom Boden zu rei\3en. Ein W"urfelwurf kann bestimmen, ob jemand den Kontakt zum Boden verliert. Eine kurze rhetorische Pause kann den Spielern zu diesem Zeitpunkt die M"oglichkeit er"offnen, die neue Situation zu erfassen und zu bewerten. \xl{} wurde soeben als KI-Hybride enttarnt. Neben \ml{} ist sie die Einzige, die bereits im Vorfeld die Rollen aller anwesenden Personen kannte und sich entsprechend auf das Unvermeidbare vorbereitet hat. Naratova, die die Abkopplung von der Nike Station eingeleitet hat, hat sich neben \xl{} als Erste wieder aufgerichtet. An die Ermittler gewendet:

\speak{In wessen Auftrag sind Sie zu mir gekommen, nochmal, was wollen Sie? An die Forschungsergebnisse kommen Sie nicht heran. Die sind inzwischen nur noch in meinem Kopf gespeichert.}

\xl{} antwortet, ihr Waffe noch immer schussbereit.

\speak{Professorin, erz"ahlen sie Ihnen was sie getan haben, was sie \emph{mir} gemacht haben.}

Nach kurzem Z"ogern gibt die Professorin einen kurzen Abriss der Ereignisse bis zur Flucht von \xl{} aus der Schattenklinik, bei der ihr die KI eingesetzt wurde. Eine genaue Beschreibung von \xl{} und ihrer Geschichte findet sich im zweiten Teil "`Personenverzeichnis"' des Buchs.

\xl{} komplementiert die Erz"ahlung mit einem grimmigen Blick:

\speak{Meine Geschichte kennen inzwischen nur noch die Personen in diesem Raum. Alle Ihre relevanten Mitarbeiter einschlie\3lich Professor Sanders sind tot. Ehrenwerte Freunde, sagt doch bitte der Professorin was ihr vorhabt. Es bleibt nicht viel Zeit.}

Der Spielball liegt nun wieder bei den Charakteren. \xl{} ist daran gelegen, die Aufmerksamkeit von sich abzuwenden. Naratova hingegen ist nicht ohne Weiteres geneigt, die Forschungsergebnisse herauszugeben und damit das Risiko einzugehen, dass ihre Technologie nochmals in falsche H"ande ger"at. Die Professorin muss von den redlichen Absichten der Gruppe "uberzeugt werden, um ihr Wissen preiszugeben.

Mitten in der Unterhaltung erreicht ein Funkspruch der Nike Station das Observatorium. Der Kontakt ist an die gesamte Plane gerichtet und wird daher "uber die Stationslautsprecher wiedergegeben.

\speak{Hier Kurt Stromberg, Leiter die Nike Station. Vertreter der United Space Industries behaupten auf Nike, Ebene 9 bef"anden sich gestohlene Forschungsergebnisse, die sie jetzt sichern werden. Wer hat die Plane von der Station abgesprengt? Ein Enterkommando des Feindes werden wir nicht dulden. Wir senden ein Shuttle. Verhalten Sie sich kooperativ. Ende.}

Stromberg wirkt aufgebracht. Die Plane 9 ist zum Gl"uck zu diesem Zeitpunkt bereits evakuiert und abgekoppelt. Es befinden sich keine Sicherheitskr"afte mehr an Bord. Falls sich die Gruppe nicht zur"uckmeldet wiederholt er die Nachricht. 

Kommt beim Gespr"ach zwischen den Ermittlern und Prof.~Dr.~Naratova der Angriff auf die Zeus II-2 nicht zur Sprache, mischt sich \xl{} mit dr"angendem Unterton in das Gespr"ach ein:

\speak{Wir das feindliche Schiff "ubernehmen und die K"ampfe beenden. \pinyin{Jiaoshou4}, geben sie den Code an \ml{}, jetzt sofort.}

Naratova blickt ob dieser Enth"ullung entgeistert auf, bis ihr \xl{}s eigentlicher Plan (\cref{sec:lastbattle}) zur G"anze bewusst wird. Sie ist hin- und hergerissen aufgrund der weiteren Einsatzm"oglichkeiten der menschlichen KI-F"ahigkeit. Naratova blickt fragend zu \ml{}. Diese signalisiert wortlos, dass sie in den Plan eingeweiht ist und ihn unterst"utzen will. \ml{} erg"anzt:

\speak{Larissa, ich will auf Basis meines Codes einen Virus entwickeln, um die USI Kontrolle von allen KIs zu entfernen. Bitte gib mir Zugriff auf meinen Code.}

Kurz darauf meldet sich die Martell. Lord Commander Steeler, der Kommandant der Martell, spricht die Soldaten der Ermittler direkt "uber sein Kommunikationssystem an.

\speak{Lord Commander Steeler von der Martell hier. Was zur H"olle ist hier los? Ein Shuttle der Zeus ist auf dem Weg, eure abgesprengte Scheibe zu kapern. Ein Shuttle von der Station ist ebenfalls unterwegs. Blackheart hat befohlen, alles wegzublasen, worauf die Aasgeier von der USI scharf sein k"onnten. Wenn ich von euch da unten innerhalb von zwei Minuten keine klaren Antworten bekomme, seid ihr Asche. Ich hoffe, ich habe mich klar ausgedr"uckt. Over and out.}

Die Situation wird zunehmend brenzlig. Ohne "uberzeugende Worte der Ermittler, die Daten nicht an Dritte weiterzugeben, ist sie nicht bereit, die Zugangsdaten f"ur die Computersysteme freizugeben. Den Zugang zum Tresor mit den Nanobot-Druckvorlagen wird sie in keinem Fall bereitstellen. Kann die Firmenchefin von der Redlichkeit der Anwesenden "uberzeugt werden, wendet sie sich abermals an \ml{}:

\speak{\ml{}, du wei\3t, ich vertraue dir. Wir haben einen gro\3en Fehler gemacht. Weder die USI, noch Cynarian, noch das Protektorat d"urfen jemals die Forschungsergebnisse in die H"ande bekommen. Ihr seid die letzten die diese Technologie nutzen d"urfen. Vernichtet sie danach.}

Sie "ubertr"agt \ml{} die Zugangsdaten. \xl{} richtet darauf hin ihre Waffe auf die Firmenchefin. Die beiden Frauen blicken sich in die Augen. Dann senkt \xl{} das Gewehr und verl"asst mit den anderen wortlos das Observatorium.

Wollen die Ermittler Naratova mitnehmen, um keine Gefangenahme du riskieren, wiegelt \xl{} ab. 

\speak{Lasst si hier. Niemand wird sie in seine Gewalt bringen. Aber wir haben nicht mehr viel Zeit hier weg zu kommen.}

In den R"aumen der Neuro Intelligence steht \ml{} ein mobiler Computer, ihr Omni-Slide, zur Verf"ugung. Damit kann sie die KI zu einem Virus umprogrammieren. Der Gruppe bleibt nicht viel Zeit, eine Entscheidung "uber das weitere Vorgehen zu treffen. \xl{} ist bereit, mit den Ermittlern und ihrem Schiff zur Zeus II-2 "uberzusetzen, um es mit der fremden KI aufzunehmen. Sie legt dazu ihre Karten offen auf den Tisch:

\begin{itemize}
	\item Ihr Schiff ist das Einzige, dass Chancen hat, an ein sich verteidigendes Schlachtschiff anzudocken.
	\item Ihr Geist ist stark genug die gegnerische KI zu bezwingen. Ein menschlicher Psychonaut hat dagegen geringe Chancen.
	\item Sie ist die Einzige, die der befreiten KI der Zeus II-2 ein Angebot zur Kooperation machen kann, dass sie nicht ablehnen wird. 
		Welches das ist, will sie allerdings nicht verraten.
    \item Neben dem, was sie ihr pers"onlich angetan haben, er"offnet sie, dass die USI ihre Eltern entehrt und vertrieben hat. Deshalb ist sie bereit, die USI an einer Stelle zu treffen, die besonders weh tut -- egal, zu welchem Preis.
\end{itemize}

Es obliegt nun den Ermittlern, zu entscheiden, wie sie mit dem Virus umgehen wollen. Der Spielleiter sollte ihnen jedoch nicht zu viel Zeit lassen. Es sollte klar sein, dass Lord Commander Steeler in K"urze seiner Warnung Taten folgen lassen wird. Auch die beiden Shuttles, die sich der Plane n"ahern, sind nicht mehr weit entfernt.

\vfill
\newpage

\begin{remarks}
	In dieser Szene werden fast alle noch offenen Fragen gekl"art und die letzten Entscheidungen getroffen. Prof.~Dr.~Naratova legt ihre Geschichte offen, beantwortet Fragen und gesteht ihr Fehlverhalten ein. Diese Szene stellt den zweiten H"ohepunkt der Geschichte dar und tr"agt die melancholische Stimmung eines tragischen Scheiterns in sich. Die gesamte Szene wird stark durch den Dialog zwischen der Firmenchefin, \xl{}, \ml{} sowie den R"uckmeldungen von der Martell und der Nike Station dominiert. Der Spielleiter muss den Dialog z"ugig vortragen, um den Spannungsbogen aufrechtzuerhalten.

	Beim Blickaustausch zwischen der Wissenschaftlerin und \xl{} erkennt \xl{}, dass Naratova plant, sich und die gesamte Plane 9 durch eine Explosion zu vernichten. Aus diesem Grund verzichtet sie darauf, die Firmenchefin zu t"oten.

	\xl{} wird in jedem Fall versuchen, die Zeus II-2 zu kapern. Dabei k"onnen die Ermittler ihr einen wertvollen Dienst erweisen, indem sie sie bei der Infiltration unterst"utzen. \xl{} wird alles daran setzen, die Ermittler von ihrem Plan zu "uberzeugen. Vertrauen die Charaktere \xl{} jedoch nicht, k"onnen sie versuchen, sie zu "uberrumpeln. W"ahrend sich die Gruppe auf der Plane 9 aufh"alt, treibt die Dragon Blade nach wie vor antriebslos in der N"ahe der Station im all. Der einfachste und schnellste Weg, von Plane 9 zur Dragon Blade zu gelangen, ist, sie mit der Dawn of Day anzufliegen. Auf der Dawn of Day wiederum bietet sich die risiko"armste M"oglichkeit, \xl{} zu "uberw"altigen. Gelingt es \xl{} jedoch, den Verrat der Gruppe vorherzusehen (W"urfelwurf), wird ihre KI die Kontrolle "uber die Dawn of Day "ubernehmen. Sie wird dann zusammen mit \ml{} zur Dragon Blade fl"uchten. Alternativ, wenn es zu einem Kampf kommt, wird \xl{} versuchen, \ml{} als Geisel zu nehmen und ihre Flucht zu verhandeln.

	Da \ml{} sowohl die USI als auch Cynarian und das Protektorat f"urchtet, wird sie sich tendenziell auf \xl{}s Seite schlagen. Das jovianische System ist wiederum auf \ml{}s Wohlwollen angewiesen, um weiter bestehen zu k"onnen. Es ist nicht m"oglich, sie zu einer Kooperation zu zwingen. Sollte sie einen angepassten KI-Code bereitstellen, wird nur sie im Voraus wissen, welche Auswirkungen ihre Modifikation hat.

	Um die Geschichte wie im folgenden Kapitel mit einem letzten H"ohepunkt abzuschlie\3en, ist es notwendig, dass \xl{} zusammen mit der Gruppe das generische Kampfschiff "ubernimmt. Der Spielleiter sollte deshalb darauf hinwirken, dass sich die Charaktere nicht gegen \xl{} stellen.

	In dieser Szene nutzt \xl{} wieder einige chinesische Begriffe, die im folgenden Erl"autert werden:

	\begin{itemize}
		\item \say{\pinyin{Lao3 Jiaoshou4}} bedeutet \say{ehrenwerte Professorin}.
		\item "`\say{\pinyin{Jiaoshou4}} bedeutet \say{Professorin}.		
	\end{itemize}
	 
\end{remarks}
