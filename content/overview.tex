%% Copyright 2019 Bernd Haberstumpf
%% License: CC BY-NC
% !TeX spellcheck = de_DE
\newchapter["Ubersicht]{"Ubersicht}

Die Rollenspielkampagne entstand urspr"unglich als ein 20-seitiger ``One-Shot''-Rollenspielplot, ausgelegt auf 4--5 Abende, der am Ende zu einer Spielzeit von zwei Jahren und diesem "uber 180-seitigen Buch gef"uhrt hat. W"ahrend dieser Spielzeit sind unz"ahlige Nichtspielercharaktere, neue dramatische Szenen und "uberraschende Wendungen entstanden.

Die Kampagne in diesem Buch ist ein durchgehender Science-Fiction-Thriller mit einem Spannungsbogen, der mit einem Spielfilm vergleichbar ist. Der Spielleiter f"uhrt die Spieler durch eine komplexe Geschichte mit vielen Querabh"angigkeiten. Daher sollte der Spielleiter bereits vor Spielbeginn einen guten "Uberblick "uber das Buch und den Verlauf der Geschichte haben, da das Buch darauf ausgelegt ist, dass die Spieler ohne gro\3e Vorbereitung direkt in die Geschichte eintauchen k"onnen.

Um dem Spielleiter den Einstieg zu erleichtern, folgt nun ein "Uberblick "uber den Aufbau des Buches und Hilfestellungen zum Spielablauf.

\newsection{Inhalt}

Des Buchs ist folgenderma\3en strukturiert:

\newcommand{\sfromref}[1]{ab \textit{S.\pageref{#1}}}
\begin{description}
    \item [Teil I] Die Geschichte

        Teil I enth"alt die Geschichte und alle f"ur das Verst"andnis der Geschichte relevanten Hintergrundinformationen. Dieser Teil steht nur dem Spielleiter zur Verf"ugung.
        \begin{description}
            \item [Story] Die eigentliche Geschichte mit Hinweisen f"ur den Spielleiter wird auf den kommenden Seiten erz"ahlt und endet \cref{sec:ace}.
            \item [Chronologie] Eine zeitliche Abfolge aller relevanten Ereignisse findet sich in der ``Chronologie'' 
                \cref{sec:chronology}. Eine Kurzzusammenfassung der Attentate, dem Auftakt der Geschichte, findet sich in \cref{sec:assassinations}.
            \item [Charaktere] Eine Beschreibung aller Nichtspielercharaktere in chronologischer Reihenfolge ihres Auftretens findet sich 
                im Personenverzeichnis  \sfromref{sec:nsc}.
            \item [Orte]  Die Geschichte spielt auf mehreren Monden und Raumstationen rund um den Jupiter. Eine Beschreibung aller Orte, 
                die Teil der Geschichte sind, findet sich \sfromref{sec:locations}.
        \end{description}
    \item [Teil II] Das Rollenspiel

        Teil II enth"alt ein eigenes Rollenspielregelwerk und eine "Ubersicht "uber das jovianische System. Dieser Teil ist sowohl f"ur den Spielleiter als auch f"ur die Spieler relevant und sollte den Spielern zur Vorbereitung an die Hand gegeben werden. 
    
        Teil II kann als separates PDF-Dokument aus dem Internet unter der Adresse 
        
        \textbf{https://github.com/poldi2015/Plane9}
        
        heruntergeladen werden. Das Charakterdatenblatt ist dort ebenfalls zu finden.
        \begin{description}
            \item [Hintergrund] Die ersten Kapitel des zweiten Teils \sfromref{sec:rpg} geben einen Einblick in die Welt des 
                23.~Jahrhunderts und im Speziellen in das jovianische System. Sie enthalten Hintergrundinformationen, beschr"ankt auf das, was die Spieler bereits wissen d"urfen. Hier werden k"unstlich gez"uchtete Menschen, die sogenannten Mutanten, Raumschiffstypen, Waffen, wichtige Pers"onlichkeiten, Orte und Institutionen beschrieben.
            \item [Regelwerk] In \sfromref{sec:rules} wird ein einfaches Rollenspielregelwerk vorgestellt, das ausreicht, um die Kampagne 
                spielen zu k"onnen. Die F"ahigkeiten der Nichtspielercharaktere im ersten Teil des Buches sind basierend auf diesem Regelwerk beschrieben.
            \item [Charakterdatenblatt] Das Charakterdatenblatt zum Regelwerk in diesem Buch findet sich in \cref{img:datasheet}.
        \end{description}        
\end{description}


\begin{remarks}
    Die Kapitel der Geschichte enthalten eine Beschreibung einzelner Szenen. Am Ende eines Kapitels findet sich meistens ein Abschnitt wie dieser hier. Er enth"alt zus"atzliche Hinweise f"ur den Spielleiter.
\end{remarks}

\newsection{Spielablauf}

Die Geschichte in diesem Buch ist in Szenen aufgeteilt. Jede Szene kann "ublicherweise an einem oder zwei Abenden durchgespielt werden. Die einzelnen Szenen sind oft recht detailliert mit Dialogen und dem Verhalten der Nichtspielercharaktere ausgeschm"uckt. Diese Ausschm"uckung dient, "ahnlich wie bei einem Roman, dazu, tiefer in das Geschehen eintauchen zu k"onnen. Das bedeutet jedoch nicht, dass die Szenen exakt so, wie beschrieben, ablaufen m"ussen. In den Spielrunden, die die Basis f"ur dieses Buch bilden, hatten die Spieler weitreichende Freiheitsgrade, die, wie anfangs beschrieben, die Inspiration f"ur die Szenen, alternativen Handlungsverl"aufe, optionale Szenen und Spielleiterhinweise geliefert haben.

Diese Kampagne enth"alt, im Gegensatz zu vielen anderen, ein paar Besonderheiten, die der Spielleiter ber"ucksichtigen sollte:

Die Spielercharaktere sind von der Regierung und dem Milit"ar des jovianischen Systems eingesetzt, um Vorf"alle aufzukl"aren, die die nationale Sicherheit gef"ahrden. Das hat zur Folge, dass ihnen weitreichende Befugnisse und Ressourcen auf Anfrage zur Verf"ugung gestellt werden. Einer der Charaktere ist dar"uber hinaus ein Omega-Mutant, dem man sich nicht so einfach entgegenstellt.

Um den Plot nicht zu gef"ahrden, kommen dem Spielleiter zwei Faktoren zugute. Das jovianische System befindet sich noch im Aufbau, und t"aglich trifft ein unkontrollierter Strom von Fl"uchtlingen im System ein. Viele Teile des Systems gleichen provisorischen Fl"uchtlingslagern und verf"ugen daher "uber keine hochentwickelte Infrastruktur.

Die zweite Einflussm"oglichkeit ist ebenfalls eine Besonderheit:

Fast w"ahrend der gesamten Handlung steht den Charakteren ein von dem Spielleiter gespielter Begleiter zur Seite. Dieser Begleiter unterst"utzt sie bei Recherchen und kann sie im Zweifelsfall auch in Konfliktsituationen unterst"utzen. W"ahrend der Begleiter den Charakteren zus"atzliche M"oglichkeiten bietet, bequem an Informationen zu gelangen, hat der Spielleiter auch die M"oglichkeit, diese Informationen zu beschr"anken oder zu verf"alschen.

Zus"atzlich gibt es eine dritte Besonderheit: Die Spieler d"urfen vor"ubergehend in andere Rollen schl"upfen, um aktiv an einer Schl"usselszene mitwirken zu k"onnen.

\vspace{1.5cm}
\begin{center}{\large{}In diesem Sinne viel Spa\3 beim Eintauchen in das jovianische System.}\end{center}
