%% Copyright 2019 Bernd Haberstumpf
%% License: CC BY-NC
% !TeX spellcheck = de_DE
\newsection{Befragung der HeM05 Besatzung}

Die vor rund 5 Tagen gerettete Mannschaft der Mine HeM05, wird wie auch die an der Rettung beteiligten J"agerpiloten erst kurz vor dem Eintreffen der Ermittler aus einer Dekompressionskammer entlassen, die sie w"ahrend des Attentats auf der Mine aufgesucht hatten. Beim Eintreffen der Charaktere auf Hellgate befinden sich die 8 geretteten Minenarbeiter in der Kantine der J"agerstaffel. Grace Anders begleitet die Charaktere zur Kantine. Vor den R"aumlichkeiten treffen die Charaktere den Flugausbilder Jos\'{e} \frqq{}Torro\flqq{} Alvarez an. Torro ist ein kleiner drahtiger Mann von fr"ohlicher Natur, dem die Jahre als Piloten schon deutlich zugesetzt haben. Torro, der die Rettungsaktion geleitet hat, kann einen ersten Einblick in die Geschehnisse geben. Nach dem Gespr"ach mit Torro k"onnen sich die Charaktere den Minenarbeitern zuwenden. Sie m"ussen zun"achst entscheiden, ob sie diese einzeln befragen wollen oder sie direkt in der Kantine aufsuchen. Sollen die Arbeiter einzeln befragt werden, bietet Torro an, Florence die Kommandantin der Mine zu den Ermittlern zu bringen. Danach kann Grace "ubernehmen und die Minenbesatzung einzeln herausbitten.

Im folgenden die Aussagen der an der Rettung beteiligten:

\begin{description}
	\item[Torro] \say{Bei einem "Ubungsflug durch die obersten Atmosph"arenschichten erhielten wir einen Notruf der Mine HeM05. Da 		
		meine Trainingsstaffel mit insgesamt vier Valkyrien, 3 Rookies und mir, der Mine am n"achsten waren, sind wir tiefer in die 		
		Atmosph"are eingetaucht und konnten dort gl"ucklicherweise die Mine nach kurzer Zeit lokalisieren. Da wir die Arbeiter nicht mit unseren Jagdmaschinen selbst retten konnten, blieb uns nur die M"oglichkeit, an die Mine selbst anzudocken. Zugegebenerweise ein recht waghalsiges und f"ur die Auszubildenden ein risikoreiches Man"over. Wir waren zu diesem Zeitpunkt bereits in einen f"ur uns kritischen Atmosph"arenbereich gesunken. Mit viel Gl"uck schafften wir es dann aber drei Maschinen an die Mine anzudocken und mit voller Schubleistung die Mine auf eine H"ohe zu bringen die es der Schlepperinsel erlaubte, die Mine in den Orbit zu ziehen. Ein hei\3er Ritt kann ich Ihnen sagen.}
	\item[Florence (Kommandantin)] \say{W"ahrend der ersten Systemmeldung, dass einer der Tr"agerballons der Station abgekoppelt wurde, 	
		befanden sich Jurij Smirnov, Blackwind, ZDee und ich auf der Br"ucke. Greydog war in der Minenanlage besch"aftigt. Die anderen waren nach ihren eigenen Angaben im oberen Bereich der Mine t"atig. Ich beauftragte als erstes ZDee damit, die Aufh"angung der Tr"agerballons au\3erhalb der Mine zu kontrollieren und sandte einen Hilferuf an die Hellgate Station. Einige Minuten sp"ater beobachteten wir auf der Br"ucke wie, von den Au\3enkameras aufgenommen, Pitch in ihrem Raumanzug in die Tiefe st"urzte. Ca.~10 Minuten sp"ater l"oste sich der zweite Tr"agerballon. Nach einem weiteren Notruf befahl ich die Evakuierung. Treffpunkt war das Rettungsshuttle. R"uckmeldung bekam ich von allen au\3er ZDee. Auf dem Weg zum Shuttle sammelten wir Salvador vor seinem Quartier ein. Er war gerade dabei fertig geworden, sich anzuziehen. Am Rettungsshuttle angekommen traf die Br"uckencrew auf Isabell und Fernandez. Das Rettungsshuttle lie\3 sich zu unserer Best"urzung nicht starten. Die Startsequenz war durch eine Manipulation blockiert. Deshalb blieb uns nichts anderes "ubrig, die Dekompressionskammer aufzusuchen und auf Rettung zu hoffen. Auf dem Weg zur Dekompressionskammer l"oste sich offensichtlich der dritte Ballon. An der Kammer trafen Greydog und Hannibal auf uns. Hannibal hatte noch versucht, "uber die Steuerung der Anlage die Manipulation der Ballons zu verhindern. ZDee war von seiner Au\3enmission nicht zur"uck gekommen.}
	\item[Jurij Smirnov, Blackwind] Die beiden best"atigen die Aussage von Florence.
	\item[Salvador] \say{Ich war in meinem Quartier, als der Aufruf zur Evakuierung kam. Die Br"uckencrew kam kurz darauf bei meinem 	
		Quartier vorbei und nahm mich mit.}
	\item[Greydog] \say{Ich war an der Raffinerie mit Wartungsarbeiten im Au\3enbereich am unteren Ende der Raffinerie besch"aftigt. 
		Dadurch habe es nicht geschafft, die anderen bereits am Shuttle zu treffen.}
	\item[Fernandez Lorend] \say{Ich habe Isabell bei der Justierung ihrer Zentrifugen f"ur die Analyse des Atmosph"arengemischs 
		unterst"utzt als der Notruf einging. Daraufhin machten wir uns unverz"uglich auf den Weg zum Rettungsshuttle.}
	\item[Isabell Sonderleiten] Isabell best"atigt die Aussage von Fernandez. Ein paar Tage vor dem Attentat vertraute Pitch ihr an, dass 
		sie auf eigene Faust gegen ein anderes Besatzungsmitglied recherchiere, weil sie glaubte, dieser sei f"ur die Havarie der HeM03 verantwortlich. Ihr Verdacht wurde durch Anpassungen an der Steuersoftware der Mine geweckt. Sie lie\3 sich deshalb auf HeM05 einschiffen, um ihren Verdacht weiterzuverfolgen und den Attent"ater selbst zur Rede zu stellen. Wen sie im Verdacht hatte, hat sie allerdings nicht verraten. Die Befragung von Isabell erfolgt nur stockend. Das Erlebte und der Verlust von Pitch machen ihr offensichtlich stark zu schaffen.
	\item[Blackwind] \say{Pitch wurde als Attent"aterin identifiziert, weil sie bei der Abkopplung des Tr"agerballons im Au\3enbereich 		
		der Mine t"atig war, was "uberhaupt nicht ihrem Arbeitsbereich entsprach. F"ur die Wartung und das Einspielen neuer Software hatte mich Pitch gebeten, ihr tempor"ar Zugang auf die Steuerung des Rettungsshuttle zu geben. Pitch war bereits vorher auf HeM03 besch"aftigt.}
	\item[Hannibal] \say{Als das Abkoppeln des ersten Ballons durchgegeben wurde, begann ich sofort die Ansteuerung der Tr"agerballons zu 
		kontrollieren, da die softwaretechnische Wartung mir und Pitch unterlag. Dabei konnte ich eine Manipulation der Steuerung entdecken, die wahrscheinlich die Abkopplung ausgel"ost hat. Als mein Rettungsversuch misslang, und der dritte Ballon sich gel"ost hatte, machte ich mich auf den Weg zur Dekompressionskammer, da Florence bereits die Manipulation des Shuttles durchgegeben hatte.}
\end{description}

W"ahrend der Befragung macht Grace Anders Meldung an ihren Vorgesetzten \emph{Karl Sandos} und "ubermittelt ihm die Aussagen der Besatzung der Mine. Karl Sandos gibt diese Informationen an \emph{Henk Arongate} weiter.

\begin{remarks}
	Die Minenarbeiter sind durch die Vorkommnisse nach wie vor angeschlagen. Die in diesem Kapitel zusammengefassten Aussagen k"onnten entsprechend emotional ausgeschm"uckt werden. Die Kommandantin Florence, eine Alpha-Mutantin, wird ihre Crew im Zweifel in Schutz nehmen und verteidigen.

	Isabel ist schon vor der Versetzung auf HeM05 mit Pitch befreundet und deshalb stark mitgenommen durch ihren Tod und dem Verdacht auf ihre T"aterschaft.

	Durch die Aussagen k"onnen die Ermittler bereits erkennen, dass Pitch aller Voraussicht nach nicht die Attent"aterin sein kann. Der zweite und vor allem der dritte Tr"agerballon l"osten sich erst nach ihrem Absturz.		
\end{remarks}
