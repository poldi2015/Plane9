%% Copyright 2019 Bernd Haberstumpf
%% License: CC BY-NC
% !TeX spellcheck = de_DE
\pageimage{images/hellgate_security_floorplan.jpg}
\newsection{Die Geiselnahme}\anchor{sec:hostage}

Haben die Charaktere ihre Befragung im Stützpunkt des Sicherheitsdienstes abgeschlossen, werden sie von Karl Sandos dem Stationsleiter wieder in den Eingangsbereich des Stützpunktes begleitet. 

Zum Einstieg in die folgende Szene sollte der Spielleiter, wenn möglich abwarten, bis sich die Charaktere außerhalb des Stützpunktes und die Charaktere im Stützpunkt in einem Informationsaustausch über das ComNetz befinden. Die vergangenen Ereignisse bieten ausreichend Stoff zu einer ausgiebigen Diskussion. Während des Gesprächs bricht dann die Verbindung zwischen den beiden Gruppen plötzlich ab. 

\newsubsection{Überfall auf den Stützpunkt}

Kurz nach dem Kommunikationsabbruch öffnet sich der Zugang zum Stützpunkt und ein Gegenstand wird in den Raum geworfen. Die \emph{Personal Area Networks (PAN)}, die Anbindungen elektronischer Systeme an das Gehirn der Personen im Raum fallen aus. Jeder Charakter in der Station muß einen Konstitutionswurf würfeln, um nicht kurzzeitig das Bewusstsein zu verlieren. Das PAN des Omega-Ermittlers, wenn er sich auf dem Stützpunkt befindet, wird als Erstes in Teilen wieder aktiv. Der Ausfall führt aber zeitweilig zu körperlichen und sensorischen Einschränkungen.

Mit dem Öffnen des Zugangs stürmen zwei bewaffnete Personen in den Raum und eröffnen sofort mit vollautomatischen Railguns das Feuer. Bei den Eindringlingen handelt es sich um Slingshot und dem Söldner \emph{Smith Handerson}. Beide Angreifer sind mit Kampfanzügen, Handerson zusätzlich mit einem Helm mit Sichtschutz gerüstet. Befindet sich der Assistenz-Ermittler des Protektorats im Eingangsbereich, wird er als Omega-Soldat sofort erkannt und von Smith Handerson ins Visier genommen. Ein weiteres Ziel ist Karl Sandos, der daraufhin hinter dem Tresen zu Boden geht. Ist kein Omega-Soldat anwesend, schießt einer der Angreifer auf einen weiteren Sicherheitsbeamten \emph{Luke Lengdon}, der dabei in die Brust und am Kopf getroffen wird und daraufhin ins Koma fällt. Beim folgenden Kampf wird nur ein Omega die Möglichkeit haben, eine Waffe zu ziehen oder in den Nahkampf zu gehen, wobei sein Handicap bestehen bleibt. Bei ihrem Angriff sollte es den Angreifern leicht fallen die überrumpelten Personen im Eingangsbereich kampfunfähig zu machen und sie danach in Schach halten zu können. Die Angreifer sammeln die zu Beginn geworfene EMP-Schockgranate ein, die die PANs der Anwesenden außer Kraft gesetzt haben.

\newsubsection[Außerhalb des Gebäudes]{Ausserhalb des Gebäudes}

Während die Angreifer den Stützpunkt überfallen, um Hannibal zu befreien, müssen die Ermittler außerhalb des Stützpunkts die neue Situation erst einmal verarbeiten. Zunächst wollen sie vermutlich in Erfahrung bringen, wodurch und wie die Verbindung zu ihren Mitstreitern ausfallen konnte. Grace Anders versucht, ihren direkten Vorgesetzten Karl Sandos und danach den Sicherheitschef Henk Arongate zu kontaktieren. Von ihm erfährt sie, dass nur das ComNetz, das Kommunikationsnetz der Station, beim Sicherheitsstützpunkt gestört ist. Er verspricht, Leute zum Stützpunkt zu schicken, weist aber darauf hin, dass die Ermittler sich am nächsten zum Stützpunkt befinden. Henk Arongate erklärt, dass im Umkreis von rund 30 Metern um den Stützpunkt das ComNetz ausgefallen wäre.

Der Stützpunkt selbst liegt an einem von zwei Seiten zugänglichen Tunnel. Vor dem geschlossenen Tor liegt der schwer verletzte Luke Lengdon. Er liegt im Koma und ist nicht ansprechbar. Andere Personen sind nicht erkennbar. Falls die Charaktere nicht selbst aktiv werden, bittet Grace Anders um Deckung und pirscht sich an ihren Ex-Freund heran, um dort Ersthilfe zu leisten.

Der Eingangsbereich ist auf den ersten Blick leer, wobei der Tresen nicht einsehbar ist. Auf dem Boden sind blutige Schleifspuren sichtbar. Um den Tresen herum sind Einschusslöcher zu erkennen. Die Scheibe ist zerstört. Nach der Erstürmung des Eingangsbereichs gilt es zunächst, das Gelände abzusichern und den verletzten Lengdon zu stabilisieren. Im Eingangsbereich und im Gangsegment vor dem Sicherheitsstützpunkt befinden sich keine weiteren Personen. Eine Erste-Hilfe-Ausrüstung mit medizinischem Expertensystem findet sich im vorderen Bereich des Stützpunkts.

\newsubsection{Rückblende}

Nachdem Handerson und Slingshot den Eingangsbereich des Stützpunkts unter ihre Kontrolle gebracht haben, öffnen sie mit dem Identitätsimplantat von Karl Sandos die Tür zu den hinteren Räumlichkeiten. Dieser Bereich umfasst Gefängniszellen, einen Verhörraum und Büros. Die Angreifer sperren alle Personen außer dem verletzten Luke Lengdon gemeinsam in eine Zelle und verlassen dann den Stützpunkt. Ihnen folgt der befreite Hannibal. Als Absicherung nehmen sie einen der Ermittler und die beiden Frauen der Minenbesatzung als Geiseln. Der Omega-Krieger der Ermittlergruppe, falls anwesend, wird in der Gefängniszelle zurückgelassen. Bei ihrer Flucht hinterlassen sie im hinteren Teil des Stützpunkts einen Störsender, der das ComNetz in der Umgebung des Stützpunkts lahmlegt, zusammen mit einem Funkgerät, um Sicherheitskräfte in die Irre führen zu können.

\newsubsection{Inspektion des Eingangsbereichs}

Nach der Absicherung des Eingangsbereichs können die Ermittler den Raum weiter untersuchen. Der Tresen kann mit der Chipkarte von Grace Anders betreten werden. Hinter dem Tresen sind weitere Blutspuren zu entdecken. Versuchen die Anwesenden, die Tür zu den inneren Bereichen zu öffnen, werden sie feststellen, dass nicht einmal Grace Anders die Tür öffnen kann. Sie vernehmen eine Stimme hinter der Tür, die ihnen droht, die Geiseln zu töten, falls jemand versucht, durch die Tür zu kommen. Bei dem Sprecher handelt es sich um Slingshot, der über Sprechfunk so lange wie möglich versucht, den Eindruck zu vermitteln, die Entführer befänden sich noch im Stützpunkt.

Kurz nachdem die Charaktere den Stützpunkt betreten, treffen weitere Mitarbeiter des Sicherheitsdienstes in Begleitung von zwei Sanitätern ein. Der Eingreiftrupp setzt sich aus zwei Norms und drei Mutanten zusammen. Sie tragen die Sicherheitswesten des Sicherheitsdienstes und jeweils eine Bolzenpistole. Die Sanitäter sind Norms. Da es sich bei Hellgate um einen Cynarian-Stützpunkt handelt, sind keine Omega-Soldaten aus den Streitkräften des Protektorats auf Hellgate im Einsatz. Angeführt wird der Trupp von einem \emph{Luke Dexter}, der sich direkt über den Verfall in Kenntnis setzen lässt. Währenddessen lässt er durch seine Leute den Eingangsbereich und die Gänge absichern. Die Sanitäter kümmern sich um den schwer verletzten Luke Lengdon.

\newsubsection{Im Zellentrakt}

Im Zellentrakt sind die übrigen Charaktere zusammen mit den Minenarbeitern eingesperrt. Insgesamt befinden sich vier Personen, einschließlich Hannibal, in den Händen der Entführer. Vom Zellentrakt aus ist nicht zu erkennen, dass die Entführer das Gebäude bereits verlassen haben. Unter den Gefangenen befindet sich der angeschossene Karl Sandos. Die Entführer haben ihren Opfern erlaubt, ein Erste-Hilfe-Kit mit in die Zelle zu nehmen. Es ist also angebracht, den schwer verletzten Stationsleiter erst einmal zu verarzten. 

Die Gefangenen haben danach Zeit, zu versuchen, sich zu befreien oder sich bemerkbar zu machen. Die Zellen sind für Randalierer und Aufständler gedacht. Dementsprechend sind sie nicht so gut gesichert wie eine reguläre Gefängniszelle. Auch wenn das ComNetz noch immer lahmgelegt ist, ist die Tür nicht ohne passendes Werkzeug zu knacken. Benötigt wird ein sogenannter \emph{Magschlossknacker}, um das elektronische Schließsystem zu knacken, oder Werkzeug, um die Türhydraulik zu überbrücken. Möglicherweise können auch Gegenstände aus dem Raum oder den Taschen der Minenarbeiter zweckentfremdet werden. Ein Luftschacht ist eine weitere Möglichkeit, zumindest Kontakt mit der Außenwelt aufzunehmen. Der Spielleiter kann großzügig kreative Ideen gewähren lassen. Befindet sich kein Charakter unter den Gefangenen, werden diese sich ruhig verhalten. Der Spielleiter kann je nach Spielfluss entscheiden, wie viel Zeit an dieser Stelle aufgebracht wird.

\newsubsection{Auf der Flucht}

Während der Vorkommnisse im Stützpunkt sind die Entführer zusammen mit ihren Geiseln auf dem Weg zu ihrem Shuttle, um Hellgate zu verlassen. Kurz nach dem Verlassen des Stützpunktes haben die Angreifer ihre Kampfausrüstung gegen schusssichere Westen und einfache Bolzenpistolen getauscht, um nicht aufzufallen. Slingshot, Smith Handerson und Hannibal sind jeweils mit einer Schusswaffe bewaffnet und treiben die Geiseln vor sich her. Um die PAN-Systeme der Geiseln zu stören, nutzen die Geiselnehmer ebenfalls einen Störsender mit kurzem Radius.

Nach der Absicherung des Eingangsbereichs können die Charaktere zusammen mit Luke Dexter, dem Anführer der eingetroffenen Sicherheitskräfte, in Verhandlung mit den Entführern treten. Die Entführer nutzen, wie schon beschrieben, ihr Sprechfunkgerät, um die Illusion aufrechtzuerhalten, sich noch im Sicherheitsstützpunkt aufzuhalten. Fordern die Charaktere nach einem Lebenszeichen ihres Freundes oder einer anderen Geisel, kann Slingshot den entführten Ermittler bitten, ein kurzes Lebenszeichen von sich zu geben. Dabei kann dieser versuchen, eine geheime Botschaft zu übermitteln.

Früher oder später wird der Eingreiftrupp wohl zusammen mit der Ermittlergruppe die hinteren Räume des Stützpunktes stürmen. Die Vorbereitung dazu wird nach wie vor durch den Störsender der Entführer erschwert. Eine Abstimmung mit anderen Kräften der Station ist nur bedingt möglich.

Zu ihrer Überraschung stoßen die Angreifer auf keine Gegenwehr. Die Gefangenen sind schnell befreit. Bei genauerer Untersuchung der eroberten Räume werden sowohl der Störsender als auch das Sprechfunkgerät sichergestellt. Charaktere mit militärischem Hintergrund identifizieren den Störsender als ein älteres Modell aus Militärbeständen, jedoch mit unbekannter Herkunft. Auch ist einem Soldaten die EMP-Schockgranate nach einer Beschreibung der Augenzeugen bekannt.
\vfill

\begin{remarks}
	\underline{Szenenwechsel:}

	Die Geiselnahme ist auf einen schnellen Szenenwechsel zwischen den Vorfällen im Stützpunkt und den anderweitigen Aktivitäten ausgelegt. Die Szenenwechsel sollten immer so gestaltet werden, dass die Spieler nur die Informationen erhalten, die auch ihren Charakteren zum jeweiligen Zeitpunkt zur Verfügung stehen. So dürfen die Spieler nach dem Überfall auf den Stützpunkt erst von der Flucht der Geiselnehmer erfahren, wenn der Zellentrakt des Stützpunktes gestürmt wurde.

	Der erste Angriff der Kidnapper sollte so ausgelegt werden, dass die Angreifer am Ende der Szene die Oberhand gewinnen.

	\underline{Vorbereitungen zum Angriff:}
	
	Im Auftrag von USI-Agenten wird Slingshot von Handerson auf Armageddon nach der Havarie der Mine HeM05 abgeholt und fliegt mit ihm nach Hellgate, um dort Hannibal abzuholen und nach Kallisto zu bringen. Sie kommen auf Hellgate an, kurz bevor die Charaktere Hellgate erreichen. Slingshot, unter dem Decknamen Drake, kann dadurch verfolgen, wie Hannibal zusammen mit den anderen Minenarbeitern zum Stützpunkt der Sicherheitsmannschaft gebracht wird.

	\underline{Beziehungen zwischen den Attentätern:}

	Slingshot steht mit Artisan, dem Stellvertreter des Protektors, in Kontakt. Dieser wiederum ist mit weiteren Attentätern und 
	USI-Agenten in Kontakt. Die Attentäter selbst sind sich allerdings nicht ihrer Aktivitäten als Attentäter vollständig bewusst. Genauso wenig nimmt ihr menschlicher Geist die anderen Attentäter als solche bewusst wahr. Alle Erinnerungen an ihre ``zweite Identität'' werden aus ihrem Geist ausgeblendet und werden auch von einem Psychonauten nicht als solche entdeckt.
\end{remarks}


\newsection{Showdown auf Hellgate}

Die Entführer haben ihr Shuttle auf der Außenseite von Adrastea bei einer Wartungsschleuse nahe dem Raumhafen verankert. Nachdem die Charaktere im Stützpunkt herausgefunden haben, dass sich die Entführer nicht mehr dort befinden, gibt es mehrere Möglichkeiten, sie aufzuspüren.

Am erfolgversprechendsten wäre es, die nach wie vor auftretenden Störungen im ComNetz der Station zu verfolgen.

Für die Navigation im Stützpunkt müssen die Kidnapper kurzzeitig den Störsender deaktivieren. Das gibt den Geiseln die Möglichkeit, eine kurze Nachricht zu übermitteln.

Folgende Szenarien, um die Entführer zu stellen, sind denkbar:

\begin{description}
	\item [Lagerkomplex] Im Lagerkomplex des Raumhafens gäbe es die Möglichkeit, einen Hinterhalt zu legen und die Entführer zu 
		überraschen. 
	\item [Landeplattform] Die Attentäter und ihre Geiseln haben Hellgate verlassen und sind dabei, ihr Shuttle zu betreten.
	\item [Shuttleverfolgung] Ist das Shuttle bereits gestartet, bleibt den Charakteren nur, das Shuttle zu verfolgen und zu entern, um die 
		Gefangenen zu befreien.
\end{description}


\newsubsection{Landeplattform}

Hellgate verfügt neben dem Raumhafen über eine Reihe von Ausgängen, die durch Tunnel zur Oberfläche des Mondes führen. Diese Tunnel dienen als Flucht- und Rettungswege oder bieten Zugang zu verschiedenen Sensorplattformen, meist in Richtung Jupiter.

Das Shuttle der Entführer hat auf der Landeplattform G.1 mit seinen Andockklammern festgemacht. Der Zugangsbereich zum Landedeck umfasst zwei Luftschleusen: eine für Personen und eine zweite für den Transport von Frachtcontainern. Eine Transportgondel fährt über 500 Meter zu einem großen Zwischenlager. Ein tenderähnliches Schienenfahrzeug ermöglicht es, Material von einem angedockten Schiff zur Schleuse zu transportieren. Der Weg zum Landebereich ist überdacht. Die Schwerkraft auf Adrastea ist vernachlässigbar; es herrscht mehr oder weniger Schwerelosigkeit. Trotz Magnetstiefeln sollten sich Personen an Handläufen festhalten.

\newsubsection{Verfolgung des Shuttles}

Die Entführer versuchen, über einen niedrigen Jupiter-Orbit zu flüchten, um eine Verfolgung innerhalb der Atmosphäre zu erschweren. In der Nähe des Jupiter sind Sensoren nur eingeschränkt nutzbar. Dies wiederum erlaubt es, Verfolgern unerkannt von hinten im Schatten des Fusionstriebwerks anzugreifen. Die Geiseln sind im Laderaum des Shuttles eingesperrt. Für ein Entermanöver ist die Dawn of Day am ehesten geeignet. Mit einem Dockingtunnel kann sie an das Shuttle der Entführer andocken. Die Schleuse in das Schiff kann entweder mit einem Magschlossknacker entriegelt oder aufgeschweißt werden. Toro Alvarez ist bereit, den Ermittlern Flankenschutz zu bieten und für ein Ablenkungsmanöver zu sorgen, indem er mit den Entführern in Verhandlungen tritt. Eine Valkyrie bietet neben einem Piloten auch Platz für eine zweite Person.

Hannibal und Slingshot werden um jeden Preis versuchen, ihrer Gefangennahme zu entgehen. Der Söldner hingegen ist nicht bereit, bei dieser Mission getötet zu werden.
\vfill

\pageimage{images/hellgate_landeplattform.jpg}


\begin{remarks}
	Bei den drei Optionen ist zu beachten, dass sie in Herausforderung und Zeitaufwand von oben nach unten gestaffelt sind. Zum Zeitpunkt der Entführung ist der Plot erst zu einem Drittel gespielt. Der Spielleiter sollte je nach angepeilter Spieldauer eine der Optionen auswählen.

	Der Showdown ist nicht darauf ausgelegt, die Entführer entkommen zu lassen. Die Hoffnung der Kidnapper, mit ihrem Täuschungsmanöver die Verfolger abzuschütteln, darf also nicht aufgehen.
	
	Wurde Hannibal keinem Gehirnscan unterzogen, sollte nachfolgend einem Psychonauten ein Gehirnscan ermöglicht werden. Wenn Hannibal bereits einem Gehirnscan unterzogen wurde, sollte der Spielleiter darauf abzielen, dass sowohl Hannibal als auch Slingshot getötet werden.
	
	Die beiden Attentäter dürfen dem Cynarian oder dem Protektorat zumindest nicht für weitere Untersuchungen geistig intakt zur Verfügung stehen. Im Zweifelsfall tötet die KI mit einem Brainburner das Gehirn des Attentäters.
\end{remarks}
