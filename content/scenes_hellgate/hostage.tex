%% Copyright 2019 Bernd Haberstumpf
%% License: CC BY-NC
% !TeX spellcheck = de_DE
\pageimage{images/cmyk/hellgate_security_floorplan_cmyk.jpg}
\newsection{Die Geiselnahme}

Haben die Charaktere ihre Befragung im St"utzpunkt des Sicherheitsdienstes abgeschlossen, werden sie von Karl Sandos dem Stationsleiter wieder in den Eingangsbereich des St"utzpunktes begleitet. 

Zum Einstieg in die folgende Szene sollte der Spielleiter, wenn m"oglich abwarten, bis sich die Charaktere au\3erhalb des St"utzpunktes und die Charaktere im St"utzpunkt in einem Informationsaustausch "uber das ComNetz befinden. Die vergangenen Ereignisse bieten ausreichend Stoff zu einer ausgiebigen Diskussion. W"ahrend des Gespr"achs bricht dann die Verbindung zwischen den beiden Gruppen pl"otzlich ab. 


\newsubsection{"Uberfall auf den St"utzpunkt}

Kurz nach dem Kommunikationsabbruch "offnet sich der Zugang zum St"utzpunkt und ein Gegenstand wird in den Raum geworfen. Die \emph{Personal Area Networks (PAN)}, die Anbindungen elektronischer Systeme an das Gehirn der Personen im Raum fallen aus. Jeder Charakter in der Station sollte nun einen Konstitutionswurf w"urfeln, um nicht kurzzeitig das Bewusstsein zu verlieren. Das PAN des Omega-Ermittlers, wenn er sich auf dem St"utzpunkt befindet, wird als Erstes in Teilen wieder aktiv. Der Ausfall f"uhrt aber f"ur eine kurze Zeit zu k"orperlichen und sensorischen Einschr"ankungen.

Mit dem "Offnen des Zugangs st"urmen zwei bewaffnete Personen in den Raum und beginnen sofort mit vollautomatischen Railguns zu feuern. Bei den Eindringlingen handelt es sich um Slingshot und dem S"oldner \emph{Smith Handerson}. Beide Angreifer sind mit Kampfanz"ugen, Handerson zus"atzlich mit einem Helm mit Sichtschutz ger"ustet. Befindet sich der Assistenz Ermittler des Protektorats im Eingangsbereich, wird er als Omega-Soldat sofort erkannt und von Smith Handerson ins Visier genommen. Ein weiteres Ziel ist Karl Sandos, der daraufhin hinter dem Tresen zu Boden geht. Ist kein Omega-Soldat anwesend, schie\3t einer der Angreifer auf einen weiteren Sicherheitsbeamten \emph{Luke Lengdon}, der dabei in die Brust und am Kopf getroffen wird und daraufhin ins Koma f"allt. Beim folgenden Kampf wird nur ein Omega die M"oglichkeit haben, eine Waffe zu ziehen oder in den Nahkampf zu gehen, wobei sein Handicap bestehen bleibt. Bei ihrem Angriff sollte es den Angreifern leicht fallen die "uberrumpelten Personen im Eingangsbereich kampfunf"ahig zu machen und sie danach in Schach halten zu k"onnen. Die Angreifer sammeln die zu Beginn geworfene EMP Schockgranate ein, die die PANs der Anwesenden au\3er Kraft gesetzt haben.

\newsubsection[Au\3erhalb des Geb"audes]{Ausserhalb des Geb"audes}
W"ahrend die Angreifer den St"utzpunkt "uberfallen, um Hannibal zu befreien, m"ussen die Ermittler au\3erhalb des St"utzpunktes die neue Situation erst einmal verarbeiten. Zun"achst wollen sie vermutlich in Erfahrung bringen, wodurch und wie die Verbindung zu ihren Mitstreitern ausfallen konnte. Grace Anders wird in diesem Zusammenhang versuchen, ihren direkten Vorgesetzten Karl Sandos und danach den Sicherheitschef Henk Arongate zu kontaktieren. Von ihm erf"ahrt sie, dass nur das ComNetz beim Sicherheitsst"utzpunkt gest"ort ist. Er verspricht Leute zum St"utzpunkt zu schicken, weist aber darauf hin, dass die Ermittler sich am n"achsten zum St"utzpunkt befinden. Henk Arongate erkl"art, dass im Umkreis von rund 30 Metern um den St"utzpunkt das ComNetz ausgefallen w"are. 

Der St"utzpunkt selbst liegt an einem von zwei Seiten zug"anglichen Tunnel. Vor dem geschlossenen Tor liegt der schwer verletzte Luke Lengdon, nach wie vor nicht ansprechbar. Andere Personen sind nicht erkennbar. Falls die Charaktere nicht selbst aktiv werden, bittet Grace Anders ihr Deckung zu geben und pirscht sich an ihren Ex-Freund heran, um dort Ersthilfe zu leisten.

Der Eingangsbereich ist auf den ersten Blick hin leer, wobei der Tresen nicht einsehbar ist. Auf dem Boden sind blutige Schleifspuren  sichtbar. Um den Tresen herum sind Einschussl"ocher zu erkennen. Die Scheibe des Tresens ist zerst"ort. Nach der Erst"urmung des Eingangsbereiches gilt es zun"achst das Gel"ande abzusichern und den verletzten Lengdon zu stabilisieren. Im Eingangsbereich und im Gangsegment vor dem Sicherheitsst"utzpunkt befinden sich keine weiteren Personen. Eine Erste-Hilfe-Ausr"ustung mit medizinischem Expertensystem findet sich im vorderen Bereich des St"utzpunktes.

\newsubsection{R"uckblende}
Nachdem Handerson und Slingshot den Eingangsbereich des St"utzpunktes unter ihre Kontrolle gebracht haben, "offnen sie mit dem Identit"atsimplantat von Karl Sandos die T"ur zu den hinteren R"aumlichkeiten. Dieser Bereich umfasst Gef"angniszellen, einen Verh"orraum und B"uros. Die Angreifer sperren alle Personen au\3er dem verletzten Luke Lengdon im Geb"aude gemeinsam in eine Zelle und verlassen dann den St"utzpunkt mit Hannibal, einem der Ermittler und den beiden Frauen der Minenbesatzung als Geiseln. Den Omega-Krieger der Ermittlergruppe, falls anwesend, werden sie in der Gef"angniszelle zur"ucklassen. Bei ihrer Flucht hinterlassen sie im hinteren Teil des St"utzpunktes einen St"orsender, der das ComNetz in der Umgebung des St"utzpunktes lahmlegt, zusammen mit einem Funkger"at, um Sicherheitskr"afte in die Irre f"uhren zu k"onnen.

\newsubsection{Inspektion des Eingangsbereichs}

Nach der Absicherung des Eingangsbereichs k"onnen die Ermittler den Raum weiter untersuchen. Der Tresen kann mit der Chipkarte von Grace Anders betreten werden. Hinter dem Tresen sind weitere Blutspuren zu entdecken. Versuchen die Anwesenden die T"ure in die inneren Bereiche zu "offnen, werden sie feststellen, dass nicht einmal Grace Anders die T"ur "offnen kann. Sie vernehmen eine Stimme hinter der T"ure, die ihnen droht die Geiseln zu t"oten, falls jemand versucht durch die T"ur zu kommen. Bei dem Sprecher handelt es sich um Slingshot, der "uber Sprechfunk so lange wie m"oglich versucht den Eindruck zu vermitteln, die Entf"uhrer bef"anden sich noch im St"utzpunkt. 

Kurz nachdem die Charaktere den St"utzpunkt betreten, treffen weitere Mitarbeiter des Sicherheitsdienstes in Begleitung von zwei Sanit"atern ein. Der Eingreiftrupp setzt sich aus zwei Norms und drei Mutant zusammen. Sie tragen die Sicherheitswesten des Sicherheitsdienstes und jeweils eine Bolzenpistole. Die Sanit"ater sind Norms. Da es sich bei Hellgate um einen Cynarian St"utzpunkt handelt, sind keine Omega-Soldaten aus den Streitkr"aften des Protektorats auf Hellgate im Einsatz. Angef"uhrt wird der Trupp durch einen \emph{Luke Dexter}, der sich direkt "uber den Verfall in Kenntnisse setzen l"asst. W"ahrenddessen l"asst er durch seine Leute den Eingangsbereich und die G"ange absichern. Die Sanit"ater k"ummern sich um den schwer verletzten Luke Langdon.

\newsubsection{Im Zellentrakt}

Im Zellentrakt sind die "ubrigen Charaktere zusammen mit den Minenarbeitern eingesperrt. Von den Minenarbeitern und der Ermittlergruppe sind insgesamt 4 Personen inklusive Hannibal in den H"anden der Entf"uhrer. Es ist vom Zellentrakt aus nicht zu erkennen, dass die Entf"uhrer das Geb"aude bereits verlassen haben. Unter den Gefangenen befindet sich der angeschossene Karl Sandos. Die Entf"uhrer haben ihren Opfern erlaubt ein Erste-Hilfe-Kit mit in die Zelle zu nehmen. Es ist also angebracht, den schwer verletzten Stationsleiter erst einmal zu verarzten. Die Gefangenen h"atten danach Zeit zu versuchen sich zu befreien oder sich bemerkbar zu machen. Die Zellen sind f"ur Randalierer und Aufst"andler gedacht. Dementsprechend sind sie nicht so gut gesichert wie eine regul"are Gef"angniszelle. Auch wenn das ComNetz noch immer lahmgelegt ist, ist die T"ur nicht ohne passendes Werkzeug zu knacken. Ben"otigt w"ar ein sogenannter \emph{Magschloss-knacker}, um des elektronischen Schlie\3systems zu knacken oder Werkzeug um die T"urhydraulik kurzzuschlie\3en. M"oglicherweise k"onnen hier aber auch Gegenst"ande aus dem Raum oder den Taschen der Minenarbeiter zweckentfremdet werden. Ein Luftschacht w"are eine weitere M"oglichkeit, zumindest Kontakt mit der Au\3enwelt aufzunehmen. Der Spielleiter kann gro\3z"ugig kreative Ideen gew"ahren lassen. Befindet sich kein Charakter unter den Gefangenen werden diese sich ruhig verhalten. Der Spielleiter sollte je nach Spielfluss entscheiden wie viel Zeit an dieser Stelle aufgebracht wird. 

\newsubsection{Auf der Flucht}

W"ahrend der Vorkommnisse im St"utzpunkt sind die Entf"uhrer zusammen mit ihren Geiseln auf dem Weg zu ihrem Shuttle, um Hellgate zu verlassen. Kurz nach dem Verlassen des St"utzpunktes haben die Angreifer ihre Kampfausr"ustung gegen schusssichere Westen und einfache Bolzenpistolen getauscht, um nicht aufzufallen. Slingshot, Smith Handerson und Hannibal sind jeweils mit einer Schusswaffe bewaffnet und treiben die Geiseln vor sich her. Um die PAN Systeme der Geiseln zu st"oren, nutzen die Geiselnehmer ebenfalls einen St"orsender mit kurzem Radius.

Nach der Absicherung des Eingangsbereichs k"onnen die Charaktere zusammen mit Luke Dexter dem Anf"uhrer der eingetroffenen Sicherheitskr"afte in Verhandlung mit den Entf"uhrern treten. Sie k"onnen dabei durch die T"ure in die hinteren R"aume mit den Entf"uhrern in Kontakt treten. Die Entf"uhrer nutzen wie schon beschrieben ihr Sprechfunkger"at, um die Illusion aufrecht zu halten sich noch im Sicherheitsst"utzpunkt aufzuhalten. Fordern die Charaktere nach einem Lebenszeichen ihres Freundes oder einer anderen Geisel, kann Slingshot den entf"uhrten Ermittler bitten ein kurzes Lebenszeichen von sich zu geben. Dabei kann dieser versuchen eine geheime Botschaft zu "ubermitteln.

Fr"uher oder sp"ater wird der Eingreiftrupp zusammen mit der Ermittlergruppe die hinteren R"aume des St"utzpunktes st"urmen. Die Vorbereitung wird nach wie durch den St"orsender der Entf"uhrer erschwert. Eine Abstimmung mit anderen Kr"aften der Station ist nur bedingt zu bewerkstelligen. 

Zu ihrer "Uberraschung sto\3en die Angreifer auf keine Gegenwehr. Die Gefangenen sind schnell befreit. Bei genauerer Untersuchungen der eroberten R"aume werden sowohl der St"orsender als auch das Sprechfunkger"at sichergestellt. Charaktere mit milit"arischem Hintergrund identifizieren den St"orsender als ein "alteres Modell aus Milit"arbest"anden mit allerdings unbekannter Herkunft. Auch ist einem Soldaten die EMP Schockgranate, nach einer Beschreibung von Augenzeugen bekannt.
\vfill

\begin{remarks}
	Die Geiselnahme ist auf einen schnellen Szenenwechsel zwischen den Vorf"allen im St"utzpunkt und den anderweitig Aktivit"aten ausgelegt. Die Szenenwechsel sollten immer so gestaltet werden, dass die Spieler nur die Information erhalten, die auch ihren Charakteren zum jeweiligen Zeitpunkt zur Verf"ugung stehen. So d"urfen die Spieler, nach dem "Uberfall auf den St"utzpunkt, erst von der Flucht der Geiselnehmer erfahren, wenn der Zellentrakt des St"utzpunktes gest"urmt wurde.
	
	Der erste Angriff der Kidnapper sollte so ausgelegt werden, dass die Angreifer am Ende der Szene die Oberhand gewinnen.

	Slingshot steht mit Artisan, dem Stellvertreter des Protektors in Kontakt. Dieser wiederum ist mit weiteren Attent"atern und USI-Agenten in Kontakt. Die Attent"ater selbst sind sich allerdings nicht ihrer Aktivit"aten als Attent"ater vollst"andig bewusst. Genauso wenig nimmt ihr menschlicher Geist die anderen Attent"ater als solches bewusst wahr. Alle Erinnerungen an ihre ``zweite Identit"at'' werden aus ihrem Geist ausgeblendet und werden auch von einem Psychonauten nicht als solches entdeckt.
	
	Im Auftrag von USI-Agenten wird Slingshot von Handerson auf Armageddon nach der Havarie der Mine HeM05 abgeholt und fliegt mit ihm nach Hellgate, um dort Hannibal abzuholen und nach Kallisto zu bringen. Sie kommen auf Hellgate an, kurz bevor die Charaktere Hellgate erreichen. Slingshot unter dem Decknamen Drake kann dadurch verfolgen, wie Hannibal zusammen mit den anderen Minenarbeitern zum St"utzpunkt der Sicherheitsmannschaft gebracht wird.
\end{remarks}


\newsection{Showdown auf Hellgate}

Die Entf"uhrer haben ihr Shuttle auf der Au\3enseite von Adrastea bei einer Wartungsschleuse nahe dem Raumhafen verankert. Nachdem die Charaktere im St"utzpunkt herausgefunden haben, dass sich die Entf"uhrer nicht mehr dort befinden, gibt es mehrere M"oglichkeiten sie aufzusp"uren. 

Am Erfolgversprechendsten w"are es, die nach wie vor auftretenden St"orungen im ComNetz der Station zu verfolgen. 

F"ur die Navigation im St"utzpunkt m"ussen die Kidnapper kurzzeitig den St"orsender deaktivieren. Das gibt den Geiseln die M"oglichkeit eine kurze Nachricht zu "ubermitteln.

Folgende Szenarien, die Entf"uhrer zu stellen, sind denkbar:

\begin{description}
	\item [Lagerkomplex] Im Lagerkomplex des Raumhafens g"abe es die M"oglichkeit einen Hinterhalt zu legen und die Entf"uhrer zu 		
		"uberraschen.
	\item [Wartungsschleuse] Die Attent"ater und ihre Geiseln haben Hellgate verlassen und sind dabei, ihr Shuttle zu betreten.
	\item [Shuttleverfolgung] Ist das Shuttle bereits gestartet, bleibt den Charakteren nur, das Shuttle zu verfolgen und zu entern, um die 	
		Gefangenen zu befreien.
\end{description}


\newsubsection{Wartungsschleuse}

Hellgate besitzt neben dem Raumhafen eine Reihe von Ausg"angen, die "uber Tunnel an die Oberfl"ache des Mondes erreichbar sind. Sie dienen als Flucht und Rettungstunnel oder bieten einen Zugang zu verschiedenen Sensorplattformen zumeist in Richtung des Jupiter. 

Das Shuttle der Entf"uhrer hat auf einer Landeplattform mit seinen Andockklammern festgemacht. Der Zugangsbereich zum Landedeck umfasst zwei Luftschleusen: Eine als Zu- und Ausgang f"ur Personen, eine zweite als Lieferweg f"ur Frachtcontainer. Eine Transportgondel f"ahrt "uber 500 Meter zu einem gro\3en Zwischenlager. Ein Tender "ahnliches Schienenfahrzeug erlaubt es Material von einem angedockten Schiff zur Schleuse zu transportieren. Der Weg zum Landebereich ist "uberdacht.  Die Schwerkraft auf Adrastea ist vernachl"assigbar. Es herrscht mehr oder weniger Schwerelosigkeit. Trotz Magnetstiefeln sollten sich Personen an Handl"aufen einhaken.

\newsubsection{Verfolgung des Shuttles}

Die Entf"uhrer versuchen "uber einen niedrigen Jupiter Orbit zu fl"uchten, um eine Verfolgung innerhalb der Atmosph"are zu erschweren. In der N"ahe des Jupiter sind Sensoren nur eingeschr"ankt nutzbar. Dies wiederum erlaubt es, Verfolgern unerkannt von hinten im Schatten des Fusionstriebwerks anzugreifen. Die Geiseln sind im Shuttle im Laderaum eingesperrt. F"ur ein Enterman"over ist die Dawn of Day am ehesten geeignet. Mit einem Dockingtunnel kann sie an das Shuttle der Entf"uhrer andocken. Die Schleuse in das Schiff kann entweder mit einem Magschlo\3knacker entriegelt oder aufgeschwei\3t werden. Torro Alvarez ist bereit, mit seinen Valkyrien den Ermittlern Flankenschutz zu bieten und f"ur ein Ablenkungsman"over zu sorgen, indem er mit den Entf"uhrern in Verhandlungen tritt. Eine Valkyrie bietet neben einem Piloten auch einer zweiten Person Platz.

Hannibal und Slingshot werden um jeden Preis versuchen ihrer Gefangennahme zu entgehen. Der S"oldner hingegen ist nicht bereit bei dieser Mission zu sterben.

\begin{remarks}
	Bei den drei Optionen ist zu beachten, dass sie in Herausforderung und Zeitaufwand von oben nach unten gestaffelt sind. Zum Zeitpunkt der Entf"uhrung ist der Plot erst zu einem Drittel gespielt. Der Spielleiter sollte je nach angepeilter Spieldauer sich f"ur eine der Optionen entscheiden.

	Der Showdown ist nicht darauf ausgelegt die Entf"uhrer entkommen zu lassen. Die Hoffnung der Kidnapper mit ihrem T"auschungsman"over
	die Verfolger abzusch"utteln darf also nicht aufgehen. 

	Wurde Hannibal keinem Gehirnscan unterzogen, sollte nachfolgend einem Psynchonauten ein Gehirnscan erm"oglicht werden. 
	Wenn Hannibal einem Gehirnscan bereits unterzogen wurde, sollte der Spielleiter anpeilen, dass sowohl Hannibal als auch Slingshot get"otet werden. 
	
	Die beiden Attent"ater d"urfen Cynarian oder dem Protektorat zumindest nicht f"ur weitere Untersuchungen
	geistig intakt zur Verf"ugung stehen. Im Zweifelsfall t"otet deren KI mit einem Brainburner das Gehirn des Attent"aters und nicht selbst.
\end{remarks}
