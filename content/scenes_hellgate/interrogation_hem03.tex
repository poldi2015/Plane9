%% Copyright 2019 Bernd Haberstumpf
%% License: CC BY-NC
% !TeX spellcheck = de_DE
\newsection{Befragung der HeM05 Besatzung}

Die 10 geretteten Minenarbeiter, werden wie die J"agerpiloten m"oglichst kurz vor dem Eintreffen der Ermittler aus der Dekompressionskammer eintlassen. Beim Eintreffen der Charaktere auf Hellgate befinden sich geretteten Mienenarbeiter in der Kantine der J"agerstaffel. Grace Anders wird die Charaktere zur Kantine begleiten. Vor den R"aumlichkeiten treffen die Charaktere auch den bekannten Pilotenausbilder Jos\'{e} \frqq{}Torro\flqq{} Alvarez. Torro ist ein kleiner drahtiger Spacer von fr"ohlicher Natur dem die Jahre als Pilot allerdings schon deutlich zugesetzt haben. Torro der die Rettungsaktion geleitet hat kann einen ersten Einblick in die Geschehnisse geben. Nach dem Gespr"ach mit Torro k"onnen sich die Charaktere den Minenarbeitern zuwenden. Sie k"onnen zun"achst entscheiden ob sie diese einzeln Interviewen wollen oder sie alle direkt in der Kantine aufsuchen. Sollen die Arbeiter einzeln befragt werden bietet Torro an Florence zu den Ermittlern zu den Ermittlern zu bringen. Danach kann Grace "ubernehmen und die Besatzung der Mine einzeln heraus bitten.

Im folgenden die Aussagen der Beteiligten:

\begin{description}
	\item[Torro] \say{Bei einem "Ubungsflug durch die obersten Athmostph"arenschichten erhielten wir einen Notruf der Mine HeM05. Da 		meine Trainngsstaffel mit insgesamt vier Valkyrien, 3 Rookies und mir, der Mine am n"achsten waren sind wir tiefer in die 		
		Athmosph"are eingetaucht und konnten dort gl"ucklicherweise die Mine nach kurzer Zeit lokalisieren. Da wir die Arbeiter nicht mit unseren Jagdmachinen selbst retten konnten blieb uns nur die M"oglichkeit an die Mine selbst anzudocken. Zugegebenerweise ein recht waghalsiges und f"ur die Auszubildenden ein risikoreiches Man"over. Wir waren zu diesem Zeitpunkt bereits in eine f"ur uns kritischen Athmosph"arenbereich besunken. Mit viel Gl"uck schafften wir es dann aber drei Maschinen an die Mine anzudocken und mit voller Schubleistung die Mine auf eine H"ohe zu bringen die es der Schlepperinsel erlaubte die Mine in den Orbit zu ziehen. Ein hei\3er Ritt kann ich Ihnen nur sagen.}
	\item[Florence (Kommandantin)] \say{W"ahrend der ersten Systemmeldung dass einer der Tr"agerbalons der Station abgekoppelt wurde 	
		befanden sich Juri Smirnov, Blackwind, ZDee und ich auf der Br"ucke. Greydog war in der Minenanlage besch"aftigt. Die anderen waren nach ihren eigenen Angaben im oberen Bereich der Mine. Ich beauftragte als erstes ZDee die Aufh"angung des Tr"agerbalons au\3erhalb der Mine zu kontrollieren und sandte einen Hilferuf an die Hellgate Station. Einige Minuten sp"ater beobachteten wir auf der Br"ucke wie von den Au\3enkameras aufgenommen, Pitch in ihrem Raumanzug in die Tiefe st"urzte. Ca.~10 Minuten sp"ater l"oste sich der zweite Tr"agerbalon. Nach einem Notruf befahl ich die Evakuierung. Treffpunkt war das Rettungsshuttle. R"uckmeldung bekam ich von allen au\3er ZDee. Auf dem Weg zum Shuttle sammelten wir noch Salvador vor seinem Quartier ein. Er war gerade dabei sich fertig anzuziehen. Am Rettungshuttle traf die Br"uckencrew auf Isabell und Fernandez. Das Rettungsshuttle lie\3 sich nicht starten. Die Startsequenz war durch eine Manipulation blockiert. Deshalb blieb uns nichts anderes "ubrig die Dekompressionskammer aufzusuchen und auf Rettung zu hoffen. Auf dem Weg zur Dekompressionskammer l"oste sich offensichtlich der dritte Ballon. An der Kammer trafen Greydog und Hanibal auf uns. Hanibal hatte noch versucht "uber die Steuerung der Anlage die Manipulation der Ballons zu verhindern. ZDee war von seiner Au\3enmission nicht zur"uck gekommen.}
	\item[Juri Smirnov, Blackwind] Die Br"uckencrew best"atigt die Aussage von Florence.
	\item[Salvador] \say{Ich war in meinem Quartier als der Aufruf zur Evakuierung kam. Die Br"uckencrew kam kurz darauf bei meinem 	
		Quartier vorbei und nahm mich mit.}
	\item[Greydog] \say{Ich war an der Raffinerie mit Wartungsarbeiten im Au\3enbereich am unteren Ende der Raffinerie besch"aftigt. 
		Dadurch habe es nicht geschafft die anderen bereits am Shuttle zu treffen.}
	\item[Fernandez Lorend] \say{Ich habe Isabell mit der Justierung ihrer Zentrifugen f"ur die Analyse des Atmosph"arengemischs 
		unterst"utzt als der Notruf einging. Darauf hin machten wir uns unverz"uglich auf den Weg zum Rettungsshuttle.}
	\item[Isabell Sonderleiten] Isabell best"atigt die Aussage von Fernandez. Ein paar Tage vor dem Attentat vertraute Pitch ihr an, dass 
		sie auf eigene Faust gegen ein anderes Besatzungsmitglied recherchierte weil sie glaubte dieser sei f"ur die Havarie der HeM03 verantwortlich. Ihr Verdacht wurde geweckt durch Unregelm"a\3igkeiten in der Steuersoftware der Mine. Sie lie\3 sich deshalb auf HeM05 einschiffen um ihren Verdacht weiter zu verfolgen und den Attent"ater selbst zur Rede zu stellen. Wen sie im Verdacht hatte hat sie allerdings nicht verraten. Die Befragung von Isabell erfolgt nur stockend. Das erlebte und der Verlust von Pitch machen ihr offensichtlich stark zu schaffen.
	\item[Blackwind] \say{Pitch wurde als Attent"aterin identifiziert, weil sie bei der Abkopplung des Tr"agerballons im Au\3enbereich 
		der Mine t"atig war, was "uberhaupt nicht ihrem Arbeitsbereich entspricht. F"ur die Wartung und das Einspielen neuer Software hatte mich Pitch gebeten ihr tempor"ar Zugang auf die Steuerung des Rettungsshuttle zu geben. Pitch war bereits vorher auf HeM03 besch"aftigt.}
	\item[Hanibal] \say{Als das Abkoppeln des ersten Ballons durchgegeben wurde begann ich sofort die Ansteuerung der Tr"agerballons zu 
		kontrollieren da die softwaretechnische Wartung mir und Pitch unterlag. Dabei konnte ich eine Manipulation der Steuerung entdecken die wahrscheinlich die Abkopplung ausgel"ost hat. Als mein Rettungsversuch mi\3lang und der dritte Ballon sich gel"ost hatte machte ich mich auf den Weg zur Dekompressionskammer da Florence bereits die Manipulation des Shuttles durchgegeben hatte.}
\end{description}

W"ahrend der Befragung macht Grace Anders Meldung an ihren Vorgesetzten \emph{Karl Sandos} und "ubermittelt ihm die Aussagen der Besatzung der Mine. Karl Sandos gibt diese Informationen an \emph{Henk Arongate} weiter.

\begin{remarks}
	Die Minenarbeiter sind durch die Vorkommnisse nach wie vor angeschlagen. Die in diesem Kapitel zusammengefassten Aussagen k"onnten entsprechen emotional ausgeschm"uckt werden. Die Kommandantin Florence, eine Beta, wird die Crew wenn n"otig in Schutz nehmen und verteidigen.

	Isabel ist schon vor der Versetzung auf HeM05 mit Pitch befreundet und deshalb stark mitgenommen durch ihren Tod und dem Verdacht die Attent"aterin zu sein.

	Durch die Aussagen k"onnen die Ermittler bereits ermitteln, dass Pitch aller Vorraussicht nicht die Attent"aterin sein kann. Der zweite und vor allem der dritte Tr"agerbalon l"osten sich erst nach ihrem Absturz.		
\end{remarks}
