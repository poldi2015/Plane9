%% Copyright 2019 Bernd Haberstumpf
%% License: CC BY-NC
% !TeX spellcheck = de_DE
\newsection{Befragung beim Sicherheitsdienst}

Treffen Ermittler bevorzugt ein Psychonaut im St"utzpunkt des Sicherheitsdienstes auf Hellgate ein, kann dort Hannibal verh"ort werden. Ein Psychonaut kann ihn dabei einem Gehirnscan unterziehen. Der St"utzpunkt ist "ahnlich einem Polizeirevier aufgebaut. Eine T"ur f"uhrt zun"achst in einen Eingangsbereich mit einem durch eine Plexiglasscheibe abgetrennten Tresen. Der diensthabende Sicherheitsbeamte empf"angt die Ermittler und fragt nach ihrem Begehren. Fordern die Charaktere eine Befragung der Minenarbeiter an, kontaktiert der Beamte den Stationsleiter Karl Sandos, der die Ermittler daraufhin pers"onlich empf"angt und sie durch den St"utzpunkt und zu den Zellen und zum Verh"orraum f"uhrt.

F"ur ein Verh"or k"onnen die Minenarbeiter einzeln in einen Verh"orraum gebracht werden. Im Folgenden wird davon ausgegangen, dass ein Psychonaut das Verh"or von Hannibal durchf"uhrt. Bei einem Verh"or durch einen Psychonauten sollte dieser alle Anwesenden au\3er evtl.~einem anderen Ermittler bitten, den Raum zu verlassen, um seine "`Befragung"' in Ruhe durchf"uhren zu k"onnen. Der milit"arische Assistenzermittler des Protektorats oder ein anderer der Spielercharaktere sollte anwesend blieben, um Hannibal zu fixieren. 

Die Aktivit"aten von Hannibal zum Zeitpunkt der Attentate auf HeM03 und HeM05 finden sich im Gehirn in derselben Form wie sie Hannibal zu Protokoll gegeben hat: Auf HeM03 "uberw"altigen Lionel Hampton, Ice Diver und er Sent, nachdem Hannibal bemerkt wie Sent in Begriff ist die Computeranlage der Mine zu manipuliert. Durch eine Messerattacke wird Lionel Hampton get"otet, Ice Diver und Hannibal wiederum k"onnen Sent t"oten, jedoch ist es bereits zu sp"at, um die Manipulation aufzuheben. Auf der HeM05 Mine wiederum erlebt der Psychonaut, wie Hannibal vergeblich versucht die Manipulation der Tr"agerballonsteuerung aufzuheben. Hannibal widersetzt sich der Untersuchung seiner Erinnerungen nicht, nachdem der Ermittler in seine Gedanken eingedrungen ist. Die geistigen F"aden sind leicht zu verfolgen. Die Bilder sind scharf gezeichnet und leicht zu verfolgen. Mit der Zeit stellt der Psychonaut allerdings fest, dass den Erinnerungen jegliche Form von Emotionen fehlen und geradlinig und detailarm wirken. Wie in einem Film wird er durch inszenierte Szenen gef"uhrt. 

Versucht er das Gehirn des Attent"aters wieder zu verlassen, wird er pl"otzlich durch die KI in Hannibals Kopf daran gehindert. Er findet sich unversehens auf einer k"unstlichen gr"unen Wiese mit blauem Himmel wieder auf der nur eine einzige vollkommen wei\3e androgyne Person steht. Er wird in die Gedanken dieser Person gezogen und erf"ahrt von einem Zwang, die Minenanlagen auf dem Jupiter zu zerst"oren. Kurz darauf nimmt er einen Countdown wahr und sollte versuchen den Geist von Hannibal wieder zu verlassen.

\begin{remarks}
	Ein Gehirnscan ist im Regelwerk im hinteren Teil des Buches beschrieben. Er kann "ahnlich zu einem Matrixkampf bei anderen Cyberpunk Rollenspiel gespielt werden. Mit dem Countdown warnt die KI in Hannibals Gehirn den Psychonaut vor einem Brainburner der, wenn ausgel"ost die Gehirne beider Personen in einer Kettenreaktion, von durch Stromst"o\3e "uberlasteten Synapsen, zerst"ort.

	In dieser Szene erfahren die Ermittler, dass das Gehirn des Attent"aters manipuliert wurde und er zu seiner Tat gezwungen wurde. Das Auftauchen der wei\3en Gestalt ist ein Hilferuf der KI die versucht sich gegen ihre Programmierung zu wehren. Der Begriff K"unstliche Intelligenz sollte zu diesem Zeitpunkt noch nicht fallen. Die Spielleiter sollten den Spielern erlauben eigenen R"uckschl"usse aus den bizarren Erlebnissen zu ziehen. Viel Zeit daf"ur wird ihnen allerdings nicht gew"ahrt.
\end{remarks}
