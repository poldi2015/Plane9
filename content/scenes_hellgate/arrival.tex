%% Copyright 2019 Bernd Haberstumpf
%% License: CC BY-NC
% !TeX spellcheck = de_DE
%\newpage
%\null
\pageimage{images/cmyk/hellgate_cmyk.jpg}
\newsection{Eintreffen auf Hellgate}

Der Flug von Armageddon nach Hellgate auf dem Mond Adrastea dauert mit dem Shuttle Dawn of Day rund zwei ereignislose Tage. W"ahrenddessen hat die Gruppe Zeit, sich mit dem Shuttle vertraut zu machen.

Die vor kurzem havarierte Mine HeM05 ist beim Eintreffen der Ermittler an einer gigantischen Schlepperinsel angedockt. Die Schlepperinsel ist ein 2 Kilometer durchmessendes Raumfahrzeug, das mit gewaltigen Schubd"usen bis in die "au\3eren Atmosph"arenregionen des Jupiter eintauchen kann, um dort HE-3-Minen abzusetzen oder einzusammeln. Sie schwebt beim Anflug auf die Hellgate-Minenkolonie majest"atisch nahe dem Mond Adrastea "uber der gewaltigen Oberfl"ache des Jupiter. Ein Staubpartikelstrom, der Hauptring des Jupiters, bildet einen feinen Nebelschleier um den Mond, der den Planeten auf einem niedrigen Orbit von 130'000 Kilometern umkreist. Hellgate befindet sich bis auf einen Anflugtunnel fast vollst"andig im Inneren des Mondes. Die Station selbst besteht aus einem Raumhafen, technischen Anlagen, einem Verwaltungskomplex, Lagerhallen, L"aden, Wohnquartieren und einem Vergn"ugungsviertel. Im Ganzen umfasst die Anlage rund~30 km$^{2}$. Wie alle eilig aufgesetzten Einrichtungen finden sich zahlreiche Provisorien, unvollst"andig eingerichtete Gangsegmente und herumstehendes Material in der Station.

Beim Ansteuern des Anflugtunnels wird die Dawn of Day von der Flugkontrolle kontaktiert und nach einer Legitimation gefragt. Nach den Formalit"aten wird das Shuttle "uber einen Leitstrahl in den Landungstunnel navigiert. Der Pilot der Ermittlergruppe kann hierbei sein K"onnen unter Beweis stellen. Dem Spielleiter bleibt "uberlassen, wie weit er den Landeanflug ausschm"uckt. Beim Eintreffen im Raumhafen herrscht reger Betrieb. Eine gro\3e F"ahre bringt neue Minenarbeiter in die Kolonie und holt Arbeiter ab, die nach Kallisto abreisen m"ochten. Mehrere Shuttles werden auf dem Landedeck instand gesetzt. In einem separaten Bereich ist die J"agerstaffel der Station untergebracht. Acht J"ager der Valkyrie-Klasse sind dort aufgereiht.

Zum Zeitpunkt des Eintreffens der Gruppe ist die Mine HeM05, vert"aut an der Schlepperinsel, teilweise zerlegt. Die Minen HeM01, HeM02 und HeM04 sind im Einsatz. Die Besatzung der zerst"orten Mine HeM03 ist teils zur Erholung auf Kallisto, teils bereits wieder im Einsatz auf den anderen Minen.

Im Raumhafen angekommen, werden die Charaktere von \emph{Grace Anders} erwartet. Grace, eine h"ubsche junge Frau mit kurzen blonden Haaren, ist Teil des lokalen Sicherheitsdienstes der Cynarian Corporation. F"ur den Aufenthalt der Charaktere ist sie von \emph{Henk Arongate} dem Chef des Sicherheitsdienstes, zur Unterst"utzung der Ermittler abgestellt. Sie steht den Ermittlern w"ahrend ihres gesamten Aufenthalts treu zur Seite, kann Recherchen beauftragen, kennt die Station mit ihren verwirrenden G"angen und kann lokale Unterst"utzung anfordern. Beim Eintreffen der Charaktere erkl"art sie den Ermittlern, dass es sich bei Hellgate bekannterma\3en um eine Minenkolonie mit entsprechenden Gepflogenheiten handelt. Auf einen ruppigen Umgang miteinander sollte man sich deshalb gefasst machen. Dementsprechend tragen die Sicherheitskr"afte Schutzkleidung, im Falle von Grace Anders auch eine Waffe. Des Weiteren erfahren die Ermittler, dass ihre Untersuchungen m"oglicherweise kritisch aufgenommen werden k"onnten, da man meint, die Vorkommnisse k"onnten auch lokal aufgekl"art werden.

\begin{remarks}
	\underline{Ausr"ustung:}

	Die Spieler k"onnen den beil"aufig eingestreuten Hinweis auf die Gepflogenheiten auf Hellgate als Ansto\3 nehmen, sich selbst passend auszur"usten. Milit"arische Schutzausr"ustung oder entsprechende Waffen sind allerdings unangebracht. Im Zweifelsfall weist Grace Anders die Ermittler freundlich, aber bestimmt darauf hin, dass es sich bei Hellgate um eine zivile, von Cynarian geleitete Einrichtung handelt und nicht um ein Kriegsgebiet.

	\underline{Henk Arongate \& Grace Anders:}
	
	Wenn die Ermittler Henk Arongate, den Sicherheitschef, direkt kontaktieren, wird er sie h"oflich begr"u\3en, verweist sie dann aber weiter an Grace. Grace Anders ist eine kompetente, lebenslustige und loyale Unterst"utzerin. Sie dient dem Spielleiter dazu, den Spielern unter die Arme zu greifen, sollten sie selbst nicht weiterkommen. Dar"uber hinaus bietet die attraktive Sicherheitsbeamtin die M"oglichkeit, zus"atzlich zur Hauptgeschichte etwas mehr Individualit"at einzuflechten. Eine ausf"uhrliche Beschreibung von Grace Anders findet sich \cref{sec:graceanders}.
\end{remarks}


