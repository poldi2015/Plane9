%% Copyright 2019 Bernd Haberstumpf
%% License: CC BY-NC
% !TeX spellcheck = de_DE
\newsection{Eintreffen auf Hellgate}

Der Flug von Armageddon dauert rund 4 ereignislose Tage  mit dem Shuttle Dawn of Day w"ahrenddessen sich die Gruppe mit dem Shuttle vertraut machen k"onnen. 

Die HeM05 ist beim Eintreffen der Ermittler an der gigantischen Schlepperinsel der Hellgate Station angedockt. Die Schlepperinsel ist ein 2 Kilometer langes und breites Raumfahrzeug das mit gewaltigen Schubd"usen bis in die "au\3eren Athmosph"arenregionen des Jupiter eintauchen kann um dort die HE-3 Mienen abzusetzen oder einzusammeln. Die Schlepperinstel schwebt beim Anflug auf Hellgate majest"atisch nahe dem Mond Adrastea "uber der gewaltigen Fl"ache des Jupiter. Kleinste Partikel bilden eine Schleier auf diesem niedrigen Orbit von 130'000 km "uber dem Planeten. Hellgate befindet sich bis auf den Anflugtunnel fast vollst"an dig im Inneren des Mondes. Die Station selbst besteht aus dem Raumhafen, technischen Anlagen, Lagerhallen und R"aumen und Wohnquartieren, Lokale, Bars und L"aden. Im Ganzen umfasst die Anlage ca.~30 km$^{2}$. Wie in alles neuen eilig aufgesetzen Einichtungen befinden sich viele Provisorien, nicht abgeschlossene G"ange und herumstehendes Material in der Station.

Beim Ansteuern des Anflugtunnels wird die Dawn of Day von der Flugkontrolle kontaktiert und nach einer Legitimation gefragt. Nach den ersten Formalit"aten wird das Shuttle "uber einen Leitstrahl in den Landungstunnel navigiert. Der Pilot in der Gruppe kann hierbei sein K"onnen unter Beweis\3 stellen. Dem Spielleiter bleibt "uberlassen wie weit er den Landeanflug ausschm"uckt. Beim Eintreffen im Raumhafen herrscht reger Betrieb, eine gro\3e F"ahre bringt gerade neue Minenarbeiter und holt Mitarbeiter die nach Kallisto abreisen m"ochten. Mehrere Shuttle werden gewartet. In einem separaten Bereich sind die Maschinen, 8 Valkyrien der J"agerstaffel untergebracht. 

Zum Zeitpunkt des Eintreffens der Gruppe ist die Mine HeM05 an der Schlepperinsel vert"aut und teilweise zerlegt. Die Minen HeM01 und HeM04 sind im Einsatz. Die Besatzung der zerst"orten HeM3 sind teils zur Erholung auf Kallisto und teils bereit wieder im Einsatz auf den anderen Minen.

Im Raumhafen angekommen werden die Charaktere bereits von \emph{Grace Anders} erwartet. Grace ist Teil des lokalen Sicherheitsdienstes der Cynarian Corporation. F"ur den Aufenthalt der Charaktere ist sie zur unterst"utzung der Ermittler von \emph{Henk Arongate} dem Chef des Sicherheitsdienstes abgestellt. Sie steht hiermit den Ermittlern w"ahrend ihres gesamten Aufenthalts treu zur Seite, kann Recherchen beauftragen, kennt die Station mit ihren verwirrenden G"angen und kann lokale Unterst"utzung anfordern. Beim Eintreffen wird sie die Ermittler aufkl"aren dass es sich um eine Minenkolonie handelt und dadurch die Gepflogenheiten etwas ruppiger seien k"onnen. Aus diesem Grunde tragen die Sicherheitskr"afte Schutzkleidung und eine Waffe. Desweiteren erfahren die Ermittler dass ihre Untersuchungen m"oglicherweise kritisch aufgenommen werden k"onnten da man meint die Vorkommnisse k"onnten auch lokal gekl"art werden.

\begin{remarks}
	Die Spieler k"onnen die ersten Information von Grace Anders dazu nutzen sich selbst passend auszur"usten. Kontaktieren die Ermittler Henk Arongate direkt wird er sie h"oflich begr"u\3en, verweist sie dann aber weiter an Grace. Grace Anders ist eine junge h"ubsche aber auch vor allem kompetente und loyale Unterst"utzerin. Sie dient dem Spielleiter den Spielern unter die Arme zu greifen sollten sie selbst nicht weiter kommen und bringt der"uber hinaus eine pers"onliche Note ins Spiel mit ein.
\end{remarks}

\pageimage{images/cmyk/hellgate_cmyk.jpg}