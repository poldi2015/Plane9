%% Copyright 2019 Bernd Haberstumpf
%% License: CC BY-NC
% !TeX spellcheck = de_DE
\newsection{Das Geschehen auf HeM05}

F"unf Tage vor dem Attentat wurde die Mannschaft der Mine durch eine Rumpfmannschaft ersetzt. Die neue Mannschaft wird 10 Tage danach erwartet und trifft etwa zeitglich mit den Charakteren ein. Die Ankunft der zugeh"origen F"ahre k"onnen die Charaktere auf dem Flugdeck miterleben. Das Attentat selbst erfolgte chronologisch folgenderma\3en:

\begin{enumerate}
	\item Ein paar Tage nach der Ankunft auf HeM05 installiert Pitch eine "Uberwachungssoftware in der Minensteuerung, die Manipulationen 	
		blockiert und sie "uber Manipulationsversuch informiert. Pitch hat zu diesem Zeitpunkt ihren Kollegen Hanibal bereits im Verdacht den Absturz der Mine HeM03 "uber eine Softwaremanipulation veranlasst zu haben.
	\item Pitch macht nach dem Abflug der Minenmannschaft das Shuttle untauglich um eine Flucht des mutma\3lichen T"aters zu verhindern.
	\item Am Tag des Attentats versucht Hanibal zun"achst wie auf HeM03 eine Fehlfunktion der Anlagensteuern hervor zu rufen. Eine solche 
		Manipulation h"atte zu einem ma\3iven Minenschaden und zur Zerst"orung eines gro\3en Teils der Mine gef"uhrt.
	\item Als die Manipulation der Minensteuerung aufgrund der Software von Pitch mi\3lingt begibt sich Hanibal in einem Raumanzug in den 
		Au\3enbereich und koppelt Tr"agerballon Nummer eins ab.
	\item Pitch verfolgt Manibal und stellt ihn auf der Au\3enballustrade der Mine zur Rede als dieser gerade den ersten Ballon abkoppelt. 
		Es kommt zum Kampf. Hanibal st"urzt Pitch in den Abgrund.
	\item W"ahrend der Abkoppeln des zweiten Ballons wird er von ZDee "uberrascht kann aber diesen ebenfalls im Kampf "uberw"altigen und 
		in die Tiefe st"urzen.
	\item Nach dem abkoppeln des dritten Tr"agerbalons betritt Hanibal  ungesehen die Mine wieder, legt den Raumanzug ab und begibt sich 
		zum zweiten Treffpunkt bei der Dekompressionskammer.
\end{enumerate}

\newsection[Weitere Nachforschungen auf Hellgate]{Weitere Nachforschungen auf Hellgate}

Nachdem die Ermittler den Raumhafen verlassen haben, wird die Minenbesatzung von f"unf Gardisten des lokalen Sicherheitsdienstes abgef"uhrt und auf den St"utzpunkt der Sicherheitskr"afte gebracht. Diese "Uberf"uhrung wurde durch Henk Arongate pers"onlich in Absprache mit dem B"uro von Vandermool veranlasst. Das Vorgehen wird von Arbeiter auf dem Raumdeck beobachtet, unter anderem von Slingshot der unter dem Namen Drake bereits vor den Charakteren auf Hellgate angekommen ist um Hanibal zu kontaktieren. Zum Zeitpunkt des Eintreffens der Ermittler hat er bereits Kontakt mit den hiesigen Arbeitern aufgebaut. 

Nach der Befragung k"onnen die Ermittler weitere Nachforschungen auf Hellgate angehen oder bei Grace beauftragen. Die n"achsten Schritte der Ermittler wewrden sie voraussichtlich entweder zu den Quartieren der Minenbesatzung der HeM05 auf Hellgate oder zu Nachforschungen auf der Mine selbst f"uhren. Zuerst sollten sie aber einen Report an ihre Vorgesetzten ab geben. Erfolgt eine Meldung an das B"uro von Vandermool werden die Ermittler davon in Kenntnis gesetzt, dass die Minenarbeiter aus Sicherheitsgr"unden verlegt werden.


Folgende Informationen k"onnen bereits "uber Nachfragen direkt beschafft werden:

\begin{description}
	\item[Tr"agerballons] Die Tr"agerbalons lassen sich "uber das Anlagensystem not entl"uften. Abkoppeln lassen sie sich nur von au\3en, 
		direkt an der Kopplungsstelle des Ballons. Diese Informationen l"asst sich "uber Grace Anders bzw.~durch R"ucksprache mit Dr.~Petrova in Erfahrung bringen. Wenn Pitch nicht die Attent"aterin ist kommen nur noch Hanibal oder Greydog als Attent"ater in Betracht.
	\item[Greydog] Greydogs Geschichte l"asst sich ebenfalls "uber Grace Anders oder durch Nachfrage bei Dr.~Petrova plausibilisieren. Die 
		Wege in der Mine von der Raffinerieanlage zu Br"ucke und Aufenthaltsr"aumen sind mehrere hundert Meter lang. Zus"atzlich mu\3te Greydog erst wieder vom Au\3enbereich in die Mine zur"uck kehren und den f"ur die Jupiteratmosph"are angepassten Raumanzug ablegen.
	\item[HeM03] Nachforschungen zu HeM03 ergeben dass sich au\3er Florence, Pitch und Hanibal kein Mannschaftsmitglied mehr auf Hellgate befindet. Auf HeM03 waren insgesamt 50 Arbeiter besch"aftigt die sich gl"ucklicherweise fast alle mit Hilfe der Rettungsshuttle retten konnten. 
	\item[HeM03 Attent"ater] Angeblich hatten Lionel Hampton, Ice Diver und Hanibal versucht den Attent"ater Sent von der Manipulation der 
		Minensoftware abzuhalten. Lionel Hamption wie auch Sent wurden dabei get"otet. Ice Diver gilt seit dem Vorfall auf der Mineals vermisst.
	\item[Hanibal] Nachfragen zum Hintergrund von Hanibal ergeben dass Hanibal vor einem dreiviertel Jahr im Raumhafen von Valhalla als 
		Software-- und Sicherheitstechniker angestellt wurde.
	\item[Pers"onliche Hintergr"unde] Nachfragen zu Hintergr"unden der Minenbesatzung f"ordern zu Tage das keine auff"alligen Gelder 
		geflossen sind oder Angeh"orige unter Druck gesetzt werden konnten.
\end{description}

Diese Informationen sollten erst nach Abschlu\3 der Befragung zug"anglich gemacht werden damit sie nicht in die Befragung mit einflie\3en k"onnen und damit sofort zur Aufkl"arung der Ereignisse f"uhren.

Bei Nachfragen zur Schlepperfehlfunktion vor 9 Wochen l"asst sich herausfinden dass ein Softwarefehler Sch"aden an zwei Tr"agerpunkten f"ur Minen verursacht hatte. Die Reparaturen dauern noch an. Gl"ucklicherweise l"asst sich die Schlepperinsel nach wie vor mit ihren drei weiteren Andockpunkten nutzen. Die Fehlfunktion wurde durch eine Manipulation ausgef"uhrt von Hanibal ausgel"ost. Zum Zeitpunkt der Fehlfunktion war er allerdings bereits auf HeM03 im Einsatz. Durch Nachfragen im Flugbereich auf Hellgate erfahren die Spieler auch von von dem Shuttleabsturz vor 10 Wochen. Der Vorfall wird nicht als Attentat eingesch"atzt sondern wird als Beispiel erw"ahnt dass nicht alle Unf"alle als Attentate eingestuft werden m"ussen. Der Shuttleabsturz oder besser Kollision mit einer F"ahre auf dem Hangardeck wurde durch ein Fehlsteuerung des "Uberm"udeten und mit Wachhaltern gedobten Betas Razor ausgel"ost. Razor wurde bereits befragt. Er ist inzwischen auf Kallisto.

Folgende Instruktionen und Information bekommen die Charaktere unaufgefordert:

\begin{description}
	\item[Chefermittler Cynarian] Der Chefermittler der Cynarian Corporation wird von Colonel Scholz angewiesen den assistierenden Ermittler alleine zu einer weiteren Befragung des verd"achtigen Hanibal in den St"utzpunkt der Sicherheitskr"afte zu schicken.
	\item[Psychonaut] Der Psynchonaut, der assistierende Ermittler der Cynarian Corporation, wird von Henry Longdale dem Sekret"ar Vandermools pers"onlich und eindringlich beauftragt den Verd"achtigen einem Gehirnscan zu unterziehen.
	\item[Chefermittler Protektorat] Der Chefermittler des Protektorat wird von Artisan beauftragt mit dem Chefermittler der Cynarian die weiteren Ermittlungen in den Quartieren der Verd"achtigen oder auf HeM05 fortzufahren.
	\item[Omega] Da Blackheart Vandarmools Truppen nicht traut beauftragt Thunderbolt seinen Untergebenen umgehend den Sicherheitsdienst abzufangen und zu begleiten oder die Aktivit"aten der Cynarian Corporation zu unterst"utzen und mitzuverfolgen. Da Blackheart ziemlich unzufrieden mit der aktuellen Entwicklung ist erfolgt der Auftrag mit deutlichem Nachdruck.	
	\item[Cowboybrigade] Der Chefermittler des Protektorats wird dar"uber informiert dass die Cowboybrigade auf dem Garnisonsst"utzpunt auf Valhalla auf Anweisung von \emph{Commander Lockhead} festgesetzt wurde.
	\item[Slingshot/Drake] Der Chefermittler des Protektorats wird von der Verwaltung des Raumhafen auf Armageddon informiert dass Slingshot kurz nach dem Frachterungl"uck im Raumhafen der Station von einer Kamera erfasst wurde. Mutma\3lich um sich von dort aus Absetzen zu k"onnen.
\end{description}
\vfill\pagebreak

\begin{remarks}
	Nach den sch"atzungsweise langatmigen Ermittlungen bietet es sich an ab hier das Schritttempo des Plots zu erh"ohen und die Informationen in kurzen Zyklen auszugeben um weitere Ermittlungen abzuk"urzen. Ausreichende Informationen sind mit den Befragungen der Minenarbeiter und den nachgefragten Informationen bereits vorhanden.

	Das nicht von den Charakteren beauftragte umverlegen der Minenbesatzung soll den Spielern zeigen, dass ihre F"uhrung jeweils ihr eigenen F"aden ohne Absprache zieht.

	Die Anweisungen den zweiten Cynarian Ermittler und den Omega zum St"utzpunkt des Sicherheitsdienstes alleine zu schicken dient dazu die Gruppe zu trennen. F"ur den weiteren Verlauf der Ereignisse auf Hellgate erlaubt das einen interessanten Spannung in den kommenden Szenen aufzubauen.
\end{remarks}

\newsection{Zusammenstoss mit Minenarbeitern}

Beim Weg zu den Quartieren der HeM05 Minenbesatzung oder anderweitig zum Lebensbereich der Hellgate Station werden die Charaktere und Grace Anders durch 7 finster hereinblickende Arbeiter mit provisorischen Schlagwaffen in Form von Werkzeugen eingekesselt. Die Arbeiter wurden von Slingshot, alias Drake, angestachelt, die Ermittler zur Rede zu stellen, wieso die Minenarbeiter wie Verbrecher auf ihre Veranlassung abgef"uhrt wurden. Den Spielern ist zu diesem Zeitpunkt noch nicht bekannt, dass die Minenarbeiter vom Sicherheitsdienst abgef"uhrt wurden. Auch Grace Anders ist nicht informiert.

\begin{remarks}
	Das Missverst"andnis l"asst sich leicht durch eine R"uckfrage bei Karl Sandos aufl"osen lassen. Eine Eskalation l"asst sich durch Androhung des Einsatzes von Sicherheitskr"aften der Station oder durch Drohungen seitens des Omegas vermeiden. Die Minen-Crew wurde auf Anweisung von Arongate auf den Sicherheitsst"utzpunkt "uberf"uhrt.

	Der Vorfall verschafft Drake und seinem Helfer Zeit f"ur eine Befreiungsaktion von Hannibal.
	
	Sollte es zu einer Ausschreitung kommen, finden sich die Details unter \emph{Pers"onlichkeiten auf Hellgate}.
\end{remarks}

\newsection{Quartiere und die HeM05 Miene}

Die Quartiere von Pitch, Hannibal, Greydog und wenn n"otig auch der Rest der Besatzung befinden sich im Wohnbereich der Station. Die Quartiere sind an den Seiten von langen G"angen aufgereiht und durch eine kleine Schleuse zu betreten. Da sich die Mitarbeiter auf Hellgate und in den Minen nur wenige Wochen aufhalten, viele davon haupts"achlich in den Minen, hat kaum jemand ein festes Quartier. Die meisten beziehen ein Quartier nur f"ur den zeitweiligen Aufenthalt auf Hellgate und geben es dann f"ur andere auf. Die Mitarbeiter haben entsprechend einen Grossteil ihrer Habseligkeiten selbst dabei. Die Quartiere bestehen aus einer einklappbaren Liege, einem kleinen Tisch mit Sitzgelegenheiten, einem Spind und einer Nasszelle inklusive Toilette. Neben dieser Einrichtung gibt es ein Nahrungsaufbereitungssystem und ein Computersystem mit Holoprojektor, das in erster Linie f"ur die Kommunikation oder als Server f"ur Datenabfragen und Unterhaltung dient.

Im Quartier von Pitch finden sich ein paar Bilder, auf denen sie vermutlich mit Freunden auf sch"atzungsweise dem Mars abgebildet ist. Das Computersystem ist dabei schon interessanter. Man findet dort ein Tagebuch mit Rechercheergebnissen und Randnotizen zu den Vorkommnissen auf HeM03 und ihrem Beschluss, Hannibal auf HeM05 zu folgen. Von einer Anzeige hatte sie wohl abgesehen, da sie sich nicht vorstellen konnte, dass Hannibal das Attentat wirklich begangen hatte und was ihn zu so einer Tat gef"uhrt haben k"onnte. Hannibal war seit der Zeit auf HeM03 nach ihren Angaben ein guter Kollege gewesen.


In den Quartieren von Hanibal und Greydog sind keine pers"onlichen Gegenst"ande zu finden.

Nachforschungen auf HeM05 k"onnen nur mit Unterst"utzung von Mitarbeitern der Mine erfolgen. Auf der Mine, die an die Schlepperinsel angedockt ist, werden derzeit einige Reparaturen durchgef"uhrt. Die Mine selbst ist ein imposantes Konstrukt, das einem auf den Kopf gestellten, abgeflachten Kegel mit einer H"ohe von "uber 500 Metern entspricht. In den oberen 20 Metern befinden sich die Wohnquartiere, Aufenthaltsr"aume, der Shuttlehangar, die Br"ucke, Arbeitsst"atten, Lager und technische Einrichtungen f"ur den Betrieb der Mine. Der darunterliegende Teil ist die eigentliche F"orderanlage und Tanks f"ur HE-3. Um die Mine sind an verschiedenen Stellen Balkone gezogen, um den Arbeitern einen einfacheren Ausseneinsatz zu erm"oglichen. Die f"unf Tr"agerballons sind rund um den oberen Teil der Mine aufgeh"angt. Derzeit besitzt die Mine weiterhin nur zwei von f"unf Tr"agerballons. Die drei zerst"orten werden derzeit auf Nike hergestellt. F"ur den Ausseneinsatz werden spezielle schwere Raumanz"uge ben"otigt, die die Arbeiter vor den Widrigkeiten der Saturnatmosph"are f"ur eine kurze Zeit sch"utzen k"onnen. F"ur die Nachforschungen auf der Mine werden Spezialisten ben"otigt, die die Anlagen der Mine erkl"aren k"onnen, sowie Softwareexperten, die Manipulationen an der Software aufdecken k"onnen. Auf der Mine k"onnen die Manipulationen des Shuttles durch Pitch zweifelsfrei festgestellt werden, ebenso wie die von Pitch installierte Sicherung der Minensteuerung und der Manipulationsversuch von Hannibal.
