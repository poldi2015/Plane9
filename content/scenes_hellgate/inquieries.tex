%% Copyright 2019 Bernd Haberstumpf
%% License: CC BY-NC
% !TeX spellcheck = de_DE
\newsection{Das Geschehen auf HeM05}

F"unf Tage vor dem Attentat wurde die Mannschaft der Mine durch eine Rumpfmannschaft, gef"uhrt von Florence der Kommandantin, ersetzt. Die nachfolgende Mannschaft wird 10 Tage nach dem Eintreffen der Rumpfmannschaft auf der Mine erwartet und trifft etwa zeitgleich mit den Charakteren auf Hellgate ein. Die Ankunft der zugeh"origen F"ahre k"onnen die Charaktere auf dem Flugdeck miterleben. Das Attentat selbst erfolgte chronologisch folgenderma\3en:

\begin{enumerate}
	\item Einen Tag nach der Ankunft auf HeM05 installierte Pitch eine "Uberwachungssoftware in der Minensteuerung, die Manipulationen 	
		blockiert und sie "uber Manipulationsversuch informiert. Pitch hat zu diesem Zeitpunkt ihren Kollegen Hannibal bereits im Verdacht, den Absturz der Mine HeM03 "uber eine Softwaremanipulation veranlasst zu haben.
	\item Pitch macht nach dem Abflug der vorherigen Minenmannschaft das Shuttle untauglich, um eine Flucht des mutma\3lichen T"aters zu 
		verhindern.
	\item Am Tag des Attentats versucht Hannibal zun"achst wie auf HeM03 eine Fehlfunktion der Anlagensteuerung hervorzurufen. Eine solche 
		Manipulation h"atte zu einem massiven Minenschaden und zur Zerst"orung eines gro\3en Teils der Mine gef"uhrt.
	\item Als die Manipulation der Minensteuerung aufgrund der Software von Pitch misslingt, begibt sich Hannibal in einem Raumanzug in den 
		Au\3enbereich und koppelt Tr"agerballon Nummer eins ab.
	\item Pitch verfolgt Hannibal und stellt ihn auf der Au\3enbalustrade der Mine zur Rede, als dieser gerade den ersten Ballon abkoppelt. 
		Es kommt zum Kampf. Hannibal st"urzt Pitch in den Abgrund.
	\item W"ahrend der Abkoppelung des zweiten Ballons wird er von ZDee "uberrascht kann aber diesen ebenfalls im Kampf "uberw"altigen und 
		in die Tiefe st"urzen.
	\item Nach dem Abkoppeln des dritten Tr"agerballons kehrt Hannibal ungesehen in die Mine zur"uck, legt den Raumanzug ab und begibt sich 
		zum Treffpunkt bei der Dekompressionskammer.
\end{enumerate}

\newsection[Nachforschungen auf Hellgate]{Nachforschungen auf Hellgate}

Nachdem die Ermittler den Raumhafen verlassen haben, wird die Minenbesatzung von f"unf Gardisten des lokalen Sicherheitsdienstes abgef"uhrt und auf den St"utzpunkt der Sicherheitskr"afte gebracht. Die "Uberf"uhrung wurde von Henk Arongate pers"onlich in Absprache mit dem B"uro von Vandermool veranlasst. Das Vorgehen wird von Arbeiter auf dem Raumdeck beobachtet, unter anderem von Slingshot, der unter dem Namen Drake bereits vor den Charakteren auf Hellgate ankam. Zum Zeitpunkt des Eintreffens der Ermittler hat er bereits Kontakt mit den hiesigen Arbeitern aufgebaut. 

Nach der ersten Befragung k"onnen die Ermittler weitere Nachforschungen auf Hellgate angehen oder bei Grace beauftragen. Als N"achstes werden die Ermittler voraussichtlich entweder die Quartiere der Minenbesatzung auf Hellgate aufsuchen oder ihr Nachforschungen auf der Mine HeM05 fortsetzen. Zuerst sollten sie aber einen Bericht an ihre Vorgesetzten abgeben. Erfolgt eine Meldung an das B"uro von Vandermool werden die Ermittler davon in Kenntnis gesetzt, dass die Minenarbeiter aus Sicherheitsgr"unden verlegt wurden.

Folgende Informationen k"onnen bereits "uber Nachfragen direkt beschafft werden:

\begin{description}
	\item[Tr"agerballons] Die Tr"agerballons lassen sich "uber das Anlagensystem Notentl"uften. Abkoppeln lassen sie sich nur von au\3en, 
		direkt an der Kopplungsstelle des Ballons. Die Informationen l"asst sich "uber Grace Anders bzw.~durch R"ucksprache mit Dr.~Petrova der technischen Leiterin der Minen in Erfahrung bringen. Wenn Pitch nicht die Attent"aterin ist, kommen nur noch Hannibal, ZDee oder Greydog als Attent"ater in Frage.
	\item[Greydog] Greydogs Geschichte l"asst sich ebenfalls "uber Grace Anders, durch Nachfrage bei Dr.~Petrova oder vor Ort in der Mine 
		plausibilisieren. Die Wege in der Mine von der Raffinerieanlage zu Br"ucke und den Aufenthaltsr"aumen sind mehrere hundert Meter lang.
	\item[HeM03] Nachforschungen zu HeM03 ergeben, dass sich au\3er Florence und Hannibal kein Mannschaftsmitglied mehr auf Hellgate befindet. Auf HeM03 waren insgesamt 50 Arbeiter besch"aftigt die sich gl"ucklicherweise fast alle mit Hilfe des Rettungsshuttles retten konnten. 
	\item[HeM03 Attent"ater] Bei der Manipulation der Mine HeM03 hatten laut Hannibals Aussage ein gewisser \emph{Lionel Hampton}, ein 
		\emph{Ice Diver} und er versucht, die Attent"aterin Sent von der Manipulation der Minensoftware abzuhalten. Lionel Hampton wie auch Sent wurden dabei get"otet. Ice Diver gilt seit dem Vorfall auf der Mine als vermisst.
	\item[Hannibal] Nachfragen zum Hintergrund von Hannibal ergeben, dass Hannibal vor einem dreiviertel Jahr im Raumhafen von Valhalla als 
		Softwaretechniker angestellt wurde. Vor 2 Monaten wechselte er Cynarian als Sicherheitstechniker f"ur den Minenbetrieb. 
	\item[Pers"onliche Hintergr"unde] Nachfragen zu Hintergr"unden der Minenbesatzung f"ordern zutage, dass keine auff"alligen Gelder 
		geflossen sind oder Angeh"orige unter Druck gesetzt werden konnten. Alle Mitarbeiter arbeiten seit Monaten im Auftrag von Cynarian.
\end{description}

Die genannten Informationen sollten erst nach Abschluss der Befragung der Mannschaft zug"anglich gemacht werden, damit sie nicht in die Befragung mit einflie\3en k"onnen und sofort zur Aufkl"arung der Ereignisse f"uhren.

Bei Nachfragen zur Schlepperfehlfunktion vor 9 Wochen ist ermittelbar, dass ein Softwarefehler Sch"aden an zwei Tr"agerpunkten f"ur Minen verursacht haben. Die Reparaturen dauern noch an. Gl"ucklicherweise l"asst sich die Schlepperinsel nach wie vor mit ihren drei weiteren Andockpunkten nutzen. Der Softwarefehler wurde von Hannibal ausgel"ost. Zum Zeitpunkt der Fehlfunktion war er allerdings bereits auf der Mine HeM03 im Einsatz. Auf Nachfragen im Flugbereich hin, erfahren die Spieler auch von einem Shuttleabsturz vor "uber 10 Wochen. Der Vorfall wird nicht als Attentat eingestuft, sondern wird als Beispiel erw"ahnt, dass nicht alle Unf"alle als Attentate ber"ucksichtigt werden m"ussen. Der Shuttleabsturz oder besser die Kollision mit einer F"ahre auf dem Hangardeck wurde durch einen Pilotenfehler des "uberm"udeten und mit Wachhaltedrogen gedopten Alpha-Mutanten Razor ausgel"ost. Razor wurde bereits befragt. Er ist inzwischen auf Kallisto.

Folgende Instruktionen und Information bekommen die Charaktere unaufgefordert:

\begin{description}
	\item[Chefermittler Cynarian] Der Chefermittler der Cynarian Corporation wird von Colonel Scholz angewiesen, den assistierenden 
		Ermittler alleine zu einer weiteren Befragung des verd"achtigen Hannibal in den St"utzpunkt der Sicherheitskr"afte zu schicken.
	\item[Assistent Cynarian] Der assistierende Ermittler der Cynarian Corporation, wird von Henry Longdale dem Sekret"ar Vandermools 	
		pers"onlich und eindringlich beauftragt, als Psychonaut die Verd"achtigen jeweils einem Gehirnscan zu unterziehen.
	\item[Chefermittler Protektorat] Der Chefermittler des Protektorats wird von Artisan beauftragt mit dem Chefermittler der Cynarian 
		Corporation zusammen die Ermittlungen in den Quartieren der Verd"achtigen fortzusetzen.
	\item[Assistent Protektorat] Da Blackheart dem Sicherheitsdienst auf Hellgate misstraut, beauftragt Thunderbolt seinen Untergebenen 
		umgehend den Sicherheitsdienst, der die gefangenen Minenarbeiter abf"uhrt, abzufangen und zu begleiten. Da Blackheart ziemlich unzufrieden mit der aktuellen Entwicklung ist, erfolgt der Auftrag mit deutlichem Nachdruck.	
	\item[Cowboybrigade] Der Chefermittler des Protektorats wird, falls noch nicht geschehen, dar"uber informiert, dass die Cowboybrigade 	
		auf dem Garnisonsst"utzpunt auf Valhalla auf Anweisung von \emph{Commander Lockhead} festgesetzt wurde.
	\item[Slingshot / Drake] Der Chefermittler des Protektorats wird von der Verwaltung des Raumhafens auf Armageddon informiert, dass 
		Slingshot vor 3 Tagen im Raumhafen der Station von einer Kamera erfasst wurde. Mutma\3lich um sich von dort aus absetzen zu k"onnen.
\end{description}

\begin{remarks}
	Nach den sch"atzungsweise langatmigen Ermittlungen bietet es sich an ab hier das Schritttempo des Plots zu erh"ohen und die Informationen in kurzen Zyklen auszugeben, um weitere Ermittlungen abzuk"urzen. Ausreichende Informationen sind mit den Befragungen der Minenarbeiter und den nachgefragten Informationen bereits vorhanden.

	Die Verlegung der Minenbesatzung soll den Spielern verdeutlichen, dass ihre F"uhrung auch eigenst"andig ins Spielgeschehen eingreift. Die Verlegung der Arbeiter wurde veranlasst, um die Arbeiter vor "Ubergriffen zu sch"utzen und potenzielle Attent"ater zu isolieren.

	Die Anweisungen die den assistierenden Cynarian Ermittler und den assistierenden Ermittler des Protektorats zum St"utzpunkt des Sicherheitsdienstes f"uhrt, dient dazu die Gruppe zu trennen. F"ur den weiteren Verlauf der Ereignisse auf Hellgate erlaubt die Trennung einen Spannungsbogen in den kommenden Szenen aufzubauen.
\end{remarks}

\newsection[Im Angesicht des Arbeiter Mobs]{Im Angesicht des Arbeiter Mobs}
                              
Beim Weg zu den Quartieren der HeM05 Minenbesatzung oder anderweitig zu Wohnbereichen auf Hellgate wird Grace Anders und die Gruppe durch 7 finster dreinblickende Arbeiter mit provisorischen Schlagwaffen in Form von Werkzeugen eingekesselt. Die Arbeiter wurden von Slingshot alias Drake dem Attent"ater aus den Reihen der Cowboy Brigade, angestachelt, die Ermittler zur Rede zu stellen, wieso die Minenarbeiter wie Verbrecher abgef"uhrt wurden. Den Ermittlern ist zu diesem Zeitpunkt vermutlich noch nicht bekannt, dass die Minenarbeiter vom Sicherheitsdienst abgef"uhrt wurden. Auch Grace Anders ist nicht informiert.
\vfill\pagebreak

\begin{remarks}
	Das Missverst"andnis l"asst sich leicht durch eine R"uckfrage bei Karl Sandos dem Chef von Grace Anders aufl"osen. Die Minen-Crew wurde auf Anweisung von Henk Arongate dem Sicherheitschef der Hellgate Station auf den Sicherheitsst"utzpunkt "uberf"uhrt. Aus den Ergebnissen der Befragung der Minenarbeiter, die Grace Anders an ihren Vorgesetzten geleitet hat, konnte die Cynarian F"uhrung schlie\3en, dass wenigstens ein Attent"ater noch Teil der Crew sein d"urfte. Eine Isolation des Minenpersonals wurde deshalb als ratsam erachtet.

	Eine Eskalation l"asst sich durch Androhung des Einsatzes von Sicherheitskr"aften der Station oder durch Drohungen seitens des Omega-Soldaten der Gruppe vermeiden. 

	Der Vorfall verschafft Drake und seinem Helfer Zeit f"ur eine Befreiung von Hannibal.
	
	Sollte es zu einer Ausschreitung kommen, finden sich Details unter \emph{Pers"onlichkeiten auf Hellgate}.
\end{remarks}

\newsection{Quartiere und die HeM05 Mine}

Die Quartiere von Pitch, Hannibal, Greydog und dem Rest der Minen Besatzung befinden sich im Wohnbereich der Station. Die Quartiere sind an den Seiten von langen G"angen aufgereiht und durch eine kleine Schleuse zu betreten. Da sich die Mitarbeiter auf Hellgate und in den Minen nur wenige Wochen aufhalten, viele davon haupts"achlich in den Minen, hat kaum jemand ein festes Quartier auf Hellgate. Die meisten beziehen ein Quartier nur f"ur den zeitweiligen Aufenthalt auf der Station und geben es dann an andere weiter. Die Mitarbeiter haben entsprechend einen Gro\3teil ihrer Habseligkeiten immer dabei oder irgendwo eingelagert. Quartiere auf der Station bestehen aus einer einklappbaren Liege, einem kleinen Tisch mit Sitzgelegenheiten, einem Spind und einer Nasszelle inklusive Toilette. Neben dieser Einrichtung gibt es eine Nahrungsaufbereitungsanlage und ein Computersystem mit Holoprojektor, das in erster Linie f"ur die Kommunikation oder als Server f"ur Datenabfragen und Unterhaltung dient.

Im Quartier von Pitch finden sich ein paar Bilder, auf denen sie vermutlich mit Freunden auf dem Mars abgebildet ist. Das Computersystem ist dabei schon interessanter. Man findet dort ein Tagebuch mit Rechercheergebnissen und Randnotizen zu den Vorkommnissen auf HeM03 und ihrem Verdacht, dass Hannibal in die Vorkommnisse verstrickt sein k"onnte. In einem weiteren Tagebucheintrag ist ihr Beschluss notiert, Hannibal auf die HeM05 Mine zu folgen. Von einer Anzeige hatte sie abgesehen, da sie sich nicht vorstellen konnte, dass Hannibal das Attentat wirklich begangen hatte. Ihr war offensichtlich unklar was ihn zu so einer Tat gef"uhrt haben k"onnte. Hannibal war ab der Zeit auf HeM03 nach ihren Angaben ein guter Kollege gewesen.

In den Quartieren von Hannibal und Greydog sind keine pers"onlichen Gegenst"ande zu finden.

Nachforschungen auf HeM05 k"onnen nur mit Unterst"utzung von Mitarbeitern der Mine erfolgen. Auf der Mine, die an die Schlepperinsel angedockt ist, werden derzeit einige Reparaturen durchgef"uhrt. Die Mine selbst ist ein imposantes Konstrukt, das einem auf den Kopf gestellten, abgeflachten Kegel mit einer H"ohe von "uber 500 Metern entspricht. In den oberen 20 Metern befinden sich die Wohnquartiere, Aufenthaltsr"aume, der Shuttlehangar, die Br"ucke, Arbeitsst"atten, Lager und technische Einrichtungen f"ur den Betrieb der Mine. Der darunterliegende Teil ist die eigentliche F"orderanlage und deren Tanks f"ur das gef"orderte HE-3. Um die Mine herum sind an verschiedenen Stellen Balkone gezogen, um den Arbeitern einen einfacheren Au\3eneinsatz zu erm"oglichen. Die f"unf Tr"agerballons sind rund um den oberen Teil der Mine aufgeh"angt. Derzeit besitzt die Mine weiterhin nur zwei von f"unf Tr"agerballons. Ein Ersatz f"ur die drei zerst"orten Ballons wird derzeit auf der Nike Station hergestellt. F"ur den Au\3eneinsatz werden spezielle schwere Raumanz"uge ben"otigt, die die Arbeiter vor den Widrigkeiten der Jupiter-Atmosph"are f"ur eine kurze Zeit sch"utzen k"onnen. F"ur die Nachforschungen auf der Mine werden Spezialisten ben"otigt, die die Anlagen der Mine erkl"aren, sowie Softwareexperten, die Manipulationen an der Software aufdecken k"onnen. Auf der Mine k"onnen die Manipulationen am Shuttles, durchgef"uhrt durch Pitch, zweifelsfrei nachgewiesen werden. Ebenso ist die von Pitch installierte Sicherung der Minensteuerung und deren Manipulationsversuch von Hannibal ersichtlich.
