%% Copyright 2019 Bernd Haberstumpf
%% License: CC BY-NC
% !TeX spellcheck = de_DE
\begin{backcover}{images/coverback-bg}
    \backcoverquote{Wissen Sie, woher er kam? \dots Jupiter. Dann sollten wir das Protektorat besuchen.}

    \vspace{5mm}
    \textbf{Die Szenerie}

    Operation P9 ist eine Rollenspielkampagne, die sich an dem Roman c23 - Band 3: In Abyssum von Ralph Edenhofer orientiert. Die Geschichte spielt im 23. Jahrhundert im jovianischen System, also rund um den Planeten Jupiter. Vor vier Jahren wurde das jovianische System durch eine Allianz der Cynarian Corporation und der jungen Nation ``das Protektorat'' zur Helium-3-Gewinnung erschlossen. Seit einigen Wochen kommt es zu verd"achtigen Zwischenf"allen. Die Lage ist brenzlig und politisch komplex. Transnationale Konzerne k"ampfen mit allen Mitteln um die Vorherrschaft im All. Alles ist erlaubt. Wird das Protektorat im Strudel der Ereignisse bestehen k"onnen?

    \medskip
    \emph{Es liegt in deiner Hand, die Vorkommnisse aufzukl"aren. Der Einsatz ist hoch.}

    \vspace{5mm}
    \textbf{Was dich erwartet}

    Das Buch bietet einen packenden Science-Fiction-Thriller auf "uber 180 Seiten, ausgelegt auf eine Spielzeit von 60 Stunden. Die Geschichte wird durch umfassende Hintergrundinformationen und eine Beschreibung von "uber 90 NSCs erg"anzt. Um das Eintauchen in die spektakul"are Umgebung der Jupitermonde, Raumschiffe und Stationen ohne gro\3e Vorbereitungen zu erleichtern, enth"alt das Buch ein einfaches W6-Rollenspielregelwerk, das den Fluss der beschriebenen Ereignisse nicht durch trockene Zahlen bremst. Zahlreiche Bilder vermitteln einen atmosph"arischen Eindruck der Szenerien. Die Kampagne ist f"ur 3-4 Spieler konzipiert.

    \vspace{12mm}
    \newcommand{\footerentry}[1]{\textit{\normalsize{}{#1}}}
    \begin{tabularx}{\textwidth} {
        >{\raggedright\arraybackslash}X
%        >{\raggedleft\arraybackslash}X
    }
        \footerentry{2019-2024} \\
        \footerentry{poldi@thatswing.de}\\
        \footerentry{https://github.com/poldi2015/Plane9}\\
        \footerentry{Creative Commons Public License} 
    \end{tabularx}

\end{backcover}
