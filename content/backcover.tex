%% Copyright 2019 Bernd Haberstumpf
%% License: CC BY-NC
% !TeX spellcheck = de_DE
\begin{backcover}{images/coverback-bg}
    \backcoverquote{Wissen Sie woher er kam? \dots Jupiter. Dann sollten wir das Protektorat besuchen.}

    \vspace{5mm}
    \textbf{Eine "uberw"altigende Szenerie}

    Operation P9 ist eine Rollenspielkampagne angelehnt an den Roman c23 - Band 3, \textit{in abyssum} von Ralph Edenhofer. Die Geschichte spielt im 23. Jahrhundert im jovianischen System, also rund um den Planeten Jupiter. Vor 4 Jahren wurde das jovianische System durch eine Allianz aus der Cynarian Cooperation und der jungen Nation "`das Protektorat"' für die Helium-3 Gewinnung erschlossen. Seit einigen Wochen kommt es zu verd"achtigen Zwischenfällen. Die Lage ist brenzlich und politisch komplex. Transnationale Konzerne k"ampfen mit allen Mitteln um die Vorherrschaft im All. Alles ist erlaubt. Wird das Protktorat im Strudel der Ereignisse bestehen k"onnen?

    \medskip
    \emph{Es liegt in deiner Hand die Vorkommnisse aufzukl"aren. Der Einsatz ist hoch.}

    \vspace{5mm}
    \textbf{Was dich erwartet}

    Das Buch enth"alt einen packenden Science Fiction Thriller auf "uber 80 Seiten ausgelegt auf eine Spielzeit von 40 Stunden. Die Geschichte wird erg"anzt durch Hintergrundinformationen und eine Beschreibung der "uber 80 NSCs. Um ohne gro\3e Vorbereitungen die spektakul"are Umgebung der Jupitermonde, Raumschiffe und Stationen einzutauchen beinhaltet das Buch ein einfaches W6 Rollenspielregelwerk das den Fluss der beschriebene Ereignisse nicht durch trockene Zahlen bremst. Das Buch ist mit zahlreichen Bildern unterlegt um einen athmosph"arischen Eindruck der Szenerien zu vermitteln. Die Kampagne ist f"ur 3-4 Spieler konzipiert.

    \vspace{15mm}
    \newcommand{\footerentry}[1]{\textit{\normalsize{}{#1}}}
    \begin{tabularx}{\textwidth} {
        >{\raggedright\arraybackslash}X
        >{\raggedleft\arraybackslash}X
    }
        & \footerentry{2019-2024} \\
        & \footerentry{poldi@thatswing.de}\\
        & \footerentry{https://github.com/poldi2015/Plane9}\\
        & \footerentry{Creative Commons Public License} 
    \end{tabularx}

\end{backcover}
