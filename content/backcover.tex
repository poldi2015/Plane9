%% Copyright 2019 Bernd Haberstumpf
%% License: CC BY-NC
% !TeX spellcheck = de_DE
\begin{backcover}{images/coverback-bg}
    \backcoverquote{Wissen Sie woher er kam? \dots Jupiter. Wir sollten das Protektorat besuchen.}
    
    \vspace{5mm}
    \textbf{Gro\3es Kino}

    P9 ist eine Rollenspielkampagne basierend auf der Romanvorlage c23 - Band 3, in abyssum von Ralph Edenhofer. Die Geschichte spielt im 23. Jahrhundert im jovianischen System, also rund um den Planeten Jupiter. Vor 4 Jahren wurde das jovianische System durch die eine Allianz auf der Cynarian Cooperation and der jungen Nation "`das Protektorat"' für die Helium-3 Gewinnung erschlossen. Seit einigen Wochen kommt es nun zu gefährlichen Zwischenfällen. Die Lage ist brenzlich und politisch komplex. Transnationale Konzerne k"ampfen mit allen Mitteln um die Vorherrschaft im All. Alles ist erlaubt. Wird das Protktorat im Strudel der Ereignisse bestehen k"onnen?
    
    \medskip
    \emph{Es liegt in eurer Hand die finsteren Vorkommnisse aufzukl"aren. Der Einsatz ist hoch.}

    \vspace{5mm}
    \textbf{Was euch erwartet}

    Das Buch enth"alt einen packenden Agententhriller auf "uber 80 Seiten ausgelegt auf eine Spielzeit auf 40 Stunden. Die Geschichte wird erg"anzt durch Hintergrundinformationen und eine Beschreibung der "uber 80 NSCs. Um ohne viel Vorbereitung f"ur die Spieler in die atmosph"arische Umgebung der Jupitermonde und Raumschiffe und Stationen einzutauchen zu k"onnen beinhaltet das Buch ein einfaches W6 Rollenspielregelwerk ausgelegt auf narrativ beschriebene Ereignisse anstatt von trockenen Zahlen.    Die Kampagne ist auf 3--4 Spieler ausgelegt.
\end{backcover}
