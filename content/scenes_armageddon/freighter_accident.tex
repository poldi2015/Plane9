%% Copyright 2019 Bernd Haberstumpf
%% License: CC BY-NC
% !TeX spellcheck = de_DE
\newsection{Frachterungl"uck auf Armageddon}

Der n"achstgelegene Ansatzpunkt der Ermittler, aufgrund des Fingerzeigs von Avenger, ist das Frachterungl"uck auf Armageddon.

Der Vorfall vor zwei Wochen ereignete sich beim Anbau eines ausgemusterten Frachters an den Habitatsring von Armageddon. Ansprechpartner hierf"ur ist der Alpha-Mutant \emph{Sunny}, der Bauleiter mit Zust"andigkeit f"ur den sogenannten blauen Sektor, Bauabschnitt 3. Der blaue Sektor umfasst die Wohnbereiche des Habitats und wird aufgrund des st"andigen Stroms von Fl"uchtlingen permanent erweitert.

Von Sunny erf"ahrt die Gruppe, dass der Frachter, der in 15 Kilometern Entfernung von Armageddon f"ur den Einbau vorbereitet wurde, mittels ferngesteuerter Drohnen in die Andockposition gebracht werden sollte. Dabei ist offensichtlich eine der Drohnen au\3er Kontrolle geraten und hat den Frachter in den Armageddon-Ring gerammt. Durch den Unfall wurden 12 Frachtcontainer, die als weitere Quartiere dienen sollten, ein Teil des Frachters und 6 bestehende Wohneinheiten zerst"ort oder stark besch"adigt. Ein Teil des Bauabschnitts 3 wurde dem Vakuum ausgesetzt; zwei Arbeiter starben, und einer der Drohnenpiloten wird vermisst. Die Reparaturarbeiten dauern noch an. Weitere Tote konnten vermieden werden, da die in Konstruktion befindlichen Bereiche weitreichend abgesperrt wurden.

Im Gespr"ach mit Sunny, das immer wieder durch andere Personen unterbrochen wird, erfahren die Charaktere, dass das Einpassen und Andocken des Frachters durch 5 Spezialisten durchgef"uhrt wurde. Diese Spezialisten waren erst rund zwei Wochen vor dem Unfall samt Equipment vom Raumhafen auf Kallisto nach Armageddon versetzt worden, um die Aufbauarbeiten mit neuer Technologie und Drohnen zu unterst"utzen.

Wenn Sunny von den Spezialisten spricht, redet er nur von der \emph{Cowboybrigade}. Die Cowboybrigade besteht aus 5 Alpha-Mutanten mit den Namen \emph{Stetson}, \emph{Quickfinger Rod}, \emph{Joe Rider}, \emph{Tom Gunslinger} und \emph{Slingshot}. Die Cowboybrigade wird von Sunny als ein lustiger Haufen bezeichnet, dessen Mitglieder sich wahlweise als betont coole Cowboys (wie aus alten Holos bekannt) geben oder mit allem Werkzeug, das sie gerade in der Hand halten, salutieren. Unabh"angig davon sind sie aber sehr gut ausgebildete und gewissenhafte Techniker.

Die Cowboybrigade war beim Einplatzieren des Frachters mit einem Wartungsshuttle der Armageddon Station unterwegs, um von dort aus die Drohnen fernzusteuern. Seit dem Vorfall wird das Gruppenmitglied \hl{Slingshot} vermisst, ein Vorfall, der den Rest der Gruppe sehr mitgenommen hat. Nachdem die Suche nach Slingshot aufgegeben wurde, hat die Cowboybrigade Armageddon verlassen und ist wieder nach Valhalla auf Kallisto zur"uckgekehrt.

Sunny kann auf R"uckfrage hin die Protokolle der Kommunikation auf dem Shuttle wie auch Kameraaufnahmen vom Shuttle und von der Station bereitstellen. Das Shuttle steht derzeit nicht zur Verf"ugung, da es sich bereits wieder im Einsatz befindet. Laut Sunny kann das Shuttle zum Unfall selbst nicht beigetragen haben.

Aus den Mitschnitten der Funkprotokolle erfahren die Ermittler, dass Slingshot kurz vor dem Andocken des Frachters seine Drohne pl"otzlich maximal beschleunigt hat. Stetson, der versuchte, ihn "uber das Helmmikrofon anzusprechen, bekam zun"achst keine Antwort. Erst eine Minute sp"ater meldete sich Slingshot mit einem panischen Aufschrei zur"uck und versuchte, seine Drohne wieder unter Kontrolle zu bekommen. Er vermeldete, dass seine Drohne eine Fehlfunktion habe. Um den Schaden wieder in Ordnung zu bringen, verlie\3 Slingshot kurze Zeit sp"ater das Shuttle, um zum Frachter "uberzusetzen und die Drohne funktionst"uchtig zu machen. Dabei geriet er aus den Aufnahmebereichen der Kameras und war ab da nicht mehr auffindbar. Kurze Zeit sp"ater rammte der Frachter die Armageddon Station. Die panischen Rufe seiner Kameraden, sich in Sicherheit zu bringen, blieben unbeantwortet.

Such- und Rettungskr"afte konnten den Unfallbereich erst betreten und absichern, nachdem der Hauptanteil der umherschwirrenden Tr"ummer au\3er Reichweite getrieben worden war.

\begin{remarks}
	\underline{Gewonnene Information:}

	\begin{itemize}
		\item Slingshot, ein Mitglied der Cowboybrigade, hat den Unfall verschuldet und ist seitdem verschollen.
		\item Die Cowboybrigade, die den Frachter steuerte, ist nach Kallisto zur"uckgekehrt.		
	\end{itemize}

	\underline{Die KI:}

	Die folgenden Informationen d"urfen zu diesem Zeitpunkt noch nicht weitergegeben werden:

	Slingshot ist einer der Attent"ater, der von einer von der USI bereitgestellten KI "ubernommen wurde. Er ist eine der ersten beiden Versuchspersonen, an denen die neue Technologie im Feld ausprobiert wird. Nach dem "Ubersetzen zum Frachter betritt Slingshot Armageddon ungesehen wieder und taucht mit Unterst"utzung von Artisan, der die Koordination der Attent"ater "ubernommen hat, auf der Station unter.

	\underline{Alternativer Spielverlauf:}

	Wird das Frachterungl"uck auf Armageddon nicht untersucht, sind den Ermittlern weder die Cowboybrigade noch Slingshot als m"ogliche Attent"ater bekannt. In diesem Fall werden die Ermittler direkt nach Hellgate aufbrechen, um dort \cref{sec:hostage} unsanft auf Slingshot zu treffen. Die Ermittler werden bei ihren Untersuchungen auf Kallisto auf die Cowboybrigade sto\3en. Die Zugeh"origkeit Slingshots zur Cowboybrigade ist neben Sunny dort dem Kommandanten der Protektoratsgarnison auf Kallisto (beschrieben \cref{sec:garnison}) und der Chief Officer \emph{Sonja Frost} des Raumhafens auf Valhalla wie \cref{sec:sonjafrost} beschrieben bekannt.
\end{remarks}

