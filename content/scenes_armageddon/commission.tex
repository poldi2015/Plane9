%% Copyright 2019 Bernd Haberstumpf
%% License: CC BY-NC
% !TeX spellcheck = de_DE
\newsection{Prolog (optional)}

Um die Ermittler in die Geschichte, ihren Charakter und auf die vorherrschende politische Gemengelage vorzubereiten, bietet es sich an, mit jedem der Spieler einzeln eine Einf"uhrungsrunde zu spielen.

\begin{description}
	\item [Cynarian, Chefermittler] Der Vertraute des Cynarian Chefermittlers ist \hl{Colonel Scholz}, der Sicherheitschef der Cynarian 
		Corporation im Jovianischen System. Er wird den Chefermittler auf das Treffen mit \hl{Vandermool}, mit dem der Ermittler selbst noch nicht viel zu tun hatte, vorbereiten. Er erkl"art, dass sich Vandermool um ein gutes Verh"altnis mit dem Protektorat bem"uht, wird aber bei einer Zusammenarbeit die F"uhrung behalten wollen. Die Zusammenarbeit ist mit Vorsicht zu genie\3en. Es ist derzeit nicht klar, inwieweit die Mutanten in die Anschl"age selbst verwickelt sind. Eventuell sind sogar Kr"afte in der Cynarian Corporation t"atig, die die Autorit"at des Direktors Vandermool zu untergraben versuchen.
	\item [Cynarian, assistierender Ermittler] F"ur den Assistenten bietet sich eine Einweisung in die T"atigkeit als Psychonaut an um 
		ihn dabei gleich direkt in die Geschichte einzubinden. Er erh"alt auf dem Mars die Aufgabe einen Agenten der der United Space Industries, dem erl"arten Gegner der Cynarian Corporation zu verh"oren. Der Agent ist bei der R"uckreise vom jovianischen System von Piraten gefangen genommen worden und wurde an Cynarian ausgeliefert. Der Agent ist dabei zuf"alligerweise Erstkontakt zu der von Prof.~Dr.~Naratova betriebenen Neuro Intelligence Forschungseinrichtung.
	\item [Protektorat, Chefermittler] Der Vertraute von Avenger trifft sich am Vorabend vor der Aufnahme der Ermittlungen mit 
		\emph{Artisan}, der rechten Hand des Protektors Avenger. Sie haben sich auf ein Synthbier, ein synthetisch gewonnenes Bier in einer kleinen Bar auf Armageddon verabredet. Artisan erkl"art ihm von den Vorf"allen, die der Ermittler jedoch schon kennt und erkl"art, dass ein Treffen mit Repr"asentanten aus der Cynarian geplant ist, um die Vorkommnisse zu untersuchen. Er warnt den Ermittler davor zu engen Kontakt mit Cynarian Mitarbeitern zu schlie\3en, weil man keinem der Konzerne in der aktuellen Situation trauen k"onne.
	\item [Protektorat, assistierender Ermittler] Den Omega-Soldat der Gruppe nimmt sein Vorgesetzter \hl{Thunderbolt} zu einem Treffen 
		mit \hl{Blackheart} am Raumhafen von Armageddon  mit. Blackheart ist die Kommandantin des Protektorat-Milit"ars mit dem Rang eines Lord-Marschalls. Sie trifft mit dem Schlachtschiff \emph{Donar} ein, um an der Beauftragung der Ermittler vor Ort teilzunehmen. Sie nimmt den Charakter zur Seite und weist ihn an, immer ein waches Auge auf den Ermittlungsvorschrift zu haben da der Cynarian Seite nicht zu trauen w"are und voraussichtlich Informationen vorenthalten werden. Der Ermittler erh"alt den Befehl t"aglich oder bei wichtigen Erkenntnissen Rapport an Thunderbolt zu leisten. In die Szene sollten milit"arische Gepflogenheiten einflie\3en. Auch sollte das Treffen ein Bild von der bereits legend"aren Anf"uhrerin der Protektoratstruppen und deren Adjutanten formen.
\end{description}

\pageimage{images/vandermool_final.png}


\newsection{Einweisung bei Cynarian}

Der Cynarian \hl{Chefermittler} wird durch Eric Vandermool, Colonel Scholz und \emph{Dr.~Petrova} der technischen Leiterin der HE-3 F"orderung in einem Konferenzraum der Cynarian Sektion auf Armageddon als erster eingewiesen. Der Charakter wird in die R"aume durch den Sekret"ar Vandermools \emph{Henry Longdale} gebracht. Das B"uro Vandermools ist sehr ger"aumig, schlicht und k"uhl aber erlesen eingerichtet. Vandermool wird das Gespr"ach von seinem Schreibtisch aus f"uhren. Vandermool ist dominant und souver"an in dem Gespr"ach. Scholz und Dr.~Petrova sind als kompetent und zielstrebig effizient bekannt. Scholz ist ein erfahrener Milit"arangeh"origer.

Der Ermittler erf"ahrt von der Sabotage auf der Mine HeM05 vor drei Tagen, der Havarie der Mine HeM03 vor 7 Wochen und der Fehlfunktion der Schlepper-Insel vor 9 Wochen. Nur der Vorfall auf der Mine HeM05 wird bereits als Attentat eingestuft. Es wird aber gemutma\3t, dass es sich auch bei den anderen Vorkommnissen um Attentate handelt. Die USI als potenzieller Drahtzieher wird direkt angesprochen. Vandermool ist offen beunruhigt und betont, dass weitere Vorkommnisse nicht tragbar w"aren. 

Nach diesem ersten Austausch wird der \hl{assistierende Ermittler} gebeten zu den Personen am Schreibtisch zu kommen. Der Assistent ist bereits seit dem Gespr"achsbeginn in den R"aumlichkeiten anwesend, h"alt sich aber im Hintergrund. Die Anwesenheit des zweiten Ermittlers sollte durch den Spielleiter  erst am Ende der Erl"auterungen der Gegebenheiten offen gelegt werden, um zu zeigen, dass er bereits eingewiesen wurde und potenziell Wissen besitzt, das dem Chefermittler nicht zug"anglich gemacht werden soll. 

Im Folgenden werden die beiden Ermittler offiziell mit Nachforschungen beauftragt. Die Ermittler werden aufgefordert, sich nach der Einweisung bei Protektor Avenger als offiziell f"uhrender Ermittler der Cynarian Corporation vorzustellen. Der Chefermittler soll Scholz "uber den Stand der Ermittlung jederzeit auf dem Laufenden halten. Kontaktmann des Ermittlers sind also Scholz oder Henry Longdale f"ur den direkten Kontakt zu Vandermool. W"ahrend der Ermittlungen stehen die Cynarian Ermittler im Dienste der inneren Sicherheit von Cynarian. Alle Ergebnisse der Mitarbeiter unterliegen der Geheimhaltung.

Vor der Verabschiedung informiert Henry Longdale den leitenden Ermittler dar"uber, dass f"ur die Ermittlungen ein Shuttle namens "`Dawn of Day"' am Raumdock von Armageddon bereitsteht.

\begin{remarks}	
	Detaillierte R"uckfragen sind bei diesem Gespr"ach unangebracht. Vandermool bittet die Ermittler sich bzgl. Fragen und t"aglichen Berichten an Colonel Scholz zu wenden. Spezielle Auftr"age, bei den Colonel au\3en vorgehalten werden sollen, l"asst Vandermool "uber Sekret"ar Henry Longdale "ubermitteln. Vandermool ist damit nie im Zugzwang irgendwelche Informationen selbst preiszugeben. Vandermool wird nur im Notfall die Ermittler kontaktieren.

	Wie vertraut die Cynarian F"uhrung mit dem Protektorat ist, ist den Ermittlern nicht bekannt. Die grobe Geschichte wie es zur Gr"undung des Protektorats gekommen ist, ist aber allgemein bekannt.

	Das Verh"altnis Vandermools zur F"uhrungsspitze der Cynarian Corporation ist nicht vollst"andig klar. Vandermool ist der Sohn eines der Vorstandmitglieder des Konzerns. Aus diesem Grund und aufgrund der urpl"otzlichen unglaublichen Machtstellung des vorher relativ unwichtigen Unterh"andlers, kann mit Neidern und Feinden im Konzern gerechnet werden.
\end{remarks}


\newsection{Einweisung beim Protektorat}

Der \hl{Chefermittler} des Protektorats wird durch Protektor Avenger, seinen Stellvertreter Artisan und seinen Leibw"achter \hl{Hato} in den Konferenzr"aumen des Protektorats auf Armageddon eingewiesen. Die Atmosph"are ist freundschaftlich. Der Ermittler erf"ahrt von den Vorkommnissen auf den Minen HeM03 und HeM05 und der Explosion beim Anbau der Habitate. Die Havarie der Mine HeM03 und das Habitatsungl"uck werden derzeit als potenzielle Attentate gewertet. N"ahere Information zu den einzelnen Vorf"allen erf"ahrt der Ermittler bei der Einweisung nicht. Er sollte aber von Avenger den Hinweis erhalten, den Unfall bei der Erweiterung Armageddons, als Erstes zu bearbeiten. Der Chefermittler wird gebeten, gewonnene Erkenntnisse an Avenger Stellvertreter Artisan zu berichten und als geheim einzustufen.

Avenger erkl"art, dass die Ermittlung, die der Charakter im folgenden f"uhren soll, mit Vandermool und Blackheart abgestimmt ist. Im Vertrauen bittet Avenger seinen Ermittler, die Vertreter der Cynarian Corporation mit Vorsicht zu genie\3en, letztendlich ist Cynarian nach wie vor ein Konzern mit undurchsichtiger Agenda. Avenger stellt daraufhin die Ermittler der Cynarian Corporation vor. Die Ermittler der Cynarian Corporation werden daf"ur in den Raum gebeten.

Nach Beendigung des Gespr"achs mit dem Protektor wird der Chefermittler des Protektorats per ComLink von Blackheart aufgefordert, sich alleine im Kommandostand auf Armageddon einzufinden. Beim Eintreffen des Ermittlers bespricht Blackheart gerade Einsatzpl"ane mit zwei anderen Omegas und einer weiteren Person am erh"oht gelegenen "`Kartentisch"'. Nach einer Minute wendet sie sich eher beil"aufig "uber den R"ucken hinweg dem Ermittler zu. Sie fragt nach seinem Auftrag und seinem Vorgehen. Dann wendet sie sich ihm direkt zu und erkl"art ihm unmissverst"andlich, dass die Vorg"ange die Sicherheit des Protektorats gef"ahrden und deshalb als Angriffe auf das Protektorat zu bewerten seien, denen mit milit"arischen Mitteln zu begegnen sei. Aus diesem Grund stellt sie dem Ermittler Team einen weiteren Ermittler aus den Reihen der Protektoratsstreitkr"afte zur Seite. Der \hl{assistierende Ermittler} ist die weitere Person am Kartentisch. F"ur Blackheart ist damit das Gespr"ach beendet und sie wendet sich wieder ohne Verabschiedung ihrem Stab zu.
\vfill

\begin{remarks}
	Die Besprechung mit Avenger kann der Chefermittler zun"achst alleine mit dem Spielleiter spielen. Die Spieler des Cynarian Ermittler Teams werden erst im zweiten Schritt dazu genommen. Der Assistent auf der Seiten des Milit"ars kommt erst beim Treffen mit Blackheart ins Spiel.
	
	Avenger ist zwar Diplomat und Staatslenker, aber im Wesen freundlich, kollegial und offen umg"anglich. Sein Leibw"achter Hato ist der Typ japanischer Samurai und h"alt sich unaufdringlich im Hintergrund.
	
	Blackheart ist eine Kommandantin mit aufbrausendem Temperament. Sie k"ampft gegen Avenger um die Kontrolle im Protektorat und setzt mit allen Mitteln ihren Willen durch. Das Treffen im Kommandostand soll zwar einsch"uchternd wirken, ist aber nicht offen feindselig.
	
	Der assistierende Ermittler aus den Reihen der Protektoratsstreitkr"afte ist dem Milit"ar und damit Blackheart verpflichtet. Er hat den Befehl, Thunderbolt auf dem Laufenden zu halten und ggf.~auch gegen den Willen der anderen Ermittler nach eigenem Ermessen oder im Auftrag der Milit"arf"uhrung Ma\3nahmen zu ergreifen.

	Die Ansprechpartner der Ermittler des Protektorats sind ihre beiden Vorgesetzten Artisan und Thunderbolt. Da Artisan im geheimen als F"uhrer der Attent"ater agiert, wird er alle Informationen, die der Aufkl"arung der Attentate dienlich sind, verschweigen oder verf"alschen. Artisan unterbindet, soweit ihm m"oglich, den direkten Kontakt zum Protektor.
\end{remarks}

\pageimage{images/blackheart.png}
