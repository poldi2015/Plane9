%% Copyright 2019 Bernd Haberstumpf
%% License: CC BY-NC
% !TeX spellcheck = de_DE
\newchapter{\@introchaptername}

Operation P9 spielt im 23. Jahrhundert, um genau zu sein im Jahre 2213. Ein Gro\3teil des Sonnensystems ist besiedelt. Nationalstaaten und transnationale Konzerne k"ampfen um Vormachtstellungen und wertvolle Ressourcen wie seltene Metalle und das zur Energiegewinnung in gro\3en Mengen ben"otigte HE-3. F"ur die schwere Arbeit auf Planeten fernab der Erde, auf Raumschiffen und Raumstationen wurden vor "uber einem halben Jahrhundert verschiedene k"unstliche Menschenrassen, im Volksmund Mutanten genannt, gez"uchtet, die in N"ahrbottichen das Licht der Welt erblicken. Auf den Kolonien au\3erhalb der Erde sind die Mutanten ein gewohnter Bestandteil der humanen Gesellschaft. Doch auf der Erde regt sich der Widerstand. In der Europ"aischen F"oderation werden Mutanten aus der Gesellschaft ausgesto\3en und sollen in Lagern interniert werden. Einer Gruppe von Mutanten unterst"utzt durch Kampfmutanten aus dem erdnahen Orbit gelingt die Flucht auf die Raumstation Aurora. Von dort aus besetzen sie weitere Stationen im terrestrischen Orbit wie die Orbitalfestung Aigis und befreien Mutanten aus Lagern auf der Erde. Kurz bevor die Situation eskaliert, greift der \emph{Transnationale Konzernrat}, der alle gro\3en Konzerne im Sonnensystem vertritt, ein. Auf Luna wird eine Friedenskonferenz einberufen. Durch einen politischen Schachzug eines bis dahin weitgehend unbekannten Unterh"andlers der \emph{Cynarian Corporation} mit Namen \emph{Eric Vandermool} gelingt es, das bisher weitestgehend unerschlossene jovianische System als neue Heimstatt der Mutanten zu gewinnen und die Sch"urfrechte f"ur HE-3 auf dem Jupiter f"ur Cynarian zu sichern. Unter der F"uhrung des \emph{Protektors Avengers} und der Oberkommandierenden der Streitkr"afte \emph{Blackheart} gr"unden die Mutanten auf dem Raumkomplex \emph{Armageddon} das \emph{Protektorat}, die erste eigenst"andige Nation der Mutanten. Zusammen mit der Cynarian Corporation gelingt es dem Protektorat in vier Jahren Arbeit die Errichtung einer florierenden Infrastruktur rund um den Jupiter. Aber der Erfolg und der Frieden sind tr"ugerisch. M"achtige Konzerne wie die \emph{United Space Industries USI}, die um ihr HE-3-Monopol f"urchten muss, sind nicht unt"atig.