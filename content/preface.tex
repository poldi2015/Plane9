%% Copyright 2019 Bernd Haberstumpf
%% License: CC BY-NC
% !TeX spellcheck = de_DE
\newchapter[Vorwort von Ralph Edenhofer]{Vorwort}

Die erste Version des Szenarios f"ur die c23-Romanreihe habe ich als Hintergrund f"ur ein Rollenspiel entworfen, lange bevor mir das erste Mal ernsthaft in den Sinn gekommen ist, B"ucher zu schreiben. Von daher kann ich es nur guthei\3en, wenn meine Werke die Leser dazu animieren, die darin beschriebene Welt wieder ihrem urspr"unglichen Zweck zuzuf"uhren. Wie oft das bereits geschehen ist, wei\3 ich nicht. Aber mit ziemlicher Sicherheit hat niemand das so ausf"uhrlich und liebevoll gemacht wie Bernd Haberstumpf mit seiner Adaption "`Operation P9"'.

Als er mir eine Vorabversion seines Kampagnenbuchs hat zukommen lassen, war ich absolut begeistert von der darin enthaltenen Detailfreude und der umfangreichen Arbeit, die deutlich sichtbar dahinter steckte. Die enthaltenen Illustrationen haben mir selbst zum ersten Mal bildlich vor Augen gef"uhrt, was ich bis dahin nur in Textform und nat"urlich meiner eigenen Vorstellung gekannt habe.

Der einzige Makel der Kampagne ist -- zumindest aus meiner pers"onlichen Perspektive --, dass ich sie mit meiner eigenen Spielrunde nicht spielen kann, da sie in Teilen stark an den Plot des dritten Romans der c23-Reihe angelehnt ist, den all meine Spieler bereits kennen. Aber Bernd hat auch zahlreiche eigene Ideen eingebracht, von denen ich die eine oder andere recht inspirierend fand. Mal sehen, ob einzelne Elemente es -- seine Erlaubnis vorausgesetzt -- vielleicht sogar in ein zuk"unftiges Buch schaffen.

Im "Ubrigen bleibt mir nur, allen Spielleitern, die dieses Kampagnenbuch in der Hand halten, viel Spa\3 dabei zu w"unschen, die Welt des 23. Jahrhunderts in ihren Runden zum Leben zu erwecken. Lasset die W"urfel rollen!

\medskip
Mai, 2024\\
\emph{Ralph Edenhofer}
