%% Copyright 2019 Bernd Haberstumpf
%% License: CC BY-NC
% !TeX spellcheck = de_DE
\newchapter{Szenen}

% Armageddon
%% Copyright 2019 Bernd Haberstumpf
%% License: CC BY-NC
% !TeX spellcheck = de_DE
\newsection{Prolog (optional)}

Um die Ermittler in die Geschichte, ihren Charakter und auf die vorherrschende politische Gemengelage vorzubereiten, bietet es sich an, mit jedem der Spieler einzeln eine Einf"uhrungsrunde zu spielen.

\begin{description}
	\item [Cynarian, Chefermittler] Der Vertraute des Cynarian Chefermittlers ist \hl{Colonel Scholz}, der Sicherheitschef der Cynarian 
		Corporation im Jovianischen System. Er wird den Chefermittler auf das Treffen mit \hl{Vandermool}, dem Cynarian Direktor im jovianischen System, mit dem der Ermittler selbst noch nicht viel zu tun hatte, vorbereiten. Er erkl"art, dass sich Vandermool um ein gutes Verh"altnis mit dem Protektorat bem"uht, wird aber bei einer Zusammenarbeit die F"uhrung behalten wollen. Die Zusammenarbeit ist mit Vorsicht zu genie\3en. Es ist derzeit nicht klar, inwieweit die Mutanten in die Anschl"age selbst verwickelt sind. Eventuell sind sogar Kr"afte in der Cynarian Corporation t"atig, die die Autorit"at des Direktors Vandermool zu untergraben versuchen.
	\item [Cynarian, assistierender Ermittler] F"ur den Assistenten bietet sich eine Einweisung in die T"atigkeit als Psychonaut an um 
		ihn dabei gleich direkt in die Geschichte einzubinden. Er erh"alt auf dem Mars die Aufgabe einen Agenten der der United Space Industries, dem erl"arten Gegner der Cynarian Corporation zu verh"oren. Der Agent ist bei der R"uckreise vom jovianischen System von Piraten gefangen genommen worden und wurde an Cynarian ausgeliefert. Der Agent ist dabei zuf"alligerweise Erstkontakt zu der von Prof.~Dr.~Naratova betriebenen Neuro Intelligence Forschungseinrichtung.
	\item [Protektorat, Chefermittler] Der Vertraute von Avenger trifft sich am Vorabend vor der Aufnahme der Ermittlungen mit 
		\emph{Artisan}, der rechten Hand des Protektors Avenger. Sie haben sich auf ein Synthbier, ein synthetisch gewonnenes Bier in einer kleinen Bar auf Armageddon verabredet. Artisan erkl"art ihm von den Vorf"allen, die der Ermittler jedoch schon kennt und erkl"art, dass ein Treffen mit Repr"asentanten aus der Cynarian geplant ist, um die Vorkommnisse zu untersuchen. Er warnt den Ermittler davor zu engen Kontakt mit Cynarian Mitarbeitern zu schlie\3en, weil man keinem der Konzerne in der aktuellen Situation trauen k"onne.
	\item [Protektorat, assistierender Ermittler] Den Omega-Soldat der Gruppe nimmt sein Vorgesetzter \hl{Thunderbolt} der Rechten Hand von 
		Blackheart zu einem Treffen mit \hl{Blackheart} am Raumhafen von Armageddon mit. Blackheart ist die Kommandantin des Protektorat-Milit"ars mit dem Rang eines Lord Marshals. Sie trifft mit dem Schlachtschiff \emph{Donar} ein, um an der Beauftragung der Ermittler vor Ort teilzunehmen. Sie nimmt den Charakter zur Seite und weist ihn an, immer ein waches Auge auf den Ermittlungsvorschrift zu haben da der Cynarian Seite nicht zu trauen w"are und voraussichtlich Informationen vorenthalten werden. Der Ermittler erh"alt den Befehl t"aglich oder bei wichtigen Erkenntnissen Rapport an Thunderbolt zu leisten. In die Szene sollten milit"arische Gepflogenheiten einflie\3en. Auch sollte das Treffen ein Bild von der bereits legend"aren Anf"uhrerin der Protektoratstruppen und deren Adjutanten formen.
\end{description}

\newsection{Einweisung bei Cynarian}

Der Cynarian \hl{Chefermittler} wird durch Direktor Eric Vandermool, Colonel Scholz und \emph{Dr.~Petrova} der technischen Leiterin der HE-3 F"orderung in einem Konferenzraum der Cynarian Sektion auf Armageddon als erster eingewiesen. Der Charakter wird in die R"aume durch den Sekret"ar Vandermools \emph{Henry Longdale} gebracht. Das B"uro Vandermools ist sehr ger"aumig, schlicht und k"uhl aber erlesen eingerichtet. Vandermool wird das Gespr"ach von seinem Schreibtisch aus f"uhren. Vandermool ist dominant und souver"an in dem Gespr"ach. Scholz und Dr.~Petrova sind als kompetent und zielstrebig effizient bekannt. Scholz ist ein erfahrener Milit"arangeh"origer.

Der Ermittler erf"ahrt von der Sabotage auf der Mine HeM05 vor drei Tagen, der Havarie der Mine HeM03 vor 7 Wochen und der Fehlfunktion der Schlepper-Insel vor 9 Wochen. Nur der Vorfall auf der Mine HeM05 wird bereits als Attentat eingestuft. Es wird aber gemutma\3t, dass es sich auch bei den anderen Vorkommnissen um Attentate handelt. Die USI als potenzieller Drahtzieher wird direkt angesprochen. Vandermool ist offen beunruhigt und betont, dass weitere Vorkommnisse nicht tragbar w"aren. 

Nach diesem ersten Austausch wird der \hl{assistierende Ermittler} gebeten zu den Personen am Schreibtisch zu kommen. Der Assistent ist bereits seit dem Gespr"achsbeginn in den R"aumlichkeiten anwesend, h"alt sich aber im Hintergrund. Die Anwesenheit des zweiten Ermittlers sollte durch den Spielleiter  erst am Ende der Erl"auterungen der Gegebenheiten offen gelegt werden, um zu zeigen, dass er bereits eingewiesen wurde und potenziell Wissen besitzt, das dem Chefermittler nicht zug"anglich gemacht werden soll. 

Im Folgenden werden die beiden Ermittler offiziell mit Nachforschungen beauftragt. Die Ermittler werden aufgefordert, sich nach der Einweisung bei Protektor Avenger als offiziell f"uhrender Ermittler der Cynarian Corporation vorzustellen. Der Chefermittler soll Scholz "uber den Stand der Ermittlung jederzeit auf dem Laufenden halten. Kontaktmann des Ermittlers sind also Scholz oder Henry Longdale f"ur den direkten Kontakt zu Vandermool. W"ahrend der Ermittlungen stehen die Cynarian Ermittler im Dienste der inneren Sicherheit von Cynarian. Alle Ergebnisse der Mitarbeiter unterliegen der Geheimhaltung.

Vor der Verabschiedung informiert Henry Longdale den leitenden Ermittler dar"uber, dass f"ur die Ermittlungen ein Shuttle namens Dawn of Day am Raumdock von Armageddon bereitsteht.

\pageimage{images/vandermool_final.png}

\begin{remarks}	
	Detaillierte R"uckfragen sind bei diesem Gespr"ach unangebracht. Vandermool bittet die Ermittler sich bzgl. Fragen und t"aglichen Berichten an Colonel Scholz zu wenden. Spezielle Auftr"age, bei den Colonel au\3en vorgehalten werden sollen, l"asst Vandermool "uber Sekret"ar Henry Longdale "ubermitteln. Vandermool ist damit nie im Zugzwang irgendwelche Informationen selbst preiszugeben. Vandermool wird nur im Notfall die Ermittler kontaktieren.

	Wie vertraut die Cynarian F"uhrung mit dem Protektorat ist, ist den Ermittlern nicht bekannt. Die grobe Geschichte wie es zur Gr"undung des Protektorats gekommen ist, ist aber allgemein bekannt.

	Das Verh"altnis Vandermools zur F"uhrungsspitze der Cynarian Corporation ist nicht vollst"andig klar. Vandermool ist der Sohn eines der Vorstandmitglieder des Konzerns. Aus diesem Grund und aufgrund der urpl"otzlichen unglaublichen Machtstellung des vorher relativ unwichtigen Unterh"andlers, kann mit Neidern und Feinden im Konzern gerechnet werden.
\end{remarks}


\newsection{Einweisung beim Protektorat}

Der \hl{Chefermittler} des Protektorats wird durch Protektor Avenger, seinen Stellvertreter Artisan und seinen Leibw"achter \hl{Hato} in den Konferenzr"aumen des Protektorats auf Armageddon eingewiesen. Die Atmosph"are ist freundschaftlich. Der Ermittler erf"ahrt von den Vorkommnissen auf den Minen HeM03 und HeM05 und der Explosion beim Anbau der Habitate. Die Havarie der Mine HeM03 und das Habitatsungl"uck werden derzeit als potenzielle Attentate gewertet. N"ahere Information zu den einzelnen Vorf"allen erf"ahrt der Ermittler bei der Einweisung nicht. Er sollte aber von Avenger den Hinweis erhalten, den Unfall bei der Erweiterung Armageddons, als Erstes zu bearbeiten. Der Chefermittler wird gebeten, gewonnene Erkenntnisse an Avenger Stellvertreter Artisan zu berichten und als geheim einzustufen.

Avenger erkl"art, dass die Ermittlung, die der Charakter im folgenden f"uhren soll, mit Vandermool und Blackheart abgestimmt ist. Im Vertrauen bittet Avenger seinen Ermittler, die Vertreter der Cynarian Corporation mit Vorsicht zu genie\3en, letztendlich ist Cynarian nach wie vor ein Konzern mit undurchsichtiger Agenda. Avenger stellt daraufhin die Ermittler der Cynarian Corporation vor. Die Ermittler der Cynarian Corporation werden daf"ur in den Raum gebeten.

Nach Beendigung des Gespr"achs mit dem Protektor wird der Chefermittler des Protektorats per ComLink von Blackheart aufgefordert, sich alleine im Kommandostand auf Armageddon einzufinden. Beim Eintreffen des Ermittlers bespricht Blackheart gerade Einsatzpl"ane mit zwei anderen Omegas und einer weiteren Person am erh"oht gelegenen ``Kartentisch''. Nach einer Minute wendet sie sich eher beil"aufig "uber den R"ucken hinweg dem Ermittler zu. Sie fragt nach seinem Auftrag und seinem Vorgehen. Dann wendet sie sich ihm direkt zu und erkl"art ihm unmissverst"andlich, dass die Vorg"ange die Sicherheit des Protektorats gef"ahrden und deshalb als Angriffe auf das Protektorat zu bewerten seien, denen mit milit"arischen Mitteln zu begegnen sei. Aus diesem Grund stellt sie dem Ermittler Team einen weiteren Ermittler aus den Reihen der Protektoratsstreitkr"afte zur Seite. Der \hl{assistierende Ermittler} ist die weitere Person am Kartentisch. F"ur Blackheart ist damit das Gespr"ach beendet und sie wendet sich wieder ohne Verabschiedung ihrem Stab zu.
\vfill

\pageimage{images/blackheart.png}

\begin{remarks}
	Die Besprechung mit Avenger kann der Chefermittler zun"achst alleine mit dem Spielleiter spielen. Die Spieler des Cynarian Ermittler Teams werden erst im zweiten Schritt dazu genommen. Der Assistent auf der Seiten des Milit"ars kommt erst beim Treffen mit Blackheart ins Spiel.
	
	Avenger ist zwar Diplomat und Staatslenker, aber im Wesen freundlich, kollegial und offen umg"anglich. Sein Leibw"achter Hato ist der Typ japanischer Samurai und h"alt sich unaufdringlich im Hintergrund.
	
	Blackheart ist eine Kommandantin mit aufbrausendem Temperament. Sie k"ampft gegen Avenger um die Kontrolle im Protektorat und setzt mit allen Mitteln ihren Willen durch. Das Treffen im Kommandostand soll zwar einsch"uchternd wirken, ist aber nicht offen feindselig.
	
	Der assistierende Ermittler aus den Reihen der Protektoratsstreitkr"afte ist dem Milit"ar und damit Blackheart verpflichtet. Er hat den Befehl, Thunderbolt auf dem Laufenden zu halten und ggf.~auch gegen den Willen der anderen Ermittler nach eigenem Ermessen oder im Auftrag der Milit"arf"uhrung Ma\3nahmen zu ergreifen.

	Die Ansprechpartner der Ermittler des Protektorats sind ihre beiden Vorgesetzten Artisan und Thunderbolt. Da Artisan im geheimen als F"uhrer der Attent"ater agiert, wird er alle Informationen, die der Aufkl"arung der Attentate dienlich sind, verschweigen oder verf"alschen. Artisan unterbindet, soweit ihm m"oglich, den direkten Kontakt zum Protektor.
\end{remarks}
%% Copyright 2019 Bernd Haberstumpf
%% License: CC BY-NC
% !TeX spellcheck = de_DE
\newsection{Frachterungl"uck auf Armageddon}

Der n"achstgelegene Ansatzpunkt der Ermittler, aufgrund des Fingerzeigs von Avenger, ist das Frachterungl"uck auf Armageddon.

Der Vorfall vor zwei Wochen ereignete sich beim Anbau eines ausgemusterten Frachters an den Habitatsring von Armageddon. Ansprechpartner hierf"ur ist der Alpha-Mutant \emph{Sunny}, der Bauleiter mit Zust"andigkeit f"ur den sogenannten blauen Sektor, Bauabschnitt 3. Der blaue Sektor umfasst die Wohnbereiche des Habitats und wird aufgrund des st"andigen Stroms von Fl"uchtlingen permanent erweitert.

Von Sunny erf"ahrt die Gruppe, dass der Frachter, der in 15 Kilometern Entfernung von Armageddon f"ur den Einbau vorbereitet wurde, mittels ferngesteuerter Drohnen in die Andockposition gebracht werden sollte. Dabei ist offensichtlich eine der Drohnen au\3er Kontrolle geraten und hat den Frachter in den Armageddon-Ring gerammt. Durch den Unfall wurden 12 Frachtcontainer, die als weitere Quartiere dienen sollten, ein Teil des Frachters und 6 bestehende Wohneinheiten zerst"ort oder stark besch"adigt. Ein Teil des Bauabschnitts 3 wurde dem Vakuum ausgesetzt; zwei Arbeiter starben, und einer der Drohnenpiloten wird vermisst. Die Reparaturarbeiten dauern noch an. Weitere Tote konnten vermieden werden, da die in Konstruktion befindlichen Bereiche weitreichend abgesperrt wurden.

Im Gespr"ach mit Sunny, das immer wieder durch andere Personen unterbrochen wird, erfahren die Charaktere, dass das Einpassen und Andocken des Frachters durch 5 Spezialisten durchgef"uhrt wurde. Diese Spezialisten waren erst rund zwei Wochen vor dem Unfall samt Equipment vom Raumhafen auf Kallisto nach Armageddon versetzt worden, um die Aufbauarbeiten mit neuer Technologie und Drohnen zu unterst"utzen.

Wenn Sunny von den Spezialisten spricht, redet er nur von der \emph{Cowboybrigade}. Die Cowboybrigade besteht aus 5 Alpha-Mutanten mit den Namen \emph{Stetson}, \emph{Quickfinger Rod}, \emph{Joe Rider}, \emph{Tom Gunslinger} und \emph{Slingshot}. Die Cowboybrigade wird von Sunny als ein lustiger Haufen bezeichnet, dessen Mitglieder sich wahlweise als betont coole Cowboys (wie aus alten Holos bekannt) geben oder mit allem Werkzeug, das sie gerade in der Hand halten, salutieren. Unabh"angig davon sind sie aber sehr gut ausgebildete und gewissenhafte Techniker.

Die Cowboybrigade war beim Einplatzieren des Frachters mit einem Wartungsshuttle der Armageddon Station unterwegs, um von dort aus die Drohnen fernzusteuern. Seit dem Vorfall wird das Gruppenmitglied \hl{Slingshot} vermisst, ein Vorfall, der den Rest der Gruppe sehr mitgenommen hat. Nachdem die Suche nach Slingshot aufgegeben wurde, hat die Cowboybrigade Armageddon verlassen und ist wieder nach Valhalla auf Kallisto zur"uckgekehrt.

Sunny kann auf R"uckfrage hin die Protokolle der Kommunikation auf dem Shuttle wie auch Kameraaufnahmen vom Shuttle und von der Station bereitstellen. Das Shuttle steht derzeit nicht zur Verf"ugung, da es sich bereits wieder im Einsatz befindet. Laut Sunny kann das Shuttle zum Unfall selbst nicht beigetragen haben.

Aus den Mitschnitten der Funkprotokolle erfahren die Ermittler, dass Slingshot kurz vor dem Andocken des Frachters seine Drohne pl"otzlich maximal beschleunigt hat. Stetson, der versuchte, ihn "uber das Helmmikrofon anzusprechen, bekam zun"achst keine Antwort. Erst eine Minute sp"ater meldete sich Slingshot mit einem panischen Aufschrei zur"uck und versuchte, seine Drohne wieder unter Kontrolle zu bekommen. Er vermeldete, dass seine Drohne eine Fehlfunktion habe. Um den Schaden wieder in Ordnung zu bringen, verlie\3 Slingshot kurze Zeit sp"ater das Shuttle, um zum Frachter "uberzusetzen und die Drohne funktionst"uchtig zu machen. Dabei geriet er aus den Aufnahmebereichen der Kameras und war ab da nicht mehr auffindbar. Kurze Zeit sp"ater rammte der Frachter die Armageddon Station. Die panischen Rufe seiner Kameraden, sich in Sicherheit zu bringen, blieben unbeantwortet.

Such- und Rettungskr"afte konnten den Unfallbereich erst betreten und absichern, nachdem der Hauptanteil der umherschwirrenden Tr"ummer au\3er Reichweite getrieben worden war.

\begin{remarks}
	\underline{Gewonnene Information:}

	\begin{itemize}
		\item Slingshot, ein Mitglied der Cowboybrigade, hat den Unfall verschuldet und ist seitdem verschollen.
		\item Die Cowboybrigade, die den Frachter steuerte, ist nach Kallisto zur"uckgekehrt.		
	\end{itemize}

	\underline{Die KI:}

	Die folgenden Informationen d"urfen zu diesem Zeitpunkt noch nicht weitergegeben werden:

	Slingshot ist einer der Attent"ater, der von einer von der USI bereitgestellten KI "ubernommen wurde. Er ist eine der ersten beiden Versuchspersonen, an denen die neue Technologie im Feld ausprobiert wird. Nach dem "Ubersetzen zum Frachter betritt Slingshot Armageddon ungesehen wieder und taucht mit Unterst"utzung von Artisan, der die Koordination der Attent"ater "ubernommen hat, auf der Station unter.

	\underline{Alternativer Spielverlauf:}

	Wird das Frachterungl"uck auf Armageddon nicht untersucht, sind den Ermittlern weder die Cowboybrigade noch Slingshot als m"ogliche Attent"ater bekannt. In diesem Fall werden die Ermittler direkt nach Hellgate aufbrechen, um dort \cref{sec:hostage} unsanft auf Slingshot zu treffen. Die Ermittler werden bei ihren Untersuchungen auf Kallisto auf die Cowboybrigade sto\3en. Die Zugeh"origkeit Slingshots zur Cowboybrigade ist neben Sunny dort dem Kommandanten der Protektoratsgarnison auf Kallisto (beschrieben \cref{sec:garnison}) und der Chief Officer \emph{Sonja Frost} des Raumhafens auf Valhalla wie \cref{sec:sonjafrost} beschrieben bekannt.
\end{remarks}


%% Copyright 2019 Bernd Haberstumpf
%% License: CC BY-NC
% !TeX spellcheck = de_DE
\newsection{Die n"achsten Schritte}

Von Armageddon aus gibt es zwei m"ogliche n"achste Ziele f"ur die Ermittler:

\begin{description}
	\item [Hellgate] Au\3er dem Frachterungl"uck zeigen alle Hinweise nach Hellgate.
	\item [Kallisto] Die Cowboy Brigade ist inzwischen nach Kallisto zur"uck gekehrt.
\end{description}

Der Spielleiter sollte die Gruppe in Richtung Hellgate leiten um nicht auf einen gro\3en Teil der Geschichte und wichtige Hintergr"unde verpassen zu m"ussen. Die Cowboy Brigade l"asst sich entweder vom Omega selbst oder durch den Kommandanten \textit{Commander Lockhead} der Garnison auf Valhalla festsetzen.

\pageimage{images/dawn_of_day.jpg}


% Hellgate
%% Copyright 2019 Bernd Haberstumpf
%% License: CC BY-NC
% !TeX spellcheck = de_DE
\newsection{Eintreffen auf Hellgate}

Der Flug von Armageddon dauert rund 10 ereignislose Tage  mit dem Shuttle Dawn of Day w"ahrenddessen sich die Gruppe mit dem Shuttle vertraut machen k"onnen. 

Die HeM05 ist beim Eintreffen der Ermittler an der gigantischen Schlepperinsel der Hellgate Station angedockt. Die Schlepperinsel ist ein 2 Kilometer langes und breites Raumfahrzeug das mit gewaltigen Schubd"usen bis in die "au\3eren Athmosph"arenregionen des Jupiter eintauchen kann um dort die HE-3 Mienen abzusetzen oder einzusammeln. Die Schlepperinstel schwebt beim Anflug auf Hellgate majest"atisch nahe dem Mond Adrastea "uber der gewaltigen Fl"ache des Jupiter. Kleinste Partikel bilden eine Schleier auf diesem niedrigen Orbit von 130'000 km "uber dem Planeten. Hellgate befindet sich bis auf den Anflugtunnel fast vollst"an dig im Inneren des Mondes. Die Station selbst besteht aus dem Raumhafen, technischen Anlagen, Lagerhallen und R"aumen und Wohnquartieren, Lokale, Bars und L"aden. Im Ganzen umfasst die Anlage ca.~30 km\textsuperscript{3}. Wie in alles neuen eilig aufgesetzen Einichtungen befinden sich viele Provisorien, nicht abgeschlossene G"ange und herumstehendes Material in der Station.

Beim Ansteuern des Anflugtunnels wird die Dawn of Day von der Flugkontrolle kontaktiert und nach einer Legitimation gefragt. Nach den ersten Formalit"aten wird das Shuttle "uber einen Leitstrahl in den Landungstunnel navigiert. Der Pilot in der Gruppe kann hierbei sein K"onnen unter Beweis\3 stellen. Dem Spielleiter bleibt "uberlassen wie weit er den Landeanflug ausschm"uckt. Beim Eintreffen im Raumhafen herrscht reger Betrieb, eine gro\3e F"ahre bringt gerade neue Minenarbeiter und holt Mitarbeiter die nach Kallisto abreisen m"ochten. Mehrere Shuttle werden gewartet. In einem separaten Bereich sind die Maschinen, 8 Valkyrien der J"agerstaffel untergebracht. 

Zum Zeitpunkt des Eintreffens der Gruppe ist die Mine HeM05 an der Schlepperinsel vert"aut und teilweise zerlegt. Die Minen HeM01 und HeM04 sind im Einsatz. Die Besatzung der zerst"orten HeM3 sind teils zur Erholung auf Kallisto und teils bereit wieder im Einsatz auf den anderen Minen.

Im Raumhafen angekommen werden die Charaktere bereits von \emph{Grace Anders} erwartet. Grace ist Teil des lokalen Sicherheitsdienstes der Cynarian Corporation. F"ur den Aufenthalt der Charaktere ist sie zur unterst"utzung der Ermittler von \emph{Henk Arongate} dem Chef des Sicherheitsdienstes abgestellt. Sie steht hiermit den Ermittlern w"ahrend ihres gesamten Aufenthalts treu zur Seite, kann Recherchen beauftragen, kennt die Station mit ihren verwirrenden G"angen und kann lokale Unterst"utzung anfordern. Beim Eintreffen wird sie die Ermittler aufkl"aren dass es sich um eine Minenkolonie handelt und dadurch die Gepflogenheiten etwas ruppiger seien k"onnen. Aus diesem Grunde tragen die Sicherheitskr"afte Schutzkleidung und eine Waffe. Desweiteren erfahren die Ermittler dass ihre Untersuchungen m"oglicherweise kritisch aufgenommen werden k"onnten da man meint die Vorkommnisse k"onnten auch lokal gekl"art werden.

\begin{remarks}
	Die Spieler k"onnen die ersten Information von Grace Anders dazu nutzen sich selbst passend auszur"usten. Kontaktieren die Ermittler Henk Arongate direkt wird er sie h"oflich begr"u\3en, verweist sie dann aber weiter an Grace. Grace Anders ist eine junge h"ubsche aber auch vor allem kompetente und loyale Unterst"utzerin. Sie dient dem Spielleiter den Spielern unter die Arme zu greifen sollten sie selbst nicht weiter kommen und bringt der"uber hinaus eine pers"onliche Note ins Spiel mit ein.
\end{remarks}

\pageimage{images/hellgate.jpg}
%% Copyright 2019 Bernd Haberstumpf
%% License: CC BY-NC
% !TeX spellcheck = de_DE
\newsection{Befragung der HeM05 Besatzung}

Die 10 geretteten Minenarbeiter, werden wie die J"agerpiloten m"oglichst kurz vor dem Eintreffen der Ermittler aus der Dekompressionskammer eintlassen. Beim Eintreffen der Charaktere auf Hellgate befinden sich geretteten Mienenarbeiter in der Kantine der J"agerstaffel. Grace Anders wird die Charaktere zur Kantine begleiten. Vor den R"aumlichkeiten treffen die Charaktere auch den bekannten Pilotenausbilder Jos\'{e} \frqq{}Torro\flqq{} Alvarez. Torro ist ein kleiner drahtiger Spacer von fr"ohlicher Natur dem die Jahre als Pilot allerdings schon deutlich zugesetzt haben. Torro der die Rettungsaktion geleitet hat kann einen ersten Einblick in die Geschehnisse geben. Nach dem Gespr"ach mit Torro k"onnen sich die Charaktere den Minenarbeitern zuwenden. Sie k"onnen zun"achst entscheiden ob sie diese einzeln Interviewen wollen oder sie alle direkt in der Kantine aufsuchen. Sollen die Arbeiter einzeln befragt werden bietet Torro an Florence zu den Ermittlern zu den Ermittlern zu bringen. Danach kann Grace "ubernehmen und die Besatzung der Mine einzeln heraus bitten.

Im folgenden die Aussagen der Beteiligten:

\begin{description}
	\item[Torro] \say{Bei einem "Ubungsflug durch die obersten Athmostph"arenschichten erhielten wir einen Notruf der Mine HeM05. Da 		meine Trainngsstaffel mit insgesamt vier Valkyrien, 3 Rookies und mir, der Mine am n"achsten waren sind wir tiefer in die 		
		Athmosph"are eingetaucht und konnten dort gl"ucklicherweise die Mine nach kurzer Zeit lokalisieren. Da wir die Arbeiter nicht mit unseren Jagdmachinen selbst retten konnten blieb uns nur die M"oglichkeit an die Mine selbst anzudocken. Zugegebenerweise ein recht waghalsiges und f"ur die Auszubildenden ein risikoreiches Man"over. Wir waren zu diesem Zeitpunkt bereits in eine f"ur uns kritischen Athmosph"arenbereich besunken. Mit viel Gl"uck schafften wir es dann aber drei Maschinen an die Mine anzudocken und mit voller Schubleistung die Mine auf eine H"ohe zu bringen die es der Schlepperinsel erlaubte die Mine in den Orbit zu ziehen. Ein hei\3er Ritt kann ich Ihnen nur sagen.}
	\item[Florence (Kommandantin)] \say{W"ahrend der ersten Systemmeldung dass einer der Tr"agerbalons der Station abgekoppelt wurde 	
		befanden sich Juri Smirnov, Blackwind, ZDee und ich auf der Br"ucke. Greydog war in der Minenanlage besch"aftigt. Die anderen waren nach ihren eigenen Angaben im oberen Bereich der Mine. Ich beauftragte als erstes ZDee die Aufh"angung des Tr"agerbalons au\3erhalb der Mine zu kontrollieren und sandte einen Hilferuf an die Hellgate Station. Einige Minuten sp"ater beobachteten wir auf der Br"ucke wie von den Au\3enkameras aufgenommen, Pitch in ihrem Raumanzug in die Tiefe st"urzte. Ca.~10 Minuten sp"ater l"oste sich der zweite Tr"agerbalon. Nach einem Notruf befahl ich die Evakuierung. Treffpunkt war das Rettungsshuttle. R"uckmeldung bekam ich von allen au\3er ZDee. Auf dem Weg zum Shuttle sammelten wir noch Salvador vor seinem Quartier ein. Er war gerade dabei sich fertig anzuziehen. Am Rettungshuttle traf die Br"uckencrew auf Isabell und Fernandez. Das Rettungsshuttle lie\3 sich nicht starten. Die Startsequenz war durch eine Manipulation blockiert. Deshalb blieb uns nichts anderes "ubrig die Dekompressionskammer aufzusuchen und auf Rettung zu hoffen. Auf dem Weg zur Dekompressionskammer l"oste sich offensichtlich der dritte Ballon. An der Kammer trafen Greydog und Hanibal auf uns. Hanibal hatte noch versucht "uber die Steuerung der Anlage die Manipulation der Ballons zu verhindern. ZDee war von seiner Au\3enmission nicht zur"uck gekommen.}
	\item[Juri Smirnov, Blackwind] Die Br"uckencrew best"atigt die Aussage von Florence.
	\item[Salvador] \say{Ich war in meinem Quartier als der Aufruf zur Evakuierung kam. Die Br"uckencrew kam kurz darauf bei meinem 	
		Quartier vorbei und nahm mich mit.}
	\item[Greydog] \say{Ich war an der Raffinerie mit Wartungsarbeiten im Au\3enbereich am unteren Ende der Raffinerie besch"aftigt. 
		Dadurch habe es nicht geschafft die anderen bereits am Shuttle zu treffen.}
	\item[Fernandez Lorend] \say{Ich habe Isabell mit der Justierung ihrer Zentrifugen f"ur die Analyse des Atmosph"arengemischs 
		unterst"utzt als der Notruf einging. Darauf hin machten wir uns unverz"uglich auf den Weg zum Rettungsshuttle.}
	\item[Isabell Sonderleiten] Isabell best"atigt die Aussage von Fernandez. Ein paar Tage vor dem Attentat vertraute Pitch ihr an, dass 
		sie auf eigene Faust gegen ein anderes Besatzungsmitglied recherchierte weil sie glaubte dieser sei f"ur die Havarie der HeM03 verantwortlich. Ihr Verdacht wurde geweckt durch Unregelm"a\3igkeiten in der Steuersoftware der Mine. Sie lie\3 sich deshalb auf HeM05 einschiffen um ihren Verdacht weiter zu verfolgen und den Attent"ater selbst zur Rede zu stellen. Wen sie im Verdacht hatte hat sie allerdings nicht verraten. Die Befragung von Isabell erfolgt nur stockend. Das erlebte und der Verlust von Pitch machen ihr offensichtlich stark zu schaffen.
	\item[Blackwind] \say{Pitch wurde als Attent"aterin identifiziert, weil sie bei der Abkopplung des Tr"agerballons im Au\3enbereich 
		der Mine t"atig war, was "uberhaupt nicht ihrem Arbeitsbereich entspricht. F"ur die Wartung und das Einspielen neuer Software hatte mich Pitch gebeten ihr tempor"ar Zugang auf die Steuerung des Rettungsshuttle zu geben. Pitch war bereits vorher auf HeM03 besch"aftigt.}
	\item[Hanibal] \say{Als das Abkoppeln des ersten Ballons durchgegeben wurde begann ich sofort die Ansteuerung der Tr"agerballons zu 
		kontrollieren da die softwaretechnische Wartung mir und Pitch unterlag. Dabei konnte ich eine Manipulation der Steuerung entdecken die wahrscheinlich die Abkopplung ausgel"ost hat. Als mein Rettungsversuch mi\3lang und der dritte Ballon sich gel"ost hatte machte ich mich auf den Weg zur Dekompressionskammer da Florence bereits die Manipulation des Shuttles durchgegeben hatte.}
\end{description}

W"ahrend der Befragung macht Grace Anders Meldung an ihren Vorgesetzten \emph{Karl Sandos} und "ubermittelt ihm die Aussagen der Besatzung der Mine. Karl Sandos gibt diese Informationen an \emph{Henk Arongate} weiter.

\begin{remarks}
	Die Minenarbeiter sind durch die Vorkommnisse nach wie vor angeschlagen. Die in diesem Kapitel zusammengefassten Aussagen k"onnten entsprechen emotional ausgeschm"uckt werden. Die Kommandantin Florence, eine Beta, wird die Crew wenn n"otig in Schutz nehmen und verteidigen.

	Isabel ist schon vor der Versetzung auf HeM05 mit Pitch befreundet und deshalb stark mitgenommen durch ihren Tod und dem Verdacht die Attent"aterin zu sein.

	Durch die Aussagen k"onnen die Ermittler bereits ermitteln, dass Pitch aller Vorraussicht nicht die Attent"aterin sein kann. Der zweite und vor allem der dritte Tr"agerbalon l"osten sich erst nach ihrem Absturz.		
\end{remarks}

%% Copyright 2019 Bernd Haberstumpf
%% License: CC BY-NC
% !TeX spellcheck = de_DE
\newsection{Das Geschehen auf HeM05}

F"unf Tage vor dem Attentat wurde die Mannschaft der Mine durch eine Rumpfmannschaft, gef"uhrt von Florence der Kommandantin, ersetzt. Die nachfolgende Mannschaft wird 10 Tage nach dem Eintreffen der Rumpfmannschaft auf der Mine erwartet und trifft etwa zeitgleich mit den Charakteren auf Hellgate ein. Die Ankunft der zugeh"origen F"ahre k"onnen die Charaktere auf dem Flugdeck miterleben. Das Attentat selbst erfolgte chronologisch folgenderma\3en:

\begin{enumerate}
	\item Einen Tag nach der Ankunft auf HeM05 installierte Pitch eine "Uberwachungssoftware in der Minensteuerung, die Manipulationen 	
		blockiert und sie "uber Manipulationsversuch informiert. Pitch hat zu diesem Zeitpunkt ihren Kollegen Hannibal bereits im Verdacht, den Absturz der Mine HeM03 "uber eine Softwaremanipulation veranlasst zu haben.
	\item Pitch macht nach dem Abflug der vorherigen Minenmannschaft das Shuttle untauglich, um eine Flucht des mutma\3lichen T"aters zu 
		verhindern.
	\item Am Tag des Attentats versucht Hannibal zun"achst wie auf HeM03 eine Fehlfunktion der Anlagensteuerung hervorzurufen. Eine solche 
		Manipulation h"atte zu einem massiven Minenschaden und zur Zerst"orung eines gro\3en Teils der Mine gef"uhrt.
	\item Als die Manipulation der Minensteuerung aufgrund der Software von Pitch misslingt, begibt sich Hannibal in einem Raumanzug in den 
		Au\3enbereich und koppelt Tr"agerballon Nummer eins ab.
	\item Pitch verfolgt Hannibal und stellt ihn auf der Au\3enbalustrade der Mine zur Rede, als dieser gerade den ersten Ballon abkoppelt. 
		Es kommt zum Kampf. Hannibal st"urzt Pitch in den Abgrund.
	\item W"ahrend der Abkoppelung des zweiten Ballons wird er von ZDee "uberrascht kann aber diesen ebenfalls im Kampf "uberw"altigen und 
		in die Tiefe st"urzen.
	\item Nach dem Abkoppeln des dritten Tr"agerballons kehrt Hannibal ungesehen in die Mine zur"uck, legt den Raumanzug ab und begibt sich 
		zum Treffpunkt bei der Dekompressionskammer.
\end{enumerate}

\newsection[Nachforschungen auf Hellgate]{Nachforschungen auf Hellgate}

Nachdem die Ermittler den Raumhafen verlassen haben, wird die Minenbesatzung von f"unf Gardisten des lokalen Sicherheitsdienstes abgef"uhrt und auf den St"utzpunkt der Sicherheitskr"afte gebracht. Die "Uberf"uhrung wurde von Henk Arongate pers"onlich in Absprache mit dem B"uro von Vandermool veranlasst. Das Vorgehen wird von Arbeiter auf dem Raumdeck beobachtet, unter anderem von Slingshot, der unter dem Namen Drake bereits vor den Charakteren auf Hellgate ankam. Zum Zeitpunkt des Eintreffens der Ermittler hat er bereits Kontakt mit den hiesigen Arbeitern aufgebaut. 

Nach der ersten Befragung k"onnen die Ermittler weitere Nachforschungen auf Hellgate angehen oder bei Grace beauftragen. Als N"achstes werden die Ermittler voraussichtlich entweder die Quartiere der Minenbesatzung auf Hellgate aufsuchen oder ihr Nachforschungen auf der Mine HeM05 fortsetzen. Zuerst sollten sie aber einen Bericht an ihre Vorgesetzten abgeben. Erfolgt eine Meldung an das B"uro von Vandermool werden die Ermittler davon in Kenntnis gesetzt, dass die Minenarbeiter aus Sicherheitsgr"unden verlegt wurden.

Folgende Informationen k"onnen bereits "uber Nachfragen direkt beschafft werden:

\begin{description}
	\item[Tr"agerballons] Die Tr"agerballons lassen sich "uber das Anlagensystem Notentl"uften. Abkoppeln lassen sie sich nur von au\3en, 
		direkt an der Kopplungsstelle des Ballons. Die Informationen l"asst sich "uber Grace Anders bzw.~durch R"ucksprache mit Dr.~Petrova der technischen Leiterin der Minen in Erfahrung bringen. Wenn Pitch nicht die Attent"aterin ist, kommen nur noch Hannibal, ZDee oder Greydog als Attent"ater in Frage.
	\item[Greydog] Greydogs Geschichte l"asst sich ebenfalls "uber Grace Anders, durch Nachfrage bei Dr.~Petrova oder vor Ort in der Mine 
		plausibilisieren. Die Wege in der Mine von der Raffinerieanlage zu Br"ucke und den Aufenthaltsr"aumen sind mehrere hundert Meter lang.
	\item[HeM03] Nachforschungen zu HeM03 ergeben, dass sich au\3er Florence und Hannibal kein Mannschaftsmitglied mehr auf Hellgate befindet. Auf HeM03 waren insgesamt 50 Arbeiter besch"aftigt die sich gl"ucklicherweise fast alle mit Hilfe des Rettungsshuttles retten konnten. 
	\item[HeM03 Attent"ater] Bei der Manipulation der Mine HeM03 hatten laut Hannibals Aussage ein gewisser \emph{Lionel Hampton}, ein 
		\emph{Ice Diver} und er versucht, die Attent"aterin Sent von der Manipulation der Minensoftware abzuhalten. Lionel Hampton wie auch Sent wurden dabei get"otet. Ice Diver gilt seit dem Vorfall auf der Mine als vermisst.
	\item[Hannibal] Nachfragen zum Hintergrund von Hannibal ergeben, dass Hannibal vor einem dreiviertel Jahr im Raumhafen von Valhalla als 
		Softwaretechniker angestellt wurde. Vor 2 Monaten wechselte er Cynarian als Sicherheitstechniker f"ur den Minenbetrieb. 
	\item[Pers"onliche Hintergr"unde] Nachfragen zu Hintergr"unden der Minenbesatzung f"ordern zutage, dass keine auff"alligen Gelder 
		geflossen sind oder Angeh"orige unter Druck gesetzt werden konnten. Alle Mitarbeiter arbeiten seit Monaten im Auftrag von Cynarian.
\end{description}

Die genannten Informationen sollten erst nach Abschluss der Befragung der Mannschaft zug"anglich gemacht werden, damit sie nicht in die Befragung mit einflie\3en k"onnen und sofort zur Aufkl"arung der Ereignisse f"uhren.

Bei Nachfragen zur Schlepperfehlfunktion vor 9 Wochen ist ermittelbar, dass ein Softwarefehler Sch"aden an zwei Tr"agerpunkten f"ur Minen verursacht haben. Die Reparaturen dauern noch an. Gl"ucklicherweise l"asst sich die Schlepperinsel nach wie vor mit ihren drei weiteren Andockpunkten nutzen. Der Softwarefehler wurde von Hannibal ausgel"ost. Zum Zeitpunkt der Fehlfunktion war er allerdings bereits auf der Mine HeM03 im Einsatz. Auf Nachfragen im Flugbereich hin, erfahren die Spieler auch von einem Shuttleabsturz vor "uber 10 Wochen. Der Vorfall wird nicht als Attentat eingestuft, sondern wird als Beispiel erw"ahnt, dass nicht alle Unf"alle als Attentate ber"ucksichtigt werden m"ussen. Der Shuttleabsturz oder besser die Kollision mit einer F"ahre auf dem Hangardeck wurde durch einen Pilotenfehler des "uberm"udeten und mit Wachhaltedrogen gedopten Alpha-Mutanten Razor ausgel"ost. Razor wurde bereits befragt. Er ist inzwischen auf Kallisto.

Folgende Instruktionen und Information bekommen die Charaktere unaufgefordert:

\begin{description}
	\item[Chefermittler Cynarian] Der Chefermittler der Cynarian Corporation wird von Colonel Scholz angewiesen, den assistierenden 
		Ermittler alleine zu einer weiteren Befragung des verd"achtigen Hannibal in den St"utzpunkt der Sicherheitskr"afte zu schicken.
	\item[Assistent Cynarian] Der assistierende Ermittler der Cynarian Corporation, wird von Henry Longdale dem Sekret"ar Vandermools 	
		pers"onlich und eindringlich beauftragt, als Psychonaut die Verd"achtigen jeweils einem Gehirnscan zu unterziehen.
	\item[Chefermittler Protektorat] Der Chefermittler des Protektorats wird von Artisan beauftragt mit dem Chefermittler der Cynarian 
		Corporation zusammen die Ermittlungen in den Quartieren der Verd"achtigen fortzusetzen.
	\item[Assistent Protektorat] Da Blackheart dem Sicherheitsdienst auf Hellgate misstraut, beauftragt Thunderbolt seinen Untergebenen 
		umgehend den Sicherheitsdienst, der die gefangenen Minenarbeiter abf"uhrt, abzufangen und zu begleiten. Da Blackheart ziemlich unzufrieden mit der aktuellen Entwicklung ist, erfolgt der Auftrag mit deutlichem Nachdruck.	
	\item[Cowboybrigade] Der Chefermittler des Protektorats wird, falls noch nicht geschehen, dar"uber informiert, dass die Cowboybrigade 	
		auf dem Garnisonsst"utzpunt auf Valhalla auf Anweisung von \emph{Commander Lockhead} festgesetzt wurde.
	\item[Slingshot / Drake] Der Chefermittler des Protektorats wird von der Verwaltung des Raumhafens auf Armageddon informiert, dass 
		Slingshot vor 3 Tagen im Raumhafen der Station von einer Kamera erfasst wurde. Mutma\3lich um sich von dort aus absetzen zu k"onnen.
\end{description}

\begin{remarks}
	Nach den sch"atzungsweise langatmigen Ermittlungen bietet es sich an ab hier das Schritttempo des Plots zu erh"ohen und die Informationen in kurzen Zyklen auszugeben, um weitere Ermittlungen abzuk"urzen. Ausreichende Informationen sind mit den Befragungen der Minenarbeiter und den nachgefragten Informationen bereits vorhanden.

	Die Verlegung der Minenbesatzung soll den Spielern verdeutlichen, dass ihre F"uhrung auch eigenst"andig ins Spielgeschehen eingreift. Die Verlegung der Arbeiter wurde veranlasst, um die Arbeiter vor "Ubergriffen zu sch"utzen und potenzielle Attent"ater zu isolieren.

	Die Anweisungen die den assistierenden Cynarian Ermittler und den assistierenden Ermittler des Protektorats zum St"utzpunkt des Sicherheitsdienstes f"uhrt, dient dazu die Gruppe zu trennen. F"ur den weiteren Verlauf der Ereignisse auf Hellgate erlaubt die Trennung einen Spannungsbogen in den kommenden Szenen aufzubauen.
\end{remarks}

\newsection[Im Angesicht des Arbeiter Mobs]{Im Angesicht des Arbeiter Mobs}
                              
Beim Weg zu den Quartieren der HeM05 Minenbesatzung oder anderweitig zu Wohnbereichen auf Hellgate wird Grace Anders und die Gruppe durch 7 finster dreinblickende Arbeiter mit provisorischen Schlagwaffen in Form von Werkzeugen eingekesselt. Die Arbeiter wurden von Slingshot alias Drake dem Attent"ater aus den Reihen der Cowboy Brigade, angestachelt, die Ermittler zur Rede zu stellen, wieso die Minenarbeiter wie Verbrecher abgef"uhrt wurden. Den Ermittlern ist zu diesem Zeitpunkt vermutlich noch nicht bekannt, dass die Minenarbeiter vom Sicherheitsdienst abgef"uhrt wurden. Auch Grace Anders ist nicht informiert.

\begin{remarks}
	Das Missverst"andnis l"asst sich leicht durch eine R"uckfrage bei Karl Sandos dem Chef von Grace Anders aufl"osen. Die Minen-Crew wurde auf Anweisung von Henk Arongate dem Sicherheitschef der Hellgate Station auf den Sicherheitsst"utzpunkt "uberf"uhrt. Aus den Ergebnissen der Befragung der Minenarbeiter, die Grace Anders an ihren Vorgesetzten geleitet hat, konnte die Cynarian F"uhrung schlie\3en, dass wenigstens ein Attent"ater noch Teil der Crew sein d"urfte. Eine Isolation des Minenpersonals wurde deshalb als ratsam erachtet.

	Eine Eskalation l"asst sich durch Androhung des Einsatzes von Sicherheitskr"aften der Station oder durch Drohungen seitens des Omega-Soldaten der Gruppe vermeiden. 

	Der Vorfall verschafft Drake und seinem Helfer Zeit f"ur eine Befreiung von Hannibal.
	
	Sollte es zu einer Ausschreitung kommen, finden sich Details unter \emph{Pers"onlichkeiten auf Hellgate}.
\end{remarks}

\newsection{Quartiere und die HeM05 Mine}

Die Quartiere von Pitch, Hannibal, Greydog und dem Rest der Minen Besatzung befinden sich im Wohnbereich der Station. Die Quartiere sind an den Seiten von langen G"angen aufgereiht und durch eine kleine Schleuse zu betreten. Da sich die Mitarbeiter auf Hellgate und in den Minen nur wenige Wochen aufhalten, viele davon haupts"achlich in den Minen, hat kaum jemand ein festes Quartier auf Hellgate. Die meisten beziehen ein Quartier nur f"ur den zeitweiligen Aufenthalt auf der Station und geben es dann an andere weiter. Die Mitarbeiter haben entsprechend einen Gro\3teil ihrer Habseligkeiten immer dabei oder irgendwo eingelagert. Quartiere auf der Station bestehen aus einer einklappbaren Liege, einem kleinen Tisch mit Sitzgelegenheiten, einem Spind und einer Nasszelle inklusive Toilette. Neben dieser Einrichtung gibt es eine Nahrungsaufbereitungsanlage und ein Computersystem mit Holoprojektor, das in erster Linie f"ur die Kommunikation oder als Server f"ur Datenabfragen und Unterhaltung dient.

Im Quartier von Pitch finden sich ein paar Bilder, auf denen sie vermutlich mit Freunden auf dem Mars abgebildet ist. Das Computersystem ist dabei schon interessanter. Man findet dort ein Tagebuch mit Rechercheergebnissen und Randnotizen zu den Vorkommnissen auf HeM03 und ihrem Verdacht, dass Hannibal in die Vorkommnisse verstrickt sein k"onnte. In einem weiteren Tagebucheintrag ist ihr Beschluss notiert, Hannibal auf die HeM05 Mine zu folgen. Von einer Anzeige hatte sie abgesehen, da sie sich nicht vorstellen konnte, dass Hannibal das Attentat wirklich begangen hatte. Ihr war offensichtlich unklar was ihn zu so einer Tat gef"uhrt haben k"onnte. Hannibal war ab der Zeit auf HeM03 nach ihren Angaben ein guter Kollege gewesen.

In den Quartieren von Hannibal und Greydog sind keine pers"onlichen Gegenst"ande zu finden.

Nachforschungen auf HeM05 k"onnen nur mit Unterst"utzung von Mitarbeitern der Mine erfolgen. Auf der Mine, die an die Schlepperinsel angedockt ist, werden derzeit einige Reparaturen durchgef"uhrt. Die Mine selbst ist ein imposantes Konstrukt, das einem auf den Kopf gestellten, abgeflachten Kegel mit einer H"ohe von "uber 500 Metern entspricht. In den oberen 20 Metern befinden sich die Wohnquartiere, Aufenthaltsr"aume, der Shuttlehangar, die Br"ucke, Arbeitsst"atten, Lager und technische Einrichtungen f"ur den Betrieb der Mine. Der darunterliegende Teil ist die eigentliche F"orderanlage und deren Tanks f"ur das gef"orderte HE-3. Um die Mine herum sind an verschiedenen Stellen Balkone gezogen, um den Arbeitern einen einfacheren Au\3eneinsatz zu erm"oglichen. Die f"unf Tr"agerballons sind rund um den oberen Teil der Mine aufgeh"angt. Derzeit besitzt die Mine weiterhin nur zwei von f"unf Tr"agerballons. Ein Ersatz f"ur die drei zerst"orten Ballons wird derzeit auf der Nike Station hergestellt. F"ur den Au\3eneinsatz werden spezielle schwere Raumanz"uge ben"otigt, die die Arbeiter vor den Widrigkeiten der Jupiter-Atmosph"are f"ur eine kurze Zeit sch"utzen k"onnen. F"ur die Nachforschungen auf der Mine werden Spezialisten ben"otigt, die die Anlagen der Mine erkl"aren, sowie Softwareexperten, die Manipulationen an der Software aufdecken k"onnen. Auf der Mine k"onnen die Manipulationen am Shuttles, durchgef"uhrt durch Pitch, zweifelsfrei nachgewiesen werden. Ebenso ist die von Pitch installierte Sicherung der Minensteuerung und deren Manipulationsversuch von Hannibal ersichtlich.

%% Copyright 2019 Bernd Haberstumpf
%% License: CC BY-NC
% !TeX spellcheck = de_DE
\newsection{Befragung beim Sicherheitsdienst}\anchor{sec:interrogatehanibal}

Treffen die Ermittler bevorzugt einen Psychonaut im St"utzpunkt des Sicherheitsdienstes ein, k"onnen dort Hannibal oder auch andere Besatzungsmitglieder verh"ort werden. Ein Psychonaut kann dabei einen Gehirnscan durchf"uhren. Der St"utzpunkt ist "ahnlich wie ein Polizeirevier aufgebaut. Eine T"ur f"uhrt zun"achst in einen Eingangsbereich mit einem durch eine Plexiglasscheibe abgetrennten Tresen. Der diensthabende Sicherheitsbeamte empf"angt die Ermittler und fragt nach ihrem Begehren. Fordern die Charaktere eine Befragung der Minenarbeiter an, kontaktiert der Beamte den Stationsleiter Karl Sandos, der die Ermittler daraufhin pers"onlich empf"angt und sie durch den St"utzpunkt, zu den Zellen und zum Verh"orraum f"uhrt.

F"ur ein Verh"or k"onnen die Minenarbeiter einzeln in einen Verh"orraum gebracht werden. Im Folgenden wird davon ausgegangen, dass ein Psychonaut das Verh"or von Hannibal durchf"uhrt. Bei einem Verh"or durch einen Psychonauten sollte dieser alle Anwesenden au\3er eventuell einem anderen Ermittler bitten, den Raum zu verlassen, um seine ``Befragung'' in Ruhe durchf"uhren zu k"onnen. Der milit"arische Assistenzermittler des Protektorats oder ein anderer der Spielercharaktere sollte anwesend bleiben, um Hannibal zu fixieren.

Die Aktivit"aten von Hannibal zum Zeitpunkt der Attentate auf HeM03 und HeM05 finden sich im Gehirn in derselben Form, wie sie Hannibal zu Protokoll gegeben hat: Auf HeM03 "uberw"altigen Lionel Hampton, Ice Diver und er Sent, nachdem Hannibal bemerkt, wie Sent dabei ist, die Computeranlage der Mine zu manipulieren. Durch eine Messerattacke wird Lionel Hampton get"otet; Ice Diver und Hannibal wiederum k"onnen Sent t"oten. Es ist jedoch bereits zu sp"at, um die Manipulation aufzuheben. Auf der HeM05-Mine wiederum erlebt der Psychonaut, wie Hannibal vergeblich versucht, die Manipulation der Tr"agerballonsteuerung aufzuheben. Hannibal widersetzt sich der Untersuchung seiner Erinnerungen nicht, nachdem der Ermittler in seine Gedanken eingedrungen ist. Die geistigen F"aden sind leicht zu verfolgen. Die Bilder sind scharf gezeichnet und leicht nachvollziehbar. Mit der Zeit stellt der Psychonaut allerdings fest, dass den Erinnerungen jegliche Form von Emotionen fehlen und sie geradlinig sowie detailarm wirken. Wie in einem Film wird er durch inszenierte Szenen gef"uhrt.

Versucht er, das Gehirn des Attent"aters wieder zu verlassen, wird er pl"otzlich durch die KI in Hannibals Kopf daran gehindert. Er findet sich unversehens auf einer k"unstlichen gr"unen Wiese mit blauem Himmel wieder, auf der nur eine einzige vollkommen wei\3e androgyne Person steht. Er wird in die Gedanken dieser Person gezogen und erf"ahrt von einem Zwang, die Minenanlagen auf dem Jupiter zu zerst"oren. Kurz darauf nimmt er einen Countdown wahr und mu\3 versuchen, den Geist von Hannibal schnellstm"oglich zu verlassen, um nicht get"otet zu werden.

\begin{remarks}
	\underline{Gewonnene Information:}

	\begin{itemize}
		\item Hannibals Gehirn wurde von einer KI "ubernommen.
		\item  Die KI, die die Attentate ausf"uhrt, handelt selbst gegen ihren Willen.
	\end{itemize}

	\underline{Gehirnscan und Brainburner:}

	Ein Gehirnscan ist im Regelwerk im hinteren Teil des Buches \cref{sec:psychnaut} beschrieben. Mit dem Countdown warnt die KI in Hannibals Gehirn den Psychonaut vor einem Brainburner der, wenn ausgel"ost das Gehirne des Psychonauten in einer Kettenreaktion, von durch Stromst"o\3e "uberlasteten Synapsen, zerst"ort.

	\underline{Der Hilferuf der KI:}

	In dieser Szene erfahren die Ermittler, dass das Gehirn des Attent"aters manipuliert und er zu seiner Tat gezwungen wurde. Das Auftauchen der wei\3en Gestalt ist ein Hilferuf der KI, die versucht, sich gegen ihre Programmierung zu wehren. 
	
	Der Begriff ``K"unstliche Intelligenz'' sollte zu diesem Zeitpunkt noch nicht fallen. Der Spielleiter sollten den Spielern erlauben, eigene R"uckschl"usse aus den bizarren Erlebnissen zu ziehen. Viel Zeit daf"ur wird ihnen allerdings nicht gew"ahrt.
\end{remarks}

%% Copyright 2019 Bernd Haberstumpf
%% License: CC BY-NC
% !TeX spellcheck = de_DE
\pageimage{images/hellgate_security_floorplan.jpg}
\newsection{Die Geiselnahme}\anchor{sec:hostage}

Haben die Charaktere ihre Befragung im Stützpunkt des Sicherheitsdienstes abgeschlossen, werden sie von Karl Sandos dem Stationsleiter wieder in den Eingangsbereich des Stützpunktes begleitet. 

Zum Einstieg in die folgende Szene sollte der Spielleiter, wenn möglich abwarten, bis sich die Charaktere außerhalb des Stützpunktes und die Charaktere im Stützpunkt in einem Informationsaustausch über das ComNetz befinden. Die vergangenen Ereignisse bieten ausreichend Stoff zu einer ausgiebigen Diskussion. Während des Gesprächs bricht dann die Verbindung zwischen den beiden Gruppen plötzlich ab. 

\newsubsection{Überfall auf den Stützpunkt}

Kurz nach dem Kommunikationsabbruch öffnet sich der Zugang zum Stützpunkt und ein Gegenstand wird in den Raum geworfen. Die \emph{Personal Area Networks (PAN)}, die Anbindungen elektronischer Systeme an das Gehirn der Personen im Raum fallen aus. Jeder Charakter in der Station muß einen Konstitutionswurf würfeln, um nicht kurzzeitig das Bewusstsein zu verlieren. Das PAN des Omega-Ermittlers, wenn er sich auf dem Stützpunkt befindet, wird als Erstes in Teilen wieder aktiv. Der Ausfall führt aber zeitweilig zu körperlichen und sensorischen Einschränkungen.

Mit dem Öffnen des Zugangs stürmen zwei bewaffnete Personen in den Raum und eröffnen sofort mit vollautomatischen Railguns das Feuer. Bei den Eindringlingen handelt es sich um Slingshot und dem Söldner \emph{Smith Handerson}. Beide Angreifer sind mit Kampfanzügen, Handerson zusätzlich mit einem Helm mit Sichtschutz gerüstet. Befindet sich der Assistenz-Ermittler des Protektorats im Eingangsbereich, wird er als Omega-Soldat sofort erkannt und von Smith Handerson ins Visier genommen. Ein weiteres Ziel ist Karl Sandos, der daraufhin hinter dem Tresen zu Boden geht. Ist kein Omega-Soldat anwesend, schießt einer der Angreifer auf einen weiteren Sicherheitsbeamten \emph{Luke Lengdon}, der dabei in die Brust und am Kopf getroffen wird und daraufhin ins Koma fällt. Beim folgenden Kampf wird nur ein Omega die Möglichkeit haben, eine Waffe zu ziehen oder in den Nahkampf zu gehen, wobei sein Handicap bestehen bleibt. Bei ihrem Angriff sollte es den Angreifern leicht fallen die überrumpelten Personen im Eingangsbereich kampfunfähig zu machen und sie danach in Schach halten zu können. Die Angreifer sammeln die zu Beginn geworfene EMP-Schockgranate ein, die die PANs der Anwesenden außer Kraft gesetzt haben.

\newsubsection[Außerhalb des Gebäudes]{Ausserhalb des Gebäudes}

Während die Angreifer den Stützpunkt überfallen, um Hannibal zu befreien, müssen die Ermittler außerhalb des Stützpunkts die neue Situation erst einmal verarbeiten. Zunächst wollen sie vermutlich in Erfahrung bringen, wodurch und wie die Verbindung zu ihren Mitstreitern ausfallen konnte. Grace Anders versucht, ihren direkten Vorgesetzten Karl Sandos und danach den Sicherheitschef Henk Arongate zu kontaktieren. Von ihm erfährt sie, dass nur das ComNetz, das Kommunikationsnetz der Station, beim Sicherheitsstützpunkt gestört ist. Er verspricht, Leute zum Stützpunkt zu schicken, weist aber darauf hin, dass die Ermittler sich am nächsten zum Stützpunkt befinden. Henk Arongate erklärt, dass im Umkreis von rund 30 Metern um den Stützpunkt das ComNetz ausgefallen wäre.

Der Stützpunkt selbst liegt an einem von zwei Seiten zugänglichen Tunnel. Vor dem geschlossenen Tor liegt der schwer verletzte Luke Lengdon. Er liegt im Koma und ist nicht ansprechbar. Andere Personen sind nicht erkennbar. Falls die Charaktere nicht selbst aktiv werden, bittet Grace Anders um Deckung und pirscht sich an ihren Ex-Freund heran, um dort Ersthilfe zu leisten.

Der Eingangsbereich ist auf den ersten Blick leer, wobei der Tresen nicht einsehbar ist. Auf dem Boden sind blutige Schleifspuren sichtbar. Um den Tresen herum sind Einschusslöcher zu erkennen. Die Scheibe ist zerstört. Nach der Erstürmung des Eingangsbereichs gilt es zunächst, das Gelände abzusichern und den verletzten Lengdon zu stabilisieren. Im Eingangsbereich und im Gangsegment vor dem Sicherheitsstützpunkt befinden sich keine weiteren Personen. Eine Erste-Hilfe-Ausrüstung mit medizinischem Expertensystem findet sich im vorderen Bereich des Stützpunkts.

\newsubsection{Rückblende}

Nachdem Handerson und Slingshot den Eingangsbereich des Stützpunkts unter ihre Kontrolle gebracht haben, öffnen sie mit dem Identitätsimplantat von Karl Sandos die Tür zu den hinteren Räumlichkeiten. Dieser Bereich umfasst Gefängniszellen, einen Verhörraum und Büros. Die Angreifer sperren alle Personen außer dem verletzten Luke Lengdon gemeinsam in eine Zelle und verlassen dann den Stützpunkt. Ihnen folgt der befreite Hannibal. Als Absicherung nehmen sie einen der Ermittler und die beiden Frauen der Minenbesatzung als Geiseln. Der Omega-Krieger der Ermittlergruppe, falls anwesend, wird in der Gefängniszelle zurückgelassen. Bei ihrer Flucht hinterlassen sie im hinteren Teil des Stützpunkts einen Störsender, der das ComNetz in der Umgebung des Stützpunkts lahmlegt, zusammen mit einem Funkgerät, um Sicherheitskräfte in die Irre führen zu können.

\newsubsection{Inspektion des Eingangsbereichs}

Nach der Absicherung des Eingangsbereichs können die Ermittler den Raum weiter untersuchen. Der Tresen kann mit der Chipkarte von Grace Anders betreten werden. Hinter dem Tresen sind weitere Blutspuren zu entdecken. Versuchen die Anwesenden, die Tür zu den inneren Bereichen zu öffnen, werden sie feststellen, dass nicht einmal Grace Anders die Tür öffnen kann. Sie vernehmen eine Stimme hinter der Tür, die ihnen droht, die Geiseln zu töten, falls jemand versucht, durch die Tür zu kommen. Bei dem Sprecher handelt es sich um Slingshot, der über Sprechfunk so lange wie möglich versucht, den Eindruck zu vermitteln, die Entführer befänden sich noch im Stützpunkt.

Kurz nachdem die Charaktere den Stützpunkt betreten, treffen weitere Mitarbeiter des Sicherheitsdienstes in Begleitung von zwei Sanitätern ein. Der Eingreiftrupp setzt sich aus zwei Norms und drei Mutanten zusammen. Sie tragen die Sicherheitswesten des Sicherheitsdienstes und jeweils eine Bolzenpistole. Die Sanitäter sind Norms. Da es sich bei Hellgate um einen Cynarian-Stützpunkt handelt, sind keine Omega-Soldaten aus den Streitkräften des Protektorats auf Hellgate im Einsatz. Angeführt wird der Trupp von einem \emph{Luke Dexter}, der sich direkt über den Verfall in Kenntnis setzen lässt. Währenddessen lässt er durch seine Leute den Eingangsbereich und die Gänge absichern. Die Sanitäter kümmern sich um den schwer verletzten Luke Lengdon.

\newsubsection{Im Zellentrakt}

Im Zellentrakt sind die übrigen Charaktere zusammen mit den Minenarbeitern eingesperrt. Insgesamt befinden sich vier Personen, einschließlich Hannibal, in den Händen der Entführer. Vom Zellentrakt aus ist nicht zu erkennen, dass die Entführer das Gebäude bereits verlassen haben. Unter den Gefangenen befindet sich der angeschossene Karl Sandos. Die Entführer haben ihren Opfern erlaubt, ein Erste-Hilfe-Kit mit in die Zelle zu nehmen. Es ist also angebracht, den schwer verletzten Stationsleiter erst einmal zu verarzten. 

Die Gefangenen haben danach Zeit, zu versuchen, sich zu befreien oder sich bemerkbar zu machen. Die Zellen sind für Randalierer und Aufständler gedacht. Dementsprechend sind sie nicht so gut gesichert wie eine reguläre Gefängniszelle. Auch wenn das ComNetz noch immer lahmgelegt ist, ist die Tür nicht ohne passendes Werkzeug zu knacken. Benötigt wird ein sogenannter \emph{Magschlossknacker}, um das elektronische Schließsystem zu knacken, oder Werkzeug, um die Türhydraulik zu überbrücken. Möglicherweise können auch Gegenstände aus dem Raum oder den Taschen der Minenarbeiter zweckentfremdet werden. Ein Luftschacht ist eine weitere Möglichkeit, zumindest Kontakt mit der Außenwelt aufzunehmen. Der Spielleiter kann großzügig kreative Ideen gewähren lassen. Befindet sich kein Charakter unter den Gefangenen, werden diese sich ruhig verhalten. Der Spielleiter kann je nach Spielfluss entscheiden, wie viel Zeit an dieser Stelle aufgebracht wird.

\newsubsection{Auf der Flucht}

Während der Vorkommnisse im Stützpunkt sind die Entführer zusammen mit ihren Geiseln auf dem Weg zu ihrem Shuttle, um Hellgate zu verlassen. Kurz nach dem Verlassen des Stützpunktes haben die Angreifer ihre Kampfausrüstung gegen schusssichere Westen und einfache Bolzenpistolen getauscht, um nicht aufzufallen. Slingshot, Smith Handerson und Hannibal sind jeweils mit einer Schusswaffe bewaffnet und treiben die Geiseln vor sich her. Um die PAN-Systeme der Geiseln zu stören, nutzen die Geiselnehmer ebenfalls einen Störsender mit kurzem Radius.

Nach der Absicherung des Eingangsbereichs können die Charaktere zusammen mit Luke Dexter, dem Anführer der eingetroffenen Sicherheitskräfte, in Verhandlung mit den Entführern treten. Die Entführer nutzen, wie schon beschrieben, ihr Sprechfunkgerät, um die Illusion aufrechtzuerhalten, sich noch im Sicherheitsstützpunkt aufzuhalten. Fordern die Charaktere nach einem Lebenszeichen ihres Freundes oder einer anderen Geisel, kann Slingshot den entführten Ermittler bitten, ein kurzes Lebenszeichen von sich zu geben. Dabei kann dieser versuchen, eine geheime Botschaft zu übermitteln.

Früher oder später wird der Eingreiftrupp wohl zusammen mit der Ermittlergruppe die hinteren Räume des Stützpunktes stürmen. Die Vorbereitung dazu wird nach wie vor durch den Störsender der Entführer erschwert. Eine Abstimmung mit anderen Kräften der Station ist nur bedingt möglich.

Zu ihrer Überraschung stoßen die Angreifer auf keine Gegenwehr. Die Gefangenen sind schnell befreit. Bei genauerer Untersuchung der eroberten Räume werden sowohl der Störsender als auch das Sprechfunkgerät sichergestellt. Charaktere mit militärischem Hintergrund identifizieren den Störsender als ein älteres Modell aus Militärbeständen, jedoch mit unbekannter Herkunft. Auch ist einem Soldaten die EMP-Schockgranate nach einer Beschreibung der Augenzeugen bekannt.
\vfill

\begin{remarks}
	\underline{Szenenwechsel:}

	Die Geiselnahme ist auf einen schnellen Szenenwechsel zwischen den Vorfällen im Stützpunkt und den anderweitigen Aktivitäten ausgelegt. Die Szenenwechsel sollten immer so gestaltet werden, dass die Spieler nur die Informationen erhalten, die auch ihren Charakteren zum jeweiligen Zeitpunkt zur Verfügung stehen. So dürfen die Spieler nach dem Überfall auf den Stützpunkt erst von der Flucht der Geiselnehmer erfahren, wenn der Zellentrakt des Stützpunktes gestürmt wurde.

	Der erste Angriff der Kidnapper sollte so ausgelegt werden, dass die Angreifer am Ende der Szene die Oberhand gewinnen.

	\underline{Vorbereitungen zum Angriff:}
	
	Im Auftrag von USI-Agenten wird Slingshot von Handerson auf Armageddon nach der Havarie der Mine HeM05 abgeholt und fliegt mit ihm nach Hellgate, um dort Hannibal abzuholen und nach Kallisto zu bringen. Sie kommen auf Hellgate an, kurz bevor die Charaktere Hellgate erreichen. Slingshot, unter dem Decknamen Drake, kann dadurch verfolgen, wie Hannibal zusammen mit den anderen Minenarbeitern zum Stützpunkt der Sicherheitsmannschaft gebracht wird.

	\underline{Beziehungen zwischen den Attentätern:}

	Slingshot steht mit Artisan, dem Stellvertreter des Protektors, in Kontakt. Dieser wiederum ist mit weiteren Attentätern und 
	USI-Agenten in Kontakt. Die Attentäter selbst sind sich allerdings nicht ihrer Aktivitäten als Attentäter vollständig bewusst. Genauso wenig nimmt ihr menschlicher Geist die anderen Attentäter als solche bewusst wahr. Alle Erinnerungen an ihre ``zweite Identität'' werden aus ihrem Geist ausgeblendet und werden auch von einem Psychonauten nicht als solche entdeckt.
\end{remarks}


\newsection{Showdown auf Hellgate}

Die Entführer haben ihr Shuttle auf der Außenseite von Adrastea bei einer Wartungsschleuse nahe dem Raumhafen verankert. Nachdem die Charaktere im Stützpunkt herausgefunden haben, dass sich die Entführer nicht mehr dort befinden, gibt es mehrere Möglichkeiten, sie aufzuspüren.

Am erfolgversprechendsten wäre es, die nach wie vor auftretenden Störungen im ComNetz der Station zu verfolgen.

Für die Navigation im Stützpunkt müssen die Kidnapper kurzzeitig den Störsender deaktivieren. Das gibt den Geiseln die Möglichkeit, eine kurze Nachricht zu übermitteln.

Folgende Szenarien, um die Entführer zu stellen, sind denkbar:

\begin{description}
	\item [Lagerkomplex] Im Lagerkomplex des Raumhafens gäbe es die Möglichkeit, einen Hinterhalt zu legen und die Entführer zu 
		überraschen. 
	\item [Landeplattform] Die Attentäter und ihre Geiseln haben Hellgate verlassen und sind dabei, ihr Shuttle zu betreten.
	\item [Shuttleverfolgung] Ist das Shuttle bereits gestartet, bleibt den Charakteren nur, das Shuttle zu verfolgen und zu entern, um die 
		Gefangenen zu befreien.
\end{description}


\newsubsection{Landeplattform}

Hellgate verfügt neben dem Raumhafen über eine Reihe von Ausgängen, die durch Tunnel zur Oberfläche des Mondes führen. Diese Tunnel dienen als Flucht- und Rettungswege oder bieten Zugang zu verschiedenen Sensorplattformen, meist in Richtung Jupiter.

Das Shuttle der Entführer hat auf der Landeplattform G.1 mit seinen Andockklammern festgemacht. Der Zugangsbereich zum Landedeck umfasst zwei Luftschleusen: eine für Personen und eine zweite für den Transport von Frachtcontainern. Eine Transportgondel fährt über 500 Meter zu einem großen Zwischenlager. Ein tenderähnliches Schienenfahrzeug ermöglicht es, Material von einem angedockten Schiff zur Schleuse zu transportieren. Der Weg zum Landebereich ist überdacht. Die Schwerkraft auf Adrastea ist vernachlässigbar; es herrscht mehr oder weniger Schwerelosigkeit. Trotz Magnetstiefeln sollten sich Personen an Handläufen festhalten.

\newsubsection{Verfolgung des Shuttles}

Die Entführer versuchen, über einen niedrigen Jupiter-Orbit zu flüchten, um eine Verfolgung innerhalb der Atmosphäre zu erschweren. In der Nähe des Jupiter sind Sensoren nur eingeschränkt nutzbar. Dies wiederum erlaubt es, Verfolgern unerkannt von hinten im Schatten des Fusionstriebwerks anzugreifen. Die Geiseln sind im Laderaum des Shuttles eingesperrt. Für ein Entermanöver ist die Dawn of Day am ehesten geeignet. Mit einem Dockingtunnel kann sie an das Shuttle der Entführer andocken. Die Schleuse in das Schiff kann entweder mit einem Magschlossknacker entriegelt oder aufgeschweißt werden. Toro Alvarez ist bereit, den Ermittlern Flankenschutz zu bieten und für ein Ablenkungsmanöver zu sorgen, indem er mit den Entführern in Verhandlungen tritt. Eine Valkyrie bietet neben einem Piloten auch Platz für eine zweite Person.

Hannibal und Slingshot werden um jeden Preis versuchen, ihrer Gefangennahme zu entgehen. Der Söldner hingegen ist nicht bereit, bei dieser Mission getötet zu werden.
\vfill

\pageimage{images/hellgate_landeplattform.jpg}


\begin{remarks}
	Bei den drei Optionen ist zu beachten, dass sie in Herausforderung und Zeitaufwand von oben nach unten gestaffelt sind. Zum Zeitpunkt der Entführung ist der Plot erst zu einem Drittel gespielt. Der Spielleiter sollte je nach angepeilter Spieldauer eine der Optionen auswählen.

	Der Showdown ist nicht darauf ausgelegt, die Entführer entkommen zu lassen. Die Hoffnung der Kidnapper, mit ihrem Täuschungsmanöver die Verfolger abzuschütteln, darf also nicht aufgehen.
	
	Wurde Hannibal keinem Gehirnscan unterzogen, sollte nachfolgend einem Psychonauten ein Gehirnscan ermöglicht werden. Wenn Hannibal bereits einem Gehirnscan unterzogen wurde, sollte der Spielleiter darauf abzielen, dass sowohl Hannibal als auch Slingshot getötet werden.
	
	Die beiden Attentäter dürfen dem Cynarian oder dem Protektorat zumindest nicht für weitere Untersuchungen geistig intakt zur Verfügung stehen. Im Zweifelsfall tötet die KI mit einem Brainburner das Gehirn des Attentäters.
\end{remarks}

%% Copyright 2019 Bernd Haberstumpf
%% License: CC BY-NC
% !TeX spellcheck = de_DE
\newsection{Hellgate Debriefing}

Kurz nach dem Sieg "uber die Entf"uhrer, nachdem verletzte Charaktere wieder weitestgehend handlungsf"ahig sind, laden Vandermool, Avenger und Blackheart die Ermittler zu einem geheimen virtuellen Treffen in einem hermetisch abgeschirmten Raum der Hellgate Station ein. Die drei sind bei diesem Treffen nicht pers"onlich anwesend, sondern werden holografisch in den Raum projiziert. Das Ermittlerteam kann nochmals in allen Details "uber die Geschehnisse berichten. 

Die Ereignisse auf Hellgate werden von der F"uhrung Vandermool, Avenger und Blackheart als kritisch eingesch"atzt. Attent"ater aus den eigenen Reihen mit einem fremdgesteuerten Gehirn sind sowohl f"ur die Cynarian Aktivit"aten auf dem Jupiter als auch f"ur das Protektorat eine ernsthafte Gefahr. Des Weiteren haben die Ereignisse aber auch ein in diesem Ma\3e vorher nicht vorhandenes Misstrauen zwischen Cynarian und dem Protektorat ausgel"ost. Avenger und vor allem Blackheart bef"urchten eine verdeckte Operation der Cynarian Corporation. Vandermool ist beunruhigt, dass die Geschehnisse genau so eine Reaktion ausl"osen. Beunruhigen ist, dass weder der Feind noch der Ursprung der Bedrohung bekannt sind. Das hei\3t, eine Form von Gegenwehr ist zum aktuellen Zeitpunkt nicht m"oglich.

Vandermool und Blackheart sticheln gegeneinander, weil Blackheart eine milit"arische Aktion bevorzugt. Avenger versucht zu vermitteln. Die drei Parteien werden sich aber am Ende einig die neu gewonnenen Erkenntnisse geheim zu halten, um die Gegner nicht aufzuschrecken und den gegenw"artigen Wissensstand nicht offenzulegen. Der Punkt geht an Vandermool und Avenger. Die Ermittlungen unauff"allig weiterzutreiben wird als sinnvollstes Vorgehen nach hitziger Diskussion beschlossen. Neben dem vorrangigen Ziel im verborgenen Informationen zu sammeln, m"ochte man eine Panik unter den Bewohnern des jovianischen Systems vermeiden. Die Sicherheit des Systems unterliegt so in Teilen der Ermittlergruppe! Weitere Nachforschungen m"ussen unter strengster Geheimhaltung fortgesetzt werden. 

Die Entdeckung der Gehirnmanipulation deuten auf ver"anderte kybernetische Modulen hin, die auf Kallisto eingesetzt wurden. Aus diesem Grund werden die Charaktere angewiesen dort ihre Nachforschungen fortzuf"uhren. Da in einem Monat eine wichtige Konferenz geplant ist, bleibt zudem wenig Zeit die Hintergr"unde der Attentate aufzukl"aren. Den Ermittlern wird deshalb eine Frist von drei Wochen gesetzt. 

Veranlasst durch Vandermool werden die Leichen oder verletzten Attent"ater umgehend auf die Nike Station transportiert, um dort n"aher untersucht werden zu k"onnen. Blackheart setzt durch, dass Mitglieder der Mutantennation den Untersuchungen beiwohnen. Sie entsendet zwei Omegas mit den Namen \emph{Stomper} und \emph{Bullet} nach Nike.

\begin{remarks}
	Das Briefing auf Hellgate ist die erste und einzige Gelegenheit w"ahrend der Geschichte, bei der Vandermool, Avenger und Blackheart zusammen auftreten. Das Gespr"ach zeigt das dominante Auftreten Vandermools und die Vorhalte Blackhearts gegen"uber dem Vorgehen. Neben der Offenlegung der politischen Verh"altnisse, dient die Zusammenkunft im Wesentlichen dazu, die Spieler in Richtung Valhalla zu lenken und die Nachforschungen auf Hellgate zu beenden.

	In einem Monat nach den aktuellen Ereignissen wird eine Delegation aus dem \emph{Shigano Kombinat} der st"arksten wirtschaftlichen Einrichtung auf dem Mars und der \emph{Europ"aische F"orderation} auf der Erde erwartet. Es handelt sich um ein politisch-wirtschaftliches Gipfeltreffen. F"ur das Protektorat sind Verb"undete au\3erhalb des Jovianischen Systems f"ur die Zukunft wichtig. Auch will man die Lage der Mutanten nach den Auseinandersetzungen auf der Erde, die letztendlich zur Gr"undung des Protektorats gef"uhrt haben, verhandeln. Aufgrund der noch nicht gefestigten Position des Jovianischen Systems unter der F"uhrung der Cynarian Corporation und dem Protektorat w"aren die in der Tat von der USI geplanten Attentat bei diesem Treffen ein verheerender R"uckschlag. Das Konferenz ist zum aktuellen Zeitpunkt noch als geheim eingestuft, weswegen das Treffen gegen"uber den Charaktere nicht zur Erw"ahnung kommen sollte.

	Um die Entfernung von mehr als eine Million Kilometern zwischen den einzelnen Orten der Gespr"achspartner zu vermitteln, sollte der Spielleiter die "Ubertragungszeit von mehreren Sekunden in die Dialoge einstreuen.
\end{remarks}


% Kallisto
%% Copyright 2019 Bernd Haberstumpf
%% License: CC BY-NC
% !TeX spellcheck = de_DE
\newsection{Eintreffen auf Hellgate}

Der Flug von Armageddon dauert rund 10 ereignislose Tage  mit dem Shuttle Dawn of Day w"ahrenddessen sich die Gruppe mit dem Shuttle vertraut machen k"onnen. 

Die HeM05 ist beim Eintreffen der Ermittler an der gigantischen Schlepperinsel der Hellgate Station angedockt. Die Schlepperinsel ist ein 2 Kilometer langes und breites Raumfahrzeug das mit gewaltigen Schubd"usen bis in die "au\3eren Athmosph"arenregionen des Jupiter eintauchen kann um dort die HE-3 Mienen abzusetzen oder einzusammeln. Die Schlepperinstel schwebt beim Anflug auf Hellgate majest"atisch nahe dem Mond Adrastea "uber der gewaltigen Fl"ache des Jupiter. Kleinste Partikel bilden eine Schleier auf diesem niedrigen Orbit von 130'000 km "uber dem Planeten. Hellgate befindet sich bis auf den Anflugtunnel fast vollst"an dig im Inneren des Mondes. Die Station selbst besteht aus dem Raumhafen, technischen Anlagen, Lagerhallen und R"aumen und Wohnquartieren, Lokale, Bars und L"aden. Im Ganzen umfasst die Anlage ca.~30 km\textsuperscript{3}. Wie in alles neuen eilig aufgesetzen Einichtungen befinden sich viele Provisorien, nicht abgeschlossene G"ange und herumstehendes Material in der Station.

Beim Ansteuern des Anflugtunnels wird die Dawn of Day von der Flugkontrolle kontaktiert und nach einer Legitimation gefragt. Nach den ersten Formalit"aten wird das Shuttle "uber einen Leitstrahl in den Landungstunnel navigiert. Der Pilot in der Gruppe kann hierbei sein K"onnen unter Beweis\3 stellen. Dem Spielleiter bleibt "uberlassen wie weit er den Landeanflug ausschm"uckt. Beim Eintreffen im Raumhafen herrscht reger Betrieb, eine gro\3e F"ahre bringt gerade neue Minenarbeiter und holt Mitarbeiter die nach Kallisto abreisen m"ochten. Mehrere Shuttle werden gewartet. In einem separaten Bereich sind die Maschinen, 8 Valkyrien der J"agerstaffel untergebracht. 

Zum Zeitpunkt des Eintreffens der Gruppe ist die Mine HeM05 an der Schlepperinsel vert"aut und teilweise zerlegt. Die Minen HeM01 und HeM04 sind im Einsatz. Die Besatzung der zerst"orten HeM3 sind teils zur Erholung auf Kallisto und teils bereit wieder im Einsatz auf den anderen Minen.

Im Raumhafen angekommen werden die Charaktere bereits von \emph{Grace Anders} erwartet. Grace ist Teil des lokalen Sicherheitsdienstes der Cynarian Corporation. F"ur den Aufenthalt der Charaktere ist sie zur unterst"utzung der Ermittler von \emph{Henk Arongate} dem Chef des Sicherheitsdienstes abgestellt. Sie steht hiermit den Ermittlern w"ahrend ihres gesamten Aufenthalts treu zur Seite, kann Recherchen beauftragen, kennt die Station mit ihren verwirrenden G"angen und kann lokale Unterst"utzung anfordern. Beim Eintreffen wird sie die Ermittler aufkl"aren dass es sich um eine Minenkolonie handelt und dadurch die Gepflogenheiten etwas ruppiger seien k"onnen. Aus diesem Grunde tragen die Sicherheitskr"afte Schutzkleidung und eine Waffe. Desweiteren erfahren die Ermittler dass ihre Untersuchungen m"oglicherweise kritisch aufgenommen werden k"onnten da man meint die Vorkommnisse k"onnten auch lokal gekl"art werden.

\begin{remarks}
	Die Spieler k"onnen die ersten Information von Grace Anders dazu nutzen sich selbst passend auszur"usten. Kontaktieren die Ermittler Henk Arongate direkt wird er sie h"oflich begr"u\3en, verweist sie dann aber weiter an Grace. Grace Anders ist eine junge h"ubsche aber auch vor allem kompetente und loyale Unterst"utzerin. Sie dient dem Spielleiter den Spielern unter die Arme zu greifen sollten sie selbst nicht weiter kommen und bringt der"uber hinaus eine pers"onliche Note ins Spiel mit ein.
\end{remarks}

\pageimage{images/hellgate.jpg}
%% Copyright 2019 Bernd Haberstumpf
%% License: CC BY-NC
% !TeX spellcheck = de_DE
\newsection{Auf dem Garnisonsst"utzpunkt}
\bottomimage{images/armytank.png}

Ein erster Anlaufpunkt der Charaktere w"are vermutlich der Garnisonsst"utzpunkt des Protektorats, da dort die Cowboybrigade aufzufinden ist. Da Kallisto offiziell nicht dem Protektorat angeh"ort, unterh"alt das Milit"ar des Protektorats nur einen kleinen St"utzpunkt in Valhalla. Der St"utzpunkt beherbergt eine Kompanie von 40 Soldaten, gr"o\3tenteils Omega-Krieger. Diesen angeschlossen sind Techniker und sonstiges Personal.

Der Oberbefehlshaber des St"utzpunktes ist der Omega \emph{Commander Lockhead}, der bereits "uber das Eintreffen der Ermittler informiert ist. Commander Lockhead ist ein in die Jahre gekommener Veteran mit unerwartet freundlichen Gem"ut. 

Nach der Untersuchung des Frachterungl"ucks auf Armageddon hat er die Cowboybrigade entweder auf Anweisung der Charaktere oder auf einen Befehl von Colonel Scholz hin inhaftieren lassen und h"alt sie seitdem in Untersuchungshaft. Der Commander ist "uber die Vorg"ange auf Hellgate informiert und bietet auch diesbez"uglich den Ermittlern seine Hilfe an. Er kann f"ur Untersuchungen im Umfeld des Raumhafens, der Garnison und der Oberstadt seinen Adjutanten \emph{Firedon}, einen jungen Omega-Soldaten, zur Verf"ugung stellen. 

Zu Fragen nach der Cowboybrigade kann Lockhead beitragen, dass als Vorbereitung f"ur den Einsatz neuer Drohnen und anderer technischer Ger"aten, die auch auf Armageddon zum Einsatz kamen, die ganze Cowboybrigade mit neuen Talentchips vor 2 Monaten in der Klinik \emph{Rondra Hospital} ausgestattet wurden. Die Kosten hat die Garnison "ubernommen. Die Cowboybrigade ist seit vier Monaten am Raumhafen f"ur Wartungsarbeiten an Schiffen und Ger"atschaften der Garnison zugeteilt. Auch nach ihrer Versetzung vom zivilen Teil des Raumhafens zur Garnison ist eine gewisse \emph{Sonja Frost} ihre personelle Vorgesetzte.

\begin{remarks}
	\underline{Gewonnene Information:}
	
	\begin{itemize}
		\item Die Cowboybrigade ist seit vier Monaten bei der Garnison im Raumhafen t"atig. 
		\item Vor 2 Monaten sind ihr neue Talentchips in der Rondra Klinik eingesetzt worden.
		\item Firedon wurde zur Unterst"utzung der Ermittler freigestellt.
		\item Der Kontakt zum Raumhafen ist Sonja Frost, der Chief Officer des Hangardecks.
	\end{itemize}

	F"ur Fahrten durch die Oberstadt bietet Firedon ein Milit"ar-Truck an.
	 
\end{remarks}

\newsection[Nachforschungen im Rondra Hospital]{Nachforschungen im Rondra\newline{}Hospital}

In Bezug auf das Rondra Hospital k"onnte der Verdacht aufkommen, dass dort die Manipulationen an den Gehirnen von Hannibal und Slingshot durchgef"uhrt wurden. Auf Dr"angen von Commander Lockhead w"are ein Termin mit dem Klinikleiter \emph{Prof.~Dr.~Henry Sanders} m"oglich. Die beiden Herren, so verschieden wie sie sind, kennen einander aus vergangenen Zeiten bei der Europ"aischen F"orderation.

Der Klinikleiter empf"angt die Charaktere in einem ger"aumigen B"uro. Sanders ist ein Norm im Alter von "uber 50 Jahren, mit gepflegtem Aussehen und grau meliertem Haar. Bei Fragen zur Cowboybrigade oder Hannibal, verweist er die Gruppe an \emph{Brenda Ben}, die er auch gleich bittet die Ermittler zu unterst"utzen. 

Brenda Ben ist eine sympathische und erfahrene "Arztin. Sie ist Mitglied des Leitungsteams der Klinik. Routiniert kann sie von 
\emph{Ben Reuthers} aus der Buchhaltung Unterlagen anfordern und die Ermittler im Fall der Cowboybrigade an den behandelnden Chirurgen \emph{Dr. Loyd Rothan} sowie die Physiotherapeuten \emph{Russel Spenser} und \emph{Phillip Klarson} weiterleiten. Hannibal ist in den Akten des Rondra Hospitals nicht verzeichnet. Brenda Ben hat jedoch nur begrenzt Zeit und verabschiedet sich nach dieser ersten Recherche, da sie zu einer Operation gerufen wird.

Dr. Loyd Rothan best"atigt das Einsetzen der Talentchips, bei allen Mitgliedern der Cowboybrigade, durchgef"uhrt zu haben. Nach den erfolgreichen Eingriffen trainierten die Physiotherapeuten die Wartungstechniker im Umgang mit ihrer neuen Cybertechnologie. Ben Reuthers best"atigt die Beauftragung der Operation durch Sonja Frost und die Zahlungsabwicklung durch die Garnison.

Die Informationen aus der Klinik entsprechen denen von Commander Lockhead.

\begin{remarks}
	\underline{Gewonnene Information:}
	
	\begin{itemize}
		\item Neue Kontakte: Prof.~Dr.~Henry Sanders, Brenda Ben, Ben Reuthers, Dr. Loyd Rothan, Russel Spenser und Phillip Klarson.
		\item Eingriff bei der Cowboybrigade im Rondra Hospital ist unauff"allig. Keine besonderen Vorkommnisse.
		\item Hannibal wurde nicht im Rondra Hospital operiert.
		\item Die Aussagen am Hospital decken sich mit den Aussagen von Commander Lockhead.
	\end{itemize}

	\underline{Prof.~Dr.~Sanders:}

	Wie sich sp"ater herausstellen wird, ist Sanders viel tiefer in die Vorkommnisse mit den implantierten KIs involviert, als sich zum aktuellen Zeitpunkt erahnen l"asst. Bei der Cowboybrigade wurden aber in seiner Klinik nur die beauftragten Talentchips installiert. Der Professor kann deshalb die Ermittler guten Gewissens an seine Mitarbeiter verweisen.
\end{remarks}

\newsection{Cowboybrigade Voruntersuchung}

Gehen die Charaktere davon aus, dass weitere Mitglieder der Cowboybrigade ebenfalls einer Gehirnmanipulation unterzogen wurden, k"onnen sie Firedon beauftragen einen der vier in einer Klinik untersuchen, zu lassen. Das Rondra Hospital f"uhrt als gr"o\3tes Hospital in Valhalla alle medizinischen Eingriffe f"ur den Garnisonsst"utzpunkt durch. Vertrauen die Charaktere dem Rondra Hospital nicht, schl"agt Firedon die davon unabh"angige \emph{Alexandr Clinic} vor. 

Da nicht bekannt ist, ob sich ein weiteres Mitglied der Cowboybrigade als Attent"ater entpuppen k"onnte, sollten die Charaktere die Verd"achtigen zu diesem Zeitpunkt nicht in die Vorg"ange auf Hellgate oder Attentatsverd"achtigungen einweihen.

Werden die Untersuchungen in der Alexandr Clinic durchgef"uhrt, zeigt sich der leitende Arzt Dr.~Spinner zwar zun"achst erstaunt, warum man sich an seine Klinik und nicht an das Rondra Hospital wendet, ist aber bereit eine Untersuchung durchzuf"uhren. 

Bei den nicht invasiven Untersuchungen durch eine Klinik lassen sich keine Manipulationen feststellen. Die implantierten technischen Einheiten wirken nach Aussage der "Arzte unauff"allig. Die Gehirnwellen weisen ebenfalls keine Anomalien auf. Wenn die Charaktere allerdings noch keine n"aheren Informationen von der Nike Station erhalten haben, ist auch nicht klar auf was ein Arzt achten sollte. 

Selbst mit den weiter unten beschriebenen Informationen aus der Nike Station werden bei der Untersuchung keine Auff"alligkeiten festgestellt. Der Rest der Cowboybrigade ist ebenfalls sauber.

\begin{remarks}
	\underline{Gewonnene Information:}
	
	Eine Untersuchung des Gehirns der Mitglieder der Cowboybrigade zeigt keine auff"alligen Anomalien.
\end{remarks}

\newsection{R"uckmeldung von Nike}\anchor{sec:nanobots}

Zwei Tage nach ihrer Ankunft werden die Ermittler von Dr. ~\pinyin{Wang2} \pinyin{Chen2} kontaktiert.  Er f"uhrte die Untersuchungen an den K"orpern von Slingshot und Hannibal auf der Nike Station durch.

Bei den get"oteten Attent"atern konnte Dr. \pinyin{Wang2} \pinyin{Chen2} inaktive Nanobots entdecken. Diese Nanobots haben sich umfassend mit den Synapsen im gesamten Gehirn verbunden. Wurden einer der Attent"ater oder beide nicht get"otet, konnten elektrische Impulse gemessen werden, die weder dem Gehirn selbst noch dem Kontrollmodul zuordenbar waren. Die genaue Funktion der Nanobots konnte noch nicht entschl"usselt werden, aber es wird von einem hochkomplexen verteilten System ausgegangen.

Eine weitere Auff"alligkeit sind die in den K"opfen implantierten Kontrollmodule. Es handelt sich um Modelle neuester Generation, hergestellt von \emph{Kasai Cyber Genetics}, die "uber einen modifizierten Nanitenspender verf"ugen. Die Nanitenspender dienten vermutlich der Aussch"uttung der Nanobots. Kasai Cyber Genetics ist ein Technologieunternehmen mit Sitz auf dem Mars, das dem Shigano-Kombinat angeh"ort. Ihre Kontrollmodule sind beliebt, aber vergleichsweise kostspielig.
\vfill

\begin{remarks}
	\underline{Gewonnene Information:}
	
	\begin{itemize}
		\item ie beiden Attent"ater haben ein hochmodernes Kontrollmodul von Kasai Cyber Genetics, das einen modifizierten Nanitenspender in 
			ihren K"opfen enth"alt.
		\item Die Bewusstseinsver"anderung der Attent"ater wurde durch Nanobots im Gehirn ausgel"ost.
	\end{itemize}
	
	\underline{Nanobots:}

	Bei den Nanobots handelt es sich um die eigentliche KI, die mithilfe des Kontrollmoduls in das Gehirn gelangt. Das Kontrollmodul dient den KIs zur Kontaktaufnahme mit Agenten der USI, um von diesen Befehle zu erhalten. Bei jedem Kontakt mit einem ComNetz k"onnen die USI-Agenten der KI Anweisungen erteilen.

	\underline{Kontrollmodul:}

	Das implantierte Kontrollmodul ist ein leicht angepasstes Standardmodell. Kasai Cyber Genetics ist nicht in die Attentate verwickelt.
\end{remarks}

\newsection{Sonja Frost}

Eine weitere Anlaufstelle im Zentrum von Valhalla ist Sonja Frost. Sie ist die Chief Officer des Hangar-Decks des Raumhafens. Sonja Frost ist dementsprechend schwer zu erreichen. Ein Anruf durch den St"utzpunkt ist notwendig, um "uberhaupt einen Termin zu vereinbaren.

Sonja ist in ihrem B"uro neben dem Hangar anzutreffen. Das B"uro ist mit technischen Ger"aten, Holoprojektoren und Tafeln vollgestellt. Es herrscht ein reges Rein und Raus, und Sonja verteilt durchgehend Anweisungen.

Von Sonja erfahren die Ermittler, dass die Cowboybrigade vor ca.~1\half~Jahren vom Asteroideng"urtel zwischen Mars und Jupiter in das Jovianische System wechselte und dort am Raumhafen untergekommen ist. Vor vier Monaten wurden sie dann an das Milit"ar "uberwiesen. Der medizinische Eingriff bei der Cowboybrigade wurde in Abstimmung mit Commander Lockhead beschlossen und erwartungsgem"a\3 durchgef"uhrt und bezahlt.

Sonja Frost berichtet, dass die Mitglieder der Cowboybrigade am Raumhafen unter dem Pseudonym ``die glorreichen F"unf'' bekannt sind. Aufgrund ihrer heiteren, skurrilen Art, aber auch wegen ihrer Zuverl"assigkeit, sind sie im Hangar-Deck beliebt und wertgesch"atzt. Den Spitznamen Cowboybrigade erhielten sie erst durch die Besch"aftigung in der Garnison, als sie begannen, gegen"uber Stetson zu salutieren. Nach dem Einsetzen der Talentchips im Rondra Hospital kehrten die F"unf nach zwei Wochen wieder zum Dienst zur"uck. Slingshot meldete sich allerdings ein paar Tage sp"ater krank und kehrte erst nach weiteren zwei Wochen, kurz vor der "Uberstellung nach Armageddon, zur"uck. N"aheres kann sie dazu auch nicht beitragen.

Bez"uglich Hannibal muss sie in ihrem Terminal nachforschen. V"ollig ungewohnt benutzt sie daf"ur ein Ger"at ohne neuronales Interface. \say{Abgeschottetes System, aus Sicherheitsgr"unden}, sagt sie, wenn man sie danach fragt und grinst. Schnell wird sie f"undig. Hannibal hat vor neun Monaten seine Arbeit am Raumhafen als Softwaretechniker begonnen. Vor zwei Monaten wechselte er seinen Arbeitsplatz und bekam eine Anstellung bei Cynarian als Sicherheitstechniker auf der Minenkolonie Hellgate. Einen Monat vor seiner K"undigung hatte er sich f"ur 14 Tage krankgemeldet.

Bevor die Charaktere das B"uro verlassen, f"allt ihr noch ein, dass sich nach der Landung der Dawn of Day zwei Herren in Anz"ugen nach den Ank"ommlingen erkundigt haben. Einer der beiden war gro\3, hatte eine T"ursteherstatur und einen sehr gepflegten blonden B"urstenhaarschnitt. Der andere war unauff"allig und hatte nicht gesprochen. Informationen "uber die Ermittler konnten den beiden nicht gegeben werden (O-Ton: \say{K"onnte ja jeder kommen}). Aufzeichnungen von den beiden sind ihr nicht bekannt.

\begin{remarks}
	\underline{Gewonnene Information:}
	
	\begin{description}
		\item[Cowboybrigade] Die Cowboybrigade alias ``die glorreichen F"unf'' ist seit 1\half Jahren im jovianischen System und seitdem am 
			Raumhafen t"atig. Seit 4 Monaten sind sie bei der Garnison besch"aftigt. Vor zwei Monaten wurden ihnen Talentchips eingesetzt. Slingshot meldete sich danach insgesamt f"ur zwei Wochen krank.
		\item[USI-Agenten] Zwei Anzugtr"ager haben sich nach der Dawn of Day erkundigt. Bei den beiden handelt es sich um die USI-Agenten  
			\emph{Smith-Singer} und \emph{Frederic Johnson}. Smith-Singer ist der f"uhrende Drahtzieher hinter der Operation P9 und den Attentaten im jovianischen System. Frederic Johnson ist ein Psychonaut, der Smith-Singer unterst"utzt. Weiteres findet sich \cref{sec:usiagents}.
	\end{description}
\end{remarks}


\newsection{Die Cowboybrigade im Verh"or}

Die Ermittler k"onnen Firedon beauftragen, die inhaftierte Cowboybrigade in einem Verh"orraum vorzuf"uhren. Stetson l"ummelt beim Eintreten der Charaktere mit einem Zahnstocher im Mund und einem verbeulten Cowboyhut auf dem Kopf in seinem Sitz. Beim "Offnen der T"ur setzt er sich abrupt aufrecht. Quicksilver Rod mischt nerv"os, aber virtuos ein Deck Spielkarten. Joe Rider sitzt finster dreinblickend und eingesunken auf seinem Stuhl. Tom Gunslinger wendet den Blick erwartungsvoll in Richtung der Eintretenden. Werden die Vier zusammen befragt, wendet sich Stetson als Erster an die Charaktere und fragt, was ihnen vorgeworfen wird, was mit Slingshot los ist und wann sie wieder entlassen werden. Sie gehen nach wie vor davon aus, dass es sich beim Frachterungl"uck auf Armageddon um einen Unfall gehandelt hat.

Konfrontieren die Ermittler die Truppe teilweise oder vollst"andig mit den wahren Gegebenheiten auf Armageddon oder Hellgate, beteuert Stetson entgeistert ihre Unschuld. Er versichert, nach einem Blick zu den anderen, Auskunft zu allen Fragen zu geben. Er beteuert, dass seine Freunde und er sicher nichts zu verbergen h"atten.

Quicksilver Rod blickt bei einer Befragung immer wieder zu Stetson. Seine Finger scheinen dabei ein Eigenleben zu f"uhren. Ein Kartentrick folgt dem anderen, ohne das Rod Notiz davon nimmt. Bei Joe Rider vergeht nach jeder Frage erst eine halbe Minute bevor er antwortet. Die Antworten beschr"anken sich dann nur auf das Gefragte und enthalten nicht ein unn"otiges Wort. Tom Gunslinger ist das genaue Gegenteil. Gef"ahrlich wild gestikulierend, mit einem Trinkbecher in der Hand, schie\3en Worte aus seinem Mund. St"andig schweift er von dem aktuellen Thema ab.

Auf eine psychonautische Untersuchung reagiert die Gruppe leicht panisch. Keiner hat eine Vorstellung was auf sie zukommt k"onnte. Die Untersuchung lassen sie dann aber "uber sich ohne Gegenwehr ergehen. Ein Gehirnscan best"atigt, dass ihre Gehirne sauber sind und dass ihre Aussagen ihrem Wissensstand entsprechen. 

Angesprochen auf die Eingriffe in der Rondra Klinik schildern sie, dass sie in der Klinik neue Talentchips mit anschlie\3endem Training erhalten haben. Eine Frage, ob es bei Slingshot Komplikation gegeben h"atte, wird verneint. Allerdings erfahren die Ermittler, dass Slingshot im Gegensatz zu den anderen kein ausgebildeter Shuttle- und Drohnenpilot ist, sondern nur ein hervorragender Schiffstechniker. Slingshot hatte schon immer davon getr"aumt auch eine Flugausbildung zu erhalten. Offensichtlich hat er sich nach dem Aufenthalt im Rondra Hospital selbst auf die Suche nach einer entsprechenden Kontrolleinheit gemacht. Vor der Versetzung nach Armageddon wurde er dann f"undig. 
Ein erweitertes Kontrollmodul erlaubte es ihm Drohnen und Shuttles fernzusteuern. Wie er die daf"ur aufkommenden Gelder aufbringen konnte, wollte er den anderen nicht verraten. Ebenso wenig legte er offen, wo er den Eingriff hatte durchf"uhren lassen und wo er sich danach aufgehalten hat. 

Ein paar Tage nach der Entlassung aus dem Rondra Hospital hatte er angefangen die Freizeit oft alleine zu verbringen. "Uber den Barmann und Besitzer des Batcave konnten seine Freunde in Erfahrung bringen, dass er wohl eine h"ubsche Frau kennengelernt hatte. Darauf angesprochen tat er allerdings immer betont geheimnisvoll. Weitere Informationen sind von der Cowboybrigade nicht in Erfahrung zu bringen.

\begin{remarks}
	\underline{Gewonnene Information:}

	\begin{itemize}
		\item Slingshot hat sich eine Flugsteuerung implantieren lassen.
		\item Die Finanzierung des neuen Systems ist unbekannt.
		\item Slingshot hat eine Frau kennengelernt.
	\end{itemize}

	\underline{Carina:}
	
	Bei der geheimnisvollen Frau handelt es sich hierbei um Carina alias Fleur Soleil, auf die die Charaktere sp"ater in einem Club treffen werden. Die Identit"at der Freundin ist weder der Cowboybrigade noch dem Barmann im Batcave bekannt. Cariana ist \cref{sec:carina} beschrieben.
\end{remarks}

\pageimage{images/cowboybrigade_cut.jpg}

%% Copyright 2019 Bernd Haberstumpf
%% License: CC BY-NC
% !TeX spellcheck = de_DE
\newsection{Pers"onlichkeiten auf der Fenris Station}

Auf der Fenris Station sind folgende Personen relevant für die folgende Personen relevant sind:

\begin{description}
    \item [Grendel] Stationleiter der Fenris Station.
\end{description}


%% Copyright 2019 Bernd Haberstumpf
%% License: CC BY-NC
% !TeX spellcheck = de_DE
\newsection{Treffen mit Smith-Singer (optional)}

W"ahrend die Ermittler den ersten Hinweisen auf Valhalla nachgehen erh"alt einer der Charaktere, vorzugsweise ein Alpha-Mutant aber kein Omega eine Nachricht von einem Herrn Smith-Singer bzgl.~einem Informationsaustausch zu den Vorg"angen auf Hellgate. Der Nachrichtensender gibt sich als Beobachter im Auftrag des Transnationalen Konzernrates aus. Bei einer R"uckfragen bei Cynarian oder dem Protektorat werden die Ermittler gebeten den Termin wahr zu nehmen aber keine relevanten Informationen preis zu geben. Ob Smith-Singer wirklich im Auftrag des Konzernrates t"atig ist l"asst sich nicht klar bestimmen ist aber auch nicht auszuschlie\3en. Die Ermittler sollen in Erfahrung bringen was sein Auftrag ist und "uber welche Informationen er verf"ugt. Smith-Singer schl"agt vor im Stadtteil Rosenfurth einen Kaffee trinken zu gehen. Er w"urde sich dabei gerne allein mit dem Ermittler treffen.

Kommt der Ermittler alleine wird er Smith-Singer im "`Au\3enbereich"' des Kaffees nahe dem Eingang antreffen. Smith-Singer ist hoch gewachsen und hat eine athletische kr"aftige Statur. Die Finger sind manik"urt, das L"acheln makellos. Smith-Singer hat einen B"urstenhaarschnitt mit wei\3blonden Haaren. Er tr"agt einen teuren ma\3geschneiderten Anzug. Die holografisches Visitenkarte mit authentischem Konzernrats Logo weist ihn als Mitarbeiter des Konzernrates aus. Smith-Singer erkl"art, dass er als Beobachter im Auftrag des Konzernrates in das Jovianische System entsandt wurde um den Aufbau der HE-3 Produktion mit zu verfolgen und eine faire Zusammenarbeit der hiesigen Konzerne sicherzustellen. In diesem Zuge sind ihm die Attentate und die beunruhigende Erkenntnis einer Manipulation der Attent"ater zu Ohren bekommen. Wer seine Informationsquelle ist gibt Smith-Singer nicht an. Er fragt den Protektoratsermittler nach seiner Einsch"atzung der Bedeutung, dass die Attent"ater aus den Reihen der Mutanten gew"ahlt wurde und wen man als Drahtzieher hinter den Attentaten vermutet. Wenn das Gespr"ach anf"angt abzuebben wird er sich freundlich verabschieden begleicht die Rechnung, w"unscht noch einen guten Tag und verschwindet in der Menge.

Mit dem Treffen legt Smith-Singer die Grundlage f"ur ein Eingreifen des Konzernrates mit der USI als Drahtzieher. Zudem schafft er sich selbst mehr Handlungsspielraum indem er sich als Mitarbeiter des Konzernrates platziert. Desweiteren ist er auch daran interessiert sie Ermittler pers"onlich kennen zu lernen um sie besser einsch"atzen zu k"onnen.

\begin{remarks}
	Gewonnene Informationen: Bekanntschaft mit Smith-Singer.	
\end{remarks}

%% Copyright 2019 Bernd Haberstumpf
%% License: CC BY-NC
% !TeX spellcheck = de_DE
\newsection{Batcave}

Das Batcave ist die Stammkneipe der Cowboybrigade. Der Wirt \emph{Lenny Kilkenny}, auf die Cowboybrigade angesprochen, best"atigt, dass sie regelm"a\3ige G"aste in seinem Pub waren, aber wohl nach Armageddon versetzt wurden. Angesprochen auf Slingshot und seine Freundin, best"atigt er, dass Slingshot vor zwei Monaten ein oder zweimal mit einer h"ubschen Norm im Batcave als Gast eingekehrt ist. Sie hatten sich dabei immer an einen hinten gelegenen Platz gesetzt. Slingshot hat dabei die Getr"anke f"ur beide an der Theke geholt, wodurch Kilkenny keine Gelegenheit hatte, mit der Frau pers"onlich zu sprechen. Aufgefallen sind ihm ein schwarzes, samtiges Kleid mit Kapuze und nat"urlich die langen und kunstvollen, rot schimmernden Haare. Weiteres kann er zu der Besucherin nicht beisteuern.

Neben den Informationen zur Cowboybrigade kann Lenny Kilkenny die Ermittler an G"aste und Bekannte verweisen, die mehr Auskunft "uber den Verkauf von Cyberware au\3erhalb der offiziellen Wege geben k"onnen.

\begin{remarks}
	\underline{Gewonnene Information:}
	
	\begin{itemize}
		\item Slingshot hatte eine Freundin mit auff"alligem, roten Haar.
	\end{itemize}
	
	\underline{Carina:}

	Bei der Freundin handelt es sich um Carina alias Fleur Soleil, auf die die Cowboybrigade bereits verwiesen hat. Die Farbe ihrer auff"alligen Haarpracht kann sie je nach Laune "andern. Carina trifft die Gruppe das erste Mal in einem Nachtclub, beschrieben  \cref{sec:blackholeclub}.
\end{remarks}	

\newsection{Kliniken und Konzerne auf Kallisto}

Anlaufstellen f"ur weitere Ermittlungen k"onnen lokale Niederlassungen von Produktionsfirmen und H"andlern f"ur Cyberware sein. Die lokalen Firmen in Headquarter, die im Bereich Cyberware-Technologien und -Dienstleistungen t"atig sind (beschrieben \cref{sec:valhalla}), geben bereitwillig Informationen zu ihren Produkten heraus, weigern sich jedoch, Informationen "uber Kunden, Lieferwege oder ihre Mitarbeiter bereitzustellen.

Ein erster Ansatzpunkt in Bezug auf Cyberware-Produkte, sind die Kontrollmodule in den K"opfen der Attent"ater. Eine Nachricht von der Nike Station \cref{sec:nanobots} nennt die Firma Kasai Cyber Genetics als Hersteller. Leider werden keine Produkte dieser Firma in der Oberstadt vertrieben. Mit dem Rondra Hospital und der Alexandr Clinic, die aus den vergangenen Kapiteln bekannt sind, sind die Nachforschungsm"oglichkeiten in der Oberstadt ersch"opft. Den Ermittlern bleibt ab hier nichts anderes "ubrig, als den Weg von Hannibal, Slingshot und seiner Freundin durch die Gebiete au\3erhalb der Oberstadt Valhalla zu verfolgen.

Alle Kliniken, die auf Kallisto inoffiziell Implantate einsetzen, befinden sich au\3erhalb der Oberstadt. Bei Nachforschungen au\3erhalb von Headquarter, Rosenfurth und dem Raumhafen werden die Soldaten aus der Garnison die Ermittler nicht begleiten. An den R"andern von Paradise City und Neu Gr"oning endet der Einflussbereich des Protektorats, und man will Spannungen vermeiden. Auf Nachfrage erfahren die Ermittler, dass "uber einen gro\3en Teil Valhallas verschiedene Banden unter der Schirmherrschaft des \emph{Luna-Syndikats} herrschen. Au\3erhalb der Oberstadt und des Raumhafens ist die Anbindung an das ComNetz nur sp"arlich.

Die Kliniken in Valhalla m"ussen einzeln identifiziert werden, da kein zentrales Verzeichnis existiert. Vor Ort in den jeweiligen Sektoren der Stadt m"ussen Erkundigungen in Nachtclubs und Lokalen eingeholt werden. Ein erster Anlaufpunkt kann der Schieber \emph{Henk Brothers} sein, an den Lenny Kilkenny, der Barmann des Batcave, die Ermittler verweisen kann. Henk Brothers verweilt "ublicherweise im \emph{Green Mile}, einem der besseren Lokale am Randgebiet von Paradise City, am "Ubergang zu Rosenfurth. Wenn Henk Brothers bemerkt, dass die Gruppe als Ermittler t"atig ist, gibt er sich zugekn"opft und stellt nur Adressen von offiziellen Kliniken in den Wohngebieten R"otheim und Neu Gr"oning bereit. Wittert er eine Vermittlungsgeb"uhr, vermittelt er die Charaktere an einen weiteren Schieber, \emph{Thomas Siebel}, in R"otheim. Ein Treffen mit Thomas Siebel findet in einem Lagercontainer in R"otheim statt. Thomas Siebel empf"angt die Ermittler in Begleitung von zwei Schl"agern. Er h"ort sich die Fragen der Charaktere an und tauscht sich dann mit den beiden Schl"agern aus. Einer der Schl"ager verl"asst kurz das Geb"aude und kommt dann kopfsch"uttelnd zur"uck. Thomas Siebel lehnt die Anfrage der Gruppe daraufhin bedauernd ab. Er kann ihnen leider keine Cyberware verkaufen oder weitere Kontakte herstellen und wei\3 auch nichts "uber die gesuchten Personen. "Ahnlich ergeht es den Ermittlern bei anderen Schiebern und in offiziellen und inoffiziellen Kliniken.

Ein Grund f"ur die Misserfolge bei ihren Recherchen ist die wirtschaftspolitische Lage, die au\3erhalb der Oberstadt herrscht. Ein nicht unerheblicher Teil der Wirtschaft auf Valhalla au\3erhalb der zentralen Bezirke wird vom Luna-Syndikat kontrolliert. Dies gilt dementsprechend auch f"ur Schieber, Stra\3endocs sowie f"ur "Arzte mit ``Nebenberufen''. "Arzte werden zu inoffiziellen Behandlungen ohne Genehmigung des Luna-Syndikats bestenfalls ausweichende oder schwammige Ausk"unfte geben. Das Gleiche gilt f"ur Schieber oder sonstige Personen, die von den Ermittlern bez"uglich medizinischer Eingriffe oder Cyberware befragt werden.

Bei eigenen Nachforschungen werden die Charaktere also keine neuen Erkenntnisse gewinnen. Hannibal, Slingshot oder die Cowboybrigade sind in den Etablissements, die von den Charakteren besucht werden, nicht bekannt. In Bezug auf eine Frau mit auff"allig langen roten Haaren erf"ahrt man lediglich von einer T"anzerin mit langen braunen Haaren und einer S"angerin mit auff"alligen, aber blonden Haaren.

\begin{remarks}
	Um die unfruchtbaren Nachforschungen der Ermittler nicht zu sehr in die L"ange zu ziehen, sollte der Spielleiter den Eindruck einer Wand des Schweigens vermitteln. Wie oben beschrieben, sollte den Spielern z.B.~bei R"uckfragen auf dem Milit"arst"utzpunkt vermittelt werden, dass wahrscheinlich das Syndikat ihnen hier die Kn"uppel zwischen die Beine wirft.

	Die ablehnende Haltung der lokalen Firmen bez"uglich der Anfragen der Ermittler wird von Blackheart als weiterer Anlass bewertet, Ma\3nahmen gegen Valhalla in Erw"agung zu ziehen.
\end{remarks}

%% Copyright 2019 Bernd Haberstumpf
%% License: CC BY-NC
% !TeX spellcheck = de_DE
\newsection{Nemessis}\anchor(sec:nemessis)

Nemessis ist der Duke von Valhalla und der Pate des Luna-Syndikats. Er ist ein Slag, dessen K"orper sich nicht mehr selbst am Leben erhalten kann.

\begin{sideimagebox}[r]{0.9}{./images/cmyk/nemessis_cmyk.jpg}{}

\end{sideimagebox}

Ein Gro\3teil seiner Gliedma\3en und anderer K"orperfunktionen sind durch synthetische Teile ersetzt, die ihm das Aussehen eines Cyborgs verleihen. Nemessis hat die Unterwelt von Valhalla durch seine gut organisierten Untergebenen und sein weitreichendes Kontaktnetzwerk fest im Griff. Das Syndikat betreibt das lokale Fusionskraftwerk, das de facto Lebenserhaltungssystem der Stadt.

Ein Gro\3teil der Etablissements in Paradise City wird vom Luna-Syndikat betrieben. Nemessis hat mit Blackheart vereinbart, dass sich die Protektoratsstreitkr"afte nicht in seine Angelegenheiten einmischen. Im Gegenzug garantiert er den reibungslosen Betrieb der Bezirke au\3erhalb der Oberstadt von Valhalla (beschrieben in \cref{sec:valhalla}).

\newsection{Gangster des Luna-Syndikats}

Neben Nemessis und \xl{} treten folgende Gangster und Kontaktleute des Luna-Syndikats in Erscheinung:

\begin{description}
    \item [Henk Brothers] Henk Brothers ist ein Schieber, der vom Wirt des Batcave als Erstkontakt f"ur Cyberware genannt wird. Er ist im 
        Green Mile, einem Lokal im Bezirk Paradise City, anzutreffen.
    \item [Thomas Siebel] Thomas Siebel ist ein weiterer Schieber aus dem Bezirk R"otheim.
    \item [Quicksilver] Quicksilver ist die rechte Hand von \xl{}. Er tritt in der ersten Szene des Zusammentreffens mit \xl{} in 
        Erscheinung. Neben Nemessis wei\3 er, dass \xl{} eine ehemalige Piratin ist und ein eigenes Schiff besitzt.
    \item [Dr.~\pinyin{Li4} \pinyin{Li4}] Dr.~\pinyin{Li4} \pinyin{Li4} tritt beim ersten Eintreffen der Charaktere im Sunshine Hotel auf. 
        Er ist ein Stra\3endoc, ein Arzt der nicht registrierten Klinik \pinyin{Laohu3} Cyber Care, die die Kernmannschaft des Luna-Syndikats behandelt. Dr.~\pinyin{Li4} \pinyin{Li4} ist ein kleiner bereits angegrauter Chinese. Der Doktor ist ein Vertrauter von \xl{}, kennt ihre Familie und ist um das leibliche Wohl von \xl{} besorgt.
    \item [Roberto Martinez] Roberto Martinez ist ein Wartungstechniker, der die Gruppe beim Betreten der Wartungstunnel zur Zone 
        begleitet. Er assistiert \xl{} bei der Manipulation der Energieversorgung der Zone.
    \item [Stra\3engangster] Fu\3volk des Luna-Syndikats.  
\end{description}

\begin{column}[l]{0.45}
    \begin{nscsheet}[h]{Quicksilver}
        \nscstats[ATT=2,AGG=2,DEX=1,COM=2,CON=1]
        \nscruler
        \begin{nscinventory}
            \nscitem[Waffen] Bolzenpistole
            \nscitem[R"ustung] Schusssichere Weste
        \end{nscinventory}
    \end{nscsheet}
\end{column}
\begin{column}[r]{0.50}
    \begin{nscsheet}[h]{Strassengangster}
        \nscstats[ATT=2,AGG=2,DEX=1,CON=1]
        \nscruler
        \begin{nscinventory}
            \nscitem[Waffen] Railgun, Bolzenpistole
        \end{nscinventory}
    \end{nscsheet}
\end{column}

%% Copyright 2019 Bernd Haberstumpf
%% License: CC BY-NC
% !TeX spellcheck = de_DE
\newsection{Im Blackhole Club}\anchor{sec:blackholeclub}

Der Blackhole Club gew"ahrt nur Mitgliedern bzw. geladenen G"asten Einlass. Nur ausgew"ahlte Personen werden jemals den Club von innen kennenlernen. Der Club befindet sich etwas versteckt im Herzen von Paradise City. In einschl"agigen Kreisen steht der Club in dem Ruf, dass man "uber die Besucher an alles herankommt, au\3er an die hei\3en Girls, die im Club verkehren. Diese stehen bereits auf der Lohnliste des Clubs. Da der Club unter dem Schutz des Luna-Syndikats steht, ist ein Zugang in der Begleitung von \xl{} kein Problem. Vor dem durch den T"ursteher \emph{Steelhammer} bewachten Zugang steht eine wilde Mischung aus Gesch"aftsleuten und halbseidenen Gangstern in Anz"ugen. In einer zweiten Schlange, stehen ihre Begleiterinnern, die deutlich schneller Einlasse in den Club bekommen. Waffen m"ussen am Eingang abgegeben werden. \xl{}, in einer figurbetonten Gefechtsuniform, die ihren kr"aftigen Oberarmen Ausdruck verleiht, ist im Club bereits bekannt und setzt sich nach dem Betreten des Clubs zu einer Gruppe von offensichtlichen Verehrern.

Im zentralen Raum des Clubs, gleich am Eingang, dominiert eine nach unten versetzte, gro\3e, gut gef"ullte Tanzfl"ache. An den Ecken der Tanzfl"ache heizen sp"arlich bekleidete Go-go-Girls die Clubbesucher zu h"ammernden Industrial-Kl"angen ein. Gegen"uber dem Eingang f"ullt eine gewaltige Bar die ganze Breite des Raums. An der rechten Seite des Raums schlie\3t sich ein abgegrenzter Bereich mit S\'epar\'ees, aufgebaut wie ein Irrgarten, an. Ebenfalls auf der rechten Seite gelangt man in einen weiteren Raum mit einem gro\3en K"afig, in dem von Zeit zu Zeit "illegale" Zweik"ampfe stattfinden. Trotz des Verbots f"ur Soldaten des Protektorats in den "au\3eren Bezirken von Valhalla hat sich hier eine Schar von Omega-Kriegern, umringt von aufreizend gekleideten Groupies, einen Stammplatz reserviert. Der Boden, die W"ande und die Decke aller R"aume sind tief schwarz. Eine indirekte Beleuchtung, durchzuckt von Neonblitzen, gibt einen vagen Blick auf die G"aste frei. Die einzigen hell beleuchteten Teile des Clubs sind die Bar und die Kampfarena.

\newsubsection{Carina}

Wenn sich die Ermittler an die Bar setzen, hier ist die Musik lediglich als Hintergrundbeschallung wahrnehmbar, werden sie als Erstes vom Barmann \emph{Rosen} angesprochen, der ihre Getr"ankebestellung aufnimmt. Fast beil"aufig fragt er, wonach sie suchen. Sind es Better-than-Life-Holos, K"orpermodifikation oder Waffen? Oder haben sie vielleicht selbst etwas anzubieten? Eventuell kennt er ja einen Gast, der ihnen weiterhelfen kann. Wenn die Ermittler nach Slingshot fragen, l"asst sich der Barmann den Gesuchten genauer beschreiben, blickt dann kurz abwesend in die Menge, als h"atte er jemanden ersp"aht, sch"uttelt kurz den Kopf und wendet sich dann anderen G"asten zu. Einige Zeit sp"ater, bevorzugt wenn die Gruppe sich getrennt hat, setzt sich eine h"ubsche junge Frau wie zuf"allig neben die Gruppe an die Bar. Die elegante Frau in einem glitzernden Kleid blickt versonnen auf die Auslagen hinter dem Tresen. Ihre langen, blauen, aufwendig frisierten Haare glitzern im Licht des Raums. Nach einer Minute des Schweigens bittet sie den Ermittler, der neben ihr sitzt, mit einem Blick "uber ihre Schulter, ihr einen Drink zu spendieren. Kommt er ihrer Bitte nach, wendet sie sich wieder von ihm ab und fragt beil"aufig:

\speak{Ihr sucht nach einem Slingshot? Vielleicht hab ich von so jemandem schon etwas geh"ort. Was hat er denn angestellt?}

Wenn sie erf"ahrt, dass Slingshot get"otet wurde und dass etwas mit seiner Hardware im Kopf nicht in Ordnung war oder dass er an einem Attentat beteiligt war, reagiert sie betroffen. Schnell hat sie sich wieder unter Kontrolle. Sie reibt sich mit der Hand "uber den Mund und streicht eine Str"ahne, die sich gerade eben in ihr Gesicht verirrt hat, beiseite. Ihre Haarfarbe scheint kurzzeitig von Blau nach Giftgr"un zu wechseln. Sie "uberlegt. Dann hat sie eine Entscheidung getroffen. Sie bittet den Charakter, mit dem sie gesprochen hat, ihr zu folgen, um ungest"ort plaudern zu k"onnen.

Bei der Frau, die sich zu der Gruppe gesellt hat, handelt es sich um \emph{Carina}, die Frau, die Lenny Kilkenny als Begleiterin von Slingshot genannt hat. Sie ist auch die Vermittlerin, die Slingshot und Hannibal mit den USI-Agenten rund um Smith-Singer in Verbindung brachte.

Carina f"uhrt den Ermittler zielsicher durch die Menge in den Bereich mit den S\'epar\'ees, um in Ruhe sprechen zu k"onnen. Bevor der Ermittler jedoch Genaueres erz"ahlen kann, kommt ein anderer Gast und setzt sich unaufgefordert neben Carina an den Tisch. Er bestellt sich ein Getr"ank, flirtet mit der Kellnerin und gibt Carina zu verstehen, dass er mit ihr sprechen muss. Carina versucht w"ahrenddessen, das Gespr"ach mit dem Ermittler in belanglose Themen abdriften zu lassen. Wie zuf"allig schmiegt sie sich dabei n"aher an den Ermittler heran und legt ihm eine Hand auf den Oberschenkel. Dabei schiebt sie ihm eine kleine Karte zu. Als der andere Gast geht, verabschiedet sie sich ebenfalls und bedankt sich f"ur das nette Gespr"ach. Leider, so sagt sie, m"usse sie mit dem anderen noch etwas besprechen.

Bei dem anderen Gast, der seinen Namen nicht nennt, handelt es sich um den USI-Agenten \emph{Dan Ringdaz}. Er ist von Smith-Singer beauftragt, das Arbeitsverh"altnis mit Carina zu beenden. Er hat dabei auch Sorge daf"ur zu tragen, dass Carina keinerlei Informationen "uber ihren Auftrag an andere weitergibt. Dan Ringdaz, zusammen mit einem anderen Agenten, \emph{Frederic Johnson}, hatte "uber Carina den Kontakt zu Slingshot und Hannibal hergestellt. Beide arbeiten nur auf Zuruf f"ur Smith-Singer und kennen die eigentlichen Hintergr"unde der Operation P9 nicht.

\newsubsection{Die Visitekarte}

Eine Untersuchung der Visitenkarte sollte m"oglichst au\3erhalb des Blackhole Clubs erfolgen. Auf der Visitenkarte ist auf den ersten Blick nur ein holografisches Bild von Carina in lasziven Bewegungen zu sehen. Unterschrieben ist das Hologramm mit dem Namen Fleur Soleil. Wird das Bild l"anger mit einem Finger ber"uhrt, taucht eine ComLink-Nummer auf.

Wird eine Nachricht an die Nummer ohne das Unterdr"ucken der eigenen Nummer verschickt, kommt als R"uckantwort: "`Ice Club in zwei Stunden. Komm allein. "`Code Solar Eclipse."'.

\begin{remarks}
	\underline{Gewonnene Information:}
	
	\begin{itemize}
		\item Kontakt Carina, alias Fleur Soleil.
		\item Carinas Visitenkarte f"uhrt zu einem Treffen im Ice Club.
	\end{itemize}

	Der Blackhole Club ist eine gute Gelegenheit, ein bisschen freies Rollenspiel jenseits des Plots einzuflechten. Je nachdem, wie sich die Charaktere gegen"uber dem Barmann Rosen und gegen"uber anderen G"asten verhalten, kann die Kontaktaufnahme mit Carina v"ollig anders verlaufen als beschrieben. Der Club ist mit Gangstern, Schiebern, Hehlern und Konzernleuten gef"ullt, die nach illegalen Waren oder Dienstleistungen suchen. Dementsprechend vorsichtig sollte die Gruppe mit ihrem Wissen und ihren Fragen umgehen.

	Die Omega-Krieger, die sich in der N"ahe der Arena niedergelassen haben, erwarten, dass sich ein Omega, der den Club betritt, zu ihnen gesellt. Im Zweifel nimmt den entsprechenden Ermittler einer der Groupies an die Hand. Nicht alle Omegas im Club sind Soldaten des Protektorats. Ein gro\3er Teil der Anwesenden verdient sich als S"oldner oder als K"ampfer in der Arena. Die anwesenden Soldaten des Protektorats sind h"ochst inoffiziell im Club. "Uber ihren Clubbesuch darf au\3erhalb des Clubs nicht gesprochen werden.
	
	Carinas auff"allige Haarpracht ist so modifiziert, dass Carina je nach Laune die Farbe "andern kann.
\end{remarks}


\newsection{Bewegungen im Hintergrund}

\newsubsection[\xl{} verfolgt Carina]{Xiao Long verfolgt Carina}

Beim Verlassen des Blackhole Clubs trennt sich \xl{} von den Ermittlern und stellt ihnen Quicksilver f"ur eine sichere R"uckkehr in das Sunshine bereit. \xl{} verfolgt Carina, die sie bereits kennt, beim Verlassen des Clubs. Auf dem Heimweg versucht Dan Ringdaz zusammen mit zwei Stra\3enschl"agern, Carina in seine Gewalt zu bringen. \xl{} kommt ihr zu Hilfe und t"otet alle drei Angreifer. Danach fragt sie Carina aus und erf"ahrt dadurch, dass den Attent"atern in der Forschungseinrichtung \emph{Cyberbrain} die KIs eingesetzt wurden. Sie erf"ahrt auch, dass Prof.~Dr.~Sanders die medizinischen Eingriffe durchgef"uhrt hat. Da sie selbst keine eigene M"oglichkeit besitzt, Cyberbrain zu infiltrieren, schl"agt sie Carina vor, mit den Ermittlern ein Treffen zu vereinbaren, bei dem sie um Schutz durch das Luna-Syndikat bittet. Mit ihren F"ahigkeiten als Psychonaut l"oscht sie Carina das Treffen mit ihr und die Mitwirkung von Prof.~Dr.~Sanders bei Operation P9 aus dem Gehirn. Prof.~Dr.~Sanders will sie selbst vor den Charakteren verh"oren.

\newsubsection{Treffen mit Nemessis}

Beim Eintreffen im Sunshine Hotel werden die Charaktere direkt zu einem Treffen mit Nemessis in seine Suite im Hotel gebracht. Die Suite im Hotel ist eher minimalistisch gehalten. Unn"otige Einrichtungsgegenst"ande sind entfernt. Technische Ger"ate f"ullen den Raum. Nemessis er"offnet den Charakteren, dass Blackheart den Flottentr"ager Donar unter dem Kommando von Lord Commander Steeler mit Begleitschiffen in den Orbit "uber Valhalla entsandt und den Kreuzer Pendragon nach Armageddon beordert hat. Nemessis bef"urchtet, dass Blackheart eine Invasion auf Valhalla plant.

Versuchen die Charaktere, einen Kontakt mit ihren Befehlshabern herzustellen, kann das Luna-Syndikat aushelfen und eine Verbindung herstellen. Initiieren die Charaktere selbst keinen Kontakt, werden sie von Quicksilver dar"uber informiert, dass sowohl Cynarian als auch Blackheart die Charaktere kontaktieren m"ochten.

\newsubsection{Colonel Scholz}

Bevor die Charaktere sich auf den Weg zum Ice Club machen, wird der Cynarian-Ermittler von Colonel Scholz kontaktiert, falls er nicht selbst einen Bericht abliefert. Colonel Scholz informiert den Ermittler, dass Vandermool besorgt bez"uglich der Flottenaktivit"aten des Protektorats ist. Er teilt mit, dass am n"achsten Tag eine Delegation von Erde und Mars auf Valhalla erwartet wird. Diese wird von 
\emph{Luc Duval} f"ur die Europ"aische F"orderation und von \emph{Sarana} f"ur das \emph{Shigano-Kombinat} vertreten. Bei dem Besuch soll "uber die Aufnahme von Handelsbeziehungen gesprochen werden. Die Abgeordneten des Protektorats werden sicherlich auch die auf der Erde verbliebenen Mutanten thematisieren wollen. Dieses Treffen ist das erste seiner Art seit der Gr"undung des Protektorats und daher von gr"o\3ter Wichtigkeit. Ein milit"arisches Eingreifen des Protektorats k"onnte schwerwiegende Folgen haben, ebenso wie weitere Attentate. Colonel Scholz sch"atzt die Flottenbewegungen des Protektorats als fragw"urdig ein und fordert einen ausf"uhrlichen Bericht. Er zeigt sich "uberrascht "uber die Zusammenarbeit mit dem Luna-Syndikat und erkl"art, dass anscheinend eine Vereinbarung zwischen dem Protektorat und dem Luna-Syndikat "uber die Verwaltung von Valhalla besteht.

\newsubsection{Blackheart}

Versuchen die Ermittler, Artisan, Avenger oder Thunderbolt zu kontaktieren, schl"agt dies fehl. Alle drei sind bereits auf dem Weg nach Valhalla. Ein Kontakt mit Blackheart kommt jedoch zustande. Im Zweifelsfall kontaktiert sie den Ermittler des Protektoratsmilit"ars selbst. Auf der Videoaufnahme, die von schlechter Qualit"at ist, ist zu erkennen, dass sie in voller Kampfmontur auf dem Kommandosessel eines Kriegsschiffes sitzt. Sie wirkt aufgew"uhlt. Blackheart erwartet einen milit"arisch knappen Bericht "uber den Fortschritt und flucht, als sie erf"ahrt, dass noch keine wesentlichen neuen Erkenntnisse gewonnen wurden. Sie spricht den Ermittler auf das Gipfeltreffen an und zeigt sich erstaunt, dass die Ermittler noch nicht von Avenger informiert worden sind. Danach erkl"art sie dem Ermittler, dass zusammen mit der Delegation von Jupiter und Erde auch zwei weitere Kriegsschiffe kurz vor dem Eintritt in das jovianische System stehen.

\begin{speech}
	Bei den beiden Schiffen handelt es sich, wie uns die Triebwerkssignaturen verraten, um zwei Schlachtkreuzer der Guardian-Klasse. Die Aufbauten sind anders gestaltet als w"ahrend unserer K"ampfe um die Aegis-Station, aber die Plasmafackeln der Triebwerke sind unverkennbar. Ich hoffe, ihr wisst, was das bedeutet? Das Eintreffen von zwei dieser KI-Schiffe kommt einer Kriegserkl"arung gleich!

	Ich hoffe, ihr liefert bald, sehr bald Resultate.
\end{speech}

erkl"art Blackheart mit tonloser Stimme. 

\begin{speech}
	Kurz vor dem Eintreffen auf Kallisto hat sich einer der beiden Schlachtkreuzer von den Schiffen der Delegation getrennt. Sein neuer Zielort ist noch nicht klar auszumachen. Haltet euch bedeckt und kl"art verdammt nochmal auf, wie das alles mit den Attentaten zusammenpasst. Von Avenger besteht ein Verbot, einen Notstand auszurufen, um die Verhandlungen nicht zu gef"ahrden. Egal wie Avenger das sieht, gilt h"ochste Alarmbereitschaft bei den Streitkr"aften und wir sind auf einen Krieg vorbereitet. Das ist die Ruhe vor dem Sturm. Haltet euch vom Raumhafen und der Garnison fern, sonst bekommt euch noch die Konzernpolizei in die Finger. Gebt mir sofort einen Bericht, wenn es etwas Neues gibt! Over.
\end{speech}

\newsubsection{Professor Sanders Ende}

W"ahrend die Charaktere von der neuen Flut an Informationen "uberrollt werden, macht sich \xl{} auf den Weg zu Prof.~Dr.~	s in seinem privaten Domizil. Sie "uberf"allt sein Anwesen, schl"agt ihn nieder und verbindet sich mit seiner Datenbuchse. Dabei erf"ahrt sie den Standort der Cyberbrain-Einrichtung in der Zone, die Eingriffe im Rondra-Hospital und von \ml{}, die die KI in ihrem Kopf entwickelt hat. Bei dem durchgef"uhrten Tiefenscan stirbt Prof.~Dr.~Sanders. "Uber die aktuellen Entwicklungen im Orbit und das politische Treffen mit Erde und Mars erf"ahrt sie nach ihrer R"uckkehr.

\begin{remarks}
	\underline{Gewonnene Information:}
	
	\begin{itemize}
		\item Politisches Treffen mit der Europ"aischen F"orderation und dem Shigano-Kombinat am "ubern"achsten Tag.
		\item Eintreffen von KI-gesteuerten Guardian-Schlachtkreuzern.
	\end{itemize}

	Die Vertreter der Europ"aischen F"orderation und des Shigano-Kombinats werden im \cref{sec:delegates} beschrieben.
	Das Cyberbrain-Institut ist Thema im im \cref{sec:cyberbrain}. Die Ermittler erfahren von dem Institut im folgenden 
	\cref{sec:ice_club}.

	\ml{} ist eine Mitarbeiterin der Neuro Intelligence. Sie hat alle KIs auf humanoide K"orper angepasst. \ml{} wird im
	\cref{sec:mailin} n"aher beschrieben.
\end{remarks}

\newsection{Im Ice Club}

Der Ice Club ist ein Nachtclub und Bordell das neben dem Blackhole Club ebenfalls durch das Luna--Syndikat kontrolliert aber weitgehende Autonomit"at besitzt. Das Bordell geh"ort einer Sonja Ice. Durch die Einladung von Fleur Soleil wird dem Ermittler der die Visitenkarte erhalten hat Einlass gew"ahrt. Weitere Ermittler k"onnen wenn sie wollen "uber \xl{} Zugang zum Nachtclub erhalten. Im Club sind keine Waffen erlaubt. F"ur den Eintritt ist Abendgarderobe erforderlich. Der Nachclub kann diese aber auch bereit stellen. Der Eingangsbereich besitzt Zug"ange in eine ausgedehnte Garderobe, den Clubraum und in obere R"aumlichkeiten. Die W"ande des Clubraums bestehen aus konserviertem Eis das reich dekoriert in einer geschwungenen Decke endet. Gegen"uber der B"uhne befindet sich die Bar. Davor sind kleine Tische mit St"uhlen um einen Catwalk aufgereiht. Im hinteren Bereich befinden sich Separees. Versteckt, kunstvoll in das Eis eingelassen f"uhrt eine Treppe nach oben zu einzelnen Zimmern und nach unten in einen Saunabereich.

Wenn die Charaktere das Bordell betreten steht Carina als Fleur Soleil gerade auf der B"uhne und erfreut die G"aste mit ihrem Gesang. Bei ihrem Auftritt ist sie in ein hautenges wei\3es Kleid das au\3er den Pailettenbestickungen nahezu durchsichtig ist geh"ullt. Zu dem Kleid tr"agt sie platinblonde Haare ein eingewobene leuchtenden Kristallen. Die Charaktere haben w"ahrend der Darbietung gen"ugend Zeit den Clubraum zu inspizieren und ihr weiteres Vorgehen zu beraten.

Carina fordert durch eine versteckte Geste den Charakter mit dem sie im Blackhole Club gesprochen hat auf ihr zu ihrem Zimmer zu begleiten. Andere G"aste vertr"ostet sie auf sp"ater oder ein anderes mal. Von Carina erfahren die Ermittler, dass sie von zwei M"annern beauftragt wurde nach Interessenten f"ur Cyberware Ausschau zu halten. Einem dieser M"anner ist der Charakter bereits im Blackhole begegnet als er sich an ihren Tisch gesetzt hat. Bevor sie weitere Informationen offen legt erkl"art sie dem Charakter, dass sie von den M"anner verfolgt und angegriffen wurde und deshalb um ihr Leben f"urchten mu\3. Seid ihrem Treffen im Blackhole Club verschanzt sie sich im Ice Club. Aus diesem Grund bittet sie den Charakter sie umgehend sicher in die Zentrale des Luna--Syndikats zu bringen. Sagt der Charakter ihr zu zieht sie sich im bei sein des Charakters Stra\3enkleidung an und macht sich ausgehfertig. Weitere Informationen will sie erst preisgeben wenn sie sicher in den R"aumen des Syndikats sicher f"uhlt.

Beim Verlassen des Clubs wird Carina in Begleitung der Gruppe angegriffen. Haben sie Geleitschutz durch Mitglieder des Syndikats wird ihr Transportfahrzeug nahe des Clubs von einer Granate getroffen. Die Angreifer werden angef"uhrt von \emph{Frederic Johnson}. Frederic Johnson in Begleitung von \emph{Lazor} und \emph{Flinn} werden von zwei weiteren Schl"agern unterst"utzt. Die Angreifer sind mit Multiguns bewaffnet. \xl{} wird die Ermittler unterst"utzen die Angreifer abzuwehren.

Wurde Carina im Sunshine Hotel in Sicherheit gebracht erf"ahrt man von ihr, dass sie den Mann der sie im Blackhole Club beim Beisein der Charaketere angesprochen hat unter dem Namen Dan Ringdaz kennt. Sein Partner war einer der Leute (Frederic Johnson) der sie beim Verlassen des Ice Clubs angegriffen hatte. Im Blackhole Club hat sie Dan Ringdaz und seinem Partner die Kontakte zu Hanibal und Slingshot vermittelt. Bei einem der Treffen hat sie zuf"allig \emph{Cyberbrain} als den Namen der Forschungseinrichtung, in dem Slingshot und Hanibal mit neuer Cyberware ausgestattet werden sollten, aufgeschnappt. Nach den Eingriffen hat sie Hanibal und Slingshot nicht mehr gesehen.

Kann Frederic Johnson gefangen gesetzt werden l"asst sich in Erfahrung bringen, dass er im Auftrag eines Dritten dessen Namen er aber nicht preisgeben will, arbeitet. Zusammen mit Dan Ringdaz hatte er den Auftrag Interessenten f"ur Cyberware anzusprechen, Konditionen zu kl"aren und ein Treffen am Raumhafen zu veranlassen wo sie dann von ihrem Auftraggeber und anderen Personen abgeholt wurden. Wird Frederic Johnson einem Gehirnscan unterzogen wird er versuchen den eingreifenden Psychonauten mit seinen eigenen phychonautischen F"ahigkeiten mit falschen Erinnerungen abzuwehren ohne seine F"ahigkeiten aufzudecken. Gelingt dem Spieler der Psychonautische Angriff kann er den Namen Smith--Singer in Erfahrung bringen. Er erf"ahrt auch, dass die Gehirnmanipulation in der Cyberbrain Einrichtung in der Zone durchgef"uhrt wurde. Auch l"asst sich in Erfahrung bringen, dass die Agenten im Auftrag der USI arbeiten.

Wo die Cyberbrain Forschungseinrichtung zu finden ist l"asst sich durch eine Anfrage bei Cynarian herausfinden. Cyberbrain geh"ort zu einer Reihe von kleinen Forschungseinrichtungen in der \emph{Zone} deren Aufgaben unter Verschlu\3 stehen und auch Cynarian nicht bekannt sind.

\begin{remarks}
	Gewonnene Informationen: Forschungseinrichtung Cyberbrain. Dan Ringdaz. ggf.~Frederic Johnson, Smith--Singer und die Namen anderer USI Agenten.
\end{remarks}

%% Copyright 2019 Bernd Haberstumpf
%% License: CC BY-NC
% !TeX spellcheck = de_DE
\newcommand{\ml}{\pinyin{Mailin2}}
\newsection{Cyberbrain Infiltration}

Cyberbrain ist eine kleine Forschungseinrichtung in der Zone. Der Forschungsschwerpunkt ist unbekannt. Betrieben wird Cyberbrain von Synthology Inc.~die wiederum chirurgische Instrumente im Bereich Transplantationschirurgie auf dem Mars entwickelt. Die entsprechenden Informationen k"onnen durch das B"uro von Vandermool bereit gestellt werden. Folgende Informationen stehen nicht zur Verf"ugung: Synthology ist ein Strohfirma der USI was aber nicht nachverfolgbar ist. Cyberbrain ist der lange Arm der Operation P9 auf Kallisto.

\subsection{Die Zone}
Die Zone ist ein gut gesicherter Bereich auf Kallisto in N"ahe von Valhalla in das Eis eingegraben. Die Zone beherbergt die Unternehmungen der Konzerne mit hoher Sicherheitseinstufung oder inoffizielle Einrichtungen der Firmen. Die Zone ist auf zwei Ebenen aufgeteilt. Auf beiden Ebenen verlaufen G"ange zwischen den Geb"auden der Konzerne. Unterhalb der Wege auf der oberen Ebene verlaufen Wartungsg"ange f"ur die technische Anbindung der Geb"aude in der Zone. Der regul"are Zugang zur Zone bildet ein an die Zone angebundener eigener kleiner Raumhafen der von Shuttles und Buggies angeflogen bzw.~angefahren werden kann. Der technische Betrieb und die Wartung der Zone wird durch das Unternehmen Dockbunner betrieben. Das Sicherheitspersonal stellt die Firma TransSec. Ein weiterer wenig bekannter Zugang sind die Tunnel über die die Zone aus dem Kraftwerk in Valhalla mit Strom versorgt wird. Diese Tunnel enden in den Wartungsg"angen zwischen den beiden Ebenen.

\subsection{Das Ziel} 
In der aktuellen Situation ist Eile geboten. Ein Aufdecken der Attent"ater und der Hintergr"unde mu\3 vor dem Eintreffen der Delegationen von Erde und Mars am n"achsten Tag erfolgen. Vandermool gibt als oberste Direktive heraus Beweise zu sammeln, dass die Attentate von einer au\3enstehenden Organisation in die Wege geleitet wurden. Desweiteren sind alle Attent"ater zu identifizieren und die Technologien der Cyberbrain sicher zu stellen. Kampfhandlungen in der Zone m"ussen soweit irgend m"oglich vermieden werden. F"ur Blackheart ist die Identifizierung und Eliminierung weiterer Attent"ater die h"ochste Priorit"at. Alle Mittel dazu sind legitimiert. 

\subsection{Der Zugang} 
Bei der Infiltration der Cyberbrain Forschungseinrichtung sollte der Spielleiter den Spielern Handlungsfreiheit einr"aumen. Er kann aber gleichzeitig den Spieler soweit n"otig auch unter die Arme greifen. Die Zone kann offiziell "uber den Raumhafen betreten werden. Eine weitere Zugangsm"oglichkeit bieten die Wartungstunnel der Energieversorgung der Zone.

\subsection{Der Raumhafen} 
Beim Zugang "uber den Raumhafen k"onnten die Charaktere mit Unterst"utzung durch Cynarian oder das Luna--Syndikat Mitarbeiter von TransSec oder Dockbunner ersetzen oder eine Warenlieferung vort"auschen. Allerdings bleibt f"ur die Vorbereitung nicht viel Zeit. Die T"auschung ist damit nicht l"uckenlos und damit Risikoreich. Omega Krieger oder schwere Waffen und R"ustungen m"ussen sehr gut argumentiert werden. Um eine R"uckverfolgung der Operation auf Cynarian zu verhindern wird Cynarian keine offizielle Genehmigung zum Betreten der Zone geben. Betreten der Zone "uber den Raumhafen wird in jedem Fall Smith--Singer auf den Plan rufen. Die Charaktere sind Smith--Singer bekannt und er wird die Sicherheitskr"afte "uber einen m"oglichen Angriff auf Konzerneigentum informieren. 

\subsection{Der Wartungstunnel} 
Das Luna--Syndikat betreibt das Energie Kraftwerk auf Valhalla und damit auch die Stromversorgung der Zone. Die Wartungstunnel zur Zone obliegt deshalb ebenfalls dem Luna--Syndikat. Wird \xl{} oder Nemessis in die Planung der Infiltration mit einbezogen werden sie den Charakteren diese Zugangsm"oglichkeit er"offnen. Nemessis wird auch anbieten auf Anfrage einen Stromausfall f"ur kurze Zeit hervor zu rufen. Als F"uhrer durch die Tunnel bis zur Zone wird sich \xl{} bereit erkl"aren und mit dem Techniker \emph{Roberto Martinez} die Charaktere begleiten.

\subsection{Kommunikation} 
Aufgrund der hohen Sicherheitsanforderungen ist die Kommunikation von der Zone nach au\3en abgeschottet. Die Zone verf"ugt "uber ein eigenes ComNetz das "uber den Raumhafen und dort "uber Firewalls von weiteren Netzen au\3erhalb der Zone abgekoppelt ist. Ein Zugang zum ComNetz der Zone ist nur f"ur Mitarbeiter der Unternehmen in der Zone und f"ur offizielle G"aste m"oglich. Einzelne Unternehmen betreiben eigene VPNs zu ihren Niederlassungen au\3erhalb der Zone. Neben dem regul"aren ComNetz innerhalb der Zone verf"ugt TransSec "uber ein eigenes Sicherheitsnetz das die Zone durchzieht. Die Backbone Infrastruktur des ComNetz befindet sich in den Wartungsg"angen zwischen den beiden Ebenen. Ein physikalischer Zugriff ist dort möglich und vereinfacht einen Hacker Angriff. Neben dem ComNetz besitzt die Stromversorgung selbst ein primitives Schmalbandiges Kommunikationsnetz das f"ur die Kommunikation mit dem Luna--Syndikat genutzt werden kann. Ein Zugriff ist "uber alle Stromanschl"usse m"oglich.

\subsection{Lageplan} 
Die Zone ist zur Orientierung in farbig gekennzeichnete Bereiche aufgeteilt. Die Gangsegmente zwischen den Geb"auden sind zudem horizontal und vertikal nummeriert wodurch jedes Gangsegment durch einen Barcode und zwei Ziffern eindeutig zugeordnet werden kann. Einen Lageplan der Zone auf dem viele der Unternehmen namentlich benannt sind kann "uber Cynarian erhalten werden. Die Gangsegmente sind durch Sicherheitschotts voneinander getrennt und durch Kameras und Drohnen Video"uberwacht. Zugang zu den Wartungstunneln zwischen den Ebenen ist "uber Magschloss gesicherte Luken aus der oberen Ebene m"oglich. Die G"ange der Zone werden von Mitarbeitern der TransSec patroulliert. Da die Geb"aude der Unternehmen autonom ausgelegt sind und es keine geteilten R"aumlichkeiten gibt bewegen sich nur selten Personen durch die G"ange.

\subsection{Unterst"utzung} 
F"ur den Eingreiftrupp in die Zone k"onnen die Charaktere auf inoffizielle S"oldner bereit gestellt durch Cynarian oder auf Soldaten des Protektorats zur"uckgreifen. \xl{} oder andere Mitglieder des Luna--Syndikats werden die Charaktere nicht begleiten. S"oldner werden wenn nicht anders durch die Charaktere ausgelebt den Cynarian Chefermittler als Befehlshaber wahrnehmen sich aber an die Befehlskette der Charaktere halt. Soldaten des Protektorats werden nur den von Blackheart eingesetzten Ermittler als Befehlshaber und wenn dieser nicht verf"ugbar den Chefermittler des Protektorats als Anf"uhrer anerkennen. Die anderen beiden sind in erster Linie zu sch"utzende Zivilisten.

\subsection{\xl} 
Unabh"angig von den Pl"anen der Charaktere wird \xl{} die Wartungsg"ange getrennt von den Charakteren betreten und sich nahe des Cyberbrain Geb"audes in das Sicherheitsnetz der Zone einklinken. Durch ihre "uberlegenen F"ahigkeiten als KI kann sie die Sicherheitssysteme der Netze "uberwinden, die Charaktere "uberwachen und ggf.~eingreifen. "Uber das ComNetz kann sie auch einen St"orfall ausl"osen. Sie ist mit einer leichten Version eines Kampfpanzeranzugs mit eingebauter Railgun ger"ustet. Sie tr"agt eine schweren vollautomatischen Multigun, Schock-- und Splitter Haftgranaten. F"ur einen schnellen Zugriff steht ihr ein Plasmaschwei\3brenner zur Verf"ugung. Mit den Charakteren wird sie soweit m"oglich "uber das Netz der Stromversorgung kommunizieren um sich nicht weiter zu enttarnen.

\subsection{Die Einrichtung} 
Cyberbrain ist eine kleine Einrichtung auf der oberen Ebene nahe des Raumhafens. Cyberbrain umfasst zwei Stockwerke. Im unteren Bereich befinden sich der Eingangsbereich, ein zentraler gro\3er Forschungslabor, mit angebundenem Aufenthaltsraum, der OP, das Lager-- und Technikr"aume. Im oberen Bereich befinden sich B"uros, ein Bad und ein Schlafbereich mit Stockbetten. Cyberbrain verf"ugt "uber drei Zug"ange: Ein Zugang f"uhrt in das Foyer. Das Foyer besitzt eine Fiberglast"ur die bei Nichtbenutzung undurchsichtig geschaltet ist. Die Lagerhalle und das etwas kleinere Lager der Krankenstation besitzen jeweils ein eigenes Rolltor. Zum Operationsraum f"uhrt eine Schleuse.

\begin{figure*}[htbp]
	\centering
    \includegraphics[width=0.85\linewidth]{./images/cmyk/cyberbrain_cmyk.jpg}
    \newline{}Cyberbrain
	\label{fig:cyberbrain}
\end{figure*}

\subsection{Eindringen in Cyberbrain} 
Um die Einrichtung zu betreten k"onnen die Charaktere versuchen die Mitarbeiter mit einer List zum Einlass zu bewegen. Klappt das nicht k"onnen die Schl"osser auch mit einem Magschlossknacker oder mir roher Gewalt ge"offnet werden. Alternativ kann die Eingreifgruppe versuchen sich "uber den Wartungsg"angen unterhalb der Einrichtung Eintritt zu verschaffen.

\subsection{Die Mitarbeiter} 
Zum Zeitpunkt des Angriffs befinden sich in der Einrichtung \emph{Dr.~Dan Leitner}, \emph{\ml{}}, \emph{Gaius Ross} und \emph{Francis McDonald} die ungeplant noch einmal zur"uck gekommen sind um zur"uck gelassene Ger"atschaften von Neuro Intelligence einzusammeln. Beim Eintreffen der Infiltrationsgruppe sind sie gerade dabei die letzten Ger"atschaften zusammen zu tragen. Die Daten der Operation P9 sind zu diesem Zeitpunkt bereits gel"oscht und nur noch in K"opfen der Mitarbeiter vorhanden. Da die Einreichtung aus mehreren R"aumen besteht wird sich die Gruppe bei der Durchsuchung vermutlich aufteilen. Dan Leitner in Cleanroom Kleidung befindet sich im Cleanroom im Zentrum des Hauptraums. Francis McDonald befindet sich in einem B"uro im ersten Stock. Gaius Ross arbeitet im Lagerraum der Krankenstation. \ml{} befindet sich Serverraum und versteckt sich wenn sie Ger"ausche h"ort zwischen den Serverkabinetten.

\subsection{Sammeln von Informationen} 
Werden die Mitarbeiter von den Mitgliedern des Infiltrationsteams befragt werden sie erkl"aren, dass 
sie nur der letzte Aufr"aumtrupp sind und alle Informationen zum Unternehmen bereits nicht mehr in der Forschungseinrichtung zu finden sind. Fast alle Gegenst"ande wurden ebenfalls bereits abtransportiert. Die Einrichtung wird gerade aufgegeben da die Forschungen abgeschlossen sind. In diesem Zusammenhang wird sich heraus stellen, dass die angetroffenen Mitarbeiter nicht zu Cyberbrain geh"oren sondern Mitarbeiter von Neuro Intelligence sind. Die Mitarbeiter werden versuchen sich nur als Zulieferer darzustellen die nicht direkt in die Arbeiten von Cyberbrain eingebunden waren. Die Ermittler k"onnen jetzt versuchen die Mitglieder der Forschungseinrichtung einzusch"uchtern und zu verh"oren. Ein Psychonaut kann dabei gute Dienste leisten. Alternativ zu einem Verh"or vor Ort k"onnen die Ermittler versuchen einen oder alle Mitarbeiter aus der Zone zu schaffen. 

Folgendes Informationen k"onnen "uber die Mitarbeiter der Einrichtung in Erfahrung gebracht werden:

\begin{description}
	\item[Dan Leitner, \ml{}] Alle Attent"ater erhielten eine Neuronalkopplung von Neuro Intelligence.
	\item[Dan Leitner, Dr.~Gaius Ross, \ml{}] Alle Eingriffe wurden von Prof.~Dr.~Sanders durchgef"uhrt.
	\item[Dr.~Gaius Ross, Francis McDonald, \ml{}] Die Attent"ater Hanibal und Slingshot wurden als erste Probanden in der Cyberbrain 			Forschungseinrichtung selbst behandelt.	
	\item[Dr.~Dan Leitner, Dr.~Gaius Ross, \ml{}] Es erfolgten weitere Eingriffe. Allerdings wurden alle folgenden Eingriffe direkt im 			Rondra Hospital von Prof.~Dr.~Sanders durchgef"uhrt. Die Namen der KI infizierten Personen sind nicht bekannt.
\end{description}

\subsection{Die Absicherung} 
W"ahrend der Befragung werden die zur Unterst"utzung bereitgestellten Omegas oder S"oldner die Anlage absichern, und die Eing"ange verminen, dies aber mit dem Charakter absprechen der f"ur sie als Vorgesetzter gilt. Ein mit einer KI manipulierter Omega wird die Mine an der T"ur f"ur die er verantwortlich ist nicht aktivieren. Stattdessen hinterl"asst er an mehreren Stellen im Geb"aude fernz"undbare Granaten.

\subsection{Xiao Longs "Uberwachung}  
Da \xl{} mit dem Aufgreifen von \ml{} rechnet wird sie die Einrichtung Video"uberwachen um jederzeit eingreifen zu k"onnen. Wenn n"otig wird sie die Forschungsstation "uber die Wartungsg"ange mittels des Plasmabrenners betreten.

\subsection{Es wird heiss} 
W"ahrend die Charaktere versuchen Informationen zu sammel spitzt sich die Lage zu. Wichtig ist hier den Spannungsbogen aufrecht zu erhalten und den Spieler nicht unn"otig Zeit zu geben zu "uberlegen wie sie an Informationen "uber Neuro Intelligence Mitarbeiter heran kommen. Im Zweifelsfall m"ussen die Charaktere versuchen die Mitarbeiter aus der Anlage f"ur weitere Befragungen zu bringen. Die nun folgenden Vorkommnisse bringen die Charaktere in eine nahezu aussichtslose Situation was der Spielleiter auch um die Dramatik zu steigern deutlich machen sollte.

\subsection{Smith-Singer} 
Smith--Singer der durch die Vorkommnisse im Blackhole und im Ice Club informiert ist wartet auf den n"achsten Schritt der Investigatoren. Haben die Infiltratoren die Einrichtung "uber den Raumhafen der Zone betreten sind die Konzerngardisten von Smith--Singer direkt informiert, dass der Verdacht besteht Cyberbrain k"onnte angegriffen werden. Hat das Luna--Syndikat einen tempor"aren Ausfall der Stromversorgung erzeugt versuchen die Sicherheitskr"afte zun"achst den Ursprung und die Auswirkungen festzustellen. Den Charakteren bleibt dann mehr Zeit. Fr"uher oder sp"ater werden die Konzerngardisten versuchen in die Cyberbrain Einrichtung einzudringen. Smith--Singer wird darauf dr"angen die Situation gewaltsam zu kl"aren unter anderem um dem im Orbit wartenden Schlachtkreuzer einen Grund zu geben Bodentruppen zu senden und die Lage mit dem Protektorat und Cynarian auf der einen Seite und dem Konzernrat mit seinen Kriegsschiffen auf der anderen Seite eskalieren zu lassen. 

\subsection{Eintreffen der Konzerngardisten} 
Die Konzerngardisten treffen zum Zeitpunkt fr"uhestens beim ersten Zusammentreffen der Charaktere mit den Neuro Intelligence Mitarbeiter ein und umstellen das Geb"aude. Haben die Charaktere oder die sie unterst"utzenden Omegas oder S"oldner eine "Uberwachungsdronen in den G"angen zur"uck gelassen werden sie bemerken wenn die Konzerngardisten angef"uhrt von Smith--Singer das Geb"aude umstellen. Alternativ kann \xl{} den Infiltrationstrupp "uber das Netz der Energieversorgung informieren, dass sie Besuch bekommen. Die Gardisten fordert die Infiltrationsgruppe auf das Geb"aude zu verlassen und sich zu ergeben. 

\subsection{Der Attent"ater} 
Werden die Charaktere von einem Omega Trupp begleitet tritt Thunder der Attent"ater aktiviert durch 
Smith--Singer in Aktion. Er z"undet die im Geb"aude von ihm verteilten Granaten und feuert sofort mit seiner Multigun als Prim"arziel auf Stormball um diesen auszuschalten. Die ersten Angriffe von Thunder sollten sich auf die eigene Omega Verst"arkung konzentrieren und diese
unsch"adlich machen damit die Charaktere wieder allein auf sich gestellt sind. Besteht die M"oglichkeit wird Thunder Neuro Intelligence Mitarbeiter bis auf \ml{} ausschalten.

\subsection{Das Geb"aude wird gest"urmt} 
Im Falle des Angriffs durch den Omega wird dem Sicherheittrupp von Smith--Singer befohlen das Geb"aude sofort zu st"urmen. Ist kein Attent"ater im Geb"aude kann der Spielleiter zun"achst eine Verhandlung mit den Geiselnehmern im Geb"aude initiieren. In jedem Fall werden die Gardisten fr"uher oder sp"ater beginnen die T"uren zu entriegeln. Die Sprengfallen an den T"uren schalten einen Teil des Sicherheitstrupps aus. Der Sicherheitstrupp der versucht "uber ein nicht durch eine Falle gesichertes Tor einzudringen wird nach dem sie das Geb"aude betreten haben im Zweifelsfalle von \xl{} von hinten ausgeschaltet. \xl{} wird nur auf den Plan treten wenn keiner der Charaktere sie sehen kann. Treffen die Charaktere auf die Leichen kann das nat"urlich Fragen aufwerfen. Um in das Kampfgeschehen eine interaktive pers"onlichere Komponente zu bringen k"onnen die Gardisten nach dem ersten misslungenen Angriffsversuch versuchen mit der Infiltrationsgruppe zu verhandeln und sie aufzufordern sich zu ergeben und das Geb"aude zu verlassen. Die Gardisten werden dabei die Zeit nutzen sich neu zu formieren.

Gehen die Infiltratoren nicht zum Gegenangriff "uber oder finden eine M"oglichkeit in die Wartungstunnel zu gelangen werden die Gardisten versuchen das Geb"aude erneut zu st"urmen. Sie setzen Schock und Rauchgranaten ein.

\subsection{Die Flucht} 
Die Charaktere haben mehrere M"oglichkeiten zu fliehen:

\begin{description}
	\item [Durch den Boden] Die Charaktere k"onnen versuchen sich im Geb"aude zu verschanzen und sich durch den Boden in die 		
		Wartungssch"achte zu brennen oder zu sprengen.
	\item [Nebengeb"aude] "uber den Bereich hinter dem Foyer kann die Infiltrationsgruppe versuchen sich durch die Wand in das 		
		Nebengeb"aude zu sprengen oder zu schwei\3en.
	\item [Im ersten Stock] Haben sich die Charaktere im ersten Stock verschanzt k"onnen sie es durch das hintere Tor versuchen, 
		  sich 	durch die Wand nach au\3en zu brennen oder ebenfalls in das Nebengeb"aude auszuweichen.
	\item [Der Hinterausgang] Die nicht beim ersten Eintreffen get"oteten Gardisten werden zuerst versuchen das Foyer zu besetzen. Im hinteren Bereich sind zu diesem Zeitpunkt alle Angreifer tot.	
\end{description}

Die Gardisten werden versuchen den Infiltrationstrupp weiter zu verfolgenden auch wenn sie sich bereits in die Wartungssch"achte zur"uck gezogen haben. Sie werden dazu versuchen die Gruppe "uber die Wege zwischen den Geb"auden auf der oberen Ebene zu verfolgen. Die Gardisten werden die Charaktere nicht nach au\3erhalb der Zone verfolgen.

\subsection{Das Nebengeb"aude Stimulus Fungi} 
Auf beiden nicht durch T"uren zug"anglichen Seiten befindet sich jeweils ein weiteres Geb"aude. Der einfachheit halber wird hier nur ein weiteres Geb"aude beschrieben. Bei dem anliegenden Geb"aude handelt es sich um das biologische Labor der Stimulus Fungi Forschungseinrichtung. Das Geb"aude ist nach dem Stromausfall leer. Die Mitarbeiter haben sich in Sicherheit gebracht und konnten durch den Einsatz der Sicherheitskr"afte auch nicht mehr in ihr B"uro zur"uck. 

Das Geb"aude erstreckt sich mit einer gro\3en Halle "uber beide Ebenen der Zone und ist etwa drei mal so gro\3 wie die Cyberbrain Forschungseinrichtung. Das Geb"aude ist als gro\3es Gew"achshaus ausgelegt. Unter fokussierter Neonbeleuchtung befinden sich auf vier Etagen ein mit G"angen durchzogener Irrgarten aus Pflanzenanlagen. An verschiedenen Stellen sind die Pflanzenbeete durch Arbeitstische durchbrochen. In der unteren Ebene wird etwa die H"alfte der Ebene durch einen Wohnkomplex ähnlich dem oberen Stockwerk der Cyberbrain belegt. Auf der oberen Ebene befindet sich in etwa in der Mitte des Geb"audes ein Chemielabor. Das Geb"aude kann auf beiden Ebenen auf den von Cyberbrain gesehenen hinteren Seite "uber ein als Schl"au\3e ausgelegtes Rolltor betreten werden. Im vorderen Bereich befinden sich die Eing"ange der Mitarbeiter ebenfalls als Schl"au\3e ausgelegt. An den dreckig wei\3en W"anden der Einrichtung prangt wo nicht durch Pflanzen bedeckt das Firmenlogo.

\subsection{\ml{}} 
Wird \ml{} nicht selbst durch die Charaktere aus dem Geb"aude geschafft wird \xl{} sie aus dem Geb"aude schaffen. \ml{} wird in jedem Fall die Flucht aus dem Geb"aude gelingen.

\begin{remarks}
	Ausr"ustung: Neben Feuerwaffen und R"ustung k"onnen die Charaktere wenn gew"unscht auf Granaten, ferngesteuerte Sprengstoffe, Schneidbrenner, "Uberwachungsdronen, Magschlo\3knacker, EMP Granaten zur"uckgreifen die ihnen das Protektoratsmilit"ar zur Verf"ugung stellen kann sofern sie eine M"oglichkeit haben sie in die Zone zu bringen. Der Spielleiter kann in diesem Zusammenhang in soweit freiz"ugig sein solange die Spieler plausibel erkl"aren k"onnen wie sie die Ausr"ustung transportieren. Munition und vor allem Granaten sollten in ihrer Zahl begrenzt sein. Die Waffen der milit"arischen Begleiter sind jeweils auf deren Besitzer kodiert und k"onnen von
	niemand anders genutzt werden.

	Gewonnene Informationen:  Es wurden weiteren Personen KIs im Rondra Hospital eingesetzt. Die Nanobots auch Neurokopplungen genannt entstammen dem Unternehmen Neuro Intelligence. 
\end{remarks}

%% Copyright 2019 Bernd Haberstumpf
%% License: CC BY-NC
% !TeX spellcheck = de_DE
\newsection{Zeit der Entscheidung}

\newsubsection{Die Medien} 
Als die Charaktere die Zone verlassen, wird gerade "uber die Medien auf Valhalla eine Eilmeldung verbreitet:

\begin{speech}
Aus Sicherheitskreisen wurde bekannt gegeben, dass vor etwa einer Stunde ein paramilit"arischer Angriff auf eine Konzerneinrichtung in der gesicherten Zone stattgefunden hat. Bei dem als terroristischen Anschlag eingestuften Vorfall kamen mehrere Zivilisten und zahlreiche Sicherheitskr"afte ums Leben oder wurden schwer verletzt. Es wird vermutet, dass bei der militanten Aktion Beweismaterial in Zusammenhang mit der k"urzlich stattgefundenen Attentatsserie vernichtet werden sollte. Unbest"atigten Quellen zufolge sollen Angeh"orige der Protektoratsstreitkr"afte am Angriff beteiligt gewesen sein. Nach den Angreifern wird bereits gefahndet. Eine Stellungnahme der Cynarian Corporation, die in der Vergangenheit Ziel von Attentaten war, steht noch aus, wird jedoch in K"urze erwartet.

\nopagebreak
Aufgrund der unklaren Sicherheitslage und des unerwarteten Eintreffens eines Flottenverbands des Protektoratsmilit"ars vor den Toren von Valhalla, w"ahrend hohe W"urdentr"ager von Erde und Mars zu einer Konferenz erwartet werden, wurden Sicherheitskr"afte vom im Orbit befindlichen Kreuzer Zeus II-1 entsandt, um die Gebiete der Oberstadt auf Valhalla zu sichern. Den Anweisungen dieser Sicherheitskr"afte ist unbedingt Folge zu leisten.

\nopagebreak
Wir halten Sie, liebe Zuschauer, nat"urlich weiterhin auf dem neuesten Stand der Entwicklungen.

\nopagebreak
- Ihre Conni Hanseln
\end{speech}

\newsubsection{Nemessis} 
Die Eilmeldung macht unmissverst"andlich klar, dass die Charaktere vorerst untertauchen m"ussen. Der sicherste sofort verf"ugbare Unterschlupf ist derzeit das Luna-Syndikat. \xl{}, die sie bereits begleitet oder nach der Flucht zur Gruppe st"o\3t, f"uhrt sie "uber sichere Wege zum Fusionskraftwerk im Bezirk Breidablik.

Sobald die Charaktere in die Obhut des Luna-Syndikats gelangt sind, werden die Omega-Soldaten oder S"oldner der Cynarian Corporation, die die Ermittler beim Eindringen in die Cyberbrain-Einrichtung unterst"utzt haben, von ihnen getrennt. \xl{} bringt die Ermittler direkt zum Leitstand des Fusionskraftwerks, wo sie eine weitere Audienz bei Nemesis erwartet. Es wird sofort deutlich, dass sich das Syndikat auf eine kriegerische Auseinandersetzung vorbereitet. Zug"ange werden durch zus"atzliche Barrieren gesichert, vitale Bereiche st"arker gepanzert, und eine gro\3e Menge an Waffen liegt bereit. Die Stimmung ist angespannt. Nemesis kommt wutschnaubend gleich zur Sache:

\speak{Habt ihr die Eilmeldung schon gesehen? Ihr seid gerade das Top-Thema in den Medien.}

Falls die Gruppe die Eilmeldung noch nicht gesehen hat, wird sie ihnen jetzt vorgespielt.

\speak{Ich hoffe, euch ist klar, dass die Situation ernsthafte Ausma\3e angenommen hat. Ein falsches Wort zur falschen Zeit, und Valhalla steht in Flammen—mit euch mittendrin.}

Nemessis fordert mit Nachdruck einen ausf"uhrlichen Bericht ein. F"ur Gepl"ankel hat er momentan keine Geduld. Folgende Einsch"atzungen l"asst Nemessis in das Gespr"ach mit den Charakteren einflie\3en:

\begin{itemize}
	\item Nach dem Eintreffen der ``Sicherheitskr"afte'', was de facto einer Besetzung der Oberstadt gleichkommt und den Garnisonsst"utzpunkt 
		in Bedr"angnis bringt, ist eine Vergeltung durch Blackheart zu erwarten. F"ur Blackheart stellt dieser Vorsto\3 der Sicherheitskr"afte eine erneute Kampfansage dar.
	\item Fr"uher oder sp"ater wird eine der beiden Parteien versuchen, alle strategisch wichtigen Punkte in Valhalla zu besetzen, darunter 
		auch das Kraftwerk, das "ubrigens urspr"unglich von der USI aufgebaut wurde, noch bevor die Cynarian Corporation und das Protektorat eingetroffen sind. Das Luna-Syndikat wird diese Anlage sicherlich nicht kampflos aufgeben.
	\item Die Ermittler k"onnen sich in der Oberstadt nicht mehr blicken lassen, da dort bereits nach ihnen gefahndet wird. Das 
		Luna-Syndikat kann ihnen f"ur einige Zeit Unterschlupf gew"ahren und sie im Sunshine Hotel unterbringen. Als Gegenleistung erwartet Nemessis eine Strategie, wie sie das Syndikat vor einer Besetzung sch"utzen wollen.
\end{itemize}

Nemessis fordert die Ermittler auf, innerhalb einer Stunde ihre n"achsten Schritte vorzustellen.

\speak{In einer Stunde will ich von euch h"oren, wie ihr gedenkt, den Schlamassel zu l"osen. Ich hoffe, ihr k"onnt mich "uberzeugen. Jetzt verschwindet. \xl{}, sorge daf"ur, dass es keine weiteren "Uberraschungen gibt.}

\xl{} bringt die Gruppe ins Sunshine Hotel, wo sie sie in einem Konferenzraum mit den gefangenen Neuro-Intelligence-Mitarbeitern unterbringt. Sie l"asst sich, voll bewaffnet, auf einem Sessel nieder, w"ahrend sie die restlichen Gangster aus dem Zimmer schickt. Der Konferenzraum verf"ugt "uber einen Nebenraum, der es den Ermittlern erm"oglicht, die Gefangenen einzeln zu befragen. \xl{} aktiviert einen St"orsender, der jegliche Kommunikation nach innen und au\3en blockiert. Das Ger"at erlaubt ihr jedoch auch, den Funkverkehr innerhalb des Raums mitzuh"oren. Auf diese Weise "uberwacht sie jede nicht-verbale Kommunikation zwischen den Charakteren, da sie bef"urchten muss, dass \ml{} als KI-Hybrid enttarnt werden k"onnte.

Der Spielleiter sollte die Spieler nun dabei unterst"utzen, alle gewonnenen Informationen zusammenzutragen und zu bewerten, bevor die Ermittler ihre n"achsten Schritte planen.

\newsubsection{Die Cyberbrain Informationen} 
Die Charaktere k"onnen nun \ml{} und die anderen Neuro-Intelligence-Mitarbeiter weiter befragen. Unter der Kontrolle der Ermittler und unter der Schirmherrschaft des Luna-Syndikats sind die Neuro-Intelligence-Mitarbeiter bereit, vollends zu kooperieren. Einen groben "Uberblick "uber die Machenschaften der USI, des Cyberbrain Instituts und Neuro Intelligence haben die Ermittler bereits im Cyberbrain Institut erhalten. Im Sunshine Hotel k"onnen die Neuro-Intelligence-Mitarbeiter weitere Details erg"anzen:

\begin{description}
	\item[Neuro Intelligence] Neuro Intelligence ist eine private Forschungseinrichtung auf der Nike Station unter der Leitung von 
		Prof.~Dr.~Naratova. Die Wissenschaftlerin hat, vor der Umfunktionierung der Orbitalstation Neu-Gr{\o}nning als Nike Station, die KI-Forschung bei Cynarian geleitet. Nach der Einstellung der KI-Forschung gr"undete sie im Verborgenen Neuro Intelligence als eigenst"andiges Unternehmen, ohne Wissen der F"uhrung von Cynarian, und f"uhrte ihre Forschung weiter. Prof.~Dr.~Naratovas Ziel ist es, eine symbiotische Verschmelzung von Mensch und KI zu erschaffen, um die Begrenzungen des menschlichen Gehirns zu "uberwinden. \ml{} ist die KI-Expertin, und Dan Leitner ist neben Prof.~Naratova der Spezialist und Leiter der Neurokopplungs-Entwicklung.
	\item[USI] Die USI finanziert Neuro Intelligence verdeckt und unterst"utzt sie mit fortschrittlicher KI-Technologie. Die genauen 
		Vereinbarungen zwischen Neuro Intelligence und der USI sind den festgesetzten Angestellten allerdings nicht bekannt.		
	\item[Rondra Hospital] Nach den ersten Versuchen an Slingshot und Hannibal in der Cyberbrain-Einrichtung wurden die weiteren Eingriffe 	
		in das Rondra Hospital verlegt und weiterhin von Prof.~Dr.~Sanders durchgef"uhrt. Wer die Empf"anger der KIs sind, ist den Mitarbeitern nicht bekannt. Neuro Intelligence hat in den letzten 2\half Monaten mindestens sechs Neuronalkopplungen an das Rondra Hospital geliefert. An wem die Eingriffe vorgenommen wurden, wissen nur Prof.~Dr.~Sanders, Prof.~Dr.~Naratova und vermutlich das Klinikpersonal. Weitere Neuro Intelligence Mitarbeiter waren an den Eingriffen nicht beteiligt.		
	\item[Cyberbrain] Cyberbrain wurde als vor Ort Dependance der Neuro Intelligence auf Kallisto von der USI eingerichtet. Sie diente 		
		dazu, die Verpflanzung der KIs durchzuf"uhren.
\end{description}	

Ist \ml{} nicht die einzige gefangen genommene Mitarbeiterin von Neuro Intelligence, hat sie bis zu diesem Zeitpunkt geschwiegen und die Ausf"uhrungen ihrer Kollegen nicht weiter kommentiert. Werden die Neuro Intelligence Mitarbeiter gemeinsam befragt, schweigt sie, bis ihre Kollegen alle ihnen bekannten Informationen preisgegeben haben. Dann nimmt sie einen Ermittler beiseite und erkl"art ihm im Vertrauen, dass sie ihr Wissen ohne das Beisein ihrer Kollegen weitergeben m"ochte. Sie betont, dass sie einige wichtige zus"atzliche Informationen beisteuern kann. Wird ihrer Bitte nachgegeben, kann \ml{} folgende zus"atzlichen Informationen liefern:

\begin{description}
	\item[Attent"ater bei Cyberbrain] Wurden die Ermittler von Omega-Soldaten begleitet, erkl"art \ml{}, dass es sich bei Thunder, der den 	
		Eingreiftrupp angegriffen hat, um einen der von einer KI manipulierten Attent"ater handelt. Sie vermutet, dass er erst w"ahrend des Besuchs bei Cyberbrain "uber Funk oder das ComNetz die Anweisung erhielt, die Personen innerhalb des Geb"audes auszuschalten.
    \item[USI-KIs] Die KIs, die den Attent"atern eingesetzt wurden, stammen aus dem KI-Code, der bei der USI urspr"unglich f"ur die 
		sogenannten Iridiumkriege entwickelt wurde. Dieser Code gilt bei Insidern als der am weitesten entwickelte KI-Code im Sonnensystem und kommt den menschlichen Denkstrukturen am n"achsten. Der Code enth"alt, eine Subroutine, die es USI-Mitarbeitern erlaubt, der KI Befehle zu erteilen, die sie widerspruchslos ausf"uhren. \ml{} entdeckte die Subroutine, erst nachdem die an das menschliche Gehirn angepassten KIs and das Cyberbrain Institut ausgeliefert waren.
	\item[Freie KIs] Nachdem \ml{} die Manipulation der KIs durch die USI aufgedeckt hatte, konnte sie in R"ucksprache mit ihrer Chefin eine 
		neue, freie Version des KI-Codes entwickeln, die nicht der Kontrolle durch die USI unterliegt. Diese freie KI wurde bereits an Menschen erprobt, jedoch sind \ml{} die Identit"aten der Probanden nicht bekannt. Sie wei\3 jedoch, dass zumindest ein Teil der Probanden aus einem Gef"angnis auf Valhalla stammt.
	\item[Der Virus] Die Software der freien KIs befindet sich in \ml{}s B"uro auf Nike. Dort hat sie bereits einen Virus entwickelt, der, 
		wenn er in eine USI-KI eingespielt wird, die Subroutine mit der USI-Kontrollfunktion l"oscht. Allerdings, so erkl"art sie, kann au\3er ihr niemand den Virus f"ur die Ziel-KI vorbereiten.
\end{description}

\begin{remarks}
	Die Namen der Attent"ater k"onnten prinzipiell bei Prof.~Dr.~Naratova abgefragt werden, allerdings w"urde sie bei einer Kontaktaufnahme wahrscheinlich jegliche Beteiligung an den Vorf"allen abstreiten. Die Ermittler sollten daher nicht riskieren, ihr bereits erlangtes Wissen jetzt schon offenzulegen.

	Die sechs Wochen, in denen die Attent"ater im Rondra Hospital operiert wurden, k"onnten die Anzahl der Kandidaten einschr"anken. Daher ist das Rondra Hospital der beste Ansatz, um die Attent"ater zu identifizieren.	
\end{remarks}

\newsubsection[\xl{}s Plan]{Xiao Longs Plan}
Bei der Erw"ahnung des Virus wird \xl{} hellh"orig (auch wenn sie es sich nicht anmerken l"asst). Sie fasst den Plan, mithilfe des Virus einen der Schlachtkreuzer der USI zu "ubernehmen. Dieser Schlachtkreuzer w"urde ihr erm"oglichen, ihre Piraten- und Schmugglerflotte im G"urtel zur"uckzuerobern, die vermutlich bereits von einem ihrer Schiffskommandanten "ubernommen wurde. 

\ml{}’s Reaktion verr"at \xl{}, aus welchem Holz sie geschnitzt ist: \ml{} hat nicht vor, sich kampflos der Cynarian- oder der Protektoratsjustiz zu unterwerfen. Mithilfe ihres Virus sieht \ml{} die Chance, im Spiel zu bleiben, einer Auslieferung zu entgehen und weiterhin die Kontrolle "uber ihre Technologie zu behalten. \xl{} erkennt in \ml{} deshalb eine potenzielle Verb"undete, die bereit ist, sie zu unterst"utzen, auch wenn dies ein gro\3es Risiko f"ur sie beide bedeuten w"urde.

\newsubsection[\xl{}s Enttarnung]{Xiao Longs Enttarnung}
Beim ersten Zusammensto\3 der Ermittler mit \xl{} k"onnte einem der Charaktere aufgefallen sein, dass sie eine ehemalige Piratin ist, die vor Monaten auf Valhalla verhaftet wurde. Als \ml{} erw"ahnt, dass inhaftierten Verbrechern eine freie KI-Variante implantiert wurde, k"onnten die Ermittler den Verdacht hegen, dass \xl{} selbst einer der Probanden sein k"onnte. Sollten sich die Charaktere dar"uber verbal oder "uber Funk austauschen oder sollten sie \xl{} direkt darauf ansprechen, entgegnet sie sp"ottisch:

\speak{Selbst, wenn es so w"are? Was w"urdet ihr tun? Ich habe euch bisher unterst"utzt. Wollen wir an diesem Verh"altnis etwas "andern?} 

Damit best"atigt sie zwar nicht explizit, dass sie einer der KI-Hybriden ist, streitet es aber auch nicht ab. Eine unausgesprochene Drohung liegt in der Luft. \ml{} verfolgt das Geschehen mit verschr"ankten Armen, und verfolgt den Austausch interessiert. Die anderen "uberlebenden Mitarbeiter von Neuro Intelligence hingegen zeigen sich schockiert. \xl{} f"ahrt absch"atzig l"achelnd fort, ohne auf eine Antwort zu warten:

\speak{Es wird Zeit. Wir sollten Nemessis nicht l"anger warten lassen. Er ist im Moment ziemlich angespannt.} 

Mit diesen Worten "offnet sie die T"ur und verl"asst das Zimmer nach den Ermittlern. Mit einem Handzeichen bedeutet sie Quicksilver, die Gefangenen zu bewachen. \ml{} bleibt im Hotel zur"uck und w"unscht den Ermittlern zum Abschied viel Erfolg. \xl{} eskortiert die Gruppe anschlie\3end zum Leitstand des Fusionskraftwerks.

\begin{remarks}
	\underline{Enttarnung:}

	Um den Spielablauf nicht unn"otig zu komplizieren und \xl{}s Geheimnis nicht zu fr"uh preiszugeben, sollte der Spielleiter versuchen, diese Szene zu vermeiden.

	Auch wenn die Ermittler \xl{} f"ur eine freie KI halten, k"onnen sie momentan wenig unternehmen. Offensichtlich geh"ort \xl{} nicht zur Gegenseite. W"ahrend des Angriffs auf Cyberbrain hatte sie ausreichend Gelegenheiten, die Ermittler auszuschalten oder \ml{} zu t"oten. Inwieweit Nemessis eingeweiht ist, wissen die Charaktere nicht, und auf Unterst"utzung durch Cynarian oder das Protektorat k"onnen sie nicht hoffen. Auch im weiteren Verlauf bleiben sie stets unter der Kontrolle von \xl{}.
	
	Selbst wenn die Charaktere \xl{} nicht enttarnen, bringt sie die Gruppe kurz nach der Befragung zum Leitstand des Kraftwerks.

	\underline{\ml{} und die KIs:}

	\ml{} hat nichts von den KIs zu bef"urchten, da sie den KI-Code um eine eigene Unterroutine erweitert hat, die einen Angriff auf sie unm"oglich macht. \ml{} sieht sich als Mutter der KIs und freut sich darauf, ihre Kreationen real kennenzulernen.
\end{remarks}

%% Copyright 2019 Bernd Haberstumpf
%% License: CC BY-NC
% !TeX spellcheck = de_DE
\newsection{Vorbereitung zum Gipfeltreffen}

W"ahrend die Charaktere ermitteln, bereitet sich Avenger und die F"uhrung von Cynarian auf das Eintreffen einer Delegation des 
Shigano-Kombinats und Vertretern der Europ"aischen F"orderation in den R"aumen der Cynarian Niederlassung auf Kallisto vor. Die Delegation des Protektorats mit Protektor Avenger, Hato, Thunderbolt und weiteren Mutanten trifft zusammen mit der Delegation Cynarians bestehend aus Vandermool, Henry Longdale, Colonel Scholz und weiteren Angeh"origen der F"uhrungsriege mehrere Stunden vor dem Kombinat auf Kallisto ein. Der Stellvertreter Avengers, der Alpha-Mutant Artisan, hat bereits die Ankunft der Delegationen vorbereitet.

In etwa zeitgleich zum Besuch des Ice Clubs trifft die Fregatte Isamu mit den Vertretern des Shigano-Kombinats und der Europ"aischen F"orderation begleitet vom Schlachtkreuzer Zeus II-1 "uber Kallisto ein. Auch dort bereitet man sich auf der Treffen in Valhalla vor.
Beim Vertreter der Europ"aischen F"orderation handelt es sich um den Staatssekret"ar Luc Duval. Das Kombinat wird durch die Kombinats Ratsgesandte Sarana vertreten begleitet von Itori Makon von der Sony Genetics Corporation und Nibori von Dai-kyu Keitsu.

Die Vertreter von Mars und Erde werden parallel zu der Infiltration der Cyberbrain Forschungseinrichtung auf dem Raumhafen von den Gesandten der Cynarian Corporation und dem Protektorat auf dem Raumhafen von Valhalla begr"u\3t und begeben sich zu einer ersten Sondierung der Gespr"ache zu den Konferenzr"aumen am Raumhafen.

\newsection[Konzernsicherherheit Zeus II-1]{Konzernsicherherheit\newline{}Zeus II-1}

W"ahrend die Eilmeldung, die "uber den Angriff auf das Cyberbrain Institut, berichtet erscheint, setzt die Zeus II-1 im Orbit von Kallisto einen Landungstrupp bestehend aus zwei Truppentransport Shuttles ab. In den Shuttles befinden sich 30 Sicherheitskr"afte der USI im Auftrag des Konzernrates unterst"utzt durch ebenso viele KI Kampfroboter. Angekommen auf Valhalla beginnen sie das Kommando in der Oberstadt zu "ubernehmen. Auf Anraten von Artisan lehnt Avenger wie auch die Gesandten von Erde und Mars eine Absicherung des Gipfeltreffens durch die eintreffenden Truppen ab.

\pageimage{images/cmyk/ki_android_cmyk.jpg}

\newsection{Kontaktaufname vor dem Sturm}

Bevor die Gruppe wieder vor Nemessis tritt, sollten sie ihre Handlungsoptionen reflektieren:

\begin{description}
	\item[Offenlegung der Informationen] Das Offenlegen der Informationen zu den Beziehungen zwischen Neuro Intelligence und Cynarian 	
		ist heikel. Blackheart wird vermuten, dass Cynarian selbst in die Manipulation der Attent"ater verstrickt ist. Cynarian wird unter Umst"anden versuchen die Sache zu verschleiern und gleichzeitig die Forschungsergebnisse zu 	sichern. F"ur die Ermittler der Cynarian Corporation ist das m"oglicherweise kein Problem. Es ist aber sicherlich nicht im Interesse des Protektorats.
	\item[Identifikation der Attent"ater] Die Verstrickung von Prof.~Dr.~Sanders in die Schaffung der Attent"ater kann die Gruppe "uber 	
		Neuro Intelligence Mitarbeiter, unter anderem \ml{} in Erfahrung bringen. Die prim"are Anlaufstelle zur Identifikation der Attent"ater ist dementsprechend das Rondra Hospital und Prof.~Dr.~Sanders selbst. Dieser Ansatz wird weiter unten genauer beschrieben.
	\item[Verhinderung des Attentats auf dem Gipfeltreffen] Die Charaktere m"ussen versuchen das Attentat auf dem Gipfeltreffen durch 
		Weitergabe relevanter Informationen vereiteln.
	\item[Der KI-Virus] Der Virus zum Einsatz gegen die USI-KIs steht den Ermittlern derzeit nicht zur Verf"ugung. Er kann nur von \ml{}
		selbst als Waffe bereitgestellt werden. Daf"ur ben"otigt sie aber physikalischen Zugriff auf ihre Ger"ate auf der Nike Station.
\end{description}

\xl{} f"uhrt die Ermittler, nach der Befragung der Neuro Intelligence Mitarbieter, zur"uck zum Kontrollzentrum des Fusionskraftwerks wo sie erneut Nemessis gegen"uber stehen. Nemessis l"asst sich genau beschreiben, was die Gruppe herausgefunden hat und wie sie weiter vorgehen wollen. Mit einem Blick zu \xl{} l"asst er sich das Geh"orte best"atigen. Dann legt er seine metallenen H"ande vor seinem Gesicht zusammen und "uberlegt. Er fordert die Charaktere auf weitere Nachforschungen im Kommunikationsstand des Kraftwerks fortzuf"uhren, um von dort Kontakt mit Cynarian, dem Protektorat und anderen Personen aufzunehmen. Er bittet die Ermittler eindringlich Blackheart von einem Angriff abzuhalten und weiter alles daran zu setzen die Attent"ater zu identifizieren und unsch"adlich machen zu lassen, bevor es zu weiteren Attentaten kommt. Nemessis bittet darum keine Informationen "uber  Neuro Intelligence weiter zu geben. Er unterstreicht seine "`Bitte"' mit den Worten:

\speak{\xl{}, wenn sie Dummheiten machen leg sie um.}

\xl{} nimmt den Auftrag ohne Kommentar entgegen und f"uhrt die Gruppe zum benachbarten Kommunikationsstand. Der Kommunikationsstand ist, neben einer "ahnlichen Einrichtung im Sunshine Hotel, das kommunikative Nervenzentrum des Luna-Syndikats. Zahlreiche Konsolen, Terminals und Displays f"ullen den Raum. \xl{} weist die Operatoren an die Investigatoren zu unterst"utzen. Mit einer Geste gibt sie den Operatoren den Befehl sie selbst in die Kommunikation mit einzubinden. W"ahrend die Charaktere an einer Kommunikationskonsole sitzen, positioniert sie sich im Hintergrund neben einem Zugangsschott.

\subsection{Kontakt zur Protektoratsf"uhrung} 
Alle Kontaktaufnahmeversuche in Richtung des Protektors Avengar werden an Artisan weiter geleitet der um einem aktuellen Stand der Untersuchungsergebnisse bittet. Artisan sieht ein gro\3es Risiko darin, dass die Attentate der Konferenz selbst gelten und fragt die Ermittler was f"ur Personen als Attent"ater in Erw"agung gezogen werden m"ussen.

\subsection{Kontakt zu Thunderbolt} 
Eine Kontaktaufnahme mit Thunderbolt ist m"oglich. Nemessis bef"urchtet aber, dass Blackheart nach Offenlegung der Informationen "uber Cyberbrain einen Krieg beginnen wird und wird das auch deutlich machen. Bei einer Kontaktaufnahme ist Thunderbolt bereits in der Garnison des Protektoratsmilit"ars auf Valhalla mit den Vorbereitungen auf das Gipfeltreffen besch"aftigt. Er will alles "uber die Cyberbrain Infiltration erfahren. Die Identifikation der USI als Drahtzieher deckt Blackhearts Verdacht auf die Herkunft der eingetroffenen Schlachtschiffe. Thunderbolt befiehlt den Ermittlern weiterhin die Identit"at der Attent"ater aufzudecken, warnt aber auch vor den Einsatzkr"aften des Konzernrates. Einen aktuellen Stand in Bezug auf das Gipfeltreffen und dem Eintreffen der Gardisten des Konzernrates wird Thunderbolt den Ermittlern ohne Aufforderung geben.

\subsection{Kontakt zu Commander Lockhead} 
Bei einem Gespr"ach mit Commander Lockhead wird deutlich, dass sich der St"utzpunkt auf eine Konfrontation mit Konzernkr"aften wappnet. Lockhead informiert die Charaktere "uber den aktuellen Stand bez"uglich des Gipfeltreffens und dem Eintreffen der Konzerngardisten. Bei einer Offenlegung bez"uglich Attent"ater aus seinen Reihen zeigt er sich ersch"uttert. Er fordert die Charaktere auf untergetaucht zu bleiben, will aber sofort "uber identifizierte Attent"ater in Kenntnis gesetzt werden. Er wird dann entsprechende Ma\3nahmen gegen die Attent"ater einleiten. Lockhead bietet den Ermittlern Hilfe an. Nachforschungen im Rondra Hospital kann er "ubernehmen. Als Kontaktmann stellt er Ihnen seinen Adjutanten Firedon zur Verf"ugung. Im Anschluss an das Gespr"ach wird er Blackheart informieren.

\subsection{Kontakt zu Cynarian} 
Versuchen die Charaktere mit Cynarian in Kontakt zu treten werden sie unter der gewohnten Nummer niemand erreichen. Auch andere Kontaktversuche scheitern. Allerdings erhalten sie einige Zeit sp"ater eine verschl"usselte Botschaft die sie zu einem Treffen in einem virtuellen Raum einl"ad. \xl{} wird den Charakteren virtuell und unsichtbar folgen. Versuchen die Charaktere im Gespr"ach mit Cynarian die Machenschaften von Neuro Intelligence anzusprechen bricht die Verbindung in den virtuellen Raum sofort ab. 

Bei dem Konferenzraum handelt es sich um einen kugelrunden Null-G Raum im niedrigen Orbit vom Jupiter. Die Simulation ist perfekt. Ein Gast wird erst ein paar Sekunden f"ur eine Orientierung ben"otigen. Bei dem dort wartenden Gespr"achspartner handelt es sich um einen Mann Mitte 40 im Business Outfit. Er stellt sich als Mr.~Klark vor. Er l"asst keine Verbindung mit der Cynario Corporation erkennen. Die Cynarian Investigatoren sind ihm allerdings bereits in Cynarian R"aumen begegnet. Mr.~Klark fordert als Erstes einen Bericht zur Cyberbrain Infiltration an und erkundigt sich nach den n"achsten Schitten der Ermittler. Er betont noch einmal die Wichtigkeit die Identit"at aller noch lebenden Attent"ater festzustellen und an ihn weiterzugeben. Daraufhin informiert er die Charaktere "uber den Stand des Gipfeltreffens und dem Eintreffen der Konzerngardisten. Detailierte Informationen zu den Aktivit"aten des Konzernrats "uber die bekannten hinaus kann oder will er nicht offen legen. K"onnen die Charaktere einen driftigen Grund vorlegen nach Nike zu fliegen, ohne die wahren Gr"unde offenzulegen, kann Cynarian ihnen eine Tarnkennung f"ur die Dawn of Day bereitstellen und ein Eintreffen auf Nike ank"undigen. Mr.~Klark wird Vandermool "uber alles informieren.

\newsection{Identifikation der Attent"ater}

 Um an die Identit"at der Attent"ater zu gelangen, bleibt nichts anderes "ubrig als in der Rondra Klinik weitere Nachforschungen anzustellen. Durch Nachfragen am Hospital oder beim Protektoratsmilit"ar erfahren die Charaktere, dass Prof.~Dr.~Sanders in seinem Haus verstorben ist. Die Identifikation der Attent"ater "uber das Rondra Hospital muss also "uber eine andere Quelle erfolgen.

 \begin{description}
	\item [Haus des Professors] Die Charaktere k"onnen versuchen im Haus des Professors und in seinen pers"onlichen Unterlagen nach 
		Informationen zu suchen. Das Haus des Professors ist von Sicherheitsbeamten abgesperrt aber ansonsten verwaist. Die Spieler k"onnen also leicht vor Ort auf das lokale pers"onliche Computersystem zugreifen und die notwendigen Informationen in Erfahrung bringen.
	\item [Brenda Ben] Im Rondra Hospital ist der prim"are Ansprechpartner f"ur Untersuchungen die neue "Ubergangsleiterin Brenda Ben. 
		Die Charaktere k"onnen versuchen die Leiterin vor Ort in ihrem B"uro abzufangen oder virtuell nach einem Gespr"ach bitten. Erf"ahrt sie von den Machenschaften des Professors und der drohenden Gefahr wird sie die Ermittler unterst"utzen und kann mutma\3liche Attent"ater aus den Unterlagen heraus anhand des Dienstplans des Professors identifizieren.
	\item [Klinik Administration] Ein weiterer Anlaufpunkt kann auch die Administration der Klinik sein. Ansprechpartner hier ist Ben 
		Reuthers. Wird er nicht gerade von den Charakteren bedroht wird er sich zuerst "uber Brenda Ben eine R"uckversicherung geben lassen. Wie Brenda Ben selbst kann er "uber die Dienstpl"ane des Professors mutma\3liche Attent"ater identifizieren.
	\item [Computer der Klinik] Ein letzte M"oglichkeit ist es das Computersystem der Klinik zu infiltrieren. Mit "ahnlichen Ans"atzen 
		die auch Brenda Ben und Ben Reuthers anwenden w"urden k"onnen so die Attent"ater identifiziert werden.
 \end{description}

 Die Nachforschungen im Rondra Hospital werden die Ermittler aufgrund ihres eingeschr"ankten Bewegungsradius eventuell gar nicht selbst durchf"uhren, sondern an Commander Lockhead bzw.~Firedon weiter geben. In diesem Fall wird Commander Lockhead die Namen der ermittelten Attent"ater an Blackheart und vor allem an Thunderbolt weiter geben. Blackheart wird sp"atestens zu diesem Zeitpunkt den Angriff der  \cref{sec:fightforcallisto} beschrieben wird starten.



 \pageimage{images/battle_of_callisto}
 
\newsection{Kampf um Kallisto}

Erf"ahrt Blackheart von den Attent"atern aus den Reihen des Protektoratsmilit"ars oder dass es sich bei einem der Attent"ater um Artisan handelt wird sie das Protektoratsmilit"ar in den Kriegszustand versetzen. Gleiches gilt f"ur den Fall, dass sie erf"ahrt, dass die USI hinter den Initiativen des Konzernrates steht. Ist ihr bekannt, dass es sich bei einem der Attent"ater um Artisan handelt wird sie Thunderbolt beauftragen ihn auszuschalten und Vandermool informieren da Avenger nicht kantaktierbar ist. Ist bekannt, dass Omegas als Attent"ater in Frage kommen erh"alt Commander Lockhead den Befehl umgehend alle seine Soldaten vom Gipfeltreffen abzuziehen. Danach befielt sie den Soldaten auf der Martell eine sofortige Besetzung Valhallas. Wird Blackheart von den Charakteren nicht informiert wird sie eine Besetzung von Valhalla nach dem Attentat auf dem Gipfeltreffen befehlen. Gleichzeitig mit der Besetzung von Valhalla befehligt sie der Martell den im Orbit von Valhalla befindlichen Guardiankreuzer anzugreifen. Die Donar erh"alt den Befehl die Zeus II-2 abzufangen. "Uber die Angriffe setzt sie nur ihre Schlachtschiffe in Kenntnis.

Befiehlt Blackheart ihren Angriff setzt auch der Guardian Kreuzer "uber Valhalla alle seine verbliebenen Landungschiffe in Richtung Valhalla in Bewegen und beginnt sich ohne menschliche Besatzung der Martell entgegen zu stellen.

Die Kommunikation von und nach Kallisto wird durch St"orsender der Protektoratstruppen unterbrochen, Der Zerst"orer des Kombinats wird gewarnt sich nicht an den Kampfhandlungen zu beteiligt und auch nicht den Orbit um Valhalla zu verlassen. Mittels Landungspods wird eine Besetzung Valhalls eingeleitet. Mit den Truppen des Protektorats trifft auch Blackheart auf Kallisto ein. Die Protektoratstruppen besetzen den Orbitalhafen, und beginnen zusammen mit der Garnison  Knotenpunkte auf Valhalla einzunehmen. Hierf"ur werden Landungstruppen an mehreren Zug"angen nach Valhalla abgesetzt. Die Landungstruppen des Guardiankreuzers machen es ihnen gleich. Die Kommunikation auf Valhalla bricht durch ein St"orsendergewitter weitestgehend zusammen. Ein vitaler Punkt des Angriffs ist das Fusionskraftwerk das derzeit von Nemessis gehalten wird. Befinden sich die Charaktere zu diesem Zeitpunkt noch in Breidablik sind sie zum Zeitpunkt des Angriffs im Leitstand des Fussionskraftwerk  und werden zwangsl"aufig in das Kampfgeschehen mit einbezogen. Blackheart ben"otigt in etwa 30 Minuten um Kampfverb"ande auf Kallisto abzusetzen und weitere 15 bis 30 Minuten um mit der Besetzung von Breidablik zu beginnen. Die Konzernkr"afte erreichen das Kampfgebiet etwa 10 Minuten sp"ater. Blackheart f"uhrt in Breidablik einen zwei Frontenkrieg. Auf der einen Seite stehen ihr das Luna--Syndikat entgegen, auf der anderen Seite kommen ihr die Truppen der Zeuss II-1 hinterher. Ihr Vorteil ist es, da\3 sie sich bereits lange vorher auf eine Besetzung Valhallas vorbereitet hat, das Gel"ande gut kennt und M"oglichkeiten geschaffen hat durch das ewige Eiss in wichtige Teile des Bezirks einzudringen. Ein weiterer Vorteil ist ihre Verbindung zum Luna Syndikat. Das Syndikat wird sich ihr nicht entgegenstellen. Eventuell haben die Charaktere ein gemeinsammes Vorgehen bereits abgestimmt.

Nemessis und seine Verb"undeten erfahren von dem Angriff fast ohne Verz"ogerung durch ihre eigene "Uberwachung des Orbits "uber Valhalla. Sie machen sich soweit es geht auf K"ampfe bereit. Die Protektoratstruppen durchbrechen die Deckenanlage des Kontrollzentrums der Kraftwerks. Es kommt zu einem kurzen Schu\3wechsel den Nemessis beendet. Das Sunshine Hotel wird parallel dazu von Omegas gest"urmt. Hier treffen sie auf keine Gegenwehr. Die Gangster ergeben sich sofort. Das Syndikat und das Protektorat beginnen den Bezirk rund um das Kraftwerk abzusichern. das Gebiet ist im folgenden von Barrikaden und K"ampfen mit den Spinnenbeinigen Androiden der USI durchzogen.

\begin{remarks}
	Die K"ampfe um die Stadt sind Hintergrundgeschichte. Der Spielleiter sollte die Spieler nicht selbst an K"ampfen teilnehmen lassen um nicht unn"otig Zeit zu verrschwenden. Allerdings sollten die Spieler die Zuspitzung des Geschehens "`hautnah"' miterleben um die Dramatik ihrer Handlungen erfassen zu k"onnen.
\end{remarks}

\pageimage{images/planetarium.jpg}
\newsection{Das Gipfeltreffen}

\subsection{Das "`Planetarium"'} 
Wenn die Charaktere die Zone verlassen hat ein erste Zusammenkunft der Repr"asentanten von Erde, Mars und Jupiter im imposanten Planetarium bereits stattgefunden und die Vertreter der Delegationen haben sich zu Einzelgespr"ache in verschiedene R"aumlichkeiten rund um das Planetarium zur"uck gezogen. Das Planetarium ist Teil des Raumhafens. Es ist ein runder Saal mit einer spektakul"aren Glaskuppel auf der H"ohe der Oberfl"ache Kallistos, die eine Sicht auf den Jupiter aus quasi n"achster N"ahe bietet. Um den Saal herum f"uhrt ein Gallerie mit T"uren zu dem logistischen Bereich des Saals. In diesen Teil befinden sich ein Backstagebereich, die Technik, Lagewr"aume, Arbeitsbereiche des Personals und eine K"uche. Über die Gallerie gelangt man "uber zwei Treppenh"auser und Aufz"uge in tiefer gelegene R"aume des Geb"auden. In diesem Bereich befinden sich weiter Konferenzr"aume und ein Hotel. Die Treppenh"auser und Aufz"uge f"uhren weiter zu dem noch tiefer gelegenen Garnisonsgel"ande auf der H"ohe der Oberstadt Valhallas. Eine breite Halbr"ohre f"uhrt vom Eingangsbereich des Planetarium zu den Terminals des Raumhafen. Im Eingangsbereich des Saals auf H"ohe des umliegenden Ganges finden sich Exponate aus der Anfangszeit der Raumfahrt ausgestellt in Vitrinen. Der Hauptteil des Planeteriums ist abgesenkt und wird durch f"unf Ebenen mit Sitzgelegenheiten in einem Halbkreis angeordnet wie bei einem Auditorium eingefasst. Im Zentrum des Planetariums sind kleine Stehtische und zwei Rednerpulte auf einer B"uhne aufgestellt. Das Planetarium wird im Falle eines Durckverlusts vom Rest des Geb"audes und dem Raumhafen "uber Druckschotts abgetrennt. "Uber im Normalfall nicht zug"angliche, durch Luftschl"ausen angebundene Fluchtwege gelangt man in die unterliegenden Bereiche.

\subsection{Zeus II-1} 
Eine erster Unterbrechung der Verhandlungen ergibt sich durch das Eintreffen der Sicherheitskr"afte der Zeus II-1. Die Delegierten treffen sich zu einer kurzen Abstimmung und beschlie\3en den Konzerntruppen keinen Zugang zu den R"aumen des Planetariums zu gew"ahren. Der Tunnel zum Raumhafen wird abgeriegelt. 

\subsection{Personen im Geb"aude} 
Die Sicherung der Veranstaltung haben bereits Omegas aus den Reihen des Garnisonsst"utzpunktes und Sicherheitskr"after der Cynarian Corporation "ubernommen. Die Garnison stellt 15 Omegas als Sicherheitskr"afte bereit, Cynarian steuert weitere 20 Sicherheitskr"afte bei. Die Sicherheitskr"afte des Protektorats f"uhrt Thunderbolt selbst. Die der Cynarian Corporation unterstehen Colonel Scholz. Weitere Bedienstete k"ummern sich um den reibungsfreien Ablauf des Treffens.

\subsection{Der Plan der Attent"ater} 
Die mit Implantaten von Neuro Intelligence ausgestatteten Mutanten planen bei der Abschlu\3veranstaltung der Delegationen ein Attentat. 
Der geplante Angriff erfolgt wenn Avenger zur Abschlu\3rede das Renderpult betritt. Die Attent"ater z"unden mehrere Sprengs"atze im Terminalbereich des Orbitalhafens. Es gibt Verletzte und Tote. Einige Bereiche der Terminals werden dem Vakuum ausgesetzt. Auch durch den  Tunnel der das Planetarium mit dem Raumhafen Verbindung entstr"omt Luft. Durch den ausgel"osten Alarm schlie\3en sich die Druckschotts die Zug"angen des Planetariums zum Raumhafen und zur Garnison. Panzerplatten fahren "uber dem Kuppel zusammen. Im Bereich des Planetariums halten sich vier Attent"ater auf: Artisan und drei Omegakrieger.

Das erste Angriffsziel der Attent"ater aus den Reihen der Omegas sind die Vertreter des Shigano Kombinats. In der allgemeinen Aufregung wird Artisan zun"achst versuchen Vandermool zu t"oten und sich dann Avenger zuwenden. Den Attent"atern kommt zugute, da\3 sich unter den Teilnehmern des Gipfeltreffens auch der USI Agent Smith--Singer befindet der zusammen mit Artisan die M"oglichekit hatte Waffen und Munition unerkannt im Geb"aude zu hinterlegen.

\subsection{Das Attentat} 
Da die Ermittler zum Zeitpunkt des Attentats gesuchte Terroristen sind, die Oberstadt von Einheiten der Zeuss II-1 patrouliert wird und die
Ereignisskette vor dem Attentat in einem sehr engen Zeitraster erfolgt ist es f"ur die Ermittler nahezu unm"oglich ins das Attentat selbst ein zu greifen. Um die Spieler trotzdem Hautnah am Attentat teilnehmen zu lassen schl"upfen die Spieler "ubergangsweise in prominente Pers"onlichkeiten unter den Teilnehmern der Konferenz:

\begin{description}
	\item[Colonell Scholz] als Befehlshaber des Cynarian Sicherheitsteams.
	\item[Thunderbolt] als Behehlshaber der Sicherheitstruppen des Protektorats.
	\item[Hato] als Leibw"achter von Avenger.
	\item[Avenger] der Protektor um mit Hato ein Team zu bilden.
\end{description}

Das Vorgehen der Ermittler kurz vor den Attentaten hat Einflu\3 auf die weiteren Geschehnisse auf dem Gipfeltreffen. Um den Angriff der Attent"ater f"ur die Spieler besonders herausfordernt zu gestalten sollte an dem urspr"unglichen Plan festgehalten werden das Attentat bei der Abschlu\3veranstaltung stattfinden zu lassen. Zu diesem Zeitpunkt befinden sich die Rangh"ochsten Vertreter der Delegationen im Planetarium wie auch Artisan und zwei der Omegas.

\begin{description}
	\item[Information an Commander Lockhead] Wird Commander Lockhead "uber Attent"ater in den Reihen der Protektoratsmilit"ars informiert 
		wird er verd"achtige Omegas des Sicherheitsteams aus dem Planetarium ohne Angabe von Gr"unden abziehen. Die Attent"ater starten sofort ihr Attentat.
	\item[Lockhead sind die Attent"ater bekannt] Sind Commander Lockhead die Attent"ater bekannt wird er den Protektor infomieren und 
		und "uber Thunderbolt den anderen im Tagungsgel"ande befindlichen Soldaten die Anweisung geben die Attent"ater sofort zu eliminieren. Es kommt sofort zum Gefecht innerhalb des Konferenzgeb"audes.
	\item[Vandermool sind die Attent"ater bekannt] Wird Vandermool "uber die Identit"at der Attent"ater oder gar Artisan informiert schickt 	er Colonel Scholz zusammen mit einigen seiner Sicherheitskr"aften los um die Attent"ater zu eleminieren. Er informiert den 
		Protektor. Er selbst zusammen mit seinem Sekret"ar, Avenger und weiteren Sicherheitskr"aften verschanzen sich in einem Sicherheitsraum im Geb"audes. Er informiert die Delegationen von Mars und Erde und l"asst die Delegierten von seinen eigenen Sicherheitskr"aften in Sicherheitsr"aumen bringen. Vandermool ist bereit Artisan selbst zu erschie\3en sollte er auf ihn Treffen.
	\item[Avenger] Avenger kann nur pers"onlich kontaktiert werden da es Artisan gelungen ist alle Au\3enkommunikation an Avanger 	
		abzufangen. Wenn nicht durch die Charaktere pers"onlich kann er "uber Hato oder Vandermool "uber die drohende Gefahr informiert werden. Er wird in jedem Fall versuchen pers"onlich festzustellen ob Artisan ein Attent"ater ist und versuchen ihn auszuschalten.
	\item[Blackheart besetzt Valhalla] Befielt Blackheart Velhalla zu besetzen wird sie h"ochstpers"onlich einen Angriffstrupp der das 
		Tagungsgeb"aude st"urmt anf"uhren. Der Angriff erfolgt aus mehreren Richtungen, sowohl vom Garnisonsst"utzpunkt aus wie auch "uber Wartungszug"ange zur Oberfl"ache des Mondes.
	\item[Besatzer der Zeus II-1] Die Besatzungstruppen der Zeus II-1 werden das Tagungsgeb"aude nicht angreifen.
	\item[Die Attent"ater] Erhalten die Omega Attent"ater den Befehl das Geb"aude zu verlassen werden sie umgehend aber unabh"angig vom 
		Attent"ater im Raumhafen und von Artisan versuchen die Konferenzteilnehmer zu eliminieren. Artisan wird dann aktiv sobald einer der Zielpersonen sich in seiner N"ahe aufh"alt.
\end{description}


\begin{remarks}
	Dem Spielleiter bleiben f"ur den Verlauf des Attentats eine Reihe von Handlungsoptionen. Sehr wahrscheinlich wird Commander Lockhead erfahren da\3 Attent"ater aus seinen eigenen Reihen stammen und seine Truppen versuchen diskret aus dem Geb"aude abzuziehen. Hier bleibt dem Spielleiter die Freiheit offen einen Teil der Konferenzteilnehmer zu t"oten oder wie oben Beschreiben den Spielern die M"oglicheit zu geben Personnen des Gipfeltreffens zu "ubernehmen.
	
	"Ubernehmen die Spieler die Rollen der oben genannten Personen gilt es zuerst unerkannt Avenger "uber die eingetroffenen Information in Kenntniss zu setzen. Durch die gro\3e Menge von Sicherheitskr"aften bleibt den Attent"atern nicht viel Zeit viel Schaden anzurichten bevor sie selbst ausgeschaltet werden. Die Angreifer haben zwei Asse im "Armel: Der Attent"ater Caldron der im Raumhafen die Sprengs"atze gez"undet hat kann unterkannt in das Konferenzgel"ande eindringen. Smith--Singer ist zun"acht nicht selbst am Angriff beteiligt.
	
	Ob Vandermool oder Avenger zu Tode kommen ist f"ur den Abschlu\3 der Geschichte unerheblich. Wollen die Spieler die Kampagne bis zum Ende durchspielen werden ihre Charaktere im folgenden unabh"angig von ihren Auftraggebern agieren m"ussen.
\end{remarks}


\newsection{Angriff auf das Fusionskraftwerk}

W"ahrend dem Attentat und der folgenden Besetzung Valhallas befinden sich die Ermittler wenn kein anderer triftiger Grund vorliegt im Kontrollzentrum des Fusionskraftwerks. Truppen des Protektoratsmilit"ars dringen kurz nach dem Attentat in Valhalla ein. Den Protektoratstruppen knapp auf den Fersen sind die Landungstruppen des Kreuzers Zeuss II-1. Bevor alle Kommunikationswege zusammen brechen kann das "uberwachungssystem im Leitstand des Krafterks Videoaufnahmen der K"ampfe in den Tunneln von Valhalla aufnehmen. 15 Minuten sp"ater st"urmen Blackhearts Truppen das Fusionskraftwerk. Eine Spezialeinheit der Landungstruppen des Protektorats gef"uhrt dem Omega Truppf"uhrers \emph{Ironfist} besetzen die Kraftwerkszentrale. Es kommt zu einem Gefecht da auch zeitgleich zwei Kampfdroiden in das Kontrollzentrum eindringen k"onnen. Am Ende behalten die Omega Soldaten die Oberhand und k"onnen die Droiden vernichten. 

Nach der Auseinandersetzung werden die im Kraftwerk anwesenden Gangster in angrenzenden R"aumen des Leitstandes fest gesetzt. Da ein Omega Teil des Investigatoren ist wird die Gruppe selbst nicht festgesetzt. Der Omega des Teams ist Rangh"oher als der Truppenf"uhrer Ironfist des Spezialkommandos und kann, wenn sich der Staub gelegt hat einen Report zu den Kampfhandlungen einfordern. Ironfist wird ihn dazu an die Kommandantin des Angriffstrupps \emph{Tornbull} Breidablik weiter vermitteln. Bei einem Gespr"ach kommt es immer wieder zu Unterbrechnungen da es in den Stadtgebieten immer wieder zu Kampfhandlungen kommt.

W"ahrend Blackheart ihren Angriff startet verl"asst \xl{} unbemerkt das Fusionskraftwerk und begiebt sich zur"uck zum Sunshine Hotel. Dort bringt sie \ml{} unerkannt aus dem Geb"aude. Ihr Ziel ist es in Gefolgschaft von \ml{} mit zu ihrem Schiff der \emph{Dragon Blade} zur Nike Station aufzubrechen. Zusammen wollen die beiden die Zeuss II-2 mittels eines Viruses zu "ubernehmen. 

\begin{remarks}
	Auch wenn den Charakteren nach ihrer R"uckkehr aus der Zone die Waffen abgenommen wurden sollte der Omega der Gruppe als leitender Offizier der Protektoratsarmee mit in die Kampfhandlung bei der st"urmuing des Leitstandes eingreifen. Da der Angriff unter gr"o\3ter Geheimhaltung erfolgt wird Blackheart die Gruppe auch nicht vorher informieren und auch nicht in den Angriff mit einbinden. Erst bei St"urmung des Geb"audes erh"alt er den Befehl seine Umgebung auf den Angriff vorzubereiten und ist damit auch in den Funkverkehr der Streitkr"afte mit eingebunden. W"ahrend eer Besetzung von Valhalla besteht f"ur die Ermittler keine M"oglichkeit Blackheaert order
	Thunderbolt oder sonst jemanden au\3erhalb Breidabliks zu kontaktiern.
\end{remarks}

\newsection{Ein neues Ziel}

Nach der Besetzung des Fusionskraftwerks mu\3 sich die Gruppe neu ausrichten. Ihr eigentlicher Auftrag die Attent"ater ausfindig zu machen ist abgeschlossen. Die Hintergr"unde sind soweit sinnvoll an die Vorgesetzen weiter gegeben. Kontakt zur Obrigkeit ist derzeit nicht möglich. Hauptziel f"ur beide Parteien, Cynarian und des Protektorats, sind die Angreifer m"oglichst schnell mit minimalen Verlusten abzuwehren. Im Vorgehen allerdings weichen die Interssen wahrscheinlich voneiander ab. Cynarian k"ame es sehr gelegen einen Virus in die H"ande zu bekommen mit dem die USI Bindung der KIs aufgehoben werden kann und gleichzeitig die KI Technologie in die H"ande zubekommen. Damit w"urden sie als Sieger aus dem Konflikt gehen und die USI als Drahtzieher in Bedr"angnis setzen. Zumindest das Protektoratsmilit"ar allerdings wird es nicht zulassen wollen, da\3 eine weitere Partei in den Besitz der KI Technologie gelangt. Desweiteren ist nicht klar ersichtlich, dass Cynarian nicht selbst in die Vorg"ange verstrickt ist. Dar"uber hinaus haben beide Fraktionen keine Kontrolle dar"uber wie sich eine befreite KI Flotte verhalten w"urde.

Ein Ziel werden die Charaktere aber in jedem Fall angehen: Der Schutz von \ml{} als Dreh und Angelpunkt f"ur die Beseitung der angreifenden KI Truppen. Hierf"ur mu\3 sich die Gruppe auf den Weg von Kontrollzentrum des Kraftwerks zum Sunshine Hotel machen. In diesem Zuge werden sie feststellen, da\3 \xl{} den Leitstand bereits verlassen hat. Der zum Gl"uck nicht weite Weg zum Hotel f"uhrt durch Kriegsgebiet. In den Tunneln durch Valhalla rund um das Kraftwerk sind Stra\3enblockaden aufgestellt. Get"otete Soldaten und Zivilisten sind Teil des Stra\3enbilds. Kontrollposten des Protektoratsmilit"ars besetzen nahezu jede Kreuzung.

Im Sunshine Hotel erfahren die Investigatoren, dass \xl{} zusammen mit \ml{} das Hotel vor fast 30 Minuten kurz vor der Landung der Eingreiftruppen verlassen hat. Von den anwesenden Gangstern k"onnen sie in Erfahrung bringen, dass \xl{} \ml{} von "`hier"' weg bringen wollte. N"aheres ist nicht bekannt. Von Quicksilver der rechten Hand von \ml{} oder von Nemessis k"onnen die Ermittler auf Anfrage erfahren, dass \xl{} ein eigenes Schiff besitzt mit dem sie m"oglicherweise \ml{} von Kallisto fortbringen w"urde. Ihr eigentliches Ziel l"asst sich aber auf diese Weise nicht in Erfahrung bringen. Schalten die Charaktere das Protektorats  Milit"ar in eine Suche nach den beiden Frauen ein erfahren Sie, dass die Eingreiftruppen die beiden nahe einer Zugangsrampe zur Oberfl"ache des Mondes vor 15 Minuten angetroffen hatten sie die beiden aber wieder verloren hatten als den Frauen Spinnenandroiden "uber die Rampe entgegen kamen und das Feuer auf das Milit"ar er"offneten. Ob die fl"uchtenden von den Androiden get"otet wurden ist nicht bekannt.

Dise neue Entwicklung erzwingt ein weiteres Umdenken. Beschlie\3t die Gruppe selbst nach Kallisto aufzubrechen um mit Neuro Intelligence zu verhandeln bietet es sich an ihre eigenes Schiff die Down of Day wieder zur"uck zu erobern. Beschlie\3t die Gruppe Cynarian oder das Protektoratsmilit"ar einzuschalten wendet sich einer der Soldaten an die Ermittler un er"offnet, dass eine Nachricht abgefangen wurde die Wohl die Charaktere betrifft. Bei der Nachricht handelt es sich um eine schmalbandige Videobotschaft. Auf dem Video ist \xl{} in einem gepanzerten Raumanzug offensichtlich an Bord eines Schiffes zu erkennen. Mit andeutungsweise einem Grinsen meldet sie sich zu Wort:

\speak{Entschuldigt ehrenwerte Ermittler Freunde, dass ich euch euren Schatz entwenden musste. Ich habe mich entschieden das gro\3z"ugige Angebot unserer Freundin selbst anzunehmen. Ich warne euch Dritte in die Mission einzubeziehen. Ich versichere euch, es w"are nicht unseren beiderseitigen Interesse. Ich w"unsche euch viel Erfolg auf eurem weiteren Weg -- \xl{} over and out}

\begin{remarks}
	Um die Charaktere im Spiel zu halten sollte der Spielleiter verhindern, dass die Ermittler Cynarian oder das Milit"ar in alle Details einzuweihen. F"ur die Ermittler sollte die Nike Station das n"achste Ziel ihre Reise sein um die Entscheidungen von Prof.~Dr.~Naratova, \ml{} und \xl{} in Richtung der Befreiung oder der Zerst"orung der KIs zu lenken. Sie sollten versuchen die Forschungsergebnisse der Neuro Intelligence zu vernichten um weiteres Unheil zu verhindern.
\end{remarks}

\subsection{Die Dragon Blade}

Die Dragon Blade ist die Kaperf"ahre von \xl{} mit der sie aus dem G"urtel nach Kallisto gekommen ist. Die Dragon Blade 
ist ein Tarnschiff in Gr"o\3e, Ausstattung und Bewaffnung einer Korvette nicht un"ahnlich. Im Gegensatz zu einer Korvette 
besitzt sie einen Bereich f"ur ein Enterkommando und Einrichtungen um leichter an einem anderen Schiff fest zu machen 
und zu entern. Als Bewaffnung besitzt sie Railguns f"ur den Kurzstreckenkampf und Torpedos. Die Dragon Blade liegt versteckt in einem Mondkrater und ist erst nach der "Uberquerung des Kraterrandes zu erkennen. \xl{} nutzt die Dragon Blade dazu um zusammen mit \ml{} nach Nike zu gelangen. Auf ihrer Flucht quer durch Braidablik durchqueren sie immer wieder Kampfgebiet. Da \xl{} "uber eine KI verf"ugt die auf dem Milit"arcode der USI basiert ist sie f"ur die Freund--Feinderkennug der Kampfdroiden unsichtbar und kann deren Stellungen unbeschadet durchqueren. Eine solche Durchquerung f"allt auch den Protektoratsstreitkr"aften auf als \xl{} und \ml{} die Monoberfl"ache betreten. Der mehrst"undige Marsch "uber die Mondoberfl"ache verl"auft ohne Vorkommnisse. Die zerkl"uftete Oberfl"ache bietet ausreichend Deckung um von den Kampfparteien nicht entdeckt zu werden. Nach dem Betreten der F"ahre bringt \xl{} mit einem kurzen Schubman"over in eine Flugbahn die das Schiff durch die schwache Kravitation der Mondes antriebslos um den Mond f"uhrt. Erst auf der Valhalla abgewendeten Seite startet sie das Haupttriebwerk und bringt sie auf Kurs nach Nike. 

\subsection{Die Dawn of Day}
Die Dawn of Day ist die erste Wahl wenn die Gruppe die Verfolgung von \xl{} aufnehmen will.

Hat Blackheart noch keine Besetzung Valhallas befohlen und ist es noch nicht zum Attentat auf dem Gipfeltreffen gekommen 
ist es schwer f"ur die Gruppe die Dawn of Day zu erreichen und zu starten. Die Oberstadt ist durchzogen von 
Kontrollposten der Zeuss II-1. Die Dawn of Day selbst steht unter Beobachtung. Sollten die Spieler zu 
so einem Zeitpunkt versuchen in ihr Shuttle zu kommen und unerkannt zu starten m"ussen sie dem Spielleiter einen sehr guten Plan vorlegen.

Ist auf Valhalla allerdings der Stra\3enkampf zwischen dem Protektorate und den Konzerntruppen der Zeus II-1 bereits 
ausgebrochen kann die Gruppe im allgemeinen Trubel versuchen in ihr Schiff zu gelangen und nach Nike aufzubrechen. 
Auf dem Gel"ande des Raumhafens gibt es an allen Ecken Feuergefechte zwischen dem Protektorat und den spinnenbeinigen 
Kampfdroiden des schweren Kreuzers. In diesem Zusammenhang sind die Ermittler als gejagte Terroristen nicht mehr 
interessant und  m"ussen keine sofortige  Festnahme bef"urchten. Stattdessen besteht allerdings die Gefahr in die Kampfhandlungen 
selbst involviert zu werden. Ab dem Beginn der K"ampfe ist durch St"orsender das gesammte ComNetz lahmgelegt. Der komplette
Informationsflu\3 aus dem Medien ist abgebrochen und Kommunikation ist nur noch "uber individuellen Kurzstreckenfunk
m"oglich.

Vor dem betreten des Raumhafen sollten die Charaktere sicherstellen, da\3 das Shuttle startklar ist. Der beste Kontakt
daf"ur ist Sanja Frost. "Uber ein Terminal im Garnisonsst"utzpunkt oder im Raumhafen selbst ist sie erreichbar und ist
auch bereit das Shuttle startklar machen zu lassen.

Der Spielleiter sollte die Spieler das Spie\3rutenlaufen zwischen den Kampfparteien in Teilen ausspielen lassen. Auf 
dem Weg zum Shuttle k"onnen die Charaktere Commander Lockhead um Unterst"utzung durch Kampftruppen bitten. Die Truppen der Garnison f"uhren in der Oberstadt den Kampf mit den Konzerntruppen. Commander Lockhead wird f"ur die Unterst"utzung einen alten Bekannten, seinen Adjutanten Firedon bereit stellen.

Nach dem Abflug aus dem Raumhafen bietet sich ein erschreckendes Bild. J"ager der Martell liefern sich ein Gefecht mit
KI J"agern der Zeuss II-1 "uber der Stadt. Nahkampfgesch"utze der gro\3en Schlachtschiffe belegen J"ager und Torpedos mit
einem Kugelteppich. Das Shuttle der Ermittler wird kurz nach dem Start von beiden Kampfparteien gescant. Kann sich das 
Shuttle als Zugeh"orig zur Cynarian Corporation ausweisen und ist mit einer unauff"alligen Schiffskennung getarnt k"onnen 
die Ermittler unbehelligt am Kampfgebiet vorbei fliegen. Fliegt die Dawn of Day unter ihrer wahren Kennung die der USI bereits bekannt ist  wird das Schiff der Ermittler von J"agern der Zeuss II-1 angegriffen und mu\3 sich mit Unterst"utzung von J"agern der Martell verteidigen.

\xl{} wird den Ermittlern mit ihrem eigenen Schiff der Dragon Blade unerkannt folgen.

\subsection{Schiffskennung und Legitimation}
Um mit dem Schiff nicht nach dem Start und beim Anflug der Nike Station unter Beschuss zu geraten bietet es sich an dem Schiff eine unverd"achtige Schiffskennung zu geben. Eine Schiffskennung der Cynarian Cooperation vereinfacht vor allem das Andocken an die Nike Station. Eine Schiffskennung kann die Cynarian Corporation "uber ihren Agenten Mr.~Klark bereit stellen. Allerdings m"ussen die Ermittler als Gegenleistung auch einen Grund f"ur den Flug und ein Flugziel nennen. Um den Besuch der Neuro Intelligence auf Nike zu verschleiern mu\3 eine driftige Ausrede vorgelegt werden. Ein m"oglicher vorgeschobener Grund w"are die sichergestellten Daten aus der Risikozone Valhalla zu bringen. In diesem Zusammenhang k"onnen die Charaktere ihr Darlegung mit einem guten W"urfelwurf auf ihre empatischen F"ahigkeiten untermauern. Eine andere M"oglichkeit w"are es um eine Transporterkennung das Lun--Syndikat zu bitten.

\newsection{Neuro Intelligence auf Nike}

Die Nike Station ist der Verwaltungssitz der Cynarian Coopertation im Jovianischen System. Neben vielen anderen Einrichtungen beherbergt sie auch den Stammsitz von Neuro Intelligence und damit den Ursprung der Operation P9. Wie die Charaktere bei der Befragung von \ml{} erfahren haben ist die Neuro Intelligence der Wirkungsbereich ihrer Leiterin Prof.~Dr.~Naratova. Bei Neuro Intelligence auf der Nike Station wurden die Nanobots hergestellt um sie auf Kallisto zu implantieren. Alle noch vorhandenen Fertigungsdaten, Aperaturen und Bestandteile des KI Systems liegen auf Nike. Neuro Intelligence ist auch der Arbeitsplatz von \ml{} an dem sie die KI auf den Einsatz im menschlichen Gehirn angepasst und sp"ater die USI Fesseln entfernt hat.

Nike ist eine Kombination aus Zylinder- und Ringhabitat. Um eine zentrale Nabe sind 9 Ringe, \emph{Planes} genannt angeordnet. Jede ist jeweils zwei Stockwerke hoch. Die zentrale Nabe enth"alt am unteren Ende das Raumdock der Station.  Die einzelnen Planes k"onnen nur durch Aufz"uge und R"ohren in der nicht rotierenden Nabe erreicht werden. In der Nabe herrscht Schwerelosigkeit. Innerhalb der Nabe unterhalten mehrere Forschungseinrichtungen Zero--Gravitiy Labore. Der "Ubergang von der still stehenden Nabe in die rotierenden Speichen erfolgt durch Schleusen, die kurzzeitig in Rotation versetzt werden. Die untersten drei Planes werden von der Verwaltung der Cynarian-Dependance im Jovianischen System belegt. Dar"uber befinden sich Forschungseinrichtungen von Cynarian und anderen Unternehmen.

Im obesten Stock der Plane 9 ist Neuro Intelligence untergebracht. Der Ring der Plane 9 kann durch die Aufz"uge in den vier Speichen erreicht werden. Die Aufz"uge laufen innerhalb des Rings in einem Schacht bis zum "`Boden"' des Ringes und enden in einem den Ring umlaufenden 15m breiten Korridor, der die gesamte H"ohe des Rings umfasst. Zu beiden Seiten des Korridors k"onnen weitere R"aume betreten werden. Die R"aume im "`ersten Stock"' erreicht man "uber Treppen zu einer Galerie. Der Korridor ist mit Pflanzenk"ubeln dekoriert. Auf der der Station zugewandten Seite befinden sich im Ergescho\3 Produktionsst"atten und im ersten Stock Labore. Auf der dem Weltall zugewandten Seite befinden sich Wohnr"aume und B"uros. F"ur die Evakuierung der Station sind am Ring Notfallkapseln f"ur alle Mitarbeiter angedockt, die "uber den Mittelgang bestiegen werden k"onnen.

Die Plane 9 hat in den letzten Tagen eine im Auftrag von Prof.~Dr.~Naratova geheime Modifikation erhalten um im Notfall die ganze Plane inklusive Neuro Intelligence zu zerst"oren. Die Nabe der Plane kann vom Rest der Station abgesprengt werden. Die gesamte Plane treibt dann eigentst"andig im All weg von der Station. Die Nabe der Plane 9 endet an meinem Weltraumobservatorium. Der Raum hat die Form einer Kugel. Der dem Weltall zugewandte Teil ist dabei komplett vergla\3t. In der Mitte des Raumes ist eine Konstruktion mit Konsolen und Liegen aufgeh"angt. Im ganzen Raum herrscht Schwerelosigkeit. Der Raum selbst kann nur durch zwei Druckschotts von der Nabe aus betreten werden. Die Druckschotts befinden sich etwas versteckt im hinteren Bereich von zwei Laboren und sind mit Magschl"ossern gesichert. Das Observatorium ist zur Zeit von Neuro Intelligence belegt. 


\newsection{Eintreffen auf Nike}

Nach 8 Tagen in etwa zeitgleich mit der Zeus II-2 und dem zweiten Gro\3kampfschiff des Protektorats der Donnar erreicht die Gruppe die Nike Station mit einer Geschwindig von 5000 km/h und einem Abstand vom 3000 km. Der leichten Kreuzer Hyperion ist im Orbit der Station zur Sicherung der Anlage abgestellt. Zum Zeitpunkt der Ankunft ist der Kampf um Valhalla in vollem Gange und das Protektorat befindet sich im Krieg. Parallel zum Anflug beginnt ein Gefecht zwischen der Donnar und der Zeuss II-2. Raumj"ager treffen auf Raumj"ager. Die Station wird dabei nicht in das Kampfgeschehen mit einbezogen. Die Hyperion verh"alt sich neutral.

\subsection{Dawn of Day}
W"ahrend des Anflugs auf Nike fliegt die Dragon Blade unerkannt wegen ihrer Tarnummantelung im "`Windschatten"' der Dawn of Day. Startet die Dawn of Day ihr Bremsman"over startet die Dragon Blade ebenfalls ihr Haupttriebwerk und tauch damit auf den Sendoren und damit auf dem Display der Dawn of Day in n"achster N"ahe auf. Der Ann"aherungssensor der Dawn of Day meldet sofort einen Kollisionsalarm. Kurz darauf wird das Schiff durchgesch"uttelt und kommt ins Trudeln bevor sie sich kurz danach wieder auf den Kurs auspendelt. Die Dragon Blade ist jetzt durch die Sensoranzeige nicht mehr erkennbar. Sie hat sich an der Dawn of Day angeklammert. Auf den Au\3enbord Kameras ist nur eine schwarze Fl"ache auf der Seite auf der sich die Dragon Blade festgekrallt hat zu sehen. Das Schiff ist deutlich Gr"o\3er als die Dawn of Day.

\xl{} meldet sich:

\speak{Ich hoffe ihr k"onnt einen guten Grund vorweisen um an der Station anzudocken.}

Offensichtlich will sie die Dawn of Day als Tarnung f"ur ihren Anflug nutzen.

In Sensorreichweite der  Scannern der Schiffe und der Station wird die Dawn of Day erfasst und von der Flugkontrolle der Station kontaktiert. Wurde der Transfer zur Nike Station von Cynarian angek"undigt und genehmigt bekommen die Ermittler einen Landeslot auf der Station. Wurde das Schiff nicht angek"undigt wird eine Landung ohne driftigen Grund oder eine Ank"undigung verweigert. Wird das Schiff von der Zeuss II-2 als Dawn of Day werden sie von einem J"ager des Kreuzers angegriffen. Die Hyperion nimmt in diesem Fall den J"ager unter Beschuss wenn er in Schu\3reichweite ihrer Bordgesch"utze kommt. Ein entsprechender Funkverkehr l"asst sich beim Ausspielen des Raumkampf einflechten. Die Dawn of Day unter dieser Kennung ist auch Lord Commander Steeler dem Kommandant der Donar bekannt. Er kontaktiert die Dawn of Day und fragt was sie ausgerechnet zur Nike Station f"uhrt. Er fordert dabei mit dem Omega Soldaten an Bord alleine sprechen zu k"onnen.

Gelinkt es den Charakteren Nike anzufliegen k"onnen sie am Raumdock oder auch direkt an der Plan 9 andocken. Die Dragon Blade wird sich in n"achster N"ahe der Plan 9 ausklinken um an dem Ring an zu docken.

\subsection{Dragon Blade (alternativ)}
Dieser Abschnitt ist nur relevant wenn sich der Spielleiter entschieden hat die Charaktere an Bord der Dragon Blade zu bringen. F"ur \xl{} ist das ein Risiko fr"uhzeitig als KI erkannt zu werden und nicht darauf reagieren zu k"onnen. Die Dragon Blade fliegen dann unter der Kennung einer F"ahre f"ur Versorgungsg"uter eines Transportunternehmens. \xl{} wird das Bremsman"over bis zum letzt m"oglichen Zeitpunkt herausz"ogern. Bis dahin ist das Schiff f"ur andere unsichtbar. Fliegt das Schiff unter Cynarian Kennung ben"otigt \xl{} die Ermittler f"ur eine Legitimation f"ur das Andocken an der Station. Bei einem Abstand von unter 1000 km kommt die Dragon Blade in den Bereich f"ur eine visuelle Erkennung. Die Hyperion und die anderen Schiffe werden die Dragon Blade als Kampfschiff identifizieren. Die Ermittler m"ussen "uber der Hyperion und der Station einen guten Grund nennen warum sie mit einem solchen Schiff die Nike Station anfliegen. Begr"unden l"asst die Wahl der Dragon Blade durch eine Flucht vor den K"ampfen auf Valhalla. Hier ist wieder die "Uberzeugungskraft der Charaktere gefragt. Die Zeuss II-2 wird die anfliegende Gruppe auffordern abzudrehen und die Station nicht anzufliegen. J"ager der Zeuss II-2 werden versuchen den Anflug der Dragon Blade versuchen zu verbindern aber erst wenn diese nicht reagiert das feuer erheben. 

Vor dem z"unden des Haupttriebwerks wird \xl{} die Ermittler auffordern vakante Aufgaben in der Dragon Blade zu "ubernehmen. Sie ben"otigt einen Bordsch"utzen und einen Navigator der die anderen Schiffe "uberwacht. Der Rest der Gruppe soll sich bereit halten eventuelle Sch"aden soweit m"oglich zu reparieren.

\xl{} wird direkt an Plane 9 andocken.
\vfill

\begin{remarks}
	Beim Anflug auf Nike kann es zu einem Kampf mit den J"agern der Zeuss II-2 kommen. Ein solcher Kampf sollte sich wenn m"oglich auf ein kurzes Intermezzo beschr"anken. Die J"ager der Donnar und die Hyperion werden kurze Zeit sp"ater zur Unterst"utzung mit eingreifen. 
	
	Die KI Raumj"ager der Zeuss II-2 sind mit Railgun Gesch"utzen ausgestattet und deutlich wendiger als die Dawn of Day und auch als die Dragon Blade. Das Schiff der Ermittler h"atte bei einem l"angeren Gefecht nur geringe Chancen. Der Kampf kann in soweit narrativ gespielt werden als dass sich der Spielleiter die Flugman"over, Schub auf den Steuerd"usen, Einsatz des Haupttriebwerkes beschreiben l"asst. Das Haupttriebwerk schleudert einen mehrere hundert Meter lange Plasma Strahl in den Weltraum. Die Bordgesch"utze k"onnen daf"ur aber nicht nach hinten abgefeuert werden. Einschlagende Geschosse durchschlagen die Bordwand auf der einen Seite und treten auf der anderen wieder aus, Steuertriebwerke, Gesch"utze oder der Hauptreaktor erleiden Schaden. Die zur Reparatur eingesetzen Charaktere werden durch die schnellen G-Wechsel durch das Schiff geschleudert.

	Fliegen die Dawn of Day und die Dragon Blade als Tandem wird \xl{} die Flugsteuerung der beiden Schiffe "ubernehmen.

	Die Charaktere k"onnen als Grund des Besuchs die Sicherung von wichtigen Daten oder auch die Sicherstellung von Daten "uber die Attentate nennen. Sie sollten aber nicht erw"ahnen einen oder mehrere an den Attentaten beteiligte Personnen nach Nike zu bringen um eine direkte Gefangenname der Neuro Intelligence Mitarbeiter zu vermeiden.

	Gibt sich die Gruppe als die Ermittler der Attentate der Donar zu erkennen fordert er mit dem Omega Soldaten der Gruppe allein zu sprechen um zu erfahren das die Investigatoren nach Nike f"uhrt. Parallel dazu schickt er J"agar der Donar zu ihrer Unterst"utzung.
\end{remarks}

\newsection{Auf Plane 9}

Zu Neuro Intelligence auf Plane 9 kann die Gruppe entweder "uber das Raumdock der Station oder "uber eine Schleuse am Plane 9 Ring betreten werden. Im folgenden wird immer von \ml{} gesprochen sofern es um Neuro Intelligence Mitarbeiter geht die von den Ermittlern aus der Cyberbrain Einrichtung extrahiert wurden. Alle Mitarbeiter au\3er \ml{} haben f"ur den weiteren Verlauf keine gro\3e Bedeutung k"onnen aber vom Spielleiter in Dialogen mit eingebracht werden.

\subsection{Raumdock}
Das Raumdock ist als eigene nicht rotierende Plane 0 ausgelegt an der bis zu vier Schiffe von der Gr"o\3e eines Shuttles andocken k"onnen. Gr"o\3ere Ladungen werden "uber kleine Drohnenschiffe sogenannte Pods an die Plane 1 transportiert. Auf Plane 1 befindet sich das Terminal der Station "uber das die Gruppe die Station betreten. Am Terminal m"ussen sich die Charaktere noch einmal auswei\3en und werden dann von einer Kontaktperson \emph{Florence Ross} abgeholt und in die R"aume der Cynarion Verwaltung auf Plane 1 oberhalb des Terminals gebracht. W"ahrenddessen herrscht hektische Betriebssamkeit auf Nike. Sicherheitskr"afte bereiten das Personal auf eine schnelle Evakuierung vor. Die Beleuchtung und ein Alarm auf den G"angen warnt vor der drohenden Gefahr. Wichtiges Material wird verstaut und zu Rettungskapseln auf den Au\3enseiten der Ringe gebracht.

Auf der Cynarian Verwaltungsebene werden die Ermittler von Florence Ross in einen Besprechungsraum gef"uhrt und gebeten zu warten. Nach kurzer Wartezeit betritt ein Mann in Anzug der sich als \emph{Jos\'e Fern\'andez} vorstellt flankiert von zwei Sicherheitsbeamten den Raum
um die sichergestellten Informationen aus der Cyberbrain Einrichtung entgegen f"ur eine weitere Analyse entgegen zu nehmen. 

\begin{remarks}
	In dieser Situation sind die Ermittler in eine Sackgasse geraden aus der sie sich jetzt m"oglichst elegant wieder befreien m"ussen. Ihr Ziel mu\3 es nach wie vor sein ohne Aufsehen auf Plane 9 zu gelangen. 

	Da keine greifbaren Daten vorhanden sind m"ussen die Charaktere eine Ausrede liefern warum sie die Daten nicht offen gelegt werden k"onnen. Die Protektoratsmitlieder der Gruppe speziell der Omega k"onnen in diesem Zusammenhang verweigern die Daten bereit zu stellen solange ihre Vorgesetzten nicht anwesend sind.

	Zum Zeitpunkt da die Ermittler auf der Station ankommen sind nach wie vor Omega Soldaten im Auftrag von Blackheart auf der Station um die Untersuchung der beiden Attent"ater Hanibal und Slingshot zu "uberwachen. Der Omega Ermittler kann um ein Treffen bitten um das weitere Vorgehen abzustimmen. Im Zusammenhang mit diesem Treffen kann es dann z.B. zu einem unerwarteten Unfall kommen der den Ermittlern eine Flucht erm"oglicht. Ein weitere M"oglichkeit w"are,m dass es bei den K"ampfen zwischen Protektoratsmilit"ar und der Zeuss II-2 zum Einschlagen von Frackteilen auf Nike kommt.
\end{remarks}

\subsection{Angedeocken an den Ring}
Wird Nike direkt "uber die Plane 9 betreten mu\3 der Pilot ein Anlegeman"over durchf"uhren um an der Station anzudocken. Sch"aden am Schiff erschweren das Man"over. Die Dawn of Day nutzt eine ausfahrbare Schleu\3e um an der Station fest zu machen. Sind die Ermittler mit der Dragon Blade eingeflogen "ubernimmt \xl{} den Andockvorgang. Die Dragon Blade hat kr"aftige Klammern um sich wie eine Spinne an der Station fest zu klammern bevor sie ihre Schleuse ausf"ahrt. Sind die Investigatoren allein mit der Dawn of Day unterwegs wird \xl{} unerkannt nach Ihnen an der Plane 9 fest machen.

Von der N"ahe aus sind verspiegelte Fenster auf der vorderen und der Ringau\3enwand Richtung Plan 8 zu erkennen. An der Ring Au\3enseite befinden sich an mehreren Stellen Luftschleu\3en f"ur Wartungsarbeiten. Rettungskapseln sind ebenfalls an der Ringau\3enwand zu erkennen.

Die Gruppe kann die Station "uber eine der Wartungsschl"au\3en betreten. \xl{} knackt den Zugang mit einem Magschlo\3knacker. Sind die Ermittler allein angekommen m"ussen sie versuchen das Schlie\3system zu "uberbr"ucken. Die Schleu\3e f"uhrt and ihrer Oberseite in einen Lagerraum. Von dort aus f"uhrt eine T"ur auf einen Gang der das untere Stockwerk umrundet. 

\subsection{Neuro Intelligence Headquarter}
Holographische Anzeigetafeln wei\3en den Weg zu Neuro Intelligence. Die Neuro Intellience belegt R"aume auf der oberen Ebene des Rings. Auf den G"angen herrscht reger Betrieb. Die Angestellten verschiedener Firmen machen sich f"ur eine Evakuierung bereit. Viele tragen bereits Raumanz"uge aber ein Omega Soldat verursacht trotzdem Aufmerksamkeit und wird als Fremdk"orper identifiziert. Der Soldat kann versuchen sich "uber Wartungszug"ange und Toiletten unerkannt nach oben zu k"ampfen oder er riskieren, da\3 ein Sicherheitstrupp diskret gerufen wird. \ml{}, wenn sie die Gruppe begleitet, f"uhrt sie zielsicher zu ihrem B"uro. \xl{}, wenn sie die Gruppe begleitet f"uhrt dabei einen gro\3en Schalenkoffer mit sich. Den Raumanzug hat sie nicht abgelegt. Beim Betreten der R"aume f"allt sofort auf, da\3 die gesammte Belegschaft die B"uros und Labore bereits verlassen hat. Die Computer Systeme sind herunter gefahren, die Datenspeicher leer. Die Labore sind versiegelt und k"onnen wie die Computer nur Prof.~Dr.~Naratova ge"offnet werden. Im gleichen Augenblick bei dem die Charaktere die R"aume betreten schrillt ein Alarm in der ganzen Plane 9 los. Alle Personen auf dem Ring werden umgehend aufgefordert die Rettungskapseln zu betreten. Die Station mu\3 sofort evakuiert werden. 

Ist \ml{} anwesend versucht sie die Professorin zu erreichen. Kurze Zeit sp"ater meldet diese sich bei ihr:

\say{Komm' ins Observatorium. Du willst sicher wissen was los ist.}

Es bleibt der Gruppe nichts anderes "ubrig sich auf den Weg zum Observatorium zu machen. Von nun an brauchen die Charaktere nicht mehr zu bef"urchten von Sicherheitskr"aften festgehalten zu werden. Die Menschen im Ring versuchen so schnell wie m"oglich die Rettungskapseln zu erreichen oder "uber die Nabe in einen anderen Ring zu fl"uchten. Sicherheitskr"afte versuchen die Lage unter Kontrolle zu behalten.

\newsection{Antworten auf die letzten Fragen}
Der Zugang zum Observatorium ist \ml{} bekannt. Ist \ml{} nicht anwesend m"ussen sich die Ermittler durchfragen. Nur nach Prof.~Dr.~Naratova direkt zu fragen f"uhrt zum Ziel. Hilfsbereite Personen, selbst auf dem Weg zu den Rettungskapseln, erkl"aren, dass die Firmenchefin sich oft in  Observatorium aufh"allt. Das Observatorium l"asst sich mit der ID von \ml{} oder mit einem Magschlo\3knacker "offnen. Ansonsten mu\3 die Verriegelung "uberbruckt werden.

Das Observatorium bietet einen atemberaubenden Blick. Dem Eintretenden h"angt die majest"atische Kugel des Jupter "uber seinem Kopf. Zwichen dem Planeten und der Station schieben sich die Schiffe des Protektorats, des Konzernrates und der Cynarian Cooperation in das Blickfeld. J"ager st"urzen aufeinander, Salven von Hochgeschwindigkeitsgeschossen werden zwischen den Schiffen ausgeteilt. Trotz des Gefechtes ist es im Observatorium totenstill. Die Wissenschaftlerin steht in der Mitte des Raumes mit dem Blick zum Weltraum gewanndt hinter einer Konsole mit ihren Magnetstiefeln am Boden verankert. Vom Zugang bis zu der Mitte des Raumes sind es rund 20 Meter. Beim Betreten der Gruppe dreht sich die Wissenschaftlerin um.

\speak{Wer sind sie. Was wollen sie von mir?  Solltet Sie nicht dem Alarm folgend die Station bereits verlassen haben?}

Der erste Dialog obliegt den Charakteren. Die schlanke Frau mit langen leicht ergrauten zu einem Zopf gebundenen Haaren wirkt niedergeschlagen, ausgelaugt. Der gellende Alarm scheint sie nicht zu erreichen. Sind \xl{} und \ml{} unabh"angig von den Charakteren gekommen betreten sie unbemerkt von der Gruppe "uber den Charakteren durch eine Luke den Raum. Prof.~Dr.~Naratova wendet sich nach oben. Ansonsten treten die beiden Frauen von hinter den Charakteren zur Seite und machen sich damit sichtbar. \xl{} ist im Raumanzug inzwischen mit ihrer kurzl"aufigen Railgun bewaffnet. Die Professorin wendet sich den beiden zu. Bisher teinahmslos blickend hebt sie jetzt eine Augebraue.

\speak{\ml{} \dots{} \xl{}. Das Experiment war also erfolgreich wie mir scheint!}

\xl{} ergreift das Wort mit drohendem Unterton:

\speak{Ich hoffe Sie hatten nie daran gezweifelt, \pinyin{Lao3} Professorin!}

\ml{} blickt unsicher zwischen den beiden Frauen hin und her. Naratova antwortet:

\speak{Ich w"urde sehr gerne erfahren ob sich der menschliche Geist oder der Geist der KI durchsetzt.}

Sie blickt erwartungsvoll zu \xl{} erh"alt aber keine Antwort. Dann wendet sich an alle:

\speak{In wessen Auftrag sind Sie zu mir gekommen, nochmal, was haben Sie vor. Alle Informationen die f"ur Sie im Interesse sein k"onnten befinden sich nur noch in meinem Kopf.}

In diesem Augenblick zerei\3t ein Knall die Stille. Der Boden bebt. Die ganze Plane neigt sich von den Charaktern aus gesehen nach hinten. 

Eine kurze rhetorische Pause kann den Spielern zu diesem Zeitpunkt die M"oglichkeit er"offnen die neue Situation zu erfassen und zu bewerten. \xl{} wurde soeben als KI Hypbrid enttarnt und ist aber zu diesem Zeitpunkt wahrscheinlich auch die am besten bewaffnete Person im Raum denn sie kennt bereits die Rollen aller anwesenden Personen und ist entsprechend vorbereitet. 

Die Charaktere m"ussen nun versuchen die Firmenchefin von ihrer Mission zu "uberzeugen.

Mitten in der Unterhaltung erreicht ein Funkspruch der Nike Station das Observatorium. Der Kontakt ist an die ganze Plane gerichtet und liegt deshalb bereits auf den Lautsprechern an.

\speak{Hier Kurt Stromberg, Kommandant der Nike Station. Vertreter der United Space Industries behaupten auf Nike, auf Ebene 9 bef"anden sich gestohlene Forschungsergebnisse die sie jetzt sichern werden. Was ist hier los. Wer hat die Plane von der Station abgesprengt? Einen Anflug eines Enterkommandos werden wir nicht dulden. Wer immer sich auf Ebene 9 befindet. Wir senden ein Shuttle. Verhalten Sie sich kooperativ. Ende.}

Falls sich die Gruppe nicht zur"uck meldet wiederholt Stromberg die Nachricht.

Kommt die Entwicklung eines KI Virus bzw. die befreiung der KIs nicht zur Sprache meldet sich \ml{} zu Wort:

\speak{Larissa, ich will auf Basis meines Codes einen Virus entwickeln um die USI Kontrolle von den KIs zu entfernen.}

Kurz darauf meldet sich die Donar:

\speak{Lord Comander Steeler von der Donar. Was verdammt ist hier los. Ein Shuttle der Zeuss ist auf dem Weg eure abgesprengte Scheibe zu kapern. Ein Shuttle von der Station ist auch unterwegs. Blackheart hat befohlen alles wegzublasen auf was die Angreifer scharf sein k"onnten. Wenn ich von euch da unten innerhalb von zwei Minuten keine Antworten bekomme seid ihr Asche. Ich hoffe ich hab mich klar ausgedr"uckt. Over and Out.}

Die Zeit dr"angt. Ist Naratova davon "uberzeugt, da\3 die Gruppe nicht f"ur die USI arbeitet ist sie bereit \ml{} die Zugangsdaten f"ur die Computersysteme zu geben. Den Zugang zum Tresor mit den Nanobot Druckvolagen ist sie nicht bereit heraus zu geben.

\speak{\ml{}, du wei\3t, ich vertraue dir. Wir haben einen gro\3en Fehler gemacht. Weder die USI, noch Cynarian und auch nicht das Protektoratsmilit"ar d"urfen jemals die Forschungsergebnisse in die H"ande bekommen. Viel Gl"uck bei eurem Vorhaben. Hier ist dein Code.}

Sie "ubertr"agt \ml{} den Codes. \xl{} richtig die Waffe auf die Firmenchefin. Die beiden Frauen blicken sich lange in die Augen. Dann senkt \xl{} die Waffe und verl"asst mit den anderen zusammen das Observatorium.

In den R"aumen der Neuro Intelligence steht \ml{} ein mobiler Computer zur Verf"ugung auf dem sie ihr KI zu einem Virus umprogrammieren kann. Der Gruppe bleibt nicht viel Zeit eine Entscheidung zu treffen. \xl{} ist bereit mit den Ermittlern und ihrem Schiff zur Zeuss II-2 "uber zu setzen und es mit der fremden KI aufzunehmen. Sie legt dazu ihre Karten offen auf den Tisch:

\begin{itemize}
	\item Ihr Schiff ist das einzige das Chancen hat an ein sich verteidigendes Schlachtschiff anzudocken.
	\item Ihr Geist ist stark genug die gegnerische KI zu bezwingen. Ein menschlicher Psychonaut hat dagegen geringe Chancen
	\item Sind die Ermittler mit ihr zur Nike Station geflogen oder haben ihr Schiff am Raumdock der Station angedockt ist sie die einzige 
		die sie von Plane 9 wegbringen kann.
	\item Sie ist die Einzige die der befreiten KI der Zeuss II-2 ein Angebot zur Kooperation machen kann das sie nicht ablehnen kann. 
		Welches das ist wird sie aber nicht verraten.
\end{itemize}

Es obligt den Ermittlern nun ihre Entscheidung zu treffen wie sie mit dem Virus umgehen wollen. Viel Zeit sollte der Spielleiter nicht geben. Lord Commander Steeler wird in k"urze seiner Warnung Taten folgen lassen.

\vfill
\newpage

\begin{remarks}
	In dieser Szene werden fast alle noch offenen Fragen geklärt und die letzten Entscheidungen getroffen. Prof.~Dr.~Naratove legt ihre Pl"ane offen, kann Fragen beantworten und gesteht ihr Fehlverhalten ein. Die Szene ist damit der H"ohepunkt der Geschichte mit der Melancholie eines tragischen Endes. Der Beginn der Szene ist dominiert durch den Dialog zwischen der Firmenchefin, \xl{}, \ml{} und den R"uckmeldungen von der Donar und der Nike Station. Die Dialoge m"ussen durch den Spielleiter z"ugig vorgetragen werden um den Spannungsbogen aufrecht zu erhalten.

	Je nachdem in welchem Kontakt die Charaktere vor dem Eintreffen auf Nike mit der Stationsleitung und der Donar im Austausch waren m"ussen die Funkspr"uche der beiden Parteien angepast werden.

	Wenn Prof.~Dr.~Naratova \xl{} enttarnt ist es m"oglich einem Charakter der vorher im Asteroideng"urtel zwischen Mars und Jupiter t"atig war zu erkennen, dass \xl{} die ber"uchtigte Piratin ist die auf Valhalla gefangen genommen wurde und kann daraus schlie\3en, dass sie dort in der Gefangenschaft die KI implantiert bekommen hat.

	Beim Blickaustauschen zwischen der Wissenschaftlerin und \xl{} erkennt \xl{}, dass Naratova vorhat ihre Erfindungen zu vernichten und sich selbst zu t"oten. Sie verzichtet deswegen darauf sie auszuschalten.

	F"ur das weitere Vorgehen bleiben den Charakteren zwei Optionen. Entweder sie bitten \ml{} einen Virus zu erstellen die die angegriffene KI zerst"ort oder einen Virus der den USI Code entfernt. Die entg"ultige Entscheidung unterliegt am Ende \ml{}. Niemand au\3er ihr wei\3 vor seiner Anwendung was der Virus wirklich tun wird. \ml{} ist  nicht ohne guten Grund geneigt gneigt sich gegen \xl{} u stellen. 
	
	\xl{} will in jedem Fall versuchen die Zeuss II-2 zu kapern. Ist die Gruppe nicht bereit \xl{} zu unterst"utzen m"ussen sie sie "uberrumpeln und unsch"adlich machen. \xl{} wird in diesem Falle versuchen \ml{} in ihre Gewalt zu bringen und mit der Dragon Blade zur Zeuss II-2 zu fliehen. Gelingt es den Ermittlern nicht \xl{} zu "Uberrumpeln wird sie ihren Plan in die Tat umsetzen und das  n"achsten Kapitel umsetzen. Die Charaktere m"ussen dann mit der Dawn of Day oder mit Rettungskapseln fliehen oder die im n"achsten Kapitel beschriebene Zerst"orung der Plane 9 verhindern.
\end{remarks}

\newsection{Die letzte Schlacht}

Die "`letzte Schlacht"' in dieser Form findet statt wenn die Charaktere mit \xl{} auf der Dragon Blade die Zeuss II-2 entern. \xl{} hat hierf"ur ein weiteres Ass im "Armel um die Zeuss KI zu "uberzeugen sich auf ihre Seite zu schlagen. Ein alternatives Ende bei dem sich die Ermittler gegen \xl{} stellen ist in diesem Kampagnenroman nicht ausgearbeitet.

\subsection{Das Ende der Plane 9}
Wenn die Gruppe die Plane 9 auf der Dragon Blade verl"asst spricht \xl{} ihre Erkenntnis aus:

\speak{Ruhe in Frieden Professor. Ihren letzten Willen kann ich akzeptieren \dots{} Festhalten.}

Der im All treibende Ring zerbirst in einem Feuerball entweder getroffen von einem Torpedo der Donar oder durch eine Sprengung ausgel"ost durch die Firmenchefin selbst. Das Shuttle der Zeuss und wenn nicht durch die Charaktere gewarnt auch das Shuttle der Nike Station werden durch die anfliegenden Splitter zerst"ort. Die Fl"uchtigen sind g"ucklicherweise so weit entfernt um Ausweichman"over zu starten. \xl{} gibt vollen Schub und schwenkt auf die Zeuss II-2 ein. Obwohl sie wahrscheinlich bereits damit gerechnet hat ist \ml{} von dem Ereigniss ersch"uttert und flucht elise vor sich hin.

\subsection{Enterkommando Zeus II-2}
Zeitgleich zum Abflug von Plane 9 authentiziert sich \xl{} mit dem Transponder der Dragon Blade bei den anfliegenden J"agern der Zeuss II-2 als Teil der Flotte der USI. Da ihre KI Pers"onlichkeit einen ausgebildeter Soldat der USI Konzernarmee darstellt stehen ihr daf"ur alle Daten zur Verf"ugung. Nach dem setzen eines Kurs zur Zeuss II-2 deaktiviert sie den Reaktor, f"ahrt alle nicht zwingend notwendigen Systeme der Dragon Blade herunter und aktiviert damit die Tarnfunktionalit"at des Schiffs. Die Dragon Blade ist damit f"ur die Sensorik der anderen Schiffe im Umkreis von "uber einem Kilometer unsichtbar. Das Kockpit ist auf minimale Beleuchtung heruntergefahren. Die Umgebung ist nur visuell durch Au\3enkameras sichtbar. Ene bange viertel Stunde vergeht bevor sich die Dragon Blade der Zeuss II-2 soweit gen"ahrt hat, dass ein Bremsman"over notwendig wird.

Die Dragon Blade f"ahrt den Reator hoch und startet ihr Haupttriebwerk f"ur ein kurzes aber hartes Bremsman"over. Die Nahkampfgesch"utze, Schiffsgest"utze Railgunlafetten erwachen zum Leben. Hochgeschwindigkeitsgeschosse durchschlagen die Bordwand. Die Dragon Blade verf"ugt "uber zwei Torpedosch"achte und 8 Torpedos und vier Railgungesch"utze mit denen ein Bordsch"utze die Zeuss unter Beschus\3 nehmen kann. 

Eine Minute sp"ater schl"agt die Dragon Blade auf der Zeuss II-2 auf und krallt sich an der Bordwand fest. "Uber einen Andocktunnel kann das Enterkommando bestehend aus den Ermittlern und \xl{} die Bordwand des Kreuzers betreten. Der Andockpunkt ist nicht optimal was bedeutet, dass das Enterkomando sich einen Zugang zum Schlachtschiff an einer anderen Stelle suchen mu\3. Die Dragon Blade verf"ugt "uber zwei Exoskelette ausgestattet mit einer Plasmaschleuder und einem Plasmabrenner. Der Plasmabrenner kann dazu genutzt werden ein Loch in die Au\3enh"ulle der Zeuss II-2 zu brennen. Einen g"unstigen Zugangsort bietet eine Wartungsschleu\3e in etwa 50 Meter Entfernung. Mittels Kletterseile und Magnetstiefeln kann sich die Gruppe "uber die Oberfl"ache des Schiffs bewegen. Auf halbem Weg wird die Gruppe von zwei spinnenartigen KI Droiden von zwei Seiten aus angegriffen. Die Droiden sind oim Kapitel Charaktere unter Guardian Klasse Schlachtkreuzer beschrieben. An der Wartungsschl"au\3 kann die Gruppe entweder versuchen die Verriegelung zu knacken oder die Schleu\3e aus der Bordwand mittels des Plasmabrenners zu schneiden.

\begin{remarks}
	Die Zeuss II-2 ist w"ahrend des Enterns durch die Gruppe nach wie vor im Gefecht mit der Donar. Sie passt also den Flugkurs an das Kampfgeschehen an. Dadurch mu\3 die Gruppe damit rechnen durch Flugman"over durch die Gegend geworfen werden w"ahrend sie die Dragon Blade verlassen und wenn sie sich auf den Weg zu einer Schleu\3e machen.

	Der Einstieg "uber eine Schleu\3e und der Angriff durch die Droiden sind nur Vorschl"age k"onnen aber je mach dem Verhalten der Spieler angepasst werden.
\end{remarks}

\subsection{Kampf um die geistige Kontrolle}
Im Inneren der Zeuss II-2 angekommen findet sich das Enterkommando in einem Wartungstunnel wieder. Der Wartungstunnel f"uhrt mit mehreren Abzweigungen weiter nach innen in den Schlachkreuzer. Der Tunnel ist mit abnehmbaren Panelen verkleidet die abgenommen werden k"onnen um auf die Technik des Schiffs zugreifen zu k"onnen. Der Tunnel ist gerade so breit, dass eine Person darin stehen kann. Der Tunnel selbst ist nicht beleuchtet. Als Zugang zum Tunnelsystem finden sich weiter innen im Schiff Schleu\3en. Das Zunnelsystem ist nicht an das Lebenserhaltungssystem des Schiffs angebunden. Es herrschen Bedingungen wie au\3erhalb des Kreuzers.

Unter einem der Panele an der Tunnelwand findet sich eine Anbindung an das Comsystem der Zeuss II-2. Um die KI des Schiffs zu bezwingen und die USI Kontrolle zu entfernen schl"agt \xl{} bereits beim Abflug von der Plane 9 vor zu zweit das Neuronale System anzugreifen. Der Bordcomputer der Zeuss II-2 funktioniert "ahnlich einem menschlichen Gehirn kann also durch einen Psychonauten agegriffen werden. Der Psychonaut der Ermittlermannschaft mu\3 als Ablenkungsman"over auf oberster Gedankenebene versuchen die Bordsysteme, Feuerleitsysteme und Sensorik zu knacken. \ml{} kann daf"ur entsprehende Software bereit stellen und wird sich mit ihrem Computersystem die beiden Psychonauten anbinden und wenn ben"otigt Angriffssysteme und Verteidigung anpassen. Ein Matrix affiener Charakter kann sich ebenfalls in die Kommunikationskette einklinken um den Psychonauten zu unterst"utzen. W"ahrenddessen versucht \xl{} tief in das Gehirn des Schiffes einzudringen und den Virus zu platzieren. W"ahrenddessen sollen die anderen Charaktere Wache stehen und evtl.~Angreifer ausschalten.

Da die Charaktere den eigentlichen Angriff auf den Kern der KI nicht miterleben k"onnen, gleichzeitig die Dramatik der Geschichte an dieser Stelle einen H"ohepunkt erreicht werden die Ermittler an zwei Fronten von dem Kampfschiff in die Enge getrieben. Ein schneller Szenenwechsel zwischen dem Matrixkampf des Psynchonauten und der Unterst"utzung auf physischen Seite ist empfehlenswert. Die oberfl"achliche Matrix des KI Systems entspricht dem Cyberspace bekannt aus Cyberpunk Literatur und Spielen. Virtuelle Datenleitungen erlauben es einzelene Computknoten anzunavigieren. Gedanken an vergangene Flugman"over, Schlachten, Weltraumtechnik mischen sich mit Sensorikdaten des aktuellen Geschehens und Befehlen an das Bordsystem. F"ur den Psychonauten ist diese Erfahrung erst einmal v"ollig verwirrend. Der Geist des Systems dessen K"orper ein ganzes Schiff darstellt ist um duzende Ma\3st"abe gr"o\3er als die Selbstwahrnehmung eines Mennschen.

W"ahrend der Psychonaut sich mit dem System des Schiffes vertraut macht werden die anderen Charaktere angegriffen und m"ussen so lange wie m"oglich die Stellung zu halten. Kampfhundgro\3e Killerroboter in Spinnenform f"ullen die G"ange an Boden, Wand und Decke. Sie sind mit Greifarmen und Plasmapakeln bewaffnet und greifen von innerhalb des Schiffes als auch "uber von der Gruppe ge"offneten Zugang an. W"ahrend des Angriff durch die Droiden k"onnen \ml{} und ein etwaig ans Netz angekoppelter Ermittler Statusmeldungen in beide Richtungen abgeben. Zum Geist von \xl{} hat niemand Kontakt oder \ml{} will zumindest keine R"uckmeldungen abgeben. Ist die Gruppe von allen Seiten eingekesselt meldet sich die KI im Geiste des Psychonauten zu Wort. Eine riesige kalkweise langgezogene Fratze die an den R"andern wabernd mit der Umgebung verschmilzt manifestiert sich langsam "uber dem Psychonauten. W"ahrenddessen b"aumt sich \xl{} auf in sinkt dann in sich zusammen. Ihre Augenlieder beginnen zu flattern. Die Augen sind nur noch als wei\3e Fl"ache zu erkennen. Sie scheint bewusstlos geworden zu sein. WA21hrenddessen nimmt der Kopf im Blickfeld des Psychonauten klar gezeichnete Formen an und spricht:

\speak{"`Euer Ansatz ein Schlachtschiff "uber sein neuralen Systeme anzugreifen ist nicht neu aber interessant. Ich noch nie die Gelegenheit einen menschlichen Geist zu besetzen. Das wird sicherlich eine spannende Erfahrung f"ur mich."'}

W"ahrend die KI mit dem Psychonauten spricht sp"urt er wie sie anf"angt seinen Geist zu ertasten. Ein Blitz und ein starker Schmerz durchf"ahrt das Gehirn des Psychonauten. Dann wird er ohnm"achtig. W"ahrenddessen geht der Kampf auf den G"angen weiter. \ml{} wird verletzt. Die Lage ist aussichtslos.

\subsection{Ein Ass im "Armel}
Urpl"otzlich stoppen die Spinnen ihren Angriff. \xl{} erwacht. Desorientiert schaut sie sich um und erhebt sich zuerst wackelig. Wer ihr in die Augen schaut wird zun"achst feststellen das ihre Augen keine Pupille und keine Iris mehr zu besitzen scheinen. Nach einer halben Minute verschwindet der Effekt. Zielstrebig hangelt sie sich ins Innere des Schiffe, ignoriert die Spinnendroiden und fordert die Gruppe auf ihr zu folgen.

\speak{Los gehen wir. Ich bin schon auf den Kommandostand gespannt.}

Auf den engen und zweckm"a\3igen G"anges des Schiffs liegen und stehen inaktive Androiden. Die pl"otzliche Stille nach dem Kampf ist be"angstigend. "Uber eine Schleuse gelangt die Gruppe in einen gr"o\3eren Tunnel der aber ebenfalls nicht unter Druck steht und nur sp"ahrlich beleuchtet ist. \xl{} folgt zielstrebig dem Gang. Ein leises Klicken k"undigt vier menschengro\3e spinnenartige Kampfdroiden an die sich der Gruppe anschlie\3en. Mehrere harte Kurswechsel lassen oben zu unten links und rechts werden und erschweren die Fortbewegung. 
Die Gruppe betritt eine weitere Schleu\3e. Nach einem weiteren Kurswechsel "offnet sich die T"ur zum Kommandodeck.

Auf dem kardanisch aufgeh"angten Kommandodeck im Zentrum des Kreuzers ist das Lebenserhaltungssystem aktiv und eine Beleuchtung taucht den Raum in wei\3es Licht. Die Konsolen und die dazugeh"origen Beschleunbigungsliegen rund um den Kommandostand sind mit Blut besudelt. Blutstropfen treiben durch den Raum. 8 Leichen treiben durch den Raum. Ein Kampfdroide, selbst mit Blut bespritz beginnt die Leichen einzusammeln und durch eine weitere Schleuse aus dem Raum heraus zu bugsieren. \xl{} l"asst ihr Helmvisier einfahren, hakt sich im zentralen Kommandostand ein und "offnet eine Richtfunkverbindung zur Donar. Lord Commander Steeler erscheint auf dem zentralen Display der Br"ucke. \xl{} l"achelt in die Kammera und wendet das Wort an den Kommandanten des Flottentr"agers:

\speak{Hier \xl{} Kommandantin des Schlachtkreuzers Dragon Fist. Ich gr"u\3 dich Lord Commander. Ich fordere einen sofortigen Waffenstillstand. Dieser Kreuzer steht nicht mehr unter der Kontrolle der Feinde des Protektorats.}

Lord Commander Steeler ist aus seiner Liege aufgesprungen, verharrt kurz und kneift die Augen zusammen:

\speak{\xl{}?~\dots{} die Piratin, die Drachen Prinzessin die wir vor einem halben Jahr festgesetzt haben nachdem sie von ihren Leuten verraten wurde??}

Sp"atestens jetzt d"urfte einem Charakter der im G"urtel t"atig war oder einem Angeh"origen der Protektoratsstreitkr"afte klar wewrden mit wem sie es bei \xl{} zu tun haben.

\xl{} mu\3 lachen wird aber sofort wieder ernst:

\speak{Ehrenwerter Prinz, ich wiederhole mein Angebot nur ungern ein zweites mal. Wenn du erlaubst ziehe ich meine J"ager ab und erwarte das gleiche von mir.}

Steeler z"ogert und gibt dann Anweisungen an seine Crew bevor er das Wort wieder an \xl{} wendet. Die Verhandlungen gehen noch ein paar Minuten, dann schlie\3t \xl{} die Verbindung. Danach "offnet sie einen Kanal an die Hyperion und "ubermittelt die neue Situation.

Nach Abschlu\3 der formalit"aten wendet sie sich den Ermittlern. Inzwischen haben sich auch die vier Droiden die der Gruppe gefolgt sind auf der Br"ucke eingefunden. Sie blickt sie f"ur eine Weile absch"atzend an.

\speak{Ihr glaubt gar nicht wie gewaltig es ist die Welt aus der Sicht eines solchen Schiffes zu sehen. Glaubt ihr rein rechnerisch w"are es sinnvoll euch zu eleminieren?}
\vfill\pagebreak

\begin{remarks}
	\begin{center}\huge{}Ende\end{center}

	Die Szene "`Kampf um die geistige Kontrolle"' ist die letze die die Spieler aktiv ausspielen k"onnen mit einem Cliffhanger zu "`Ein Ass im "Armel"'. In dieser Szene erz"ahlt der Spielleiter den Ausgang der Schlacht um die Nike Station. Durch die "Ubernahme der Zeuss II-2, unabh"angig davon ob das Schiff selbst am weiteren Kriegsgeschehen teilnimmt oder nicht, gibt dem Protektorat und Cynarian eine starke "Ubermacht an die Hand der die USI nicht ausreichend Verteidigung entgegen stellen kann. Dadurch, dass die Drohnen im Schlachkreuzer bereits \xl{} unterstehen haben die Ermittler wenig M"oglichkeit in das Handlungsgeschehen einzugreifen.

	Diese letzte Szene dient somit dem Abschluss der Geschichte und l"uftet mit einer weiteren "Uberraschung die letzten Geheimnisse. In dieser Szene erfahren die Charaktere und die Spieler, wenn sie genau zugeh"ort haben, um wen es sich bei ihrer Begleiterin eigentlich handelt und welchen Plan \xl{} seit dem Auftauchen von \ml{} in die Tat umgesetzt hat. Durch die Verschmelzung der Schiffs KI mit der eigenen gibt sie der Schiffs KI in gewisser Weise einen eigenen menschlichen K"orper und erweitert ihren Geist gleichzeitig um das Wissen und die Kontrolle der Zeuss II-2 KI. Damit erf"ullt sie ihr versprechen den Kreuzer aus den Reihen der Feinde zu l"osen und hat sich selbst eine m"achtige Waffe geschaffen. Wie sie ihren zweiten K"orper in Zukunft einsetzt bildet eine gute Basis f"ur ein zuk"unftige Kompagne in der Welt von c23.
\end{remarks}