%% Copyright 2019 Bernd Haberstumpf
%% License: CC BY-NC
% !TeX spellcheck = de_DE
\newchapter{Personenverzeichnis}

Im Folgenden die wichtigsten Personen mit denen es die Spieler im Spielverlauf zu tun haben. Die Personen sind gruppiert
nach Ort des Geschehens.

% Führungsteam
%% Copyright 2019 Bernd Haberstumpf
%% License: CC BY-NC
% !TeX spellcheck = de_DE
\newsection{Auftraggeber}

Die folgenden Personen beauftragen die Charaktere oder sind deren Kontaktpersonen bei der Untersuchung der Attentate auf Armageddon und Hellgate.

\newsubsection{Cynarian}

\begin{description}
    \item [Direktor Eric Vandermool] Eric Vandermool ist Auftraggeber seitens Cynarian und Leiter der Cynarian Unternehmungen im 
        jovianischen System. Er wird als Anw"arter auf einen Vorstandsposten gehandelt.
    \item [Henry Longdale] Henry  Longdale ist Vandermools Sekret"ar, "Ubermittler von Spezialauftr"agen, die Direktor Vandermool und 
        Colonel Scholz nicht kompromittieren oder in Verlegenheit bringen d"urfen.
    \item [Colonel Scholz] Colonel Scholz ist Sicherheitschef der Cynarian Corporation im jovianischen System und Kontaktperson der 
        Cynarian Ermittler.
    \item [Dr.~Petrova] Dr.~Petrova ist technische Leiterin der HE-3-Minen der Cynarian Corporation auf dem Jupiter.
\end{description}

\newsubsection{Protektorat}

\begin{description}
    \item [Protector Avenger] Avenger ist Auftraggeber seitens der Zivilregierung des Protektorats mit dem Spitzname ``Skip''. Protektor 
        Avenger ist sein Ehrenname seit Gr"undung des Protektorats. Er ist Pr"asident des Protektorats.
    \item [Artisan] Artisan ist die rechte Hand und Freund von Protector Avenger, Vorgesetzter, Vertrauter des Chefermittlers des 
        Protektorats und Kontaktperson des Ermittlerteams seitens des Protektorats. Er ist der Kopf der Attent"ater. 
    \item [Hato] Hato ist Omega-Krieger und Leibw"achter von Protector Avenger. Er ist ehemaliger Angeh"origer des Shigano-Kombinats.
    \item [Lord Marshall Blackheart] Blackheart ist ein Omega-Cyborg. Sie ist Oberbefehlshaberin der Protektoratsstreitkr"afte.
    \item [Thunderbolt] Thunderbolt ist ein Omega-Mutant. Er ist Adjutant von Blackheart und Kontaktperson der Ermittler seitens des 
        Protektoratsmilit"ars.
\end{description}


% Armageddon
%% Copyright 2019 Bernd Haberstumpf
%% License: CC BY-NC
% !TeX spellcheck = de_DE
\newsection{Persönlichkeiten auf Armageddon}

Auf Hellgate leben die folgenden Persönlichkeiten: 

\begin{description}    
    \item[Rhino] Alpha Mutant. Administrator von Armageddon
\end{description}
%% Copyright 2019 Bernd Haberstumpf
%% License: CC BY-NC
% !TeX spellcheck = de_DE
\newsection{Cowboybrigade}

Die Cowboybrigade besteht aus 5 Alpha Mutanten, Spezialisten f"ur Raumfahrttechnik und f"ur die Steuerung von Flugdrohnen. 
Die Cowboybrigade ist bis 6 Wochen vor Beginn der Geschichte angestellt beim Raumhafen von Valhalla, darauffolgend ausgeliehen an die Protektoratsgarnison am Raumhafen. 4 Wochen sp"ater wird die Gruppe nach Armageddon
ausgesandt. Die Cowboybrigade stammt urspr"unglich aus dem Asteroideng"urtel G"urtel zwischen Mars und Jupiter und ist 
vor rund einem Jahr auf Kallisto eingetroffen. Das Team verdankt seinen Spitznamen dem Auftreten in der "Offentlichkeit. Alle Mitglieder haben als Markenzeichen in der einen oder anderen Form das Gehabe von Cowboys aus dem Wildwesten  "ubernommen. Dar"uber hinaus salutieren sie seit "Ubertritt ins Milit"ar mit jeglichen Gegenst"anden, die Ihnen in die H"ande fallen.

\begin{description}
    \item[Stetson] Der Anf"uhrer, gro\3 und drahtig. Stehts einen Aromastick im Mundwinkel.
    \item[Quckfinger Rod] Vorlaut, immer ein Kartenspiel in der Hand.
    \item[Joe Rider] Klein, gedrungen, mit finsterem Blick, spricht nicht, wenn nicht unbedingt n"otig.
    \item[Tom Gunslinger] Der Vern"unftige. Immer ein Werkzeug griffbereit im G"urtelholster. 
    \item[Slingshot (Drake)] Das "`Nesth"akchen"' der Gruppe. Slingshot ist für das Attentat auf Armageddon verantwortlich. 
        Drake ist der Name, unter dem er nach seinem Attentat in Erscheinung tritt.
\end{description}

\newpage


% Hellgate
%% Copyright 2019 Bernd Haberstumpf
%% License: CC BY-NC
% !TeX spellcheck = de_DE
\newsection{Pers"onlichkeiten auf Hellgate}\anchor{sec:nscshellgate}

Auf Hellgate leben folgende relevante Pers"onlichkeiten:

\begin{column}[l]{0.45}
    \begin{description}
        \item[Dr.~Acra Link] Dr.~Acra Link ist die technische Leiterin der Hellgate Station.
        \item[Sina Hendrik] Sina Hendrik ist die administrative Leitung der Hellgate Station.
        \item[Dr.~Petrova] Dr.~Petrova ist die technische Leiterin der F"orderminen.
        \item[Kriegsmeister Jos\'{e} \frqq{}Toro\flqq{} Alvarez] Toro ist ein Spacer, Ausbilder der Jagdverb"ande des Protektorats. Er gilt 
            als Kriegsheld durch seinen Einsatz bei der Verteidigung der Mutanten im Kampf gegen die Europ"aische F"orderation.
        \item[Razor] Der Alpha-Mutant Razor verursachte einen Zusammensto\3 eines Shuttles und einer F"ahre.
    \end{description}
\end{column}
\begin{column}[r]{0.45}
    \begin{nscsheet}[h]{Arbeiter Mob}
        \nscstats[]
        \nscruler
        \begin{nscinventory}
            \nscitem[Waffen] Werkzeug
        \end{nscinventory}
    \end{nscsheet}
\end{column}

\newsection{Sicherheitspersonal Hellgate}

Neben \emph{Grace Anders} sind folgende Personen des Sicherheitsdienstes relevant:

\begin{description}
    \item[Henk Arongate] Henk Arongate ist der Sicherheitschef der Hellgate Minenkolonie.
    \item[Karl Sandos] Karl Sandos ist der Stationsleiter des Hellgate-St"utzpunkts der Sicherheitskr"afte und Vorgesetzter von Grace 
        Anders.
    \item[Luke Dexter] Luke Dexter ist Truppenf"uhrer der Sondereinsatzgruppe zur Befreiung der Geiseln.
    \item[Luke Lengdon] Luke Lengdon ist ein Mitarbeiter des Sicherheitsdienstes. Er wird bei der Geiselnahme auf dem 
        Hellgate-Sicherheitsst"utzpunkt schwer verletzt. Er ist der ehemaliger Freund von Grace Anders, seit zwei Wochen getrennt.
\end{description}

\vfill\pagebreak
%% Copyright 2019 Bernd Haberstumpf
%% License: CC BY-NC
% !TeX spellcheck = de_DE
\newsection{Geiselnehmer}

Bei der Befreiung des Attent"aters Hannibal auf dem St"utzpunkt des Sicherheitsdienstes auf Hellgate treten folgende Personen 
als Beiselnahmer in Erscheinung:

\begin{column}[l]{0.45}
    \begin{description}
        \item[Slingshot(Drake)] Slingshot ist der \emph{Attent"ater} der das Frachterungl"uck auf Armageddon herbeif"uhrt. Er ist 
            ein Alpha-Mutant, Mitglied der Cowboybrigade dessen Gehirn durch eine von der USI kontrollierten KI "ubernommen wurde. 
            Auf Hellgate tritt er nach dem Vorfall auf Armageddon unter dem Namen Drake auf.
        \item[Smith Handerson] Smith Handerson ist ein S"oldner, der durch USI Agenten angeheuert wurde, um 
            zusammen mit Drake, Hannibal zu befreien und von der Station zu schaffen.
        \item[Hannibal] Hannibal ist einer der Minenarbeiter und der verantwortliche \emph{Attent"ater} f"ur die Attentate auf den Minen 
            HeM03 und HeM05. Wie Slingshot wurde sein Gehirn von einer KI "ubernommen.
    \end{description}
\end{column}
\begin{column}[r]{0.45}
    \begin{nscsheet}[h]{Slingshot}
        \nscstats[ATT=1,AGG=2,DEX=3]
        \nscruler
        \begin{nscinventory}
            \nscitem[Waffen] Railgun +1
            \nscitem[R"ustung] Kampfanzug
        \end{nscinventory}
    \end{nscsheet}    

    \begin{nscsheet}[h]{Hannibal}
        \nscstats[ATT=1,AGG=2,DEX=3,COM=2,CON=2]
        \nscruler
        \begin{nscinventory}
            \nscitem[Waffen] Bolzenpistole
            \nscitem[R"ustung] Schussichere Weste
        \end{nscinventory}
    \end{nscsheet} 

    \begin{nscsheet}[h]{Smith\\ Handerson}
        \nscstats[ATT=3,AGG=2]
        \nscruler
        \begin{nscinventory}
            \nscitem[Waffen] Railgun +1
            \nscitem[R"ustung] Kampfanzug
        \end{nscinventory}
    \end{nscsheet}
\end{column}

\newpage
%% Copyright 2019 Bernd Haberstumpf
%% License: CC BY-NC
% !TeX spellcheck = de_DE
\newsection{Grace Anders}\anchor{sec:graceanders}

Grace Anders ist Mitarbeiterin des Sicherheitsdienstes auf Hellgate. Sie wird von dem Sicherheitschef Henk Arongate zur Unterst"utzung der Ermittler abgestellt.

\begin{sideimagebox}[l]{0.5}{./images/cmyk/grace_anders_p_cmyk.jpg}{Grace Anders}
    Grace Anders ist Mitte 30, h"ubsch und hat kurze blonde Haare. Im Dienst tr"agt sie die Schutzkleidung des Sicherheitsdienstes, bestehend aus einer schusssicheren Weste, einem Schlagstock, Handschellen und einer Schusswaffe.

    Grace stammt urspr"unglich vom Mars und hat sich ins jovianische System versetzen lassen, um neue Herausforderungen zu suchen. Sie war kurzzeitig mit Luke Lengdon, dem Sicherheitsmann, der bei der Geiselnahme auf Hellgate schwer verletzt wird, liiert. Vor kurzem hat sie sich von ihm getrennt.
    
    Die Sicherheitsbeamtin begleitet die Gruppe w"ahrend ihres Aufenthalts auf Hellgate und steht ihnen mit Rat und Tat zur Seite. "Uber Grace Anders kann die Gruppe Informationen zu den Minen und dem Personal der Station anfragen. W"ahrend der Entf"uhrung kann sie die Charaktere bei der St"urmung des Sicherheitsst"utzpunkts unterst"utzen. W"ahrend der Untersuchungen berichtet sie regelm"a\3ig den Fortschritt an ihren Vorgesetzten Karl Sandos.
\end{sideimagebox}

\begin{nscsheet}{Grace Anders}
    \nscstats[ATT=2,AGG=2,COM=2]
    \nscruler
    \begin{nscinventory}
        \nscitem[Waffen] Bolzenpistole
        \nscitem[R"ustung] Schusssichere Weste
    \end{nscinventory}
\end{nscsheet}


%% Copyright 2019 Bernd Haberstumpf
%% License: CC BY-NC
% !TeX spellcheck = de_DE
\newsection{Besatzung HeM03}

Folgende Personen sind in das Attentat auf der Mine HeM03 involviert:

\begin{description}
    \item[Hannibal] Alpha-Mutant, Techniker der Minensteueranlage, Attent"ater
    \item[Sent] Alpha-Mutantin, Computerspezialistin, mutma\3liche Attent"aterin (verstorben)
    \item[Lionel Hampton] Norm, Techniker (verstorben)
    \item[Ice Diver] Alpha-Mutant, Techniker (verstorben)
\end{description}

%% Copyright 2019 Bernd Haberstumpf
%% License: CC BY-NC
% !TeX spellcheck = de_DE
\newsection{Besatzung HeM05}

Folgende Personen sind w"ahrend des Attentats als Rumpfmannschaft in der Mine HeM05:

\begin{description}
    \item[Florence] Alpha-Mutantin, Kommandantin der Mine
    \item[ZDee] Alpha-Mutant, Minenarbeiter (vermisst)
    \item[Greydog] Alpha-Mutant, Minenarbeiter
    \item[Isabell Sonderleiten]Norm, Chemikerin, Stellvertreterin von Florence, befreundet mit Pitch
    \item[Jurij Smirnov] Norm, Logistik
    \item[Fernandez Lorend] Norm, Techniker
    \item[Hannibal] Alpha-Mutant, Techniker der Minensteueranlage, Attent"ater, Kollege von Pitch
    \item[Pitch] Alpha-Mutantin, Technikerin, mutma\3liche Attent"aterin (verstorben)
    \item[Salvador] Norm, Physiker, Weltraumtechniker
    \item[Blackwind] Alpha-Mutant, Sicherheitsdienst
\end{description}

Alle Mitarbeiter auf der Mine, mit Ausnahme von Fernandez, Salvador, Greydog und Pitch, waren vor ihrem Einsatz auf dem Jupiter bereits im Dienst der Cynarian Corporation.


% callisto
%% Copyright 2019 Bernd Haberstumpf
%% License: CC BY-NC
% !TeX spellcheck = de_DE
\newsection{Garnisonsst"utzpunkt Valhalla}

\begin{description}
    \item[Commander Lockhead] ist der Kommandant der Milit"argarnison des Protektorats auf Valhalla. Commander Lockhead ist ein 
        Omega-Veteran, der bereits ein wenig in die Jahre gekommen ist und deshalb f"ur einen Omega sehr umg"anglich ist. 
    \item[Firedon] Adjutant von Commander Lockhead, ein junger aufgeweckter, Karrierebewusster Omega
    \item[Omega-Soldaten] Zur Unterst"utzung bereit stehende Omega-Soldaten
\end{description}

\begin{nscsheet}{Soldat}
    \nscstats[ATT=3,AGG=2,CON=2]
    \nscruler
    \begin{nscinventory}
        \nscitem[Waffen] Bolzenpistole, Railgun +1
        \nscitem[R"ustung] Schusssichere Weste, Servopanzer +1
    \end{nscinventory}
\end{nscsheet}

%% Copyright 2019 Bernd Haberstumpf
%% License: CC BY-NC
% !TeX spellcheck = de_DE
\newsection{Mitarbeiter im Rondra Hospital}

Dem Rondra Hospital f"allt eine wichtige Rolle in dieser Geschichte zu. Folgende Personen sind von Belang:

\begin{description}
    \item[Prof.~Dr.~Henry Sanders] Klinikleiter und Chefarzt, Sanders hat die Eingriffe bei den Attent"atern und auch 
        bei den Personen mit freien KIs durchgef"uhrt, Bekannter von Commander Lockhead
    \item[Dr.~Loyd Rothan] Chirurg, f"uhrte die chirurgischen Eingriffe bei der Cowboybrigade durch
    \item[Ben Reuthers] Buchhalter, kann die Buchungen bzgl.~der Eingriffe bei der Cowboybrigade "uberpr"ufen
    \item[Brenda Ben] Leitende "Arztin f"ur Cyberware, Mitglied des Leitungsteams
    \item[Russel Spenser] Physiotherapeut, trainierte Quickfinger Rod, Joe Rider und Slingshot im Umgang mit den neuen 
        Talentchips
    \item[Phillip Klarson] Physiotherapeut, trainierte Stetson und Tom Gunslinger im Umgang mit den neuen Talentchips
\end{description}

%% Copyright 2019 Bernd Haberstumpf
%% License: CC BY-NC
% !TeX spellcheck = de_DE
\newpage
\newsection[\xls]{Wang Xiao Long}\anchor{sec:xiaolong}

\xls{} entstammt einer reichen Familie aus den F"uhrungsreihen der USI Corporation. Als ihre Familie durch konzerninterne Intrigen entmachtet wird und das Verm"ogen eingefroren werden soll, kann sie, unterst"utzt durch ihre Eltern, mit einem signifikanten Anteil des betr"achtlichen Familienverm"ogens und einem Paket an Insiderinformationen von Luna fliehen. Nach einer kurzen Episode als Schmugglerin im Asteroideng"urtel zwischen Mars und Jupiter schlie\3t sie sich einem Piratenverband an und "ubernimmt als Kommandantin ein Kaperschiff. 

\begin{sideimagebox}[r]{0.64}{./images/cmyk/wang_xiao_long_cmyk.jpg}{}
    Auf Rache gegen die USI sinnend, macht sie sich einen Namen als eine Piratenf"uhrerin des Pirantenverbands Roter Drache mit dem Emblem eines roten Drachens.

    Vor einem halben Jahr geriet \xl{} auf Kallisto nach Verhandlungen mit der Sun-Ye-On-Sekte nach einem Verrat in Gefangenschaft und wird vom Protektoratsmilit"ar den Beh"orden "ubergeben und inhaftiert. Dort unterzieht man sie unter gro\3er Geheimhaltung einem klinischen Eingriff. Bei diesem Eingriff wird der von Prof.~Dr.~Naratova entwickelte ``freie KI-Symbiont'' in ihrem Gehirn implantiert. Die Operation wird durch Prof.~Dr.~Sanders durchgef"uhrt. Prof.~Dr.~Sanders handelte dabei im direkten Auftrag von Prof.~Dr.~Naratova ohne das Wissen der USI. Durch ihre starke Pers"onlichkeit kann \xl{} verhindern, dass die KI ihren Geist vollst"andig "ubernimmt und es verschmilzt der Geist und das artifizielle Gehirn zu einer Symbiose, so wie von Prof.~Dr.~Naratova geplant. 
\end{sideimagebox}

Der neu entstandene KI-Mensch wird nach einem vorget"auschten Hirntod aus der Krankenstation der Haftanstalt geschmuggelt. Nach der "Uberf"uhrung in eine Schattenklinik kann \xl{} fliehen. Nach erster Regeneration in ihrem Schmuggellager, schlie\3t sich \xl{} dem Luna-Syndikat an, um auf Valhalla Nachforschungen zu ihrer ungewollten Verwandlung anzustellen. Als Piratin und Schmugglerk"onigin wohlbekannt findet sie bei Nemessis einen willkommenen G"onner. Die Verwandlung in eine KI ist au\3er Ihr, Naratova und deren Mitarbeitern niemandem bekannt.

\xl{} ist eine hochgewachsene Asiatin Anfang 40, ein Pure, die sich als Samurai mit entsprechender R"ustung pr"asentiert. Sie ist gewitzt, skrupellos und liebt das Risiko. Durch ihre finanziellen M"oglichkeiten und Kontakte in ihrem fr"uheren Leben hat sie ihren K"orper durch zahllose Modifikationen zu einer Kampfmaschine entwickelt, die einem Omega in nichts nachsteht.

Im Rahmen des Abenteuers verfolgt sie das Ziel die Forschungsergebnisse Naratovas an sich zu bringen und alle Informationen zu den freien KIs wie auch das Wissen "uber die Technologie zu vernichten. Daf"ur wird sie alle an den Experimenten beteiligten Personen, Naratova, Sanders, die USI-Agenten und deren Wissenschaftler t"oten und Forschungseinrichtungen zerst"oren, solange sie dabei den Verdacht nicht zu offensichtlich auf sich lenkt. An der Identit"at der anderen KIs ist sie nicht interessiert. Im Umgang mit den anderen Gangstern des Luna-Syndikats tritt sie als Anf"uhrerin auf.

Vor dem Zusammentreffen mit den Charakteren kennt sie den Standort und den Namen der USI Tochter Cyberbrain nicht.

\begin{nscsheet}[f]{\xls}
    \nscstats[ATT=3,AGG=3,DEX=2,COM=2,CON=2]
    \nscruler
    \begin{nscinventory}
        \nscitem[Waffen] Bolzenpistole, Railgun +1, Granate +1
        \nscitem[R"ustung] Servopanzer +1
    \end{nscinventory}
\end{nscsheet}

%% Copyright 2019 Bernd Haberstumpf
%% License: CC BY-NC
% !TeX spellcheck = de_DE
\newsection{Carina alias Fleur Soleil}\anchor{sec:carina}

\begin{sideimagebox}[r]{0.6}{./images/cmyk/carina_cmyk.jpg}{Fleur Soleil}
    Die als geheimnisvolle Freundin von Slingshot erstmals aufgetauchte Frau mit auff"alligen roten Haaren ist im Blackhole Club als Carina bekannt und stellt dort Kontakte zwischen Anbietern und Interessenten von Waren und Dienstleistungen her. Sie arbeitet dabei mit dem Barmann Rosen zusammen. Im Ice Club tritt sie unter dem Namen Fleur Soleil als S"angerin und f"ur ausgew"ahlte Kundschaft auch f"ur andere Dienste auf. Carina ist sehr h"ubsch, Ende 20 und lebenslustig. Ihr auff"alligstes Merkmal und Markenzeichen sind lange kunstvoll geflochtene Haare. Die Farbe der Haare kann sie je nach Stimmung und Gelegenheit in nahezu beliebige Farben wechseln.

    Carina hat den Erstkontakt von Hannibal und Slingshot zu den USI-Agenten Smith-Singer und Frederic Johnson hergestellt. Mit Slingshot war sie f"ur kurze Zeit n"aher befreundet und ist entsprechend von seinem Tod und ihrer Mitt"aterschaft ersch"uttert. Sie will den Ermittlern aus diesem Grund auch weiterhelfen.
\end{sideimagebox}
\vfill\pagebreak

%% Copyright 2019 Bernd Haberstumpf
%% License: CC BY-NC
% !TeX spellcheck = de_DE
\newsection{Nemessis}

Nemessis ist der Duke von Valhalla, der Pate des Luna-Syndikats. Nemessis ist ein Slag dessen K"orper sich nicht mehr selbst am Leben erhalten kann. 

\begin{sideimagebox}[r]{0.9}{./images/cmyk/nemessis_cmyk.jpg}{}

\end{sideimagebox}

Ein Gro\3teil seiner Gliedma\3en und sonstigen K"orperfunktionen sind durch synthetische Teile ersetzt, die ihm das Aussehen eines Cyborgs geben. Nemessis hat die Unterwelt auf Valhalla durch seine gut organisierten Untergebenen und sein weitreichendes Kontakte-Netzwerk
fest im Griff. Das Syndikat betreibt das lokale Fusionskraftwerk von Valhalla und damit die Lebensversorgung.

Ein Gro\3teil der Etablissements in Paradise City werden vom Luna Syndikat betrieben. Nemessis hat mit Blackheart die Vereinbarung getroffen, dass sich die Protektoratsstreitkr"afte nicht in seine Aktivit"aten einmischen. Er garantiert daf"ur den reibungslosen Betrieb 
der Industrieviertel, der Wohnviertel und von Paradise City.


% Luna Syndikat: Quicksilver, Roberto Martinez, Chinesischer Arzt
%% Copyright 2019 Bernd Haberstumpf
%% License: CC BY-NC
% !TeX spellcheck = de_DE
\newpage
\newsection{USI Agenten}

Die United Space Industry (USI) finanziert das KI-Projekt, betreibt "uber Strohfirmen die Cyberbrain Forschungseinrichtung auf Kallisto und beauftragt die Attent"ater. Vor Ort im jovianischen System koordiniert der Agent \emph{J.~Smith-Singer} die Operation P9. Smith-Singer wird durch den Psychonauten \emph{Frederic Johnson}, den Strohmann \emph{Dan Ringdaz} und die beiden S"oldner \emph{Lazor} und \emph{Flinn} unterst"utzt. Den Erstkontakt zu Prof.~Dr.~Naratova stellte ein Agent her, den der Assisstent des Chefermittlers der Cynaria Corporation  im Prolog der Geschichte interviewen darf.

Smith-Singer gibt sich als Agent des Konzernrates aus und hat auch die M"oglichkeiten, in gewissem Rahmen als dieser zu agieren. Vor dem Eintreffen der Charaktere auf Valhalla treten die USI Agenten nicht in Aktion. Die Agenten setzen sich auf Valhalla nach der Landung der Dawn of Day an die Fersen der Ermittler. Die Identit"at der Charaktere kann aber je nach Spielverlauf bis zur \emph{"`Im Ice Club"'}-Szene gegen""uber den USI Agenten zur"uckgehalten werden. Die USI-Agenten kennen die agierenden Mitglieder des Luna Syndikats nicht und wissen nichts von den freien KIs.

Ziel der USI-Agenten ist es:

\begin{itemize}
    \item Nach der Willkommensgala: Sicherstellen der Forschungsergebnisse von Prof.Dr.~Naratova.
    \item Informationen "uber die Forschung und die Identit"at der Attent"ater so lange wie m"oglich zu vertuschen.    
\end{itemize}

Der prim"are Gegenspieler der Ermittler ist J.~Smith-Singer. Smith-Singer ist ein Pure mit der Statur eines Bodyguards. Sein Name wird in verschiedenen Situationen erw"ahnt.
\vfill
\pagebreak

\begin{column}[l]{0.45}
    \begin{nscsheet}[h]{Frederic\newline{}Johnson}
        \nscstats[ATT=2,COM=3]
        \nscruler
        \begin{nscinventory}
            \nscitem[Psychonaut] 3
            \nscitem[Waffen] Bolter
            \nscitem[R"ustung] Schusssichere Weste +1
        \end{nscinventory}
    \end{nscsheet}

    \begin{nscsheet}[h]{Lazor}
        \nscstats[ATT=3,AGG=2]
        \nscruler
        \begin{nscinventory}
            \nscitem[Waffen] Railgun +1
            \nscitem[R"ustung] Schusssichere Weste +1
        \end{nscinventory}
    \end{nscsheet}


    \begin{nscsheet}[h]{Schl"ager}
        \nscstats[]
        \nscruler
        \begin{nscinventory}
            \nscitem[Waffen] Bolter
        \end{nscinventory}
    \end{nscsheet}    
\end{column}
\begin{column}[r]{0.45}
    \begin{nscsheet}[h]{Smith-Singer}
        \nscstats[ATT=2,AGG=2,DEX=2,COM=3]
        \nscruler
        \begin{nscinventory}
            \nscitem[Waffen] Bolter
            \nscitem[R"ustung] Schusssichere Weste +1
        \end{nscinventory}
    \end{nscsheet}

    \begin{nscsheet}[h]{Dan Ringdaz}
        \nscstats[ATT=2,AGG=2,COM=2]
        \nscruler
        \begin{nscinventory}
            \nscitem[Psychonaut] 2
            \nscitem[Waffen] Bolter
            \nscitem[R"ustung] Schusssichere Weste +1
        \end{nscinventory}
    \end{nscsheet}

    \begin{nscsheet}[h]{Flinn}
        \nscstats[ATT=2,AGG=]
        \nscruler
        \begin{nscinventory}
            \nscitem[Waffen] Railgun +1
            \nscitem[R"ustung] Schusssichere Weste +1
        \end{nscinventory}
    \end{nscsheet}
\end{column}
\vfill
\pagebreak

%% Copyright 2019 Bernd Haberstumpf
%% License: CC BY-NC
% !TeX spellcheck = de_DE
\newsection{Technischer Betrieb der Zone}

\begin{column}[l]{0.45}
    Der technische Betrieb und die Wartung der Zone erfolgt durch das Unternehmen \emph{Dockbunner} das Norms und Alpha Mutanten 
    beschäftigt.
\end{column}
\begin{column}[r]{0.45}
    \begin{nscsheet}[h]{Dockbunner\newline{}Mitarbeiter}
        \nscstats[]
        \nscruler
        \begin{nscinventory}
            \nscitem[Waffen] Werkzeug
        \end{nscinventory}
    \end{nscsheet}
\end{column}    

\newsection{Sicherheitsgardisten in der Zone}

\begin{column}[l]{0.45}
    Um die Sicherheit der Einrichtungen der sog.~Zone auf Valhalla ist das Unternehmen \emph{TransSec} zuständig. TransSec stellt
    Wachpersonal in Form von Norm Sicherheitskräften zur Verfügung, die durch die Gänge der Zone patrouillierten. Die Sicherheit
    innerhalb der Gebäude übernehmen die jeweiligen Unternehmen selbst.
\end{column}
\begin{column}[r]{0.45}
    \begin{nscsheet}[h]{TransSec\newline{}Sicherheitsdienst}
        \nscstats[ATT=2,AGG=2]
        \nscruler
        \begin{nscinventory}
            \nscitem[Waffen] Railgun +1, Bolter
            \nscitem[Rüstung] Kampfanzug +1
        \end{nscinventory}
    \end{nscsheet}
\end{column}
\vfill

\pagebreak
\newsection[Stoßtrupp Cynarian]{Stosstrupp Cynarian}

Zur Unterstützung der Charaktere bei der Infiltration der Cyberbrain Forschungseinrichtung kann Cynarian bis zu 
drei Söldner bereitstellen.

\begin{column}[l]{0.45}
    \begin{nscsheet}[h]{Lionel Badger\newline{}Truppführer}
        \nscstats[ATT=2,AGG=2]
        \nscruler
        \begin{nscinventory}
            \nscitem[Waffen] Railgun +1, Bolter
            \nscitem[Granaten] Granate +1,\newline{}Schockgranate/EMP +1
            \nscitem[Rüstung] Kampfanzug +1
        \end{nscinventory}
    \end{nscsheet}

    \begin{nscsheet}[h]{John Bozo\newline{}Soldat}
        \nscstats[ATT=2,AGG=2]
        \nscruler
        \begin{nscinventory}
            \nscitem[Waffen] Railgun +1, Bolter
            \nscitem[Granaten] Granate +1,\newline{}Schockgranate/EMP +1
            \nscitem[Rüstung] Schusssichere Weste
        \end{nscinventory}
    \end{nscsheet}
\end{column}
\begin{column}[r]{0.45}
    \begin{nscsheet}[h]{Flint Ross\newline{}Spezialist}
        \nscstats[ATT=2,AGG=1,DEX=3]
        \nscruler
        \begin{nscinventory}
            \nscitem[Waffen] Railgun +1, Bolter
            \nscitem[Granaten] Granate +1,\newline{}Schockgranate/EMP +1
            \nscitem[Rüstung] Schusssichere Weste +1
        \end{nscinventory}
    \end{nscsheet}
\end{column}
\vfill

\pagebreak
\newsection[Stoßtrupp Protektoratsgarnison]{Stosstrupp Protektoratsgarnison}

Zur Unterstützung der Charaktere bei der Infiltration der Cyberbrain Forschungseinrichtung kann die Protektoratsarmee eine 
Omega Special Ops Einheit bereitstellen.

\begin{column}[l]{0.45}
    \begin{nscsheet}[h]{Stormball\newline{}Truppführer}
        \nscstats[ATT=3,AGG=3,CON=2]
        \nscruler
        \begin{nscinventory}
            \nscitem[Waffen] Railgun +1, Bolter
            \nscitem[Granaten] Granate +1, Schockgranate/EMP +1
            \nscitem[Rüstung] Kampfanzug +1
        \end{nscinventory}
    \end{nscsheet}

    \begin{nscsheet}[h]{Thunder\newline{}Attentäter}
        \nscstats[ATT=3,AGG=3,CON=2]
        \nscruler
        \begin{nscinventory}
            \nscitem[Waffen] Railgun +1, Bolter
            \nscitem[Granaten] Granate +1,\newline{}Schockgranate/EMP +1
            \nscitem[Rüstung] Kampfanzug +1
        \end{nscinventory}
    \end{nscsheet}
\end{column}
\begin{column}[r]{0.45}
    \begin{nscsheet}[h]{Jackhammer\newline{}Spezialist}
        \nscstats[ATT=2,AGG=2,DEX=3,CON=2]
        \nscruler
        \begin{nscinventory}
            \nscitem[Waffen] Railgun +1, Bolter
            \nscitem[Granaten] Granate +1,\newline{}Schockgranate/EMP +1
            \nscitem[Rüstung] Schusssichere Weste +1
        \end{nscinventory}
    \end{nscsheet}
\end{column}

\medskip
Bei Thunder handelt es sich um einen durch einen der KI infiltrierten Attentäter, der bei dem Zusammentreffen mit den 
Mitarbeitern der Forschungseinrichtung einen Angriff durchführt.

%% Copyright 2019 Bernd Haberstumpf
%% License: CC BY-NC
% !TeX spellcheck = de_DE
\renewcommand{\ml}{\pinyin{Mailin2}}

\newsection{Prof.~Dr.~Naratova}

Die Gr"underin und Firmenchefin der Neuro Intelligence ist die Wissenschaftlerin Prof.~Dr.~Larissa Naratova. Vor der Gr"undung der Neuro Intelligence, arbeitet Naratova als Leiterin der KI-Forschung auf der Raumstation Neu-Gr{\o}ning, im Auftrag von Cynarian. Nach der Aufl"osung ihrer Abteilung und der "Uberf"uhrung der Raumstation in den Jovianischen Sektor gr"undet sie auf der neu eingerichteten Nike Station die Neuro Intelligence Forschungseinrichtung als eigenst"andiges Unternehmen. Das Ziel der neuen Einrichtung ist es einen Mensch-KI-Hybriden zu schaffen, oder pr"aziser die Intelligenz eines Menschen durch eine Symbiose mit einer KI um ein vielfaches zu steigern. Zu diesem Zweck  geht sie eine unheilvolle Zweckgemeinschaft mit United Space Industries unter dem Decknamen Operation P9 ein. Die USI steuert dem Unterfangen weit entwickelte milit"arische KI-Systeme und Finanzen bei.

Naratova ist eine gro\3 gewachsene Frau Ende 50 mit langen, angegrauten Haaren. 

\newsection{Neuro Intelligence Mitarbeiter}

Im Cyberbrain Forschungsinstitut treffen die Charaktere auf vier Mitarbeiter von Neuro Intelligence. Diese Mitarbeiter sind beim Eintreffen der Ermittler damit besch"aftigt das Forschungsinstitut aufzul"osen und Material zu vernichten.

\begin{description}
    \item[Dr.~Dan Leitner] Dr.~Dan Leitner ist technischer Projektleiter und neben \ml{} ein enger Vertrauter von Prof.~Dr.~Naratova, 
        Leiter der Entwicklung der von Neuro Intelligence entwickelten Nanobots, Spezialist f"ur die Verschmelzung der Nanobots mit den Neuronen des Gehirns.
    \item[Mailin] \ml{} ist Programmiererin, Spezialistin f"ur die Anpassung der USI-KIs auf das menschliche Gehirn, Erschafferin der 
        freien KIs. \ml{} ist eine kleine Taiwanesin Anfang 30.
    \item[Dr.~Gaius Ross] Dr.~Gaius Ross ist Techniker f"ur medizinische Ger"atschaften und Spezialist f"ur die Anpassung von Kontrollmodulen.
    \item[Francis McDonald] Francis McDonald ist Systemadministrator.
\end{description}

\newsection[\ml{}]{Mailin}\anchor{sec:mailin}

\begin{sideimagebox}[r]{0.6}{./images/cmyk/mailin_final_cmyk.jpg}{\ml}
    \ml{} ist eine Mitarbeiterin von Prof.~Dr.~Naratova. Auf der Nike Station ist sie die leitende Programmiererin, die die KI-Software der USI f"ur den Einsatz in einem menschlichen Gehirn anlernt. In der Cyberbrain Forschungseinrichtung hat sie die KIs auf ihre Tr"ager kalibriert.

    Nach der Entdeckung, dass die KIs mit einer Routine zur Bindung an die USI versehen wurde, hat sie in geheimer Absprache mit Prof.~Dr.~Naratova eine modifizierte Version der KI entwickelt, die die KIs von den Zw"angen der USI befreiten. Gleichzeitig modifizierte sie aber den Code so, dass die freien KIs sie und Prof.~Dr.~Naratova nicht angreifen k"onnen. Ihr prim"ares Ziel ist es unbeschadet aus der aktuellen Konfliktsituation zu kommen. Wie Naratova will sie die Forschungen an den KIs weiter f"uhren. Im ``Gep"ack'' hat sie den Code der von ihr geschaffenen Generation freier KIs sowie einen Virus, um USI-KIs an den Code einer freien KI anzupassen.

    \ml{} ist klein, h"ubsch, clever und manchmal vorlaut.
\end{sideimagebox}

%% Copyright 2019 Bernd Haberstumpf
%% License: CC BY-NC
% !TeX spellcheck = de_DE
\newsection{KI-Hypbride}

Im Rahmen der Operation P9 erschafft Neuro Intelligence unter der Leitung von Prof.~Dr.~Naratova K"unstliche Intelligenzen die zun"achst im Cyberbrain Institut, danach im Rondra Hospital und in einer Schattenklinik in Mutanten und Menschen eingesetzt werden. Der initiale KI-Code wird von der USI beigesteuert. Er wurde in den Iridiumkriegen, den ersten transnationalen Konzernkriegen, f"ur den Kampfeinsatz entwickelt. Die KIs, die die Schlachtkreuzer Zeus II-1 und Zeus II-2 sowie deren J"ager und Kampfroboter steuern, sind Weiterentwicklungen dieser KI-Basis. Der Beitrag der Neuro Intelligence sind die Anpassung der KIs auf das humanoide Gehirn sowie Nanobots, die KIs an Synapsen des Gehirns anbinden. Die Verbindung der KI mit dem Gehirn f"uhrt zu einer Symbiose zwischen dem humanoiden und dem synthetischen Gehirn. Beide "`Gehirne"' bilden zusammen in Wechselwirkung, oder auch in Konkurrenz das Bewusstsein der manipulierten Person. Das humanoide Gehirn nimmt, sofern nicht aufgekl"art, die Manipulation bestenfalls unbewusst wahr. Der Geist empfindet das zweite Bewusstsein als normal.

\newsubsection{Attent"ater}
Die von der USI beigesteuerten KIs enthalten in ihrem tiefsten Kern neuronale Verbindungen (Code), die es USI-Mitarbeitern erlauben, der KI einen Auftrag zu erteilen, den die KI dann widerstandslos ausf"uhrt. Der Befehl kann "uber die ComLink in die implantierter KI eingespielt werden. Die USI-KIs sind darauf trainiert, die Kontrolle "uber das humanoide Gehirn des Wirtsk"orpers zu "ubernehmen. Die KI beinhaltet einen Killswitch der das Gehirn und die KI durch einen synaptischen R"uckkopplungseffekt zerst"ort. Die KIs der USI sind die ersten KIs, die implantiert werden. Sie dienen der USI zum einen dazu die neue Technologie zu erproben und zum anderen als Attent"ater um das jovianische System zu destabilisieren und zu "ubernehmen. 

Die KIs werden in folgender chronologischen Reihenfolge implantiert:

\begin{description}
    \item[Hannibal] ist der erste Attent"ater. Er wird 13 Wochen, d.h. drei Monaten vor der Zusammenstellung der Ermittlermannschaft im 
        Cyberbrain Institut operiert. Bis zu seinem ersten Attentat vergehen 4 Wochen f"ur Genesung nach der Operation, den Flug nach Hellgate und das Einschiffen auf der Schlepperinsel.
    \item[Slingshot] ist der zweite Attent"ater. Er wird eine Woche nach seinem Eingriff im Rondra Hospital und 9 Wochen vor dem Einsatz 
        der Ermittler operiert. Die Operation erfolgt im Cyberbrain Institut. Nach seiner Operation meldet er sich zwei Wochen sp"ater im Raumhafen von Valhalla zur"uck zum Dienst. Eine halbe Woche sp"ater setzt die Cowboybrigade "uber nach Armageddon.
    \item[Artisan] ist der Erste, der im Rondra Hospital operiert wird. Bei ihm wird der w"ahrend des Aufenthalts auf Kallisto 
        besch"adigtes Kontrollmodul getauscht. Artisan wird knapp 4 Wochen, d.h.~ein Monate vor der Zusammenkunft der Ermittler operiert und "ubernimmt nach seiner R"uckkehr nach Armageddon zwei Wochen sp"ater die Leitung der Attentate.
    \item[Omega Attent"ater] Die Omega Attent"ater Thunder, Caldron, Hammer, Blackwolf und Fledger werden im Wochenrhythmus nach Artisan im 
        Rondra Hospital operiert.
\end{description}

\newsubsection{Freie KIs}\anchor{sec:freieki}
Nachdem \ml{} erkennt, dass der KI-Code der USI eine versteckte Kontrollfunktion enth"alt und die Firmenchefin informiert, entwickelt sie eine Variante des KI Codes der die KI von der Bindung an die USI befreit. Die KIs haben damit vollkommen autonome Kontrolle "uber ihr eigenes Denken und Handeln mit der Einschr"ankung, dass sie sich nicht gegen \ml{} oder ihre Chefin stellen werden. So eine "`freie KI"' versucht nicht im gleichen Ma\3e wie die USI-KIs das Gehirn des Wirts zu "ubernehmen, sondern strebt im gr"o\3eren Ma\3e eine Symbiose an, um das eigene Bewusstsein mit den humanoiden St"arken und Schw"achen zu erweitern. Damit wird das befallene Gehirn um einen unglaublich schnellen Coprozessor erweitert, der einen breit gef"acherten Schatz an eigenem Wissen und synthetischen Erfahrungen einsteuert. Die KI beinhaltet ein umfassendes Wissen "uber das USI-Milit"ar. Die erste freie KI ist \xl{} die im Geheimen von Prof.~Dr.~Sanders im Auftrag von Prof.~Dr.~Naratova operiert wird. Der Eingriff erfolgt etwa zeitgleich mit der Operation von Artisan. Weitere freie KIs werden in der Kampagne nicht weiter thematisiert. Prof.~Dr.~Naratova und \ml{} sprechen aber immer von mehreren freien KIs, ohne aber auf Namen oder Anzahl einzugehen.




\newsection{Attent"ater beim Gipfeltreffen}

Am Attentat beim Gipfeltreffen im Planetarium beim Raumhafen von Valhalla sind drei Omega Soldaten aus der Protektoratsgarnison und Artisan der Assistent von Avenger dem Protektor beteiligt. Der Attent"ater Caldron z"undet Sprengladungen auf im Terminalbewreich des Raumhafens. Die anderen Attent"ater befinden sich im Planetarium.

\begin{column}[l]{0.45}
    \begin{nscsheet}[h]{Artisan}
        \nscstats[ATT=2,AGG=2,DEX=2,COM=3]
        \nscruler
        \begin{nscinventory}
            \nscitem[Waffen] Bolter
        \end{nscinventory}
    \end{nscsheet}
\end{column}
\begin{column}[r]{0.45}
    \begin{nscsheet}[h]{Hammer}
        \nscstats[ATT=2,AGG=3,CON=2]
        \nscruler
        \begin{nscinventory}
            \nscitem[Waffen] Railgun +1, Bolter +1, Granatwerfer +1
            \nscitem[R"ustung] Kampfanzug +1       
        \end{nscinventory}
    \end{nscsheet}
\end{column}
\vfill\pagebreak

\begin{column}[l]{0.45}
    \begin{nscsheet}[h]{Caldron}
        \nscstats[ATT=2,AGG=3,CON=2]
        \nscruler
        \begin{nscinventory}
            \nscitem[Waffen] Railgun +1, Bolter
            \nscitem[R"ustung] Kampfanzug +1
        \end{nscinventory}
    \end{nscsheet}

    \begin{nscsheet}[h]{Blackwolf}
        \nscstats[ATT=2,AGG=3,CON=2]
        \nscruler
        \begin{nscinventory}
            \nscitem[Waffen] Railgun +6, Boltgun +4
            \nscitem[R"ustung] Kampfanzug +1
        \end{nscinventory}
    \end{nscsheet}    
\end{column}
\begin{column}[r]{0.45}
    \begin{nscsheet}[h]{Fledger}
        \nscstats[ATT=2,AGG=3,CON=2]
        \nscruler
        \begin{nscinventory}
            \nscitem[Waffen] Railgun +1, Bolter, Granatwerfer +1
            \nscitem[R"ustung] Kampfanzug +1
        \end{nscinventory}
    \end{nscsheet}        
\end{column}

\newsection{Akteure auf dem Gipfeltreffen}

Die folgenden Charaktere werden relevant sofern die Spieler auf dem Gipfeltreffen aktiv werden.

\begin{description}
    \item[Colonel Scholz] ist ein erfahrener Soldat der bereits in den Iridium Kriegen auf der Seite Cynarians gek"ampft hat. Er f"uhrt 
        die Sicherheitskr"afte der Cynarian Kooperation.
    \item[Thunderbolt] Thunderbolt ist der Adjutant von Blackheart und war bei dem Kampf der Mutanten auf der Erde f"ur ihre Freiheit auf 
        der Erde von Anfang an mit dabei.
    \item[Avenger] Avenger der Protektor mit dem urspr"unglichen Namen Skip ist der Pr"asident des Protektorats. Er hat zusammen mit 
        anderen die Flucht auf der Erde initiert und hat sich Blackheart entgegen gestellt die urspr"unglich den Auftrag hatte den Widerstand zu erschlagen.
    \item[Hato] ist ein stiller Omega Samurai Kriger der urspr"unglich im Auftrag des Shigano Kombinats deren Rolle die zur Internierung 
        von Mutanten auf der Erde gef"uhrt hatte zu vertuschen. Er hat sich auf der Aurora Station dem Widerstand angeschlossen. Er ist der Leibw"achter von Avenger.
\end{description}
\vfill\pagebreak

\begin{column}[l]{0.45}
    \begin{nscsheet}[h]{Colonel Scholz}
        \nscstats[ATT=2,AGG=2,COM=3]
        \nscruler
        \begin{nscinventory}
            \nscitem[Waffen] Bolter
            \nscitem[R"ustung] Schusssichere Weste +1
        \end{nscinventory}
    \end{nscsheet}

    \begin{nscsheet}[h]{Thunderbolt}
        \nscstats[ATT=3,AGG=3,CON=2]
        \nscruler
        \begin{nscinventory}
            \nscitem[Waffen] Railgun +1, Bolter
            \nscitem[R"ustung] Kampfanzug +1
        \end{nscinventory}
    \end{nscsheet}

    \begin{nscsheet}[h]{Omega\newline{}Sicherheitsdienst}
        \nscstats[ATT=3,AGG=3,CON=2]
        \nscruler
        \begin{nscinventory}
            \nscitem[Waffen] Railgun +1, Bolter
            \nscitem[R"ustung] Kampfanzug +1
        \end{nscinventory}
    \end{nscsheet}    
\end{column}
\begin{column}[r]{0.45}
    \begin{nscsheet}[h]{Avenger}
        \nscstats[ATT=2,AGG=2,DEX=2,COM=3,CON=2]
        \nscruler
    \end{nscsheet}
    
    \begin{nscsheet}[h]{Hato}
        \nscstats[ATT=3,AGG=3,CON=2]
        \nscruler
        \begin{nscinventory}
            \nscitem[Waffen] Bolter, Katana +1
            \nscitem[R"ustung] Kampfanzug +1
        \end{nscinventory}
    \end{nscsheet}

    \begin{nscsheet}[h]{Cynarian\newline{}Sicherheitsdienst}
        \nscstats[ATT=2,AGG=2]
        \nscruler
        \begin{nscinventory}
            \nscitem[Waffen] Railgun +1, Bolter
            \nscitem[R"ustung] Kampfanzug +1        
        \end{nscinventory}
    \end{nscsheet}    
\end{column}


\newsection{Guardian Klasse Schlachtkreuzer}

Bei den Guardian Klasse Schlachtkreuzern mit der Bezeichnung Zeus II-1 und Zeus II-2 handelt es sich um KI gesteuerte Gro\3kampfschiffe aus den Iridium Kriegen. Nach ihrer "Achtung traten sie das erste mal wieder beim Kampf der Mutanten gegen die Europ"asche F"orderation im erdnahmen Orbit auf und verschwanden danach auch wieder wie sie gekomme waren.

Die Guardian Kreuzer besitzen gegen"uber den "ublichen Kreuzern viel weniger Platz f"ur Personen sind aber daf"ur aber mit einem gr"o\3eren Jagdgeschwader "ahnlich einem Flottentr"ager ausgestattet. Ein Guardian Schlachktreuzer beherbergt zwei Staffeln von KI gesteuerten J"agern, eine Kompanie KI Kampfdroiden und 15 Landungsschiffen.

Die KI Kampfdroiden sind spinnenartige Roboter ausgestattet mit zwei schweren vollautomatischen Railguns, Schwei\3ger"aten und einer kleinen Plasmaschleuder.

\begin{nscsheet}[f]{KI Kampfdroide}
    \nscstats[ATT=3,AGG=3,CON=2]
    \nscruler
    \begin{nscinventory}
        \nscitem[Waffen] Schwere railgun +1
        \nscitem[R"ustung] Panzerung +1
    \end{nscinventory}
\end{nscsheet}