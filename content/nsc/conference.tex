%% Copyright 2019 Bernd Haberstumpf
%% License: CC BY-NC
% !TeX spellcheck = de_DE
\newsection{Attent"ater auf der Konferenz}\anchor{sec:attentaeter}

Am Attentat auf der Konferenz in Valhalla sind drei Omega-Soldaten aus der Protektoratsgarnison und Artisan, der Assistent des Protektors beteiligt. Der Attent"ater Caldron z"undet Sprengladungen im Terminalbereich des Raumhafens. Die anderen Attent"ater befinden sich im Planetarium. Hat Thunder (beschrieben \cref{cref:stosstruppprotektorat}) die Gruppe bei der Infiltration der Cyberbrain-Forschungseinrichtung nicht begleitet, ist er Teil der Attent"ater auf der Konferenz.

\begin{column}[l]{0.45}
    \begin{nscsheet}[h]{Artisan}
        \nscstats[ATT=2,AGG=2,DEX=2,COM=3]
        \nscruler
        \begin{nscinventory}
            \nscitem[Waffen] Bolzenpistole
        \end{nscinventory}
    \end{nscsheet}

    \begin{nscsheet}[h]{Caldron}
        \nscstats[ATT=2,AGG=3,CON=2]
        \nscruler
        \begin{nscinventory}
            \nscitem[Waffen] Railgun +1, Bolzenpistole
            \nscitem[R"ustung] Kampfanzug
        \end{nscinventory}
    \end{nscsheet}
\end{column}
\begin{column}[r]{0.45}
    \begin{nscsheet}[h]{Hammer}
        \nscstats[ATT=2,AGG=3,CON=2]
        \nscruler
        \begin{nscinventory}
            \nscitem[Waffen] Bolzenpistole, Granatwerfer +1            
        \end{nscinventory}
    \end{nscsheet}

    \begin{nscsheet}[h]{Fledger}
        \nscstats[ATT=2,AGG=3,CON=2]
        \nscruler
        \begin{nscinventory}
            \nscitem[Waffen] Railgun +1, Bolzenpistole, Granatwerfer +1
            \nscitem[R"ustung] Kampfanzug
        \end{nscinventory}
    \end{nscsheet}            
\end{column}
\vfill\pagebreak

\begin{column}[l]{0.45}
    \begin{nscsheet}[h]{Blackwolf}
        \nscstats[ATT=2,AGG=3,CON=2]
        \nscruler
        \begin{nscinventory}
            \nscitem[Waffen] Railgun +1, Bolzenpistole
            \nscitem[R"ustung] Kampfanzug
        \end{nscinventory}
    \end{nscsheet}    
\end{column}
\begin{column}[r]{0.45}

\end{column}

\newsection{Akteure auf der Konferenz}

Die Konferenz wird von Sicherheitskr"aften abgesichert. Daf"ur stellt das Protektoratsmilit"ar 15 Soldaten bereit. Die 
Cynarian Corporation hat 20 Konzernsicherheitskr"afte abgestellt.

Die folgenden Charaktere werden relevant, sofern die Spieler auf der Konferenz aktiv werden.

\begin{description}
    \item[Colonel Scholz] Colonel Scholz ist ein erfahrener Soldat, der bereits in den Iridiumkriegen auf der Seite Cynarians gek"ampft hat. 
        Er f"uhrt die Sicherheitskr"afte der Cynarian Kooperation.
    \item[Thunderbolt] Thunderbolt ist der Adjutant von Blackheart und war bei dem Kampf der Mutanten auf der Erde von 
        Anfang an mit dabei.
    \item[Avenger] Avenger, der Protektor mit dem urspr"unglichen Namen Skip, ist der Pr"asident des Protektorats. Er hat zusammen mit 
        anderen die Flucht von der Erde initiiert und sich Blackheart entgegengestellt, die urspr"unglich den Auftrag hatte, den Widerstand niederzuschlagen.
    \item[Hato] Hato ist ein stiller Omega-Samurai-Krieger der urspr"unglich im Auftrag des Shigano Kombinats deren Rolle, die zur 
        Internierung von Mutanten auf der Erde gef"uhrt hat, zu vertuschen. Er hat sich auf der Aurora Station dem Widerstand angeschlossen. Er ist der Leibw"achter des Protektors.
\end{description}

\begin{column}[l]{0.45}
    \begin{nscsheet}[h]{Colonel Scholz}
        \nscstats[ATT=2,AGG=2,COM=3]
        \nscruler
        \begin{nscinventory}
            \nscitem[Waffen] Bolzenpistole
            \nscitem[R"ustung] Schusssichere Weste
        \end{nscinventory}
    \end{nscsheet}
\end{column}
\begin{column}[r]{0.45}
    \begin{nscsheet}[h]{Avenger}
        \nscstats[ATT=2,AGG=2,DEX=2,COM=3,CON=2]
        \nscruler
    \end{nscsheet}
\end{column}
\vfill\pagebreak

\begin{column}[l]{0.45}
    \begin{nscsheet}[h]{Thunderbolt}
        \nscstats[ATT=3,AGG=3,CON=2]
        \nscruler
        \begin{nscinventory}
            \nscitem[Waffen] Railgun +1, Bolzenpistole
            \nscitem[R"ustung] Kampfanzug
        \end{nscinventory}
    \end{nscsheet}

    \begin{nscsheet}[h]{Omega\newline{}Sicherheitsdienst}
        \nscstats[ATT=3,AGG=3,CON=2]
        \nscruler
        \begin{nscinventory}
            \nscitem[Waffen] Railgun +1, Bolzenpistole
            \nscitem[R"ustung] Kampfanzug
        \end{nscinventory}
    \end{nscsheet}    
\end{column}
\begin{column}[r]{0.45}    
    \begin{nscsheet}[h]{Hato}
        \nscstats[ATT=3,AGG=3,CON=2]
        \nscruler
        \begin{nscinventory}
            \nscitem[Waffen] Bolzenpistole, Katana
            \nscitem[R"ustung] Kampfanzug
        \end{nscinventory}
    \end{nscsheet}

    \begin{nscsheet}[h]{Cynarian\newline{}Sicherheitsdienst}
        \nscstats[ATT=2,AGG=2]
        \nscruler
        \begin{nscinventory}
            \nscitem[Waffen] Railgun +1, Bolzenpistole
            \nscitem[R"ustung] Kampfanzug        
        \end{nscinventory}
    \end{nscsheet}    
\end{column}

\newsection{Delegierte}\anchor{sec:delegates}

Auf der Konferenz nehmen Vertreter der Europ"aischen F"orderation von der Erde und Mitarbeiter des Shigano-Kombinats vom Mars teil. Die Europ"aische F"orderation und das Shigano-Kombinat werden \cref{sec:institutions} beschrieben.

\begin{description}
    \item [Luc Duval] Luc Duval ist Staatssekret"ar der Europ"aischen F"orderation.
    \item [Sarana] Sarana ist Ratsgesandte des Shigano Kombinats.
    \item [Itori Makon] Itori Makon ist Mitarbeiter der Sony Genetics Corporation im Shigano Kombinat.
    \item [Nibori] Nibori ist Mitarbeiter von Dai-kyu Keitsu im Shigano Kombinat.
\end{description}
