%% Copyright 2019 Bernd Haberstumpf
%% License: CC BY-NC
% !TeX spellcheck = de_DE
\newsection{Technischer Betrieb der Zone}

\begin{column}[l]{0.45}
    Der technische Betrieb und die Wartung der Zone erfolgt durch das Unternehmen \emph{Dockbunner}, das Norms und Alpha-Mutanten 
    besch"aftigt.
\end{column}
\begin{column}[r]{0.45}
    \begin{nscsheet}[h]{Dockbunner\newline{}Mitarbeiter}
        \nscstats[]
        \nscruler
        \begin{nscinventory}
            \nscitem[Waffen] Werkzeug
        \end{nscinventory}
    \end{nscsheet}
\end{column}    

\newsection{Sicherheitsgardisten in der Zone}

\begin{column}[l]{0.45}
    Um die Sicherheit der Einrichtungen der sog.~Zone auf Valhalla ist das Unternehmen \emph{TransSec} zust"andig. TransSec stellt
    Wachpersonal in Form von Norm-Sicherheitskr"aften zur Verf"ugung, die durch die G"ange der Zone patrouillierten. Die Sicherheit
    innerhalb der Geb"aude "ubernehmen die jeweiligen Unternehmen selbst.
\end{column}
\begin{column}[r]{0.45}
    \begin{nscsheet}[h]{TransSec\newline{}Sicherheitsdienst}
        \nscstats[ATT=2,AGG=2]
        \nscruler
        \begin{nscinventory}
            \nscitem[Waffen] Railgun +1, Bolzenpistole
            \nscitem[R"ustung] Kampfanzug
        \end{nscinventory}
    \end{nscsheet}
\end{column}
\vfill

\pagebreak
\newsection[Sto\3trupp Cynarian]{Stosstrupp Cynarian}

Zur Unterst"utzung der Charaktere bei der Infiltration der Cyberbrain Forschungseinrichtung kann Cynarian bis zu 
drei S"oldner bereitstellen.

\begin{column}[l]{0.45}
    \begin{nscsheet}[h]{Lionel Badger\newline{}Truppf"uhrer}
        \nscstats[ATT=2,AGG=2]
        \nscruler
        \begin{nscinventory}
            \nscitem[Waffen] Railgun +1, Bolzenpistole
            \nscitem[Granaten] Granate +1,\newline{}Schockgranate/EMP +1
            \nscitem[R"ustung] Kampfanzug
        \end{nscinventory}
    \end{nscsheet}

    \begin{nscsheet}[h]{John Bozo\newline{}Soldat}
        \nscstats[ATT=2,AGG=2]
        \nscruler
        \begin{nscinventory}
            \nscitem[Waffen] Railgun +1, Bolzenpistole
            \nscitem[Granaten] Granate +1,\newline{}Schockgranate/EMP +1
            \nscitem[R"ustung] Schusssichere Weste
        \end{nscinventory}
    \end{nscsheet}
\end{column}
\begin{column}[r]{0.45}
    \begin{nscsheet}[h]{Flint Ross\newline{}Spezialist}
        \nscstats[ATT=2,AGG=1,DEX=3]
        \nscruler
        \begin{nscinventory}
            \nscitem[Waffen] Railgun +1, Bolzenpistole
            \nscitem[Granaten] Granate +1,\newline{}Schockgranate/EMP +1
            \nscitem[R"ustung] Schusssichere Weste
        \end{nscinventory}
    \end{nscsheet}
\end{column}
\vfill

\pagebreak
\newsection[Sto\3trupp Protektoratsgarnison]{Stosstrupp Protektoratsgarnison}

Zur Unterst"utzung der Charaktere bei der Infiltration der Cyberbrain Forschungseinrichtung kann die Protektoratsarmee eine 
Omega Special-Ops-Einheit bereitstellen.

\begin{column}[l]{0.45}
    \begin{nscsheet}[h]{Stormball\newline{}Truppf"uhrer}
        \nscstats[ATT=3,AGG=3,CON=2]
        \nscruler
        \begin{nscinventory}
            \nscitem[Waffen] Railgun +1, Bolzenpistole
            \nscitem[Granaten] Granate +1,\newline{}Schockgranate/EMP +1
            \nscitem[R"ustung] Kampfanzug
        \end{nscinventory}
    \end{nscsheet}

    \begin{nscsheet}[h]{Thunder\newline{}Attent"ater}
        \nscstats[ATT=3,AGG=3,CON=2]
        \nscruler
        \begin{nscinventory}
            \nscitem[Waffen] Railgun +1, Bolzenpistole
            \nscitem[Granaten] Granate +1,\newline{}Schockgranate/EMP +1
            \nscitem[R"ustung] Kampfanzug
        \end{nscinventory}
    \end{nscsheet}
\end{column}
\begin{column}[r]{0.45}
    \begin{nscsheet}[h]{Jackhammer\newline{}Spezialist}
        \nscstats[ATT=2,AGG=2,DEX=3,CON=2]
        \nscruler
        \begin{nscinventory}
            \nscitem[Waffen] Railgun +1, Bolzenpistole
            \nscitem[Granaten] Granate +1,\newline{}Schockgranate/EMP +1
            \nscitem[R"ustung] Schusssichere Weste
        \end{nscinventory}
    \end{nscsheet}
\end{column}

\medskip
Bei Thunder handelt es sich um einen durch einen der KI infiltrierten Attent"ater, der bei dem Zusammentreffen mit den 
Mitarbeitern der Forschungseinrichtung einen Angriff durchf"uhrt.
