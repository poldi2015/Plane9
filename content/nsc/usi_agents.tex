%% Copyright 2019 Bernd Haberstumpf
%% License: CC BY-NC
% !TeX spellcheck = de_DE
\newsection{USI-Agenten}\anchor{sec:usiagents}

Die United Space Industry (USI) finanziert das KI-Projekt, betreibt "uber Strohfirmen die Cyberbrain-Forschungseinrichtung auf Kallisto und kontrolliert die Attent"ater. Vor Ort im jovianischen System koordiniert der Agent \emph{J.~Smith-Singer} die Operation P9. Smith-Singer wird dabei durch den Psychonauten \emph{Frederic Johnson}, den Strohmann \emph{Dan Ringdaz}, sowie die beiden S"oldner \emph{Lazor} und \emph{Flinn} unterst"utzt. Den Erstkontakt zu Prof.~Dr.~Naratova stellte ein Agent her, den der Assistent des Chefermittlers der Cynaria Corporation im Prolog der Geschichte interviewen darf.

Smith-Singer gibt sich als Agent des Konzernrats aus und hat auch die M"oglichkeit, in gewissem Rahmen in dessen Namen zu agieren. Vor dem Eintreffen der Ermittler auf Valhalla treten die USI-Agenten nicht in Erscheinung. Nach der Landung der Dawn of Day auf Valhalla heften sich die Agenten an die Fersen der Ermittler. Die Identit"at der Charaktere kann je nach Spielverlauf bis zum Abschnitt  ``Im Ice Club'' \cref{sec:ice_club} plausibel zur"uck gehalten werden. Die USI-Agenten kennen die agierenden Mitglieder des Luna-Syndikats nicht und wissen nichts von den freien KIs.

Ziele der USI-Agenten sind:

\begin{itemize}
    \item Unruhe stiften durch Attentate durchgef"uhrt von Hannibal und Slingshot.
    \item Ein Attentat auf der Konferenz mit dem Shigano Kombinat und der Europ"aischen F"oderation.
    \item Einen Krieg im Jovianischen System provozieren.
    \item Sicherstellung der Forschungsergebnisse von  Prof.~Dr.~Naratova.
\end{itemize}

Der prim"are Gegenspieler der Ermittler ist J.~Smith-Singer. Smith-Singer ist ein Pure mit der Statur eines Bodyguards. Sein Name wird in verschiedenen Situationen erw"ahnt.
\vfill
\pagebreak

\begin{column}[l]{0.45}
    \begin{nscsheet}[h]{Frederic\newline{}Johnson}
        \nscstats[ATT=2,COM=3]
        \nscruler
        \begin{nscinventory}
            \nscitem[Psychonaut] 3
            \nscitem[Waffen] Bolzenpistole
            \nscitem[R"ustung] Schusssichere Weste
        \end{nscinventory}
    \end{nscsheet}

    \begin{nscsheet}[h]{Lazor}
        \nscstats[ATT=3,AGG=2]
        \nscruler
        \begin{nscinventory}
            \nscitem[Waffen] Railgun +1
            \nscitem[R"ustung] Schusssichere Weste
        \end{nscinventory}
    \end{nscsheet}


    \begin{nscsheet}[h]{Schl"ager}
        \nscstats[]
        \nscruler
        \begin{nscinventory}
            \nscitem[Waffen] Bolzenpistole
        \end{nscinventory}
    \end{nscsheet}    
\end{column}
\begin{column}[r]{0.45}
    \begin{nscsheet}[h]{Smith-Singer}
        \nscstats[ATT=2,AGG=2,DEX=2,COM=3]
        \nscruler
        \begin{nscinventory}
            \nscitem[Waffen] Bolzenpistole
            \nscitem[R"ustung] Schusssichere Weste
        \end{nscinventory}
    \end{nscsheet}

    \begin{nscsheet}[h]{Dan Ringdaz}
        \nscstats[ATT=2,AGG=2,COM=2]
        \nscruler
        \begin{nscinventory}
            \nscitem[Psychonaut] 2
            \nscitem[Waffen] Bolzenpistole
            \nscitem[R"ustung] Schusssichere Weste
        \end{nscinventory}
    \end{nscsheet}

    \begin{nscsheet}[h]{Flinn}
        \nscstats[ATT=2,AGG=]
        \nscruler
        \begin{nscinventory}
            \nscitem[Waffen] Railgun +1
            \nscitem[R"ustung] Schusssichere Weste
        \end{nscinventory}
    \end{nscsheet}
\end{column}
\vfill
\pagebreak
