%% Copyright 2019 Bernd Haberstumpf
%% License: CC BY-NC
% !TeX spellcheck = de_DE
\newpage
\newsection{USI Agenten}

Die United Space Industry (USI) finanziert das KI-Projekt, betreibt "uber Strohfirmen die Cyberbrain Forschungseinrichtung auf Kallisto und beauftragt die Attent"ater. Vor Ort im jovianischen System koordiniert der Agent \emph{J.~Smith-Singer} die Operation P9. Smith-Singer wird durch den Psychonauten \emph{Frederic Johnson}, den Strohmann \emph{Dan Ringdaz} und die beiden S"oldner \emph{Lazor} und \emph{Flinn} unterst"utzt. Den Erstkontakt zu Prof.~Dr.~Naratova stellte ein Agent her, den der Assisstent des Chefermittlers der Cynaria Corporation  im Prolog der Geschichte interviewen darf.

Smith-Singer gibt sich als Agent des Konzernrates aus und hat auch die M"oglichkeiten, in gewissem Rahmen als dieser zu agieren. Vor dem Eintreffen der Charaktere auf Valhalla treten die USI Agenten nicht in Aktion. Die Agenten setzen sich auf Valhalla nach der Landung der Dawn of Day an die Fersen der Ermittler. Die Identit"at der Charaktere kann aber je nach Spielverlauf bis zur \emph{"`Im Ice Club"'}-Szene gegen""uber den USI Agenten zur"uckgehalten werden. Die USI-Agenten kennen die agierenden Mitglieder des Luna Syndikats nicht und wissen nichts von den freien KIs.

Ziel der USI-Agenten ist es:

\begin{itemize}
    \item Nach der Willkommensgala: Sicherstellen der Forschungsergebnisse von Prof.Dr.~Naratova.
    \item Informationen "uber die Forschung und die Identit"at der Attent"ater so lange wie m"oglich zu vertuschen.    
\end{itemize}

Der prim"are Gegenspieler der Ermittler ist J.~Smith-Singer. Smith-Singer ist ein Pure mit der Statur eines Bodyguards. Sein Name wird in verschiedenen Situationen erw"ahnt.
\vfill
\pagebreak

\begin{column}[l]{0.45}
    \begin{nscsheet}[h]{Frederic\newline{}Johnson}
        \nscstats[ATT=2,COM=3]
        \nscruler
        \begin{nscinventory}
            \nscitem[Psychonaut] 3
            \nscitem[Waffen] Bolter
            \nscitem[R"ustung] Schusssichere Weste +1
        \end{nscinventory}
    \end{nscsheet}

    \begin{nscsheet}[h]{Lazor}
        \nscstats[ATT=3,AGG=2]
        \nscruler
        \begin{nscinventory}
            \nscitem[Waffen] Railgun +1
            \nscitem[R"ustung] Schusssichere Weste +1
        \end{nscinventory}
    \end{nscsheet}


    \begin{nscsheet}[h]{Schl"ager}
        \nscstats[]
        \nscruler
        \begin{nscinventory}
            \nscitem[Waffen] Bolter
        \end{nscinventory}
    \end{nscsheet}    
\end{column}
\begin{column}[r]{0.45}
    \begin{nscsheet}[h]{Smith-Singer}
        \nscstats[ATT=2,AGG=2,DEX=2,COM=3]
        \nscruler
        \begin{nscinventory}
            \nscitem[Waffen] Bolter
            \nscitem[R"ustung] Schusssichere Weste +1
        \end{nscinventory}
    \end{nscsheet}

    \begin{nscsheet}[h]{Dan Ringdaz}
        \nscstats[ATT=2,AGG=2,COM=2]
        \nscruler
        \begin{nscinventory}
            \nscitem[Psychonaut] 2
            \nscitem[Waffen] Bolter
            \nscitem[R"ustung] Schusssichere Weste +1
        \end{nscinventory}
    \end{nscsheet}

    \begin{nscsheet}[h]{Flinn}
        \nscstats[ATT=2,AGG=]
        \nscruler
        \begin{nscinventory}
            \nscitem[Waffen] Railgun +1
            \nscitem[R"ustung] Schusssichere Weste +1
        \end{nscinventory}
    \end{nscsheet}
\end{column}
\vfill
\pagebreak
