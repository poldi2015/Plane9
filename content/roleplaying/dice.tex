%% Copyright 2019 Bernd Haberstumpf
%% License: CC BY-NC
% !TeX spellcheck = de_DE
\newsection{W"urfelw"urfe}

Der Erfolg einer \emph{Aktion} wird, wie beschrieben, durch W"urfel beeinflusst. Das W"urfelergebnis bestimmt, wie erfolgreich eine Aktion gewesen ist. W"urfelw"urfe werden mit einem bis f"unf W6-W"urfeln durchgef"uhrt, wobei das h"ochste W"urfelergebnis, beziehungsweise die h"ochsten W"urfelergebnisse z"ahlen.

Es wird zwischen den folgenden Schwierigkeitsgraden unterschieden: \emph{``Alltagswurf''}, \emph{``Einfacher Wurf''}, \emph{``Risikowurf''} und \emph{``Vergleichender Wurf''}.

\newsubsection{Alltagswurf}
Ein ``Alltagswurf'' kann bei einer allt"aglichen Situation zum Einsatz kommen. Eine allt"agliche Situation ist z.B. eine Suche im ComNetz. Bei einem Alltagswurf werden drei W"urfelergebnisse unterschieden:

\begin{diceroles}
    \sdice{1} & Die Aktion ist fehlgeschlagen, oder dem Ausf"uhrenden ist m"oglicherweise sogar ein Missgeschick unterlaufen.\\
    \rdice{2}{5} & Die Aktion hat Erfolg. \\
    \sdice{6} & Die Aktion hat einen herausragenden Erfolg. \\
\end{diceroles}

\begin{ruleexample}
    Rondra ist auf der Suche nach Informationen in einer einschl"agigen Diskothek. Sie begibt sich erst einmal auf die Tanzfl"ache. Der Spielleiter entscheidet sich f"ur einen Alltagswurf. Rondra w"urfelt:

    \begin{diceroles}
        \sdice{1} & Rondra rempelt mehrere Discobesucher an und ger"at damit ins Visier der T"ursteher.\\
        \rdice{2}{5} & Rondra tanzt ungezwungen und kann dabei andere Discobesucher ansprechen. \\
        \sdice{6} & Rondra legt eine beeindruckende Performance aufs Parkett und erregt das Interesse des Barmanns Bruno, der ihr nur 
            allzu gerne Antworten auf ihre Fragen gibt.\\
    \end{diceroles}
\end{ruleexample}

\newsubsection{Einfacher Wurf}
Ein ``Einfacher Wurf'' beeinflusst das Ergebnis einer nicht allt"aglichen Herausforderung. Folgende W"urfelergebnisse werden unterschieden:

\begin{diceroles}
    \sdice{1} &  Die Aktion ist fehlgeschlagen und hat negative Auswirkungen.\\
    \rdice{2}{3} & Die Aktion hat keinen Erfolg. \\
    \rdice{4}{5} & Die Aktion hat einen Teilerfolg. \\
    \sdice{6} & Die Aktion hat vollen Erfolg.\\
    \tdice{6}{6} & Die Aktion hat einen herausragenden Erfolg.\\
\end{diceroles}

\begin{ruleexample}
    Hektor ist auf der Suche nach Informationen im Raumhafen von Valhalla. Er wird von dem Agenten Johnson beschattet. Der Spielleiter l"asst den Spieler w"urfeln. Wird er den Agenten als solchen erkennen?

    \begin{diceroles}
        \sdice{1} & Der Agent rempelt Hektor unerkannt an und heftet ihm eine Nanowanze an den Overall.\\
        \rdice{2}{3} & Hektor bemerkt nichts.\\
        \rdice{4}{5} & Hektor bemerkt, dass er verfolgt wird verliert den Agenten aber zun"achst wieder aus den Augen.\\
        \sdice{6} & Hektor hat den Agenten entdeckt, ohne das dieser seine Entdeckung bemerkt hat.\\
        \tdice{6}{6} & Hektor hat den Agenten entdeckt und schafft es ihn auf eine falsche F"ahrte zu locken.\\
    \end{diceroles}
\end{ruleexample}

\newsubsection{Risikowurf}
Ein ``Risikowurf'' ist ein W"urfelwurf, bei dem der W"urfelnde geringe Chancen auf Erfolg hat. Folgende W"urfelergebnisse kommen in Betracht:

\begin{diceroles}
    \sdice{1} & Die Aktion ist fehlgeschlagen und hat negative Auswirkungen.\\
    \rdice{2}{5} & Die Aktion hat keinen Erfolg.\\
    \sdice{6} & Die Aktion hat einen Teilerfolg erzielt.\\
    \tdice{6}{6} & Die Aktion hat einen vollen Erfolg.\\
    \hdice{6}{6}{6} & Die Aktion hat einen herausragenden Erfolg.\\
\end{diceroles}

\begin{ruleexample}
    In einem Raumgefecht erleidet Hektors Shuttle einen Schaden an der Steuereinheit und taumelt durchs All. Hektor beschlie\3t, den Schaden an der Bordwand zu reparieren, und verl"asst im Raumanzug das Innere des Schiffs. Der Spielleiter l"asst w"urfeln:
    
    \begin{diceroles}
        \sdice{1} & Hektors Magnetstiefel verlieren die Haftung auf der Bordwand und er schl"agt mit dem Kopf auf. Nun h"angt er          
            bewusstlos am Rettungsseil. Hoffentlich ist noch eine zweite Person an Bord, die ihn retten kann.  \\
        \rdice{2}{5} & Hektor kann den Schaden nicht beheben.\\
        \sdice{6} & Hektor schafft es, einen Teil der Steuereinheit notd"urftig zu reparieren. Hoffentlich reicht es, das Schiff aus        
            der Schusslinie zu bringen.\\
        \tdice{6}{6} & Hektor kann die Steuereinheit wieder in Betrieb nehmen.\\
        \hdice{6}{6}{6} & Hektor repariert die Steuereinheit und koppelt sie gleich noch mit der Waffenkontrolle.\\
    \end{diceroles}
\end{ruleexample}

\newsubsection{Vergleichender Wurf}
Ein ``Vergleichender Wurf'' ist ein regul"arer Wurf, bei dem die Anzahl der verf"ugbaren W"urfel und der Schwierigkeitsgrad durch Aktionen eines ``Gegners'' negativ beeinflusst werden k"onnen.

Die Anzahl der W"urfel, die f"ur eine Aktion bereitstehen, wird durch die Anzahl der W"urfel, die dem Gegner f"ur seine Aktion zur Verf"ugung stehen, reduziert.

\begin{itemize}
    \item Aktionsw"urfel +1 werden um die Anzahl der W"urfel des Gegners reduziert.
    \item Resultat $\leq$ 0 erh"oht die Schwierigkeit.
    \item Es wird mit mindestens einem W"urfel gew"urfelt.
\end{itemize}

Wenn weniger als ein W"urfel "ubrig bleibt, wird mit einem W"urfel gew"urfelt und es erh"oht sich die Schwierigkeit nach folgender Tabelle:

\begin{center}\begin{tabular}{m{3cm} m{5.5ex} m{3.5cm}}
    \textbf{Schwierigkeit} & \textbf{wird} & \textbf{Schwierigkeit} \\\hline
    Alltagswurf            & $\rightarrow$ & Einfacher Wurf \\
    Einfacher Wurf         & $\rightarrow$ & Risikowurf \\
    Risikowurf             & $\rightarrow$ & Risikowurf \\
\end{tabular}\end{center}

\underline{Anmerkung:} Vergleichende W"urfe kommen vor allem bei K"ampfen zum Einsatz. Im Kampf bedeutet dies, dass ein Charakter ohne Kampferfahrung gegen einen kampferfahrenen Gegner immer einen Risikowurf w"urfelt.

\medskip
\begin{ruleexample}
    Hektor verfolgt den Agenten Johnson in den Raumhafen. Johnson ist sich der Gefahr bewusst und will sich der Verfolgung entziehen. Der Spielleiter entscheidet sich f"ur einen Vergleichenden Wurf mit der Schwierigkeit Einfacher Wurf.

    \underline{Reduzierte W"urfel}

    Hektor stehen regul"ar zwei W"urfel zur Verf"ugung. Dem Agenten stehen f"ur sein Versteck auch zwei W"urfel zur Verf"ugung. Hektors W"urfelanzahl wird um einen W"urfel reduziert (Hektors W"urfel +1 minus W"urfel des Gegners). Die Schwierigkeit erh"oht sich nicht, da nach Abzug der gegnerischen W"urfel immer noch ein W"urfel zur Verf"ugung steht.

    \underline{Erh"ohte Schwierigkeit}

    Hektor stehen regul"ar zwei W"urfel zur Verf"ugung. Der Agent ist ein Meister der Tarnung und hat drei W"urfel zur Verf"ugung. Hektors W"urfelanzahl w"urde auf null reduziert. Die Schwierigkeit, den Agenten zu entdecken, steigt auf Risikowurf. Hektor w"urfelt mit einem W"urfel.
\end{ruleexample}

\newsubsection{Kr"afte B"undeln}\anchor{sec:bundleforce}
Bei einer Aktion k"onnen mehrere Personen ihre Kr"afte b"undeln, um ihre Erfolgschancen zu erh"ohen. Das B"undeln ist jedoch nur dann m"oglich, wenn die Spieler glaubhaft erkl"aren k"onnen, wie sie gemeinsam eine Aktion ausf"uhren.

Beim B"undeln einer Aktion werden die W"urfel aller beteiligten Personen einschlie\3lich Boni zusammen addiert. Die maximale Anzahl der W"urfel ist auf f"unf W"urfel, ohne Boni auf drei W"urfel begrenzt. Bei einem vergleichenden Wurf wird die maximale Anzahl an W"urfeln \emph{nach} Abzug der gegnerischen W"urfel auf die maximal zul"assige Anzahl beschr"ankt. B"undeln die Gegner ebenfalls ihre W"urfel, wird die geb"undelte Anzahl an W"urfeln des Gegners abgezogen.

\medskip
\begin{ruleexample}
    Hektor und Rondra versuchen, einen guten Preis f"ur den Flug auf dem Frachter "`Leviathan"' von Valhalla zum Mars auszuhandeln. Der Eigner Dudelwald und seine Frau Sandmann halten dagegen. Beide Gruppen nutzen die F"ahigkeit \stat{Communication} und b"undeln ihre Kr"afte. Hektor hat nur einen W"urfel zur Verf"ugung, Rondra dagegen zwei. Die beiden Eigner haben jeweils einen W"urfel zur Verf"ugung. Hektor und Rondra gehen also mit drei W"urfeln ins Rennen, von denen ein W"urfel abgezogen wird (die Eigner haben $1+1$ W"urfel minus ein W"urfel zur Berechnung den Vergleichenden Wurf). Damit k"onnen Hektor und Rondra zumindest einen Wurf mit zwei W"urfeln durchf"uhren.
\end{ruleexample}
