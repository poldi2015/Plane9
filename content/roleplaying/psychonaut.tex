%% Copyright 2019 Bernd Haberstumpf
%% License: CC BY-NC
% !TeX spellcheck = de_DE
\newsection{Psychonauten}
Psychonauten, wie \cref{sec:technology} beschrieben, k"onnen mithilfe spezieller Ausr"ustung in das Gehirn einer anderen Person eindringen. "Ahnlich wie beim Eintauchen in das ComNetz besteht dann eine vollsensorische Verbindung zwischen beiden Personen. Der Psychonaut ``denkt'' quasi im Gehirn des anderen mit. Zun"achst dringt der Psychonaut in die oberfl"achlichen Gedanken des Hier und Jetzt ein, kann aber auch tiefer in das Unterbewusstsein vordringen, um Wissen und Gef"uhle zu erkunden. Der Psychonaut kann spezifische Informationen anfordern und nimmt diese "ahnlich wie in einem Traum durch Bilder und Sinneseindr"ucke wahr.

Ein Gehirnscan wird mit der F"ahigkeit \textbf{Research} gew"urfelt. Das Opfer des Gehirnscans kann sich mit \textbf{Empathy} bei dem vergleichenden Wurf verteidigen, im Falle eines Psychonauten ebenfalls mit \textbf{Research}.

Ein Gehirnscan ist sowohl f"ur das gescannte Opfer als auch f"ur den Psychonauten gef"ahrlich. Wehrt sich das Opfer bei einem Tiefenscan und erzielt einen Misserfolg, kann sein Gehirn psychischen Schaden erleiden. Auf der anderen Seite ist die Gehirnverbindung bidirektional. Ein Misserfolg beim Scan kann dazu f"uhren, dass das Opfer selbst Informationen aus dem Gehirn des Psychonauten erh"alt oder dass der Psychonaut psychischen Schaden nimmt.

Die F"ahigkeit ``Psychonaut'' ist ist ein \stat{Merit}, das in Absprache mit dem Spielleiter vergeben werden kann. Psychonauten sind in ihrer F"ahigkeit ausgebildet. Psychonaut wird ebenfalls unter \stat{BACKGROUND} als ``Psychonaut[D]'' vergeben.
