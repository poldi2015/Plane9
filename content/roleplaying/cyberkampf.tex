%% Copyright 2019 Bernd Haberstumpf
%% License: CC BY-NC
% !TeX spellcheck = de_DE
\newsection{Cyberkampf}\anchor{sec:cyberkampf}
Cyberkampf ist der Kampf im ComNetz. Ein Charakter kann sich in das ComNetz mittels eines station"aren Terminals, eines Computers oder seines ComLinks einloggen. Weitaus effektiver ist jedoch das vollsensorische Eintauchen mittels eines Kontrollmoduls. Der Charakter taucht dann mit all seinen Sinnen in die virtuelle Realit"at des Datennetzes ein. Mittels Gedankenkontrolle lassen sich Befehle absetzen, und Informationen k"onnen "uber alle Sinnesorgane aufgenommen werden. F"ur diese Art des Einstiegs ins Netz ist eine breitbandige Kabelverbindung notwendig.

Zur Informationsbeschaffung oder "Uberwachung im Netz k"onnen Agentensysteme mit Auftr"agen losgeschickt werden. Andere Systeme im Netz k"onnen "uber eine Cyberattacke, unterst"utzt durch Agentensysteme und Viren, angegriffen werden.

Auch das ComLink zusammen mit dem Personal Area Network (PAN) eines Netzteilnehmers oder das Kontrollmodul sind Systeme im Netz, die angegriffen werden k"onnen. Eingespeiste Viren k"onnen die Headware des Teilnehmers beeintr"achtigen und dadurch Sinneswahrnehmungen beeinflussen, Hormoneinspeisungen einleiten oder k"unstliche K"orperfunktionen st"oren. Firewallsoftware sch"utzt alle Formen von Systemen im Netz vor Cyberattacken.

Eine Cyberattacke wird wie eine physische Attacke durch vergleichende W"urfelw"urfe durchgef"uhrt. Cyberangriffe und das Programmieren von Viren und Agentensystemen werden mit der F"ahigkeit \textbf{Technics} gew"urfelt.
