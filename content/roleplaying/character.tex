%% Copyright 2019 Bernd Haberstumpf
%% License: CC BY-NC
% !TeX spellcheck = de_DE
\newsection{Der Charakter}\anchor{sec:character}

Die F"ahigkeiten und Eigenschaften eines Charakters sind, neben den menschlichen Grundf"ahigkeiten, durch seinen Hintergrund und spezielle St"arken und Schw"achen bestimmt. 

\begin{description}
    \item[Grundf"ahigkeiten] Grundf"ahigkeiten sind F"ahigkeiten die jeder Humanoiden mehr oder weniger beherrscht. Alles, wof"ur keine
        spezielle Ausbildung, kein spezielles ethnischer oder kulturelles Hintergrundwissen und keine speziellen K"orpereigenschaften ben"otigt werden, f"allt in diese Kategorie. Jeder der es bis ins All geschafft hat, kann lesen, schreiben und rechnen. Auch kann jeder eine entsicherte Feuerwaffe abfeuern. Alle Bewohner der au\3er terrestrischen Kolonien k"onnen sich miteinander irgendwie unterhalten. Gesprochen wird ein Kauderwelsch aus Englisch mit starken Einfl"ussen von Russisch, Chinesisch und Spanisch. Ausfl"uge mit einem Raumanzug ins All sind hingegen nicht selbstverst"andlich. 
        
        Aktionen in einer Grundf"ahigkeit werden mit \textbf{1W6} gew"urfelt. 
    \item[Hintergr"unde] Hintergr"unde beginnen mit einer Kulturkreis entsprechender Muttersprache, fortgef"uhrt durch Schulbildung und 
        Profession mit entsprechendem Wissen und entsprechenden F"ahigkeiten wie auch Kontakten. Der Hintergrund erweitert Grundf"ahigkeiten des Charakters individuell. Hintergr"unde werden ebenfalls mit \textbf{1W6} gew"urfelt. Was ein Hintergrund im Detail umfasst, kann in einer Szene individuell durch den Spielleiter passend zum Spielverlauf bestimmt werden. Kann der Spieler aufgrund seines Hintergrunds glaubhaft darlegen, dass er eine F"ahigkeit beherrscht, kann er diese auch anwenden.
    \item[St"arken] St"arken beschreiben was ein Charakter besonders gut kann. St"arken erlauben es mit zus"atzlich W"urfeln zu w"urfeln. 
    \item[Talente \& Schw"achen] Talente sind besondere Eigenschaften wie eine fotografisches Ged"achtnis. Schw"achen k"onnen k"orperliche  
        Gebrechen, Phobien, eine Sucht aber auch eine rachs"uchtige Exfreundin sein. Sie k"onnen durch den Spielleiter oder auch vom Spieler eingesetzt werden, um dem Charakter mehr pers"onlichen Tiefgang zu geben.
\end{description}        
