%% Copyright 2019 Bernd Haberstumpf
%% License: CC BY-NC
% !TeX spellcheck = de_DE
\newsection{Die Helden}\anchor{sec:helden}

In einer Rollenspielkampagne "ubernehmen die Spieler die Rollen eines Helden. Als Held treten die Charaktere aus der Masse der Normalsterblichen heraus. Helden k"onnen ausweglose Situationen meistern, denn sie haben \emph{Gl"uck}, mehr Gl"uck als normale Personen. 

\begin{column}[l]{0.52}
    Im Regelwerk wird das Gl"uck durch Schicksalspunkt, \stat{FATE} auf dem Charakterblatt repr"asentiert, die w"ahrend des Spiels ausgegeben werden k"onnen.
\end{column}
\begin{column}[r]{0.48}
    \centering
    \includegraphics[width=0.80\textwidth]{images/character_fate.jpg}    
\end{column}

Schicksalspunkte werden nach jeder Szene wieder aufgef"ullt. Das Ausgeben eines Schicksalspunktes, erlaubt es dem Spieler, einen oder mehrere \textbf{W"urfel noch einmal zu w"urfeln oder} alternativ \textbf{einen W"urfel zus"atzlich} zu werfen. Ein W"urfelwurf darf dabei allerdings maximal mit vier W"urfeln, bei Anwendung von Waffen- oder R"ustungensbonus mit f"unf W"urfeln ausgef"uhrt werden. F"ur jede Aktion darf nur maximal ein Schicksalspunkt ausgegeben werden. F"ur jeden ausgegebenen Schicksalspunkt wird ein K"astchen auf dem Charakterdatenblatt unter FATE abgestrichen.

Die Helden sind im Normalfall keine ungebildeten Tagel"ohner. Sie besitzen eine gute Ausbildung, Kontakte und Ressourcen. K"orpermodifikationen wie ein Kontrollmodul d"urfen vorausgesetzt werden. Helden sind zumindest rudiment"ar mit dem Umgang eines Raumanzugs vertraut.

Helden sind z"ah. Wie in einem Roman stehen Helden in jeder neuen Szene frisch auf der Matte, um die n"achste Herausforderung zu meistern. Die Szenen in einer Geschichte sind meist so gestaltet, dass dazwischen Zeit vergeht und die Charaktere sich in einer sicheren Umgebung aufhalten k"onnen. So k"onnen sie sich erholen. Verletzungen, die ein Weiterspielen behindern oder unglaubw"urdig machen, k"onnen in einer sicheren Umgebung geheilt werden. Die Medizin wird sich in den zuk"unftigen 200 Jahren stark weiterentwickelt. KI gest"utzte Autobots assistieren bei medizinischen Eingriffen oder f"uhren Operationen selbst durch. Innere Organe lassen sich durch k"unstliche Organe ersetzen. Gez"uchtete und k"unstliche Gliedma\3en lassen nat"urliche Gliedma\3en voll funktionsf"ahig ersetzen. Mikrobenkulturen erlauben, Verletzte zu stabilisieren.
