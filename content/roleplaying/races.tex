%% Copyright 2019 Bernd Haberstumpf
%% License: CC BY-NC
% !TeX spellcheck = de_DE
\newsection{Rassenbesonderheiten}

Den Spielern stehen Norms, Alpha-Mutanten und Omega-Krieger zur Auswahl. Bez"uglich dem Regelwerk gelten folgende Besonderheiten:

\begin{description}
    \item[Norms] Norms sind normale Menschen. Im Falle der Spielercharaktere handelt es sich normalerweise um Menschen mit einer 
        Hochschulausbildung im zivilen oder milit"arischen Sektor. Ihr St"arken liegen in den Attributen \stat{EMPATHY} und \stat{EDUCATION}. In einem dieser Bereiche k"onnen bis zu zwei zus"atzliche Punkte bei den F"ahigkeiten vergeben werden. 
        
        Die Charaktere k"onnen in ihrer Laufbahn eine leitende Stelle eingenommen haben und werden von Konzernmitarbeitern als Ansprechpartner wahrgenommen. 
        
        Die Charaktere haben in ihrer Vergangenheit m"oglicherweise hochwertige K"orpermodifikationen durchf"uhren lassen.
      
        Norms besitzen einen famili"areren Hintergrund, auch wenn der Kontakt abgebrochen sein mag, bestehen, wenn n"otig, Kontakte aus Schulzeit und Hochschulzeiten.
    \item[Alpha-Mutanten] Alpha-Mutanten sind in einer Zuchtfarm auf dem Mars als Arbeiter ausgebildet worden und haben keine Eltern. 
        Sie geh"oren der Arbeiterklasse an. Sie sind handwerklich gut ausgebildet. Aufgrund ihres genetischen Zuchtmaterials sind sie gr"o\3er, st"arker und widerstandsf"ahiger als Norms. Bei k"orperlichen T"atigkeiten wie auch bei K"orperbelastungen sollten diese k"orperlichen Besonderheiten bei den Auswirkungen von Aktionen ber"ucksichtigt werden. Die St"arken der Alpha-Mutanten liegen damit auf dem Attribut \stat{BODY}. Bei den entsprechenden F"ahigkeiten k"onnen dadurch jeweils bis zu zwei Punkte zus"atzlich ausgegeben werden. 
        
        Alphas werden als Teil der Zivilgesellschaft wahrgenommen und werden aufgrund ihrer meist umg"anglichen Art im extraterrestrischen Umfeld freundschaftlich behandelt und als zuverl"assig gesch"atzt. 
        
        Alpha-Mutanten sind im Normalfall mit dem Leben in Schwerelosigkeit vertraut. Alpha-Mutanten haben eine handwerkliche und meist eine logistische Ausbildung erhalten. 
        
        Sie haben innerhalb ihrer Zuchtreihe jeweils eine eigene Sprache entwickelt und k"onnen sich damit mit ihren ``Artgenossen'' aber auch anderen Alpha-Mutanten, unverst"andlich f"ur Norms und Omega-Mutanten unterhalten. 
    \item[Omega-Krieger] Omega-Krieger sind entweder auf dem Mars oder in Zuchtfarmen im erdnahen Orbit aufgewachsen. Sie sind   
        gr"o\3er und widerstandsf"ahiger als Alpha-Mutanten und Norms. Bei allen k"orperlichen Aktionen und Folgen von Verletzungen m"ussen diese "uberlegenen Eigenschaften ber"ucksichtigt werden. Ihre St"arke im W"urfelsystem liegt wie bei Alpha-Mutanten im Attribut \stat{BODY}. Hier k"onnen auf alle F"ahigkeiten bis zu zwei Punkte vergeben werden. 
        
        Omega-Krieger werden von klein auf den Krieg vorbereitet und ausgebildet. Nah- und Fernkampff"ahigkeiten sind vorauszusetzen. 
        Omega-Krieger sind mit milit"arischen K"orpermodifikationen ausgestattet. Omega-Krieger haben eine strategische und logistische Ausbildung f"ur Krisensituationen erhalten. Sie sind f"ur die m"uhelose Bewegung in Schwerelosigkeit vorbereitet. Sie haben auch eine medizinische Ausbildung als Sanit"ater erhalten. Viele Omega-Krieger sind ausgebildete Piloten oder haben eine Ausbildung als Schiffskommandant absolviert. 
        
        Aufgrund ihrer genetischen Programmierung sind Omega Mutanten leicht reizbar und gehen schnell offensiv mit einer Provokation oder einer Bedrohung um. Sie unterliegen deshalb dem \stat{FLAW} ``reizbar''. Von anderen Personen werden sie oft als militanten Bedrohung wahrgenommen.

        Omega-Krieger sind in den meisten F"allen Teil einer Armee, d.h. sie arbeiten als Soldaten oder S"oldner, in den meisten F"allen f"ur einen Konzern. In den Wirren des letzten Jahrhunderts sind aber auch eine Reihe von Omega-Kriegern desertiert oder ihre Einheiten wurden aufgel"ost.
\end{description}

\underline{Anmerkung:} Eine komplette "Ubersicht "uber alle Rassen findet sich \cref{sec:humanraces}.
