%% Copyright 2019 Bernd Haberstumpf
%% License: CC BY-NC
% !TeX spellcheck = de_DE
\newsection{Raumkampf}
Bei einer Auseinandersetzung im Weltall k"onnen K"ampfe mit den \cref{sec:spaceship} beschriebenen Schiffsklassen ausgefochten werden.

Den Charakteren stehen "ublicherweise F"ahren, Frachter oder eventuell auch eine Korvette zur Verf"ugung. Korvetten und J"ager sind mit Railguns oder Gau\3kanonen ausgestattet, gelegentlich auch mit Raketen oder Torpedos. Mehrere Hundert Meter gro\3e Kriegsschiffe wie Fregatten, Schlachtkreuzer und Flottentr"ager sind mit Railguns, Gau\3kanonen, Raketen und Torpedos best"uckt. Railguns werden bei Schiffen gr"o\3er als eine Korvette aufgrund ihrer hohen Feuerfrequenz prim"ar als Verteidigungswaffen genutzt. Sonden k"onnen zur Aufkl"arung "uber gro\3e Entfernungen eingesetzt werden.

Technisch besteht ein Schiff "uberwiegend aus einem Fusionstriebwerk als Hauptantrieb, einer Reihe von Steuerungstriebwerken, einem Lebenserhaltungssystem und einer gro\3en Anzahl von Sensoren. Alle Systeme, bis auf Triebwerke, sind mit Redundanzsystemen ausgestattet. Da die Mannschaft im freien Raum komplett auf sich allein gestellt ist, k"onnen die meisten technischen Systeme eines Schiffs autonom repariert und ausgetauscht werden. Ein Weltraumspaziergang ist f"ur die Crew eines Schiffes kein au\3ergew"ohnlicher Vorfall und kann auch bei hoher Fluggeschwindigkeit durchgef"uhrt werden. Ein Gro\3teil der Zeit befindet sich ein Schiff im Driftflug, d.h. mit ausgeschalteten Triebwerken. Es herrscht Schwerelosigkeit an Bord. Nur unbedingt notwendige Systeme sind aktiv. Der Reaktor des Haupttriebwerks kann innerhalb einer Minute hochgefahren werden und bietet dann ein Vielfaches an Schub im Vergleich zu einem klassischen Raketenantrieb. Ein Fusionstriebwerk wie auch die Steuerd"usen ben"otigen Treibmasse, allerdings in deutlich geringerer Menge als bei heutigen Raketen.

Im freien Raum sind die Gegebenheiten viele Stunden vor einem Zusammentreffen einsehbar. Der taktische Computer des Schiffes zeigt alle "uber die Sensorik erfassten Daten in einer 3D-Ansicht an. Auf dem Display oder eingeblendet im Gehirn sind die aktuellen und prognostizierten Flugbahnen von Schiffen, Planeten, Monden und anderen Himmelsk"orpern einsehbar, sofern sie erfasst werden k"onnen.

Ein strategisches Planen von Raumk"ampfen auf einem Schlachtplan w"are m"oglich. Ist der Kampf jedoch nicht der Fokus der Geschichte, sollten nur einige kurze Kampfszenen ausgespielt werden. Die Schiffskontrolle und die Kontrolle der Waffensysteme werden durch die F"ahigkeit \textbf{Dexterity} abgebildet. F"ur notwendiges Fachwissen bei Reparaturen wird die F"ahigkeit \textbf{Technics} herangezogen. F"ur Flugplanung und Raumkartografie wird die F"ahigkeit \textbf{Knowledge} ben"otigt. F"ur einen Au\3enspaziergang w"urfelt man auf \textbf{Dexterity}.
