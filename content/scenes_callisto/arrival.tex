%% Copyright 2019 Bernd Haberstumpf
%% License: CC BY-NC
% !TeX spellcheck = de_DE
\pageimage{images/dawn_of_day.jpg}

\newsection{Im Zentrum von Valhalla}

Im Orbit "uber Valhalla halten sich gro\3e Frachter in Position. Shuttles und F"ahren verkehren zwischen Kallisto, den gro\3en Schiffen und anderen Zielorten im jovianischen System. Der Anflug auf Valhalla bringt die Ermittler in das gesellschaftliche Zentrum der Stadt, dem Raumhafen. Bei der Ankunft im Raumhafen, hundert Meter unterhalb der eisigen Oberfl"ache des Mondes herrscht entsprechend reges Treiben. Der Dawn of Day wird routiniert eine Andockposition zugewiesen. Mit gro\3en Formalien m"ussen die Ermittler nicht rechnen. Sie sind bereits angek"undigt. 

F"ur Valhalla wie auch f"ur nahezu alle Siedlungen und Stationen im Sonnensystem ist ein k"unstlicher Tag- und Nachtzyklus durch passende Lichtverh"altnisse in der Oberstadt, d.h. in den Bereichen um den Raumhafen eingerichtet. F"ur den Raumhafen gilt so ein Zyklus nicht. Im Raumhafen herrscht Tag und Nacht Hochbetrieb. Im unteren Teil des Raumhafens ist der kleine Garnissonsst"utzpunkt des Protektorats untergebracht. Dem Raumhafen schlie\3t sich der Stadtteil Headquarter an. Hier sind alle kleinen und gro\3en Konzerne untergebracht. Zusammen mit Rosenfurth dem "`Down-Town"' von Valhalla bilden er die Oberstadt. Die daran anschlie\3enden Stadtgebiete \emph{Paradise City}, \emph{Neu Gr"oning} und \emph{R"otheim} bilden einen herben Kontrast zu den "`zivilisierten"' Stadtteilen der Oberstadt. Eine urbane Infrastruktur fehlt hier weitestgehend. Ein Schmelztiegel aus Fl"uchtlingen, ethnischer Gruppen und Glaubensgemeinschaften bev"olkern das weit reichende Tunnel und R"ohrensystem.

\begin{remarks}
    Die Stadtgebiete von Valhalla werden im Detail \cref{sec:valhalla} beschrieben.
\end{remarks}