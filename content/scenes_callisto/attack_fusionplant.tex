%% Copyright 2019 Bernd Haberstumpf
%% License: CC BY-NC
% !TeX spellcheck = de_DE
\newsection{Angriff auf das Fusionskraftwerk}

W"ahrend des Attentats, das aus dem Fusionskraftwerk nicht verfolgt werden kann, kehren die Ermittler vom Kommunikationsstand aus zur"uck zum Kontrollzentrum. Bei ihrem Eintreten dr"angt sich einer der Kommunikationsoffiziere an ihnen vorbei und verk"undet bleich:

\say{Es geht los. Landungspods von beiden Schiffen steuern Kallisto an. Die sind in ein paar Minuten da!} Nemessis herrscht ihn an: \say{Los auf den Schirm!}

Aber es bedarf gar keiner Aufforderung. Ein gro\3er Wandschirm erwacht zum Leben und zeigt in starker Vergr"o\3erung die anfliegen Landungsboote und eine schematische Darstellung der K"ampfe im Orbit.

Truppen des Protektoratsmilit"ars dringen kurze Zeit sp"ater in Valhalla ein. Den Protektoratstruppen knapp auf den Fersen sind die Landungstrupps des Kreuzers Zeus II-1. Bevor alle Kommunikationswege durch St"orfunk zusammen brechen, kann das "Uberwachungssystem im Leitstand des Kraftwerks Videoaufnahmen der K"ampfe in den Tunneln von Valhalla liefern. 15 Minuten sp"ater st"urmen Blackhearts Truppen das Fusionskraftwerk. Eine Spezialeinheit des Protektorats, gef"uhrt vom Omega \emph{Ironfist}, besetzen die Kraftwerkszentrale. Es kommt zu einem Gefecht, als zeitgleich Kampfdroiden in das Kontrollzentrum eindringen. Am Ende behalten die Soldaten des Protektorats die Oberhand und k"onnen die Droiden vernichten. 

Nach der Auseinandersetzung werden die im Kraftwerk anwesenden Gangster in angrenzenden R"aumen des Leitstandes festgesetzt. Nemessis ist zu diesem Zeitpunkt nicht mehr aufzufinden. Da ein Omega Teil des Investigatoren ist, wird die Gruppe selbst nicht in Gewahrsam genommen. Der Omega des Teams ist im Zweifel rangh"oher als der Truppf"uhrer Ironfist und kann, wenn sich der Staub gelegt hat, eine Zusammenfassung der Kampfhandlungen einfordern. Ironfist wird ihn dazu an die Kommandantin der Besatzungstruppen \emph{Tornbull} weiter vermitteln. Bei einem Gespr"ach mit der Kommandantin, kommt es immer wieder zu Unterbrechungen, da es in den Stadtgebieten weiterhin zu Kampfhandlungen kommt.

W"ahrend das Protektorat seinen Angriff startet, verl"asst \xl{} unbemerkt das Fusiundonskraftwerk und begibt sich zur"uck zum Sunshine Hotel. Dort bringt sie \ml{} unerkannt aus dem Geb"aude. Ihr Ziel ist es in Gefolgschaft von \ml{} mit ihrem Schiff der \emph{Dragon Blade} zur Nike Station aufzubrechen. Wenn nicht durch die Ermittler aufgedeckt, ahnt \ml{} bereits, wer ihr wirklich gegen"uber steht. Ein wortloser Augenkontakt unterstreicht ihre Vermutung. Zusammen beschlie\3en die beiden Frauen einen der Guardian KI-Kreuzer mittels \ml{}s Virus, den sie auch schon den Ermittlern angeboten hat, von den USI Fesseln zu befreien und zu "ubernehmen. 

\begin{remarks}
	Auch wenn den Charakteren nach ihrer R"uckkehr aus der Zone die Waffen abgenommen wurden, sollte der Omega der Gruppe als leitender Offizier mit in die Kampfhandlung, bei der Erst"urmung des Leitstandes, eingreifen. Das gebietet die Ehre der Omega Krieger.
	
	Da der Angriff unter gr"o\3ter Geheimhaltung vorbereitet wurde, wird Blackheart die Gruppe nicht vorher informieren und auch nicht in den Angriff mit einbinden. Erst bei St"urmung des Geb"audes findet ein Kontakt mit den Ermittlern statt. Die Freund-Feind Erkennung identifiziert den Omega der Gruppe vor dem Betreten des Geb"audes. 
	
	W"ahrend der Besetzung von Valhalla besteht f"ur die Ermittler keine M"oglichkeit Blackheart, Thunderbolt oder sonst jemanden au\3erhalb Breidabliks zu kontaktiert.

	Kurz vor der Erst"urmung des Kraftwerks verl"asst Nemessis zusammen mit einer Handvoll Getreuer den Leitstand "uber geheime Wege um einen Wiederstand aufzubauen. Die zur"uckgelassenen Gangster haben bereits die Anweisung erhalten die Protektoratsstreitkr"afte zu unterst"utzen, sollten diese das Kraftwerk besetzen. Nemessis steht den Charakter also nicht mer direkt zur Verf"ugung. Wenn f"ur das weitere Spielgeschehen hilfreich, k"onnen die Ermittler allerdings in Verhandlungen zwischen dem Milit"ar und Nemessis eingebunden werden und so eine Kommunikation erm"oglicht werden. 
\end{remarks}
