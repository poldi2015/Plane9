%% Copyright 2019 Bernd Haberstumpf
%% License: CC BY-NC
% !TeX spellcheck = de_DE
\newsection{Zwischenfall auf Fenris (optional)}
% Gestrichehn da: Derzeit gibt es kein gutes Konzept bei dem Tiger nicht im Rondra Hospital operiert wurde aufmerksam
% ohne auf das Rondra Hospital zu machen.

Beim Anflug oder w"ahrend der ersten Nachforschungen auf Kallisto werden die Charaktere von Lord Commander Grendel dem Kommandanten der Fenris Station kontaktiert. Auf der Basis der Streitkr"afte konnte ein Attent"ater dingfest gemacht werden, der die Computersysteme der Station zu manipulieren versuchte. Beim Attent"ater handelt es sich um einen Omega-Soldaten. Der Soldat mit dem Namen \emph{Tiger} beteuert, sagt Grendel, von dem Attentat nichts zu wissen, obwohl er sich seiner Festnahme widersetzte und einen Kameraden schwer verletzt hat. Die Ermittler werden gebeten, sich schnellstm"oglich auf der Fenris Station einzufinden, um dem Verh"or beizuwohnen.

Beim Landeanflug auf die Fenris Station kommt es dann zu einem unerwarteten Zwischenfall. Die Verteidigungsanlagen der Station nehmen das Shuttle der Ermittler unter Beschuss. Geschosse durchschlagen die Bordwand, es kommt zu einem Druckverlust, der Zentralcomputer f"allt aus, was zu einer Notabschaltung des Fusionstriebwerks f"uhrt. Nach der Abschaltung des Triebwerks tritt im Shuttle Schwerelosigkeit ein. Der sekund"are Bordcomputer "ubernimmt. Niemand wird verletzt, aber das Schiff wird ordentlich durchgesch"uttelt und die Passagiere werden unsanft aus der virtuellen Realit"at des Bordsystems gerissen. Notbeleuchtung und der Schiffsalarm weisen unmissverst"andlich auf den Ernst der Lage hin. Die Ermittler tragen gl"ucklicherweise einen Druckanzug, um die Kr"afte des Anflugs zu kompensieren, m"ussen aber noch die Atemmaske anlegen, die jeweils in einem Fach ihrer Beschleunigungsliegen bereitliegt. Bei einem Abstand von rund 1'200 Kilometern rast das Shuttle mit rund 600 m/s auf die Fenris Station zu. Die Ermittler haben nur noch eine halbe Stunde Zeit um das Haupttriebwerk zu reaktivieren und das Bremsman"over abzuschlie\3en, um zu verhindern, dass ihr Shuttle an der Station zerschellt.

W"ahrend das Fusionstriebwerk reaktiviert wird, muss die Funkanlage, die ebenfalls ausgefallen ist, wieder in Betrieb genommen werden. Die Anlage muss auf die Notantenne umgeschaltet werden, da die Hauptantenne beim Angriff besch"adigt wurde. Ist die Funkanlage wieder verf"ugbar, kann die Fenris Station kontaktiert werden. Die Flugleitung antwortet auf den Ruf. Mit geknickter Stimme fragt der Flugleitstand nach der Situation auf der Dawn of Day und meldet, dass die Verteidigung der Station aufgrund einer noch nicht gekl"arten Fehlfunktion das Shuttle unter Beschuss genommen hat. Ein Anflug des Landedecks ist derzeit nur manuell m"oglich. Wenn N"otig entsendet die Station ein Shuttle, um die Dawn of Day abzuschleppen.

Lord Commander Grendel, in Begleitung von zwei weiteren Omega-Soldaten, nimmt die Ermittler pers"onlich in Empfang. Er erkl"art den Besuchern, dass vermutet wird, dass der Angriff auf das Shuttle mit dem Sabotageakt in Zusammenhang steht. Die Verteidigungsanlage ist derzeit komplett deaktiviert, heruntergefahren und vom Rest der Stationssysteme getrennt. Lord Marshal Blackheart ist bereits "uber die Vorkommnisse auf der Station informiert und hat angek"undigt, sich selbst vor Ort ein Bild machen zu wollen.

Laut Lord Commander Grendel wurde Tiger "uberrascht, w"ahrend er sich im zentralen Computerkabinett an den Speicherb"anken zu schaffen machte. Als Waffentechniker hatte er zu diesem Bereich keinen Zugriff und sollte eigentlich auch keine Expertise f"ur die Rechenanlage besitzen. Bei seiner Entdeckung griff er sofort zu seiner Waffe und feuerte mehrere Sch"usse auf den Sergeant der Patrouille ab, die ihn entdeckt hatte. Der Sergeant ging zu Boden. Seine Begleiter konnte Tiger "uberw"altigen und Hilfe anfordern. Bei der Erstbefragung beteuerte der Gefangene, sich in keinster Weise an die Vorg"ange erinnern zu k"onnen.

Tiger ist in einer Arrestzelle zum Verh"or festgesetzt worden und wird dort von zwei Soldaten bewacht. Der Gefangene sitzt in einem durch Gitter abgesperrten Teil der Zelle. Im Besucherteil halten zwei bewaffnete Omega-Soldaten Wache. Der Attent"ater wirkt beim Eintreffen der Ermittler stark angespannt, bei\3t die Z"ahne zusammen und antwortet auf keine Fragen. Lord Commander Grendel erw"agt, Tiger durch Wahrheitsdrogen gespr"achig zu machen. Bestehen die Ermittler darauf, einen Gehirnscan durchzuf"uhren, wird ihnen diese Bitte widerwillig gew"ahrt. Der Psychonaut des Teams muss daf"ur den abgesperrten Teil der Zelle betreten. Tiger ist mit Hand- und Fu\3fesseln auf einem Stuhl fixiert. Betritt der Psychonaut den Gefangenenteil l"osen sich die elektronisch verriegelten Fesseln, und der Gefangene st"urzt sich auf den Ermittler. Die Wachen ziehen ihre Waffen und werden den Gefangenen niederschie\3en, sofern niemand eingreift.

Kann der Gefangene lebendig "uberw"altigt werden, kann der Psychonaut zur Tat schreiten. Zum Attentat finden sich keinerlei Erinnerungen in den Gedanken des Soldaten. Allerdings st"o\3t der Psychonaut auf seltsam artifiziell wirkende Gedankenfragmente. Nach dieser Erkenntnis wird der Psychonaut mental angegriffen. Wie bei den beiden Attent"atern auf Hellgate wurde auch Tiger eine KI in das Gehirn eingesetzt.  Kann der Psychonaut die KI "uberwinden und die Gedankenwelt der KI durchdringen, durchfluten die panischen Gedanken Tigers sein Gehirn. Eine Szene dr"angt sich in den Vordergrund: Tiger tritt aus einem K"afig, umringt von jubelnden Fans, Omegas und Norms. Ein Norm in Anzug tritt ihm entgegen und spricht ihn. Abrupt brechen die Gedankenstr"ome ab. Der Geist Tigers beginnt sich aufzul"osen. Der Psynchonaut wei\3, was das bedeutet. Ein Brainburner beginnt das Gehirn der Soldaten zu zerst"oren. Der Psychonaut muss so schnell wie m"oglich das Gehirn verlassen, um nicht ebenfalls vom Brainburner erfasst zu werden.

Erkundigen sich die Ermittler nach Tigers Vergangenheit, erfahren sie, dass Tiger erst seit kurzem den Protektoratsstreitkr"aften beigetreten ist. Vor zwei Monaten hatte er sich bei der Garnison auf Valhalla verpflichtet und wurde aufgrund seiner Erfahrung als Waffentechniker auf der Fenris Station stationiert. Seine Vorgeschichte ist nicht weiter bekannt. Ger"uchteweise hat er vorher irgendwo auf Valhalla seinen Lebensunterhalt mit illegalen Zweik"ampfen verdient. 

Wollen die Ermittler die Fehlfunktionen im Computersystem untersuchen, wird das einige Stunden in Anspruch nehmen. Im Computersystem finden sich Spuren eines komplexen, selbst denkenden Virus, der etwaige Analysten angreift. Danach l"oscht sich der Virus selbst. W"ahrend sich die Ermittler dem Computersystem widmen, trifft Lord Marshal Blackheart auf der Station ein und l"asst sich im Beisein der Ermittler des Protektorats "uber den Stand der Ermittlungen informieren. Die Cynarian-Ermittler werden dabei ausgeschlossen (milit"arische Angelegenheiten).

\begin{remarks}
	\underline{Gewonnene Information:}

	Je nach Vorgehen der Charaktere:
	
	\begin{itemize}
		\item Tiger Gehirn wurde manipuliert.
		\item Teile des Computersystems auf der Fenris Station sind von einem hochintelligenten Virus befallen.
	\end{itemize}

	\underline{Optionaler Spielverlauf:}

	Der Zwischenfall auf Fenris kann genutzt werden, wenn die Charaktere auf Hellgate nicht mit den KIs in den K"opfen der beiden Attent"ater in Kontakt kommen konnten. Der Vorfall motiviert ein aggressives Vorgehen von Blackheart im sp"ateren Spielverlauf. Er ist eine gute M"oglichkeit, die kosmonautischen F"ahigkeiten der Charaktere zum Einsatz zu bringen. Auf der anderen Seite verl"angert der Zwischenfall den Spielablauf signifikant, ohne zwingende Relevanz f"ur die Geschichte zu haben. Er ist deshalb als optional gekennzeichnet.

	\underline{Gehirnscan:}

	Das Gedankenduell ist im Regelwerk im hinteren Teil des Buches \cref{sec:cyberkampf} beschrieben. Eine "ahnliche Szene wie das Duell auf der Fenris Station findet bei einem Gehirnscan von Hannibal, dem Attent"ater auf Hellgate \cref{sec:interrogatehanibal} statt.

	Wie auch beim Hannibals Gehirnscan auf Hellgate erw"ahnt sollte der Spielleiter den Begriff K"unstliche Intelligenz (KI) vermeisen um dem Geger nicht sofort einen Namen zu geben.

	\underline{Tigers Zweikampf:}

	Die kurze Szene, die sich in den Tigers Gedanken abspielt, findet im Blackhole Club (beschrieben \cref{sec:blackholeclub}) in der Arena des Clubs, in der Tiger gerade einen Zweikampf gewonnen hat, statt. Tiger trifft in dieser Szene das erste Mal auf den USI-Agenten \emph{Frederic Johnson}, der ihm ein Upgrade seiner Cyberware anbietet. Die in dieser Geschichte auftretenden USI-Agenten sind \cref{sec:usiagents} beschrieben.

	\underline{Der Virus:}

	W"ahrend des Verh"ors ist der selbst denkende Virus im Computerystem der Station nach wie vor aktiv und befreit Tiger von seinen Fesseln. Die KI in Tigers Sch"adel "ubernimmt daraufhin und greift die Ermittler in der Zelle an.

	% TODO: Anpassungen: Dieses Kapitel, Mailin Kapitel, Cyberbrain Kapitel?, Tiger in Personenverzeichnis, BlackholeClub (incl. K"ampfe von xl)	
\end{remarks}
