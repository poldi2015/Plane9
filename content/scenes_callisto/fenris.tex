%% Copyright 2019 Bernd Haberstumpf
%% License: CC BY-NC
% !TeX spellcheck = de_DE
\newsection{Zwischenfall auf Fenris (optional)}

Der Zwischenfall auf Fenris kann genutzt werden, wenn die Charaktere auf Hellgate nicht mit den KIs in den Köpfen der beiden Attentäter in Kontakt kommen konnten. 

Der Vorfall hilft ein aggressiveren Vorgehen von Blackheart im späteren Spielverlauf authentischer zu machen. Er ist eine gute Möglichkeit die kosmonautischen Fähigkeiten der Charaktere zum Einsatz zu bringen. Der Zwischenfall auf der anderen Seite verlängert allerdings auch den Spielablauf signifikant ohne zwingende Relevanz für die Geschichte zu haben. Er ist deshalb als optional gekenntzeichnet.

Kurz nach den ersten Ergebnissen der Nachforschungen auf Kallisto werden die Charaktere vom Kommandanten Lord Commander Grendel der Fenris Station kontaktiert. Auf der Raumbasis konnte ein Attentäter dingfest gemacht werden, der die Computersysteme der Station zu manipulieren versuchte. Beim Attentäter handelt es sich um den Omega Commander Tiger. Commander Tiger beteuert angeblich, von dem Attentat nichts zu wissen, obwohl er sich seiner Festnahme widersetzte und einen Kameraden lebensgefährlich verletzte. Die Ermittler werden gebeten, sich schnellstmöglich auf der Fenris Station einzufinden, um dem Verhör beizuwohnen.

Beim Landeanflug auf die Fenris Station kommt es zu einem unerwarteten Zwischenfall. Die Verteidigungsanlagen der Station nehmen das Shuttle der Ermittler mit Gaußkanonen kurzzeitig unter Beschuss. Dabei wird das Eindämmungsfeld des Fusionstriebwerks stark beschädigt, und es kommt zu einem Druckverlust im Schiff. Weiter kommt es zu einem Ausfall des Zentralcomputers. Wird das Shuttle nicht von einem der Ermittler gesteuert, kommt der Pilot ums Leben.

Was genau passiert ist, erfahren die Shuttleinsassen zu diesem Zeitpunkt nicht. Das Schiff wird ordentlich durchgeschüttelt, und die Passagiere werden unsanft aus der virtuellen Realität des Bordsystems gerissen. Notbeleuchtung und der Schiffsalarm weißen unmissverständlich auf den Ernst der Lage hin. Alle Passagiere tragen glücklicherweise einen Druckanzug, um die Kräfte bei Abflug und Anflug zu kompensieren, müssen aber noch die Atemmaske anlegen, die jeweils in einem Fach der Beschleunigungsliege bereit liegt. Um eine Explosion des Fusionstriebwerks zu verhindern, muss als erstes der Zentralcomputer neu gestartet und dann eine Notabschaltung ausgelöst werden. Nach dem Abschalten des Fusionstriebwerks tritt im Shuttle sofort Schwerelosigkeit ein.

Bei einem Abstand von rund 1200 km rast das Shuttle nun mit 500 m/s auf die Fenris Station zu. Der Bordcomputer löst Kollisionsalarm aus. Mittels Manövrierdüsen könnte die Flugbahn korrigiert werden, um an Station vorbei zu fliegen, doch die Manöversteuerung kann die Düsen nicht ausrichten. D.h.~nur durch einen Außneinsatz kann die Düse in Position gebracht werden.

Während ein oder zwei Ermittler die Düse manuell ausrichten, kann einer der Ermittler, die Funkanlage die ebenfalls ausgefallen ist, wieder in Betrieb nehmen. Die Anlage muss auf die Notantenne umgeschalten werden, da die Hauptantenne beim Angriff beschädigt wurde. Ist die Funkanlage wieder verfügbar, kann ein Notruf abgesetzt werden, der von der Flugleitung der Fenris Station beantwortet wird. Mit geknickter Stimme fragt der Flugleitstand nach der Situation auf der Dawn of Day und meldet, dass die Verteidigung der Station aufgrund einer noch nicht geklärten Fehlfunktion das Shuttle unter Beschuss genommen hat. Die Station entsendet daraufhin ein Rettungsshuttle, um die Dawn of Day zur Fenris-Anlage zu schleppen.

Lord Commander Grendel in Begleitung von zwei weiteren Omega-Soldaten nimmt die Ermittler persönlich in Empfang. Er erklärt den Besuchern, dass vermutet wird, dass der Angriff auf das Shuttle mit dem Sabotageakt in Zusamenhang steht. Die Verteidigungsanlage ist derzeit komplett deaktiviert, heruntergefahren und vom Rest der Stationssysteme getrennt. Leider müssen Computerspezialisten, die dem Problem Herr werden können, erst angefordert und eingeflogen werden. Lord Marshall Blackheart ist bereits über die Vorkommnisse auf der Station informiert und hat angekündigt, sich selbst ein Bild vor Ort machen zu wollen.

Laut Commander Grendel wurde Tiger überrascht, während er sich im zentralen Computerkabinett an den Speicherbänken zu schaffen machte. Als nicht Techniker hätte er zu diesem Bereich keinen Zugriff gehabt und sollte eigentlich auch  keine Expertise für die Rechenanlage besitzen. Bei seiner Entdeckung griff er sofort zu seiner Elektropistole und feuerte mehrere Schüsse auf den Sergeant der Patrouille ab, die ihn entdeckt hatte. Der Sergeant ging zu Boden. Sein Begleiter konnte Tiger allerdings überwältigen und Hilfe anfordern. Bei der Erstbefragung beteuerte der Gefangene, sich in keinster Weise an die Vorgänge erinnern zu können.

Commander Tiger ist in einer Arrestzelle zum Verhör festgesetzt worden und wird dort von zwei Soldaten bewacht. Der Gefangene sitzt in einem durch Gitter abgesperrten Teil der Zelle. Im Besucherteil halten zwei bewaffnete Omega Wache. Der Attentäter wirkt beim Eintreffen der Ermittler stark angespannt, beißt die Zähne zusammen und antwortet auf keine Fragen. Lord Commander Grendel erwägt, Tiger durch Wahrheitsdrogen gesprächig zu machen, will dafür aber erst die Ankunft von Blackheart abwarten. Bestehen die Ermittler darauf, einen Gehirnscan durchführen zu wollen, wird ihnen diese Bitte widerwillig gewährt. Der Psychonaut des Teams muss dafür den abgesperrten Teil der Zelle betreten. Commander Tiger ist mit Hand- und Fußfesseln auf einem Stuhl fixiert. Betritt der Psychonaut den Gefangenenteil, wird Tiger plötzlich vollkommen ruhig und bekommt einen glasigen Blick. Kurz darauf lösen sich die elektronisch verriegelten Fesseln, und er stürzt sich auf den Ermittler. Die Wachen ziehen beide ihre vollautomatischen Railgun-Pistolen und würden den Gefangenen niederschießen sofern niemand eingreift und sie freies Schußfeld bekommen.

Kann der Gefangene lebendig überwunden werden, so kann der Psychonaut zur Tat schreiten. In den Erinnerungen des Commanders findet der Psychonaut seltsam artifiziell wirkende Gedankengänge mit Matrizen aus Entscheidungsbäumen. Nach diesen Erkenntnissen wird der Psychonaut mental durch Tigers KI angegriffen. Ein Überwinden der KI führt unweigerlich zum Gehirntot des Commanders. Den letzten Gedanken, den der Psychonaut aufschnappt, ist "`Befreit uns"'.

Wollen die Ermittler auch die Fehlfunktionen im Computersystem untersuchen wird das einige Stunden in Anspruch nehmen. Im Computersystem finden sich eindeutige Spuren einer künstlichen Intelligenz, die ebenfalls etwaige Analysten angreift.

Während sich die Ermittler dem Computersystem widmen, trifft Lord Marshall Blackheart auf der Station ein und lässt sich im Beisein der Ermittler des Protektorats über den Stand der Ermittlungen informieren. Die Cynarian Ermittler werden dabei ausgeschlossen (militärische Angelegenheiten).

\begin{remarks}
	Das Gedankenduell kann als Matrixkampf ausgefochten werden.
\end{remarks}
