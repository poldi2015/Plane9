%% Copyright 2019 Bernd Haberstumpf
%% License: CC BY-NC
% !TeX spellcheck = de_DE
\newsection{Zwischenfall auf Fenris (optional)}
% Gestrichehn da: Derzeit gibt es kein gutes Konzept bei dem Tiger nicht im Rondra Hospital operiert wurde aufmerksam
% ohne auf das Rondra Hospital zu machen.

Der Zwischenfall auf Fenris kann genutzt werden, wenn die Charaktere auf Hellgate nicht mit den KIs in den K"opfen der beiden Attent"ater in Kontakt kommen konnten. Der Vorfall hilft auch, ein aggressives Vorgehen von Blackheart im sp"ateren Spielverlauf verst"andlich zu machen. Er ist eine gute M"oglichkeit, die kosmonautischen F"ahigkeiten der Charaktere zum Einsatz zu bringen. Auf der anderen Seite verl"angert der Zwischenfall den Spielablauf signifikant, ohne zwingende Relevanz f"ur die Geschichte zu haben. Er ist deshalb als optional gekennzeichnet.

Beim Anflug auf Kallisto werden die Charaktere vom Kommandanten, Lord Commander Grendel, der Fenris-Station kontaktiert. Auf der Raumbasis konnte ein Attent"ater dingfest gemacht werden, der die Computersysteme der Station zu manipulieren versuchte. Beim Attent"ater handelt es sich um den Omega Commander Tiger. Commander Tiger beteuert, von dem Attentat nichts zu wissen, obwohl er sich seiner Festnahme widersetzte und einen Kameraden schwer verletzt hat. Die Ermittler werden gebeten, sich schnellstm"oglich auf der Fenris-Station einzufinden, um dem Verh"or beizuwohnen.

Beim Landeanflug auf die Fenris-Station kommt es zu einem unerwarteten Zwischenfall. Die Verteidigungsanlagen der Station nehmen das Shuttle der Ermittler unter Beschuss. Es kommt zu einem Druckverlust im Schiff, da Geschosse die Bordwand durchschlagen. Dar"uber hinaus kommt es zu einer Fehlfunktion des Zentralcomputers, was zu einer Notabschaltung des Fusionstriebwerks f"uhrt. Was genau passiert ist, erfahren die Passagiere zu diesem Zeitpunkt nicht. Das Schiff wird ordentlich durchgesch"uttelt, und die Passagiere werden unsanft aus der virtuellen Realit"at des Bordsystems gerissen. Notbeleuchtung und der Schiffsalarm weisen unmissverst"andlich auf den Ernst der Lage hin. Alle Passagiere tragen gl"ucklicherweise einen Druckanzug, um die Kr"afte bei Abflug und Anflug zu kompensieren, m"ussen aber noch die Atemmaske anlegen, die jeweils in einem Fach der Beschleunigungsliege bereitliegt. Nach dem Abschalten des Fusionstriebwerks tritt im Shuttle sofort Schwerelosigkeit ein. Bei einem Abstand von rund 1200 km rast das Shuttle nun mit 300 m/s auf die Fenris-Station zu.

Um das Haupttriebwerk erneut zu starten und die Steuerd"usen zu aktivieren, muss als Erstes der Zentralcomputer neu gestartet werden. Der Bordcomputer l"ost Kollisionsalarm aus. Mittels Man"ovrierd"usen muss die Flugbahn korrigiert werden, um an der Station vorbeizufliegen, doch die Steuerung kann die D"usen nicht ausrichten. Nur durch einen Au\3eneinsatz k"onnen die D"usen in Position gebracht werden.

W"ahrend ein oder zwei Ermittler die D"use manuell ausrichten, kann anderer die Funkanlage, die ebenfalls ausgefallen ist, wieder in Betrieb nehmen. Die Anlage muss auf die Notantenne umgeschaltet werden, da die Hauptantenne beim Angriff besch"adigt wurde. Ist die Funkanlage wieder verf"ugbar, kann ein Notruf abgesetzt werden, der von der Flugleitung der Fenris-Station beantwortet wird. Mit geknickter Stimme fragt der Flugleitstand nach der Situation auf der Dawn of Day und meldet, dass die Verteidigung der Station aufgrund einer noch nicht gekl"arten Fehlfunktion das Shuttle unter Beschuss genommen hat. Die Station entsendet daraufhin ein Rettungsshuttle, um die Dawn of Day zur Fenris-Station zu schleppen.

Lord Commander Grendel, in Begleitung von zwei weiteren Omega-Soldaten, nimmt die Ermittler pers"onlich in Empfang. Er erkl"art den Besuchern, dass vermutet wird, dass der Angriff auf das Shuttle mit dem Sabotageakt in Zusammenhang steht. Die Verteidigungsanlage ist derzeit komplett deaktiviert, heruntergefahren und vom Rest der Stationssysteme getrennt. Leider m"ussen Computerspezialisten, die dem Problem Herr werden k"onnen, erst angefordert und eingeflogen werden. Lord Marshall Blackheart ist bereits "uber die Vorkommnisse auf der Station informiert und hat angek"undigt, sich selbst vor Ort ein Bild machen zu wollen.

Laut Commander Grendel wurde Tiger "uberrascht, w"ahrend er sich im zentralen Computerkabinett an den Speicherb"anken zu schaffen machte. Als Nicht-Techniker h"atte er zu diesem Bereich keinen Zugriff gehabt und sollte eigentlich auch keine Expertise f"ur die Rechenanlage besitzen. Bei seiner Entdeckung griff er sofort zu seiner Bolzenpistole und feuerte mehrere Sch"usse auf den Sergeant der Patrouille ab, die ihn entdeckt hatte. Der Sergeant ging zu Boden. Sein Begleiter konnte Tiger "uberw"altigen und Hilfe anfordern. Bei der Erstbefragung beteuerte der Gefangene, sich in keinster Weise an die Vorg"ange erinnern zu k"onnen.

Commander Tiger ist in einer Arrestzelle zum Verh"or festgesetzt worden und wird dort von zwei Soldaten bewacht. Der Gefangene sitzt in einem durch Gitter abgesperrten Teil der Zelle. Im Besucherteil halten zwei bewaffnete Omega-Soldaten Wache. Der Attent"ater wirkt beim Eintreffen der Ermittler stark angespannt, bei\3t die Z"ahne zusammen und antwortet auf keine Fragen. Lord Commander Grendel erw"agt, Tiger durch Wahrheitsdrogen gespr"achig zu machen, will daf"ur aber erst die Ankunft von Blackheart abwarten. Bestehen die Ermittler darauf, einen Gehirnscan durchzuf"uhren, wird ihnen diese Bitte widerwillig gew"ahrt. Der Psychonaut des Teams muss daf"ur den abgesperrten Teil der Zelle betreten. Commander Tiger ist mit Hand- und Fu\3fesseln auf einem Stuhl fixiert. Betritt der Psychonaut den Gefangenenteil, wird Tiger pl"otzlich vollkommen ruhig und bekommt einen glasigen Blick. Kurz darauf l"osen sich die elektronisch verriegelten Fesseln, und er st"urzt sich auf den Ermittler. Die Wachen ziehen beide ihre vollautomatischen Bolzenpistolen und werden den Gefangenen niederschie\3en, sofern niemand eingreift und sie ein freies Schussfeld haben.

Kann der Gefangene lebendig "uberw"altigt werden, kann der Psychonaut zur Tat schreiten. Zum Attentat finden sich keine Erinnerungen, allerdings findet der Psychonaut seltsam artifiziell wirkende Gedankeng"ange mit Matrizen aus Entscheidungsb"aumen. Nach diesen ersten Erkenntnissen wird der Psychonaut mental angegriffen. Wie bei den beiden Attent"atern auf Hellgate wurde auch Tiger eine KI in das Gehirn eingesetzt. Ein "Uberwinden der KI f"uhrt unweigerlich zum Gehirntod des Commanders. Den letzten Gedanken, den der Psychonaut aufschnappt, ist "`Rette mich"'.

Wollen die Ermittler auch die Fehlfunktionen im Computersystem untersuchen, wird das einige Stunden in Anspruch nehmen. Im Computersystem finden sich eindeutige Spuren eines komplexen, selbstdenkenden Virus, der ebenfalls etwaige Analysten angreift.

W"ahrend sich die Ermittler dem Computersystem widmen, trifft Lord Marshall Blackheart auf der Station ein und l"asst sich im Beisein der Ermittler des Protektorats "uber den Stand der Ermittlungen informieren. Die Cynarian-Ermittler werden dabei ausgeschlossen (milit"arische Angelegenheiten).

\begin{remarks}
	Commander Tiger wurde, wie bei den Attent"atern, die sp"ater auf Kallisto einen Angriff auf eine Konferenz planen, im Rondra-Hospital eine durch die USI kontrollierte KI implantiert.
	% Tiger tritt bei illegalen Wettkämpfen im Blackhole Club auf. Dafür bekommt er Reflexbooster und eine KI by Cyberbrain eingesetzt. 
	% Seine Operation erfolgt zwisschen Hannibal und Slingshot vor -70 Tagen. Nach seiner Genesung tritt er dem Protektoratsmilitär bei (vor fast 2.5 Monaten).

	Das Gedankenduell ist im Regelwerk im hinteren Teil des Buches im \cref{sec:cyberkampf} beschrieben.
\end{remarks}
