%% Copyright 2019 Bernd Haberstumpf
%% License: CC BY-NC
% !TeX spellcheck = de_DE
\newsection{Auf dem Garnisonsst"utzpunkt}\anchor{sec:garnison}

Ein erster Anlaufpunkt der Charaktere ist sinnvollerweise der Garnisonsst"utzpunkt des Protektorats, da dort die Cowboybrigade aufzufinden ist. Da Kallisto offiziell nicht dem Protektorat angeh"ort, unterh"alt das Milit"ar des Protektorats nur einen kleinen St"utzpunkt in Valhalla. Der St"utzpunkt beherbergt eine Kompanie von 40 Soldaten, gr"o\3tenteils Omega-Krieger. Diesen angeschlossen sind Techniker und sonstiges Personal.

Der Oberbefehlshaber des St"utzpunktes ist der Omega \emph{Commander Lockhead}, der bereits "uber das Eintreffen der Ermittler informiert ist. Commander Lockhead ist ein in die Jahre gekommener Veteran mit unerwartet freundlichem Gem"ut.

Nach der Untersuchung des Frachterungl"ucks auf Armageddon hat er die Cowboybrigade entweder auf Anweisung der Charaktere oder auf einen Befehl von Colonel Scholz hin inhaftieren lassen und h"alt sie seitdem in Untersuchungshaft. Der Commander ist "uber die Vorg"ange auf Hellgate informiert und bietet auch diesbez"uglich den Ermittlern seine Hilfe an. Er kann f"ur Untersuchungen im Umfeld des Raumhafens, der Garnison und der Oberstadt seinen Adjutanten \emph{Firedon}, einen jungen Omega-Soldaten, zur Verf"ugung stellen.

Zu Fragen nach der Cowboybrigade kann Lockhead beitragen, dass als Vorbereitung f"ur den Einsatz neuer Drohnen und anderer technischer Ger"ate, die auch auf Armageddon zum Einsatz kamen, die ganze Cowboybrigade vor 2 Monaten in der Klinik \emph{Rondra Hospital} mit neuen Talentchips ausgestattet wurde. Die Kosten hat die Garnison "ubernommen. Die Cowboybrigade ist seit vier Monaten am Raumhafen f"ur Wartungsarbeiten an Schiffen und Ger"atschaften der Garnison zugeteilt. Auch nach ihrer Versetzung vom zivilen Teil des Raumhafens zur Garnison ist eine gewisse \emph{Sonja Frost} ihre personelle Vorgesetzte.

\begin{remarks}
	\underline{Gewonnene Information:}
	
	\begin{itemize}
		\item Die Cowboybrigade ist seit vier Monaten bei der Garnison im Raumhafen t"atig. 
		\item Vor 2 Monaten sind ihr neue Talentchips in der Rondra Klinik eingesetzt worden.
		\item Firedon wurde zur Unterst"utzung der Ermittler freigestellt.
		\item Der Kontakt zum Raumhafen ist Sonja Frost, der Chief Officer des Hangardecks.
	\end{itemize}

	F"ur Fahrten durch die Oberstadt bietet Firedon ein Milit"ar-Truck an.
	 
\end{remarks}

\newsection[Im Rondra Hospital]{Im Rondra Hospital}
\bottomimage{images/armytank.png}

In Bezug auf das Rondra Hospital k"onnte der Verdacht aufkommen, dass dort die Manipulationen an den Gehirnen von Hannibal und Slingshot durchgef"uhrt wurden. Auf Dr"angen von Commander Lockhead w"are ein Termin mit dem Klinikleiter \emph{Prof.~Dr.~Henry Sanders} m"oglich. Die beiden Herren, so verschieden wie sie sind, kennen einander aus vergangenen Zeiten bei der Europ"aischen F"orderation.

Der Klinikleiter empf"angt die Charaktere in einem ger"aumigen B"uro. Sanders ist ein Norm im Alter von "uber 50 Jahren, mit gepflegtem Aussehen und grau meliertem Haar. Bei Fragen zur Cowboybrigade oder Hannibal verweist er die Gruppe an \emph{Brenda Ben}, die er auch gleich bittet, die Ermittler zu unterst"utzen.

Brenda Ben ist eine sympathische und erfahrene "Arztin. Sie ist Mitglied des Leitungsteams der Klinik. Routiniert kann sie von \emph{Ben Reuthers} aus der Buchhaltung Unterlagen anfordern und die Ermittler im Fall der Cowboybrigade an den behandelnden Chirurgen \emph{Dr. Loyd Rothan} sowie die Physiotherapeuten \emph{Russel Spenser} und \emph{Phillip Klarson} weiterleiten. Hannibal ist in den Akten des Rondra Hospitals nicht verzeichnet. Brenda Ben hat jedoch nur begrenzt Zeit und verabschiedet sich nach dieser ersten Recherche, da sie zu einer Operation gerufen wird.

Dr. Loyd Rothan best"atigt, das Einsetzen der Talentchips bei allen Mitgliedern der Cowboybrigade durchgef"uhrt zu haben. Nach den erfolgreichen Eingriffen trainierten die Physiotherapeuten die Wartungstechniker im Umgang mit ihrer neuen Cybertechnologie. Ben Reuthers best"atigt die Beauftragung der Operation durch Sonja Frost und die Zahlungsabwicklung durch die Garnison.

Die Informationen aus der Klinik entsprechen denen von Commander Lockhead.

\begin{remarks}
	\underline{Gewonnene Information:}
	
	\begin{itemize}
		\item Neue Kontakte: Prof.~Dr.~Henry Sanders, Brenda Ben, Ben Reuthers, Dr. Loyd Rothan, Russel Spenser und Phillip Klarson. 
		\item Der Eingriff bei der Cowboybrigade im Rondra Hospital verlief unauff"allig. Keine besonderen Vorkommnisse.
		\item Hannibal wurde nicht im Rondra Hospital operiert.
		\item Die Aussagen des Hospitals decken sich mit denen von Commander Lockhead. 
	\end{itemize}

	\underline{Prof.~Dr.~Sanders:}

	Wie sich sp"ater herausstellen wird, ist Sanders viel tiefer in die Vorkommnisse mit den implantierten KIs involviert, als sich zum aktuellen Zeitpunkt erahnen l"asst. In seiner Klinik wurden bei der Cowboybrigade nur die beauftragten Talentchips installiert. Der Professor kann deshalb die Ermittler guten Gewissens an seine Mitarbeiter verweisen.
\end{remarks}

\newsection{Cowboybrigade Voruntersuchung}

Gehen die Charaktere davon aus, dass weitere Mitglieder der Cowboybrigade ebenfalls einer Gehirnmanipulation unterzogen wurden, k"onnen sie Firedon beauftragen, einen der vier in einer Klinik untersuchen zu lassen. Das Rondra Hospital f"uhrt als gr"o\3tes Hospital in Valhalla alle medizinischen Eingriffe f"ur den Garnisonsst"utzpunkt durch. Vertrauen die Charaktere dem Rondra Hospital nicht, schl"agt Firedon die davon unabh"angige \emph{Alexandr Clinic} vor.

Da nicht bekannt ist, ob sich ein weiteres Mitglied der Cowboybrigade als Attent"ater entpuppen k"onnte, sollten die Charaktere die Verd"achtigen zu diesem Zeitpunkt nicht in die Vorg"ange auf Hellgate oder Attentatsverd"achtigungen einweihen.

Werden die Untersuchungen in der Alexandr Clinic durchgef"uhrt, zeigt sich der leitende Arzt Dr.~Spinner zwar zun"achst erstaunt, warum man sich an seine Klinik und nicht an das Rondra Hospital wendet, ist aber bereit, eine Untersuchung durchzuf"uhren.

Bei den nicht-invasiven Untersuchungen in der Klinik lassen sich keine Manipulationen feststellen. Die implantierten technischen Einheiten wirken nach Aussage der "Arzte unauff"allig. Die Gehirnwellen weisen ebenfalls keine Anomalien auf. Wenn die Charaktere allerdings noch keine n"aheren Informationen von der Nike Station erhalten haben, ist auch nicht klar, worauf ein Arzt achten sollte.

Selbst mit den weiter unten beschriebenen Informationen aus der Nike Station werden bei der Untersuchung keine Auff"alligkeiten festgestellt. Der Rest der Cowboybrigade ist ebenfalls sauber.

\begin{remarks}
	\underline{Gewonnene Information:}
	
	Eine Untersuchung des Gehirns der Mitglieder der Cowboybrigade zeigt keine auff"alligen Anomalien.
\end{remarks}

\newsection{Sonja Frost}\anchor{sec:sonjafrost}

Eine weitere Anlaufstelle im Zentrum von Valhalla ist Sonja Frost. Sie ist der Chief Officer des Hangardecks des Raumhafens. Sonja Frost ist dementsprechend schwer zu erreichen. Ein Anruf durch den St"utzpunkt ist notwendig, um "uberhaupt einen Termin zu vereinbaren.

Sonja ist in ihrem B"uro neben dem Hangar anzutreffen. Das B"uro ist mit technischen Ger"aten, Holoprojektoren und Tafeln vollgestellt. Es herrscht ein reges Ein und Aus, und Sonja verteilt durchgehend Anweisungen.

Von Sonja erfahren die Ermittler, dass die Cowboybrigade vor ca.~1\half~Jahren vom Asteroideng"urtel zwischen Mars und Jupiter in das jovianische System wechselte und dort am Raumhafen untergekommen ist. Vor vier Monaten wurden sie dann an das Milit"ar "uberwiesen. Der medizinische Eingriff bei der Cowboybrigade wurde in Abstimmung mit Commander Lockhead beschlossen und erwartungsgem"a\3 durchgef"uhrt und bezahlt.

Sonja Frost berichtet, dass die Mitglieder der Cowboybrigade am Raumhafen unter dem Pseudonym ``die glorreichen F"unf'' bekannt sind. Aufgrund ihrer heiteren, skurrilen Art, aber auch wegen ihrer Zuverl"assigkeit, sind sie im Hangardeck beliebt und wertgesch"atzt. Den Spitznamen Cowboybrigade erhielten sie erst durch die Besch"aftigung in der Garnison, als sie begannen, gegen"uber Stetson zu salutieren. Nach dem Einsetzen der Talentchips im Rondra Hospital kehrten die F"unf nach zwei Wochen wieder zum Dienst zur"uck. Slingshot meldete sich allerdings ein paar Tage sp"ater krank und kehrte erst nach weiteren zwei Wochen, kurz vor der "Uberstellung nach Armageddon, zur"uck. N"aheres kann sie dazu auch nicht beitragen.

Bez"uglich Hannibal muss sie in ihrem Terminal nachforschen. V"ollig ungewohnt benutzt sie daf"ur ein Ger"at ohne neuronales Interface. \say{Abgeschottetes System, aus Sicherheitsgr"unden}, sagt sie, wenn man sie danach fragt, und grinst. Schnell wird sie f"undig. Hannibal hat vor neun Monaten seine Arbeit am Raumhafen als Softwaretechniker begonnen. Vor zwei Monaten wechselte er seinen Arbeitsplatz und bekam eine Anstellung bei Cynarian als Sicherheitstechniker auf der Minenkolonie Hellgate. Einen Monat vor seiner K"undigung hatte er sich f"ur zwei Wochen krankgemeldet.

Bevor die Charaktere das B"uro verlassen, f"allt ihr noch ein, dass sich nach der Landung der Dawn of Day zwei Herren in Anz"ugen nach den Ank"ommlingen erkundigt haben. Einer der beiden war gro\3, hatte eine T"ursteherstatur und einen gepflegten blonden B"urstenhaarschnitt. Der andere war unauff"allig und hatte auch nicht gesprochen. Informationen "uber die Ermittler konnten den beiden nicht gegeben werden (O-Ton: \say{K"onnte ja jeder kommen}). Aufzeichnungen von den beiden stehen leider nicht zur Verf"ugung.

\begin{remarks}
	\underline{Gewonnene Information:}
	
	\begin{description}
		\item[Cowboybrigade] Die Cowboybrigade alias ``die glorreichen F"unf'' ist seit 1\half Jahren im jovianischen System und seitdem am 
			Raumhafen t"atig. Seit 4 Monaten sind sie bei der Garnison besch"aftigt. Vor zwei Monaten wurden ihnen Talentchips eingesetzt. Slingshot meldete sich danach insgesamt f"ur zwei Wochen krank.
		\item[Hannibal] Hannibal war 7 Monate am Raumhafen angestellt. Vor zwei Monaten wechselte er zu Cynarian. Vor dem Wechsel war er 	
			zwei Wochen krankgemeldet.
		\item[USI-Agenten] Zwei Anzugtr"ager haben sich nach der Dawn of Day erkundigt. Bei den beiden handelt es sich um die USI-Agenten  
			\emph{Smith-Singer} und \emph{Frederic Johnson}. Smith-Singer ist der f"uhrende Drahtzieher hinter der Operation P9 und den Attentaten im jovianischen System. Frederic Johnson ist ein Psychonaut, der Smith-Singer unterst"utzt. Weiteres findet sich \cref{sec:usiagents}.
	\end{description}
\end{remarks}

\newsection{Die Cowboybrigade im Verh"or}

Die Ermittler k"onnen Firedon beauftragen, die inhaftierte Cowboybrigade in einem Verh"orraum vorzuf"uhren. Stetson l"ummelt beim Eintreten der Charaktere mit einem Zahnstocher im Mund und einem verbeulten Cowboyhut auf dem Kopf in seinem Sitz. Beim "Offnen der T"ur setzt er sich abrupt auf. Quickfinger Rod mischt nerv"os, aber virtuos ein Deck Spielkarten. Joe Rider sitzt finster dreinblickend und eingesunken auf seinem Stuhl. Tom Gunslinger wendet den Blick erwartungsvoll in Richtung der Eintretenden. Werden die Vier zusammen befragt, wendet sich Stetson als Erster an die Charaktere und fragt, was ihnen vorgeworfen wird, was mit Slingshot los ist und wann sie wieder entlassen werden. Sie gehen nach wie vor davon aus, dass es sich beim Frachterungl"uck auf Armageddon um einen Unfall gehandelt hat.

Konfrontieren die Ermittler die Truppe teilweise oder vollst"andig mit den wahren Gegebenheiten auf Armageddon oder Hellgate, beteuert Stetson entgeistert ihre Unschuld. Er versichert, nach einem Blick zu den anderen, Auskunft zu allen Fragen zu geben. Er beteuert, dass seine Freunde und er sicher nichts zu verbergen h"atten.

Quickfinger Rod blickt bei einer Befragung immer wieder zu Stetson. Seine Finger scheinen dabei ein Eigenleben zu f"uhren. Ein Kartentrick folgt dem anderen, ohne dass Rod Notiz davon nimmt.

Bei Joe Rider vergeht nach jeder Frage erst eine halbe Minute, bevor er antwortet. Die Antworten beschr"anken sich dann nur auf das Gefragte und enthalten kein unn"otiges Wort.

Tom Gunslinger ist das genaue Gegenteil. Gef"ahrlich wild gestikulierend, mit einem Trinkbecher in der Hand, schie\3en Worte aus seinem Mund. St"andig schweift er vom Thema ab.

\pageimage{images/cowboybrigade_cut.jpg}

Auf eine psychonautische Untersuchung reagiert die Gruppe leicht panisch. Keiner hat eine Vorstellung davon, was auf sie zukommen k"onnte. Die Untersuchung lassen sie dann aber ohne Gegenwehr "uber sich ergehen. Ein Gehirnscan best"atigt, dass ihre Gehirne sauber sind und dass ihre Aussagen ihrem Wissensstand entsprechen.

Angesprochen auf die Eingriffe in der Rondra-Klinik schildern sie, dass sie in der Klinik neue Talentchips mit anschlie\3endem Training erhalten haben. Eine Frage, ob es bei Slingshot Komplikationen gegeben h"atte, wird verneint. Allerdings erfahren die Ermittler, dass Slingshot im Gegensatz zu den anderen kein ausgebildeter Shuttle- und Drohnenpilot ist, sondern nur ein hervorragender Schiffstechniker. Slingshot hatte schon immer davon getr"aumt, auch eine Flugausbildung zu erhalten. Offensichtlich hat er sich nach dem Aufenthalt im Rondra Hospital selbst auf die Suche nach einer entsprechenden Kontrolleinheit gemacht. Vor der Versetzung nach Armageddon wurde er dann f"undig. Ein erweitertes Kontrollmodul erlaubte es ihm, Drohnen und Shuttles fernzusteuern. Wie er die daf"ur aufkommenden Gelder aufbringen konnte, wollte er den anderen nicht verraten. Ebenso wenig legte er offen, wo er den Eingriff hatte durchf"uhren lassen und wo er sich danach aufgehalten hat.

Ein paar Tage nach der Entlassung aus dem Rondra Hospital hatte er angefangen, die Freizeit oft alleine zu verbringen. "Uber den Barmann und Besitzer des Batcave konnten seine Freunde in Erfahrung bringen, dass er wohl eine h"ubsche Frau kennengelernt hatte. Darauf angesprochen tat er allerdings immer betont geheimnisvoll.

Weitere Informationen sind von der Cowboybrigade nicht in Erfahrung zu bringen.

\begin{remarks}
	\underline{Gewonnene Information:}

	\begin{itemize}
		\item Slingshot hat sich eine Flugsteuerung implantieren lassen. 
		\item Die Finanzierung des neuen Systems ist unbekannt.
		\item Slingshot hat eine Frau kennengelernt.
	\end{itemize}

	\underline{Carina:}
	
	Bei der geheimnisvollen Frau handelt es sich um Carina alias Fleur Soleil, auf die die Charaktere sp"ater in einem Club treffen werden. Die Identit"at der Freundin ist weder der Cowboybrigade noch dem Barmann im Batcave bekannt. Carina ist \cref{sec:carina} beschrieben.
\end{remarks}
