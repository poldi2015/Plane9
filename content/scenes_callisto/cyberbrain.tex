%% Copyright 2019 Bernd Haberstumpf
%% License: CC BY-NC
% !TeX spellcheck = de_DE
\newsection{Cyberbrain Infiltration}\anchor{sec:cyberbrain}

Cyberbrain ist eine kleine Forschungseinrichtung in der Zone. Der Forschungsschwerpunkt ist unbekannt. Betrieben wird Cyberbrain von Synthology Inc., die chirurgische Instrumente im Bereich Transplantationschirurgie auf dem Mars entwickelt. Die entsprechenden Informationen können durch das Büro von Vandermool bereitgestellt werden. Folgende Informationen stehen nicht zur Verfügung: Synthology ist eine Strohfirma der USI, was aber nicht nachverfolgbar ist. Cyberbrain ist der lange Arm der Operation P9 auf Kallisto.

\subsection{Die Zone}
Die Zone ist ein gut gesicherter Bereich auf Kallisto, in der Nähe von Valhalla, der in das Eis eingegraben ist. Sie beherbergt Unternehmungen der Konzerne mit hoher Sicherheitseinstufung oder inoffizielle Einrichtungen der Firmen. Die Zone ist auf zwei Ebenen aufgeteilt. Auf beiden Ebenen verlaufen Wege zwischen den Gebäuden der Konzerne. Unterhalb der Wege verlaufen Wartungsgänge, die für die technische Anbindung der Gebäude in der Zone zuständig sind. Der reguläre Zugang zur Zone erfolgt über einen eigenen kleinen Raumhafen, der an die Zone angebunden ist und von Shuttles und Buggies angeflogen bzw. angefahren werden kann. Der technische Betrieb und die Wartung der Zone werden von der Firma \emph{Dockbunner} durchgeführt. Das Sicherheitspersonal wird von der Firma \emph{TransSec} gestellt. Ein weiterer, wenig bekannter Zugang sind die Tunnel, über die die Zone aus dem Kraftwerk in Valhalla mit Strom versorgt wird. Diese Tunnel enden in den Wartungsgängen zwischen den beiden Ebenen.

\subsection{Das Ziel} 
In der aktuellen Situation ist Eile geboten. Die Aufklärung der Attentäter und ihrer Hintergründe muss vor dem Eintreffen der Delegationen von Erde und Mars am nächsten Tag erfolgen. Colonel Scholz fordert als oberste Direktive, Beweise zu sammeln, dass die Attentate von einer außenstehenden Organisation initiiert wurden. Des Weiteren müssen alle Attentäter identifiziert und die KI-Technologien in der Cyberbrain-Einrichtung sichergestellt werden. Kampfhandlungen in der Zone sind, soweit möglich, zu vermeiden. Für Blackheart hat die Identifizierung und Eliminierung weiterer Attentäter höchste Priorität. Alle Mittel dazu sind legitimiert.

\subsection{Der Zugang} 
Bei der Infiltration der Cyberbrain-Forschungseinrichtung sollte der Spielleiter den Spielern weitreichende Handlungsfreiheiten einräumen, gleichzeitig jedoch auch gegebenenfalls unterstützend eingreifen. Die Zone kann offiziell über den Raumhafen betreten werden. Eine weitere Zugangsmöglichkeit bieten, wie weiter unten beschrieben, die Wartungstunnel der Energieversorgung der Zone.

\subsection{Der Raumhafen} 
Beim Zugang über den Raumhafen können die Ermittler mit Unterstützung von Cynarian oder dem Luna-Syndikat versuchen, Mitarbeiter von TransSec oder Dockbunner zu ersetzen oder eine Warenlieferung vorzutäuschen. Allerdings bleibt nicht viel Zeit für die Vorbereitung, und die Täuschung ist daher riskant und nicht lückenlos. Omega-Krieger sowie schwere Waffen und Rüstungen müssen besonders gut argumentiert werden. Um eine Rückverfolgung der Operation auf Cynarian zu verhindern, wird Cynarian keine offizielle Genehmigung zum Betreten der Zone erteilen. Ein Zugang über den Raumhafen wird in jedem Fall Smith-Singer auf den Plan rufen. Die Charaktere sind Smith-Singer bekannt, und er wird die Sicherheitskräfte vor einen möglichen Angriff auf Konzerneigentum warnen. 

\subsection{Der Wartungstunnel} 
Das Luna-Syndikat betreibt das Fusionskraftwerk in Valhalla und stellt damit auch die Stromversorgung der Zone sicher. Daher obliegt die Wartung der Tunnel zur Zone ebenfalls dem Luna-Syndikat. Wenn \xl{} oder Nemessis in die Planung der Infiltration einbezogen werden, können sie den Ermittlern diese Zugangsmöglichkeit eröffnen. Nemessis wird auch anbieten, auf Anfrage einen kurzzeitigen Stromausfall zu verursachen. Als Führer durch die Tunnel bis zur Zone wird sich \xl{} bereit erklären und zusammen mit dem Techniker \emph{Roberto Martinez} die Ermittler begleiten.

\begin{remarks}
	Der Wartungstunnel ist die einfachste und sicherste Möglichkeit, in die Zone zu gelangen. Er wird voraussichtlich alle anderen Ansätze übertreffen. Aus diesem Grund sollte der Spielleiter diese Alternative erst spät anbieten, um den Spielern zunächst die Gelegenheit zu geben, kreativ zu werden.
\end{remarks}

\subsection{Kommunikation}
Aufgrund der hohen Sicherheitsanforderungen ist die Kommunikation in und aus der Zone heraus gut abgesichert. Die Zone verfügt über ein eigenes ComNetz, das nur über den Raumhafen und von dort aus über Firewalls mit weiteren Netzen außerhalb der Zone verbunden ist. Ein Zugang zum ComNetz der Zone ist nur für Mitarbeiter der Unternehmen in der Zone und für offizielle Gäste möglich. Einzelne Unternehmen betreiben eigene VPNs zu ihren Niederlassungen außerhalb der Zone. Neben dem regulären ComNetz innerhalb der Zone betreibt TransSec ein zusätzliches Sicherheitsnetz, das die Zone durchzieht. Die Backbone-Infrastruktur des ComNetzwerks befindet sich in den Wartungsgängen zwischen den beiden Ebenen, wo ein physikalischer Zugriff möglich ist und einen Hackerangriff erleichtert. Zusätzlich zum ComNetz verfügt die Stromversorgung über ein primitives, schmalbandiges Kommunikationsnetz, das nur noch dem Luna-Syndikat bekannt istEin Zugriff auf dieses Netzwerk ist über alle Stromanschlüsse möglich.

\subsection{Lageplan} 
Die Zone ist zur Orientierung in farblich gekennzeichnete Bereiche aufgeteilt. Die Gangsegmente zwischen den Gebäuden sind horizontal und vertikal nummeriert, sodass jedes Segment durch einen Barcode und eine Ziffer eindeutig identifiziert werden kann. Einen Lageplan der Zone, auf dem viele der Unternehmen namentlich benannt sind, kann über Cynarian erhalten werden. Die Gangsegmente sind durch Sicherheitsschotts voneinander getrennt und werden durch Kameras und Drohnen videoüberwacht. Der Zugang zu den Wartungstunneln zwischen den Ebenen erfolgt über mit Magschlössern gesicherte Luken aus der oberen Ebene. Die Gänge der Zone werden von Mitarbeitern der TransSec patrouilliert. Da die Gebäude der Unternehmen autonom ausgelegt sind und es keine geteilten Räumlichkeiten gibt, bewegen sich nur selten Personen durch die Gänge.

\subsection{Unterstützung} 
Für einen Eingreiftrupp können die Ermittler auf inoffizielle Söldner, bereitgestellt durch Cynarian, oder auf Soldaten des Protektorats zurückgreifen. Mitglieder des Luna-Syndikats, wie \xl{}, werden die Charaktere nicht begleiten. Söldner werden, sofern nicht anders von den Charakteren gewünscht, den Cynarian-Chefermittler als Befehlshaber anerkennen. Soldaten des Protektorats werden nur den von Blackheart eingesetzten Ermittlern als Befehlshaber akzeptieren und falls diese nicht greifbar ist, den Chefermittler des Protektorats als Anführer anerkennen. Die anderen beiden Gruppenmitglieder sind in erster Linie zu schützende Zivilisten. Die Söldner sind in \cref{cref:stosstruppcynarian}, die Soldaten des Protektorats in \cref{cref:stosstruppprotektorat}.

\subsection[\xl{}]{Xiao Long} 
Unabhängig von den Plänen der Ermittler wird \xl{} die Wartungsgänge getrennt von der Gruppe betreten und sich in der Nähe des Cyberbrain-Gebäudes in das Sicherheitsnetz der Zone einklinken. Durch ihre KI-Unterstützung kann sie die Sicherheitssysteme der Netze überwinden, den Trupp überwachen und gegebenenfalls eingreifen. Über das ComNetz kann sie einen Störfall auslösen. Neben ihrer leichten Version eines Kampfpanzers, ihrer kurzläufigen, vollautomatischen Multigun, sowie Schock- und Haftminen, steht ihr auch ein Plasmabrenner für einen schnellen Zugriff zur Verfügung. Mit der Gruppe wird sie, wie vorher abgesprochen, über das Netz der Stromversorgung kommunizieren.

\subsection{Das Gebäude} 
Cyberbrain ist eine kleine Einrichtung auf der oberen Ebene der Zone, nahe dem Raumhafen. Die Einrichtung umfasst zwei Stockwerke. Im unteren Bereich befinden sich der Eingangsbereich, ein zentrales Reinraum-Forschungslabor, Aufenthaltsräume, eine Krankenstation mit OP-Saal sowie Lager- und Technikräume. Im oberen Bereich sind Büros, Lagerräume, ein Bad und ein Schlafbereich mit Stockbetten untergebracht. Das Forschungslabor überspannt beide Stockwerke.

Cyberbrain verfügt über vier Zugänge:

\begin{description}
	\item[Foyer] Das Foyer ist durch eine Fiberglastür gesichert, die bei Nichtbenutzung undurchsichtig geschaltet ist.
	\item[Lagerhalle] Im Untergeschoss befindet sich eine Lagerhalle, die über ein eigenes Rolltor verfügt.
	\item[Medizinisches Lager] Ebenfalls im Untergeschoss gibt es das Lager der Krankenstation, das ebenfalls durch ein Rolltor zugänglich 		ist.
	\item[Obergeschoss] Zum ersten Stock führt eine außen gelegene Rampe, die über ein Rolltor den Zugang zum Gebäude im Obergeschoss 	
		gewährt.
\end{description}

\begin{figure*}[htbp]
	\centering
    \includegraphics[width=0.85\linewidth]{./images/cmyk/cyberbrain_cmyk.jpg}
    \newline{}Cyberbrain
	\label{fig:cyberbrain}
\end{figure*}

\subsection{Zugang zum Gebäude} 
Um bis zum Cyberbrain-Gebäude zu gelangen, muss sich der Eingreiftrupp zumindest auf den letzten Metern durch die videoüberwachten Gänge der Zone bewegen. Hat der Trupp die Zone über die Wartungstunnel betreten, benötigen sie entweder eine Verkleidung als Dockbunner-Mitarbeiter oder müssen einen temporären Ausfall der Sicherheitssysteme provozieren. Beim Angriff auf die Sicherheitssysteme können ihnen der von Cynarian bereitgestellte Söldner \emph{Flint Ross} oder der Omega-Soldat \emph{Jackhammer} helfen. TransSec-Patrouillen können den Weg durch oder unterhalb der Gänge erschweren.

Um die Mitarbeiter des Instituts davon abzuhalten, einen Alarm auszulösen, muss das Gebäude entweder vom Sicherheitsnetz und dem regulären ComNetz der Zone getrennt werden, oder Roberto Martinez, der Techniker des Luna-Syndikats, löst kurzzeitig einen Stromausfall in der Zone aus. Um zu verhindern, dass Mitarbeiter, die sich im Gebäude befinden, fliehen können, muss das Gebäude von allen Seiten abgesichert werden. Kommen die Spieler selbst nicht auf diese Idee, kann dieser Einsatzplan auch von den Söldnern beziehungsweise Protektorats-Soldaten vorgeschlagen werden.

\subsection{Im Cyberbrain Institut}
Nach den ersten erfolgreichen Versuchen bei Hannibal und Slingshot wurde die Verpflanzung der Attentäter-KIs in das Rondra Hospital verlagert, und die Forschung an der Technologie wurde beendet. Damit wurde der Betrieb des Cyberbrain Instituts überflüssig, und es wurde begonnen, die Forschungseinrichtung aufzulösen. Parallel zum Angriff auf das Cyberbrain Institut schicken die USI-Agenten eine kleine Gruppe von Neuro Intelligence Mitarbeitern in das Institut, um zurückgelassene Daten und Gerätschaften zu vernichten. Da nach den Ereignissen im Blackhole Club und im Ice Club mit einem Angriff auf das Institut zu rechnen ist, verzichten die USI-Agenten darauf, eigene Mitarbeiter einzusetzen. Stattdessen nehmen sie lieber in Kauf, die lästigen Mitwisser von Neuro Intelligence im Zweifel auszuschalten.

Zeitgleich mit dem Eintreffen des Eingreiftrupps befinden sich \emph{Dr. Dan Leitner}, \emph{\ml{}}, \emph{Dr. Gaius Ross} und \emph{Francis McDonald} in der Einrichtung. Dr. Dan Leitner, in Reinraumkleidung, befindet sich im Forschungslabor im Erdgeschoss; Francis McDonald in einem Büro im ersten Stock. Gaius Ross arbeitet im Lagerraum der Krankenstation. \ml{} befindet sich im Serverraum und wird sich dort auch verstecken. Zu diesem Zeitpunkt sind bereits alle Gerätschaften untauglich gemacht. \ml{} gelingt es gerade noch, die Daten in den Servern zu löschen bevor sie entdeckt wird.

Wenn die Gruppe das Gebäude stürmt, werden sie früher oder später auf das Personal treffen, das sich nur ungenügend in den wenigen Räumen des Gebäudes verstecken kann. Ohne Gegenwehr lassen sich die verängstigten Wissenschaftler zusammentreiben. Nur \ml{} schimpft, man solle die Finger von ihr lassen, sie käme ja mit.

Bei der folgenden Befragung stellt sich Dr. Dan Leitner schützend vor seine Untergebenen und beteuert, dass sie lediglich für Aufräumarbeiten beauftragt wurden und keine Kenntnis über die Forschungsinhalte des Instituts hätten. Er erklärt, dass alle wichtigen Geräte und Daten bereits vorher von den Cyberbrain-Mitarbeitern selbst entfernt wurden, nachdem die Einrichtung vor einigen Tagen aufgelöst wurde. Seine zunehmende Nervosität ist bei seinen Ausführungen nicht zu übersehen. Dr. Ross und McDonald bestätigen seine Geschichte kurz und knapp. Im Einzelgespräch erklären sie, dass sie nur den Auftrag haben, die letzten Gerätschaften aus dem Gebäude zu entfernen. Sie können ebenfalls keine Auskunft darüber geben, woran gearbeitet wurde. Die junge \ml{} schweigt und weigert sich beharrlich, Fragen zu beantworten.

Gefragt nach ihrer Firmenzugehörigkeit nennen die Gefangenen nach kurzem Zögern die Firma Neuro Intelligence. Zum aktuellen Zeitpunkt ist Neuro Intelligence für die Ermittler und die Spieler noch ein unbeschriebenes Blatt, sodass sie nicht einschätzen können, welche Bedeutung diese Information hat. Der Name der Firma klingt aber in den Ohren der Ermittler nicht nach einer Organisation, die lediglich für Aufräumarbeiten tätig sind. Leider bietet die Zone keine Möglichkeit, weiterführende Informationen zu beschaffen, und auch eine schnelle Anfrage beim Luna-Syndikat liefert keine zusätzlichen Erkenntnisse.

\newsubsection{Weitere Befragung}
Während die Ermittler die eingeschüchterten Mitarbeiter von Neuro Intelligence befragen, sichern die unterstützenden Truppmitglieder das Gebäude ab. Den Ermittlern bleiben folgende Optionen:

\begin{itemize}
	\item Sie versuchen die Neuro Intelligence Mitarbeiter mitzunehmen, um sie außerhalb der Zone in Ruhe weiter zu verhören.
	\item Sie versuchen die Mitarbeiter vor Ort weiter unter Druck zu setzen und die Wahrheit aus ihnen heraus pressen. Aufgrund 
		der Situation werden die Gefangenen schnell einknicken.
	\item Der Psychonaut unter den Ermittlern kann vor Ort einen Gehirnscan durchführen.
\end{itemize}

Es empfiehlt sich, die Räume im Gebäude weiter zu durchsuchen, um belastendes Material zu finden. Parallel dazu sollten die Mitarbeiter von Neuro Intelligence weiterhin befragen. Bei der Durchsuchung der Räumlichkeiten kann festgestellt werden, dass zwar noch nicht alle Gerätschaften entfernt wurden, jedoch kein wirklich verwertbares Material gefunden werden kann. Dennoch kann aus den medizinischen Geräten bestätigt werden, dass hier Gehirnoperationen durchgeführt wurden. Zudem finden sich Belege dafür, dass Neuro Intelligence Material für die Integration in ein Kontrollmodul bereitgestellt hat. Werden die Mitarbeiter anschließend ausführlicher befragt und weiter unter Druck gesetzt, bröckelt die Geschichte von den Aufräumarbeiten. Leitner gibt zu, dass Neuro Intelligence einen Großteil der bei Cyberbrain eingesetzten Technologie beigesteuert hat. Er beteuert jedoch vehement, dass ihnen die finsteren Pläne der USI zunächst nicht bekannt waren und erst während der Operationen im Cyberbrain-Institut entdeckt wurden.

Bei einem Gehirnscan kommen folgende Erkenntnisse ans Licht:

\begin{itemize}
	\item Dr. Dan Leitner ist leitender Angestellter bei Neuro Intelligence und langjähriger Vertrauter von \emph{Prof.~Dr.~Naratova}, der 
		Inhaberin der Firma. Zusammen mit ihr hat er die Neuronal-Kopplungen entwickelt, also die Nanobots, die die KI mit dem menschlichen Gehirn über die Synapsen verbinden.
	\item Die Eingriffe am Gehirn, einschließlich des Einsetzens des Kontrollmoduls, das von Kasai Cyber Genetics geliefert wurde, wurden 
		im Cyberbrain-Institut von Prof. Dr. Sanders durchgeführt.
	\item Hannibal und Slingshot wurden als erste Probanden bei Cyberbrain behandelt.
	\item Weitere Eingriffe wurden direkt im Rondra Hospital von Prof. Dr. Sanders durchgeführt. Die Namen der manipulierten Personen sind 
		nicht bekannt.
	\item \ml{} ist die Chefentwicklerin, die die KIs der USI für den Einsatz im menschlichen Gehirn trainiert hat. \ml{} ist die zweite 
		enge Vertraute  von Prof. Dr. Naratova.
	\item Den Mitarbeitern von Neuro Intelligence wurde erst nach den ersten Eingriffen im Cyberbrain-Institut bewusst, dass die Eingriffe 
		ohne Wissen der Probanden durchgeführt wurden und die Opfer später gegen ihren Willen als Attentäter eingesetzt werden sollten.
\end{itemize}

Bei einem Gehirnscan von \ml{} wird der Psychonaut auf unerwarteten Widerstand stoßen. Im Kontrollmodul ist eine hochentwickelte, KI-gestützte Firewall installiert, die versucht, das Kontrollmodul des Psychonauten zum Absturz zu bringen. Ein Gehirnscan bei \ml{} ist daher nahezu unmöglich.

\begin{remarks}
	Während die Charaktere versuchen, Informationen zu sammeln, spitzt sich die Lage zu. Es ist wichtig, den Spannungsbogen aufrechtzuerhalten und den Spielern nicht zu viel Zeit zu lassen, um zu überlegen, wie sie an Informationen über die Neuro Intelligence-Mitarbeiter gelangen können. Im Zweifelsfall sollten die Charaktere versuchen, die Mitarbeiter aus der Anlage zu bringen, um sie für weitere Befragungen zu nutzen. Alle relevanten Informationen kann im Zweifel auch \ml{} bereitstellen, sobald sie aus der Einrichtung heraus in Sicherheit gebracht wurde.
\end{remarks}

\subsection{Die Absicherung} 
Während der Befragung sichern die unterstützenden Truppmitglieder die Anlage ab und verminen die Eingänge, wobei sie sich mit dem Charakter abstimmen, den sie als Vorgesetzten akzeptieren. Ein von einer KI manipulierter Omega-Soldat wird die Mine an einer der hinteren Türen, für die er verantwortlich ist, nicht aktivieren. Stattdessen platziert er an mehreren Stellen im Gebäude fernzündbare Sprengsätze.

\subsection[\xl{}s Überwachung]{Xiao Longs Überwachung}  
Da \xl{} damit rechnet, \ml{}, deren Existenz ihr durch ihren Besuch bei Prof. Dr. Sanders bekannt ist, aufgreifen zu müssen, überwacht sie die Einrichtung kontinuierlich per Video, um jederzeit eingreifen zu können. Mit dem Plasmabrenner ist es möglich aus den Wartungsschächsten von unten in das Institut einzudringen. Für \xl{} hat \ml{} oberste Priorität. Nach ihrer Einschätzung kann \ml{} alle für sie wichtigen Informationen über die \emph{Freien KI} (beschrieben im \cref{sec:freieki}) in ihrem Kopf liefern.

\subsection{Smith-Singer} 
Smith-Singer, der über die Vorfälle im Blackhole und im Ice Club informiert ist, wartet auf den nächsten Schritt der Ermittler. Sobald der Eingreiftrupp die Einrichtung über den Raumhafen der Zone betritt, werden die Konzerngardisten von Smith-Singer sofort darüber informiert, dass ein Angriff auf Cyberbrain stattfinden wird. Falls das Luna-Syndikat einen temporären Ausfall der Stromversorgung verursacht, versuchen die Sicherheitskräfte zunächst, den Ursprung und die Auswirkungen zu ermitteln. In diesem Fall haben die Charaktere etwas mehr Zeit. Früher oder später werden die Konzerngardisten jedoch das Cyberbrain-Institut umstellen. 

Als Gründer des Cyberbrain Instituts wird Smith-Singer darauf bestehen, die Sicherheitsgardisten zu begleiten. Beim Eintreffen vor dem Institut wird Smith-Singer darauf drängen, die Situation gewaltsam zu klären, um ein Blutbad in der Zone zu provozieren und damit eine spätere Besetzung Valhallas durch Kräfte des Konzernrats zu legitimieren. Sein zweites Ziel ist es, sowohl die Ermittler als auch die Mitarbeiter von Neuro Intelligence auszuschalten.

\subsection{Eintreffen der Sicherheitsgardisten} 
Die TransSec Sicherheitsgardisten treffen frühestens während der ersten Befragung der Neuro Intelligence Mitarbeiter ein und umstellen das Gebäude. Haben die Ermittler oder die unterstützenden Soldaten oder Söldner eine Überwachungsdrohne in den Gängen postiert oder Zugriff auf die Überwachungsvideos, werden sie bemerken, wenn die Konzerngardisten unter der Führung von Smith-Singer das Gebäude einkesseln. Alternativ kann \xl{} den Trupp über das Netz der Energieversorgung darüber informieren, dass sie Besuch bekommen.

Die Sicherheitsgardisten sind wie in \cref{sec:Sicherheitsgardisten} ausgerüstet. Vor und während der Auseinandersetzung mit dem Eingreiftrupp muss sich der Spielleiter nicht auf die Anzahl der Gardisten festlegen.

\subsection{Der Attent"ater} 
Werden die Charaktere von einem Trupp der Protektorats-Soldaten begleitet, tritt \emph{Thunder}, der von Smith-Singer befehligte Attentäter, in Aktion. Smith-Singer nutzt das ComNetz, um Thunder den Befehl zu erteilen, alle Personen im Gebäude zu töten.

Thunder zündet die von ihm platzierten Sprengsätze und eröffnet sofort das Feuer. Sein primäres Ziel ist es zunächst, seine Kameraden auszuschalten. Wenn möglich, wird Thunder als Nächstes die Mitarbeiter von Neuro Intelligence eliminieren, wobei \ml{} zufällig verschont bleibt.

\subsection{Das Geb"aude wird gest"urmt} 
Sollte Thunder das Feuer innerhalb des Gebäudes eröffnen, werden die Konzerngardisten sofort stürmen. Falls kein Attentäter im Gebäude ist, kann der Spielleiter zunächst eine Verhandlung mit den Geiselnehmern initiieren. Smith-Singer wird jedoch darauf drängen, möglichst schnell das Gebäude einzunehmen.

In jedem Fall werden die Gardisten früher oder später beginnen, die Türen zu entriegeln. Die Sprengfallen an den Türen werden einige der Gardisten ausschalten. Falls erforderlich, wird \xl{} den Eingreiftrupp unterstützen und die TransSec-Mitarbeiter von hinten angreifen. \xl{} wird nur dann aktiv werden, wenn keiner der Ermittler sie sehen kann. Sollten die Charaktere auf die Leichen stoßen, kann dies Fragen aufwerfen.

Um dem Kampfgeschehen eine kommunikative Komponente hinzuzufügen, können die Gardisten nach einem ersten misslungenen Angriff versuchen, erneut mit den Eindringlingen zu verhandeln. Die Gardisten werden diese Zeit nutzen, um sich neu zu formieren.

Geht der Eingreiftrupp nicht zum Gegenangriff über, werden die Gardisten das Gebäude erneut stürmen und Schock- sowie Rauchgranaten einsetzen.

\subsection{Die Flucht}
Müssen sich die Gardisten aus dem Gebäude zurückziehen, werden sie sich in den Gängen verschanzen und auf die nächste Reaktion der aus ihrer Sicht Terroristen im Gebäude warten, bevor sie einen erneuten Angriff starten. Das gibt der Gruppe Zeit, sich für einen Fluchtweg zu entscheiden. Die Charaktere haben mehrere Möglichkeiten zu fliehen:

\begin{description}
	\item [Die Ausgänge] Der Eingreiftrupp kann durch einen der Eingänge im Erdgeschoss zurück in die Wartungsgänge gelangen.
	\item [Im ersten Stock] Haben sich die Charaktere im ersten Stock verschanzt, können sie wie im Erdgeschoss durch das Tor auf der
		Rückseite über die Rampe auf den dahinterliegenden Gang gelangen.
	\item [Durch den Boden] Der Eingreiftrupp kann versuchen, sich im Gebäude zu verschanzen und sich mit Plasmabrennern durch den Boden in	
		die Wartungsschächte zu schneiden.
	\item [Nebengebäude] Über den Bereich hinter dem Foyer ist es möglich, sich durch die Wand in das Nebengebäude zu sprengen oder zu	
		schweißen. Das Gleiche ist auch im ersten Stock möglich. Die anderen Seiten des Gebäudes führen auf die umliegenden Gänge.
\end{description}

Flüchtet der Trupp in die Wartungsgänge, werden sie dort möglicherweise entdeckt. Die Sicherheitsgardisten werden zunächst versuchen, sie von oben zu beschießen, bevor sie ebenfalls in die Gänge eindringen.

\subsection{Das Nebengebäude Stimulus Fungi} 
Auf der vom Foyer rechten Seite befindet sich ein weiteres Gebäude. Dieses anliegende Gebäude ist das biologische Labor der Stimulus Fungi Forschungseinrichtung. Das Gebäude ist leer, da die Mitarbeiter während der Umstellung des Cyberbrain Instituts durch die Sicherheitskräfte evakuiert wurden.

Das Gebäude erstreckt sich mit einer großen Halle über beide Ebenen der Zone und ist etwa dreimal so groß wie die Cyberbrain Forschungseinrichtung. Es ist als großes Gewächshaus ausgelegt. Unter fokussierter Neonbeleuchtung befindet sich ein mit Gängen durchzogener Irrgarten aus Pflanzenbeeten auf vier Etagen. An verschiedenen Stellen werden die Pflanzenbeete durch Arbeitstische unterbrochen. Etwa die Hälfte der unteren Ebene wird von einem Wohnkomplex eingenommen, der dem oberen Stockwerk des Cyberbrain Gebäudes ähnelt. In der obersten Ebene befindet sich etwa in der Mitte des Gebäudes ein Chemielabor.

Das Gebäude kann auf beiden Ebenen auf der vom Cyberbrain Gebäude aus gesehen hinteren Seite über ein als Schleuse ausgelegtes Rolltor betreten werden. Im vorderen Bereich befinden sich die Mitarbeitereingänge, ebenfalls als Schleuse ausgelegt. An den dreckig weißen Wänden der Einrichtung prangt, wo nicht durch Pflanzen bedeckt, das Firmenlogo.

\subsection[\ml{}]{Mailin}
Wird \ml{} nicht von den Ermittlern aus dem Gebäude gebracht, wird \xl{} sie mit sich nehmen. In jedem Fall muss \ml{} die Flucht aus dem Gebäude gelingen.

%
% Nutzen die Charaktere wie beim hereinkommen die Tunnel zwischen den beiden Ebenen der Zone oder müssen die von \xl{} gerettet werden
% wird sie sie beim Rückweg aus der Zone begleiten und sie nach Breidablick zum Luna-Syndikat bringen.
%

\begin{remarks}
	Gewonnene Information:  Es wurden weiteren Personen KIs im Rondra Hospital eingesetzt. Die Nanobots auch Neuronal-kopplungen genannt entstammen dem Unternehmen Neuro Intelligence. 

	Die Cyberbrain Infiltration ist darauf ausgelegt, dass TransSec Mitarbeiter getöted werden, um die Gruppe im folgenden glaubwürdiger als Terroristen abstempeln zu können. Die Neuro Intelligence Mitarbeiter bis auf \ml{} in dieser Szene auszuschalten vereinfacht den folgenden Verlauf der Geschichte. Die Cynarian Söldner bzw. Protektorats Soldaten frühzeitig aus dem Spiel zu nehmen erhöht die Dramatik.
	
	Wichtig ist nur, dass weder \xl{} noch \ml{} getötet werden und \ml{} aus der Zone extrahiert wird. Beide Figuren sind für den Rest 

	Ausrüstung: Neben Feuerwaffen und Rüstung können die Charaktere wenn gewünscht auf Granaten, ferngesteuerte Sprengstoffe, Schneidbrenner, Überwachungsdronen, Magschloßknacker, EMP Granaten zurückgreifen die ihnen das Protektoratsmilitär zur Verfügung stellen kann sofern sie eine Möglichkeit haben sie in die Zone zu bringen. Der Spielleiter kann in diesem Zusammenhang in soweit freizügig sein solange die Spieler plausibel erklären können wie sie die Ausrüstung transportieren. Munition und vor allem Granaten sollten in ihrer Zahl begrenzt sein. Die Waffen der militärischen Begleiter sind jeweils auf deren Besitzer kodiert und können von
 	niemand anders genutzt werden.
\end{remarks}
