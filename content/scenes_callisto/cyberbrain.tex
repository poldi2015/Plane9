%% Copyright 2019 Bernd Haberstumpf
%% License: CC BY-NC
% !TeX spellcheck = de_DE
\newsection{Cyberbrain Infiltration}\anchor{sec:cyberbrain}

Cyberbrain ist eine kleine Forschungseinrichtung in der Zone. Der Forschungsschwerpunkt ist unbekannt. Betrieben wird Cyberbrain von Synthology Inc., die chirurgische Instrumente im Bereich Transplantationschirurgie auf dem Mars entwickelt. Die entsprechenden Informationen k"onnen durch das B"uro von Vandermool bereitgestellt werden. Folgende Informationen stehen nicht zur Verf"ugung: Synthology ist eine Strohfirma der USI, was aber nicht nachverfolgbar ist. Cyberbrain ist der lange Arm der Operation P9 auf Kallisto.

\subsection{Die Zone}
Die Zone ist ein gut gesicherter Bereich auf Kallisto, in der N"ahe von Valhalla, der in das Eis eingegraben ist. Sie beherbergt Unternehmungen der Konzerne mit hoher Sicherheitseinstufung und inoffizielle Einrichtungen der Firmen. Die Zone ist auf zwei Ebenen aufgeteilt. Auf beiden Ebenen verlaufen Wege zwischen den Geb"auden der Konzerne. Unterhalb der Wege verlaufen Wartungstunnel, die f"ur die technische Anbindung der Geb"aude in der Zone zust"andig sind. Der regul"are Zugang zur Zone erfolgt "uber einen eigenen kleinen Raumhafen, der an die Zone angebunden ist und von Shuttles und Buggies angeflogen bzw. angefahren werden kann. Der technische Betrieb und die Wartung der Zone werden von der Firma \emph{Dockbunner}, beschrieben \cref{sec:Dockbunner}, durchgef"uhrt. Das Sicherheitspersonal wird von der Firma \emph{TransSec}, beschrieben \cref{sec:Sicherheitsgardisten}, gestellt. Ein weiterer, wenig bekannter Zugang sind die Tunnel, "uber die die Zone aus dem Kraftwerk in Valhalla mit Strom versorgt wird. Diese Tunnel enden in den Wartungstunneln zwischen den beiden Ebenen.

\subsection{Das Ziel} 
In der aktuellen Situation ist Eile geboten. Die Aufkl"arung der Attent"ater und ihrer Hintergr"unde muss vor dem Eintreffen der Delegationen von Erde und Mars am n"achsten Tag erfolgen. Colonel Scholz fordert als oberste Direktive, Beweise zu sammeln, dass die Attentate von einer au\3enstehenden Organisation initiiert wurden. Des Weiteren m"ussen alle Attent"ater identifiziert und die KI-Technologien in der Cyberbrain-Einrichtung sichergestellt werden. Kampfhandlungen in der Zone sind, soweit m"oglich, zu vermeiden. F"ur Blackheart hat die Identifizierung und Eliminierung weiterer Attent"ater h"ochste Priorit"at. Alle Mittel dazu sind legitimiert.

\subsection{Der Zugang} 
Bei der Infiltration der Cyberbrain-Forschungseinrichtung sollte der Spielleiter den Spielern weitreichende Handlungsfreiheiten einr"aumen, gleichzeitig jedoch auch gegebenenfalls unterst"utzend eingreifen. Die Zone kann offiziell "uber den Raumhafen betreten werden. Eine weitere Zugangsm"oglichkeit bieten, wie weiter unten beschrieben, die Wartungstunnel der Energieversorgung der Zone.

\subsection{Der Raumhafen} 
Beim Zugang "uber den Raumhafen k"onnen die Ermittler mit Unterst"utzung von Cynarian oder dem Luna-Syndikat versuchen, Mitarbeiter von TransSec oder Dockbunner zu ersetzen oder eine Warenlieferung vorzut"auschen. Allerdings bleibt nicht viel Zeit f"ur die Vorbereitung, und die T"auschung ist daher riskant und nicht l"uckenlos. Omega-Krieger sowie schwere Waffen und R"ustungen m"ussen besonders gut argumentiert werden. Um eine R"uckverfolgung der Operation auf Cynarian zu verhindern, wird Cynarian keine offizielle Genehmigung zum Betreten der Zone erteilen. Ein Zugang "uber den Raumhafen wird in jedem Fall Smith-Singer auf den Plan rufen. Die Charaktere sind Smith-Singer bekannt, und er wird die Sicherheitskr"afte vor einen m"oglichen Angriff auf Konzerneigentum warnen. 

\subsection{Der Wartungstunnel} 
Das Luna-Syndikat betreibt das Fusionskraftwerk in Valhalla und stellt damit auch die Stromversorgung der Zone sicher. Daher obliegt die Wartung der Tunnel zur Zone ebenfalls dem Luna-Syndikat. Wenn \xl{} oder Nemessis in die Planung der Infiltration einbezogen werden, k"onnen sie den Ermittlern diese Zugangsm"oglichkeit er"offnen. Nemessis wird auch anbieten, auf Anfrage einen kurzzeitigen Stromausfall zu verursachen. Als F"uhrer durch die Tunnel bis zur Zone wird sich \xl{} bereit erkl"aren und zusammen mit dem Techniker \emph{Roberto Martinez} die Ermittler begleiten.

\begin{remarks}
	Der Wartungstunnel ist die einfachste und sicherste M"oglichkeit, in die Zone zu gelangen. Er wird voraussichtlich alle anderen Ans"atze "ubertreffen. Aus diesem Grund sollte der Spielleiter diese Alternative erst sp"at anbieten, um den Spielern zun"achst die Gelegenheit zu geben, kreativ zu werden.
\end{remarks}

\subsection{Kommunikation}
Aufgrund der hohen Sicherheitsanforderungen ist die Kommunikation in und aus der Zone heraus gut abgesichert. Die Zone verf"ugt "uber ein eigenes ComNetz, das nur "uber den Raumhafen und von dort aus "uber Firewalls mit weiteren Netzen au\3erhalb der Zone verbunden ist. Ein Zugang zum ComNetz der Zone ist nur f"ur Mitarbeiter der Unternehmen in der Zone und f"ur offizielle G"aste m"oglich. Einzelne Unternehmen betreiben eigene VPNs zu ihren Niederlassungen au\3erhalb der Zone. Neben dem regul"aren ComNetz innerhalb der Zone betreibt TransSec ein zus"atzliches Sicherheitsnetz, das die Zone durchzieht. Die Backbone-Infrastruktur des ComNetzwerks befindet sich in den Wartungstunneln zwischen den beiden Ebenen, wo ein physikalischer Zugriff m"oglich ist und einen Hackerangriff erleichtert. Zus"atzlich zum ComNetz verf"ugt die Stromversorgung "uber ein primitives, schmalbandiges Kommunikationsnetz, das nur noch dem Luna-Syndikat bekannt ist. Ein Zugriff auf dieses Netzwerk ist "uber alle Stromanschl"usse m"oglich.

\subsection{Lageplan} 
Die Zone ist zur Orientierung in farblich gekennzeichnete Bereiche aufgeteilt. Die Gangsegmente zwischen den Geb"auden sind horizontal und vertikal nummeriert, sodass jedes Segment durch einen Barcode und Ziffern eindeutig identifiziert werden kann. Einen Lageplan der Zone, auf dem viele der Unternehmen namentlich benannt sind, kann "uber Cynarian erhalten werden. Die Gangsegmente sind durch Sicherheitsschotts voneinander getrennt und werden durch Kameras und Drohnen video"uberwacht. Der Zugang zu den Wartungstunneln zwischen den Ebenen erfolgt "uber mit Magschl"ossern gesicherte Luken aus der oberen Ebene. Die G"ange der Zone werden von Mitarbeitern der TransSec patrouilliert. Da die Geb"aude der Unternehmen autonom ausgelegt sind und es keine geteilten R"aumlichkeiten gibt, bewegen sich nur selten Personen durch die G"ange.

\subsection{Unterst"utzung} 
F"ur einen Eingreiftrupp k"onnen die Ermittler auf inoffizielle S"oldner, bereitgestellt durch Cynarian, oder auf Soldaten des Protektorats zur"uckgreifen. Mitglieder des Luna-Syndikats, wie \xl{}, werden die Charaktere nicht begleiten. S"oldner werden, sofern nicht anders von den Charakteren gew"unscht, den Cynarian-Chefermittler als Befehlshaber anerkennen. Soldaten des Protektorats werden nur den von Blackheart eingesetzten Ermittler als Befehlshaber akzeptieren und falls diese nicht greifbar ist, den Chefermittler des Protektorats als Anf"uhrer anerkennen. Die anderen beiden Gruppenmitglieder sind in erster Linie zu sch"utzende Zivilisten. Die S"oldner sind  \cref{cref:stosstruppcynarian}, die Soldaten des Protektorats \cref{cref:stosstruppprotektorat} beschrieben.

\subsection[\xl{}]{Xiao Long} 
Unabh"angig von den Pl"anen der Ermittler wird \xl{} die Wartungstunnel getrennt von der Gruppe betreten und sich in der N"ahe des Cyberbrain-Geb"audes in das Sicherheitsnetz der Zone einklinken. Durch ihre KI-Unterst"utzung kann sie die Sicherheitssysteme der Netze "uberwinden, den Trupp "uberwachen und gegebenenfalls eingreifen. "Uber das ComNetz kann sie einen St"orfall ausl"osen. Neben ihrer leichten Version eines Kampfpanzers, ihrer kurzl"aufigen, vollautomatischen Multigun, sowie Schock- und Haftminen, steht ihr auch ein Plasmabrenner f"ur einen schnellen Zugriff zur Verf"ugung. Mit der Gruppe wird sie, wie vorher abgesprochen, "uber das Netz der Stromversorgung kommunizieren.

\subsection{Das Geb"aude} 
Cyberbrain ist eine kleine Einrichtung auf der oberen Ebene der Zone, nahe dem Raumhafen. Die Einrichtung umfasst zwei Stockwerke. Im unteren Bereich befinden sich der Eingangsbereich, ein zentrales Reinraum-Forschungslabor, Aufenthaltsr"aume, eine Krankenstation mit OP-Saal sowie Lager- und Technikr"aume. Im oberen Bereich sind B"uros, Lagerr"aume, ein Bad und ein Schlafbereich mit Stockbetten untergebracht. Das Forschungslabor "uberspannt beide Stockwerke.

Cyberbrain verf"ugt "uber vier Zug"ange:

\begin{description}
	\item[Foyer] Das Foyer ist durch eine Fiberglast"ur gesichert, die bei Nichtbenutzung undurchsichtig geschaltet ist.
	\item[Lagerhalle] Im Untergeschoss befindet sich eine Lagerhalle, die "uber ein eigenes Rolltor verf"ugt.
	\item[Medizinisches Lager] Ebenfalls im Untergeschoss gibt es das Lager der Krankenstation, das ebenfalls durch ein Rolltor zug"anglich 		ist.
	\item[Obergeschoss] Zum ersten Stock f"uhrt eine au\3en gelegene Rampe, die "uber ein Rolltor den Zugang zum Geb"aude im Obergeschoss 	
		gew"ahrt.
\end{description}

\begin{figure*}[htbp]
	\centering
    \includegraphics[width=0.85\linewidth]{./images/cmyk/cyberbrain_cmyk.jpg}
    \newline{}Cyberbrain
	\label{fig:cyberbrain}
\end{figure*}

\subsection{Zugang zum Geb"aude} 
Um bis zum Cyberbrain-Geb"aude zu gelangen, muss sich der Eingreiftrupp zumindest auf den letzten Metern durch die video"uberwachten G"ange der Zone bewegen. Hat der Trupp die Zone "uber die Wartungstunnel betreten, ben"otigen sie entweder eine Verkleidung als Dockbunner-Mitarbeiter oder m"ussen einen tempor"aren Ausfall der Sicherheitssysteme provozieren. Beim Angriff auf die Sicherheitssysteme k"onnen ihnen der von Cynarian bereitgestellte S"oldner \emph{Flint Ross} oder der Omega-Soldat \emph{Jackhammer} helfen. TransSec-Patrouillen k"onnen den Weg durch oder unterhalb der G"ange erschweren.

Um die Mitarbeiter des Instituts davon abzuhalten, einen Alarm auszul"osen, muss das Geb"aude entweder vom Sicherheitsnetz und dem regul"aren ComNetz der Zone getrennt werden, oder Roberto Martinez, der Techniker des Luna-Syndikats, l"ost kurzzeitig einen Stromausfall in der Zone aus. Um zu verhindern, dass Mitarbeiter, die sich im Geb"aude befinden, fliehen k"onnen, muss das Geb"aude von allen Seiten abgesichert werden. Kommen die Spieler selbst nicht auf diese Idee, kann dieser Einsatzplan auch von den S"oldnern beziehungsweise Protektorats-Soldaten vorgeschlagen werden.

\subsection{Im Cyberbrain Institut}
Nach den ersten erfolgreichen Versuchen bei Hannibal und Slingshot wurde die Verpflanzung der Attent"ater-KIs in das Rondra Hospital verlagert und die Forschung an der Technologie beendet. Damit wurde der Betrieb des Cyberbrain Instituts "uberfl"ussig, und es wurde begonnen, die Forschungseinrichtung aufzul"osen. 

Parallel zum Angriff auf das Cyberbrain Institut schicken die USI-Agenten eine kleine Gruppe von Neuro Intelligence Mitarbeitern in das Institut, um zur"uckgelassene Daten und Ger"atschaften zu vernichten. Da nach den Ereignissen im Blackhole Club und im Ice Club mit einem Angriff auf das Institut zu rechnen ist, verzichten die USI-Agenten darauf, eigene Mitarbeiter einzusetzen. Stattdessen nehmen sie in Kauf, die l"astigen Mitwisser von Neuro Intelligence im Zweifel auszuschalten.

Zeitgleich mit dem Eintreffen des Eingreiftrupps befinden sich \emph{Dr.~Dan Leitner}, \emph{\ml{}}, \emph{Dr. Gaius Ross} und \emph{Francis McDonald} in der Einrichtung. Dr.~Dan Leitner, in Reinraumkleidung, befindet sich im Forschungslabor im Erdgeschoss; Francis McDonald in einem B"uro im ersten Stock. Gaius Ross arbeitet im Lagerraum der Krankenstation. \ml{} befindet sich im Serverraum und wird sich dort auch verstecken. Zu diesem Zeitpunkt sind bereits alle Ger"atschaften untauglich gemacht. \ml{} gelingt es gerade noch, die Daten in den Servern zu l"oschen bevor sie entdeckt wird.

Wenn die Gruppe das Geb"aude st"urmt, werden sie fr"uher oder sp"ater auf das Personal treffen, das sich nur ungen"ugend in den wenigen R"aumen des Geb"audes verstecken kann. Ohne Gegenwehr lassen sich die ver"angstigten Wissenschaftler zusammentreiben. Nur \ml{} schimpft, man solle die Finger von ihr lassen, sie k"ame ja mit.

Bei der folgenden Befragung stellt sich Dr.~Dan Leitner sch"utzend vor seine Untergebenen und beteuert, dass sie lediglich f"ur Aufr"aumarbeiten beauftragt wurden und keine Kenntnis "uber die Forschungsinhalte des Instituts h"atten. Er erkl"art, dass alle wichtigen Ger"ate und Daten bereits vorher von den Cyberbrain-Mitarbeitern selbst entfernt wurden, nachdem die Einrichtung vor einigen Tagen aufgel"ost wurde. Seine zunehmende Nervosit"at ist bei seinen Ausf"uhrungen nicht zu "ubersehen. Dr. Ross und McDonald best"atigen seine Geschichte kurz und knapp. Im Einzelgespr"ach erkl"aren sie, dass sie nur den Auftrag haben, die letzten Ger"atschaften aus dem Geb"aude zu entfernen. Sie k"onnen ebenfalls keine Auskunft dar"uber geben, woran gearbeitet wurde. Die junge \ml{} schweigt und weigert sich beharrlich, Fragen zu beantworten.

Gefragt nach ihrer Firmenzugeh"origkeit nennen die Gefangenen nach kurzem Z"ogern die Firma Neuro Intelligence. Zum aktuellen Zeitpunkt ist Neuro Intelligence f"ur die Ermittler und die Spieler noch ein unbeschriebenes Blatt, sodass sie nicht einsch"atzen k"onnen, welche Bedeutung diese Information hat. Der Name der Firma klingt aber in den Ohren der Ermittler nicht nach einer Organisation, die lediglich f"ur Aufr"aumarbeiten beauftragt ist. Leider bietet die Zone keine M"oglichkeit, weiterf"uhrende Informationen zu beschaffen, und auch eine schnelle Anfrage beim Luna-Syndikat liefert keine zus"atzlichen Erkenntnisse.

\newsubsection{Weitere Befragung}
W"ahrend die Ermittler die eingesch"uchterten Mitarbeiter von Neuro Intelligence befragen, sichern die unterst"utzenden Truppmitglieder das Geb"aude ab. Den Ermittlern bleiben folgende Optionen:

\begin{itemize}
	\item Sie versuchen die Neuro Intelligence Mitarbeiter mitzunehmen, um sie au\3erhalb der Zone in Ruhe weiter zu verh"oren.
	\item Sie versuchen die Mitarbeiter vor Ort weiter unter Druck zu setzen und die Wahrheit aus ihnen heraus pressen. Aufgrund 
		der Situation werden die Gefangenen schnell einknicken.
	\item Der Psychonaut unter den Ermittlern kann vor Ort einen Gehirnscan durchf"uhren.
\end{itemize}

Es empfiehlt sich, die R"aume im Geb"aude weiter zu durchsuchen, um belastendes Material zu finden. Parallel dazu sollten die Mitarbeiter von Neuro Intelligence weiterhin befragt werden. Bei der Durchsuchung der R"aumlichkeiten kann festgestellt werden, dass zwar noch nicht alle Ger"atschaften entfernt wurden, jedoch kein wirklich verwertbares Material gefunden werden kann. Dennoch kann aus den medizinischen Ger"aten best"atigt werden, dass hier Gehirnoperationen durchgef"uhrt wurden. Zudem finden sich Belege daf"ur, dass Neuro Intelligence Material f"ur die Integration in ein Kontrollmodul bereitgestellt hat. Werden die Mitarbeiter anschlie\3end ausf"uhrlicher befragt und weiter unter Druck gesetzt, br"ockelt die Geschichte von den Aufr"aumarbeiten. Leitner gibt zu, dass Neuro Intelligence einen Gro\3teil der bei Cyberbrain eingesetzten Technologie beigesteuert hat. Er beteuert jedoch vehement, dass ihnen die finsteren Pl"ane der USI zun"achst nicht bewusst waren und erst w"ahrend der Operationen im Cyberbrain-Institut nach und nach aufgedeckt wurden.

Bei weiteren Befragungen unterst"utzt durch Gehirnscans kommen folgende weitere Erkenntnisse ans Licht:

\begin{itemize}
	\item Dr.~Dan Leitner ist leitender Angestellter bei Neuro Intelligence und langj"ahriger Vertrauter von \emph{Prof.~Dr.~Naratova}, der 
		Inhaberin der Firma. Zusammen mit ihr hat er die Neuronalkopplung entwickelt, also die Nanobots, die die KI mit dem menschlichen Gehirn "uber die Synapsen verbinden.
	\item Die Eingriffe am Gehirn, einschlie\3lich des Einsetzens des Kontrollmoduls, das von Kasai Cyber Genetics geliefert wurde, wurden 
		im Cyberbrain-Institut von Prof.~Dr.~Sanders durchgef"uhrt.
	\item Hannibal und Slingshot wurden als erste Probanden bei Cyberbrain behandelt.
	\item Weitere Eingriffe wurden direkt im Rondra Hospital von Prof.~Dr.~Sanders durchgef"uhrt. Die Namen der manipulierten Personen sind 
		nicht bekannt.
	\item \ml{} ist die leitende Programmiererin, die die KIs der USI f"ur den Einsatz im menschlichen Gehirn trainiert hat. \ml{} ist die zweite 
		enge Vertraute  von Prof.~Dr.~Naratova.
	\item Den Mitarbeitern von Neuro Intelligence wurde erst nach den ersten Eingriffen im Cyberbrain-Institut bewusst, dass die Eingriffe 
		ohne Wissen der Probanden durchgef"uhrt wurden und die Opfer sp"ater gegen ihren Willen als Attent"ater eingesetzt werden sollten.
\end{itemize}

Bei einem Gehirnscan von \ml{} wird der Psychonaut auf unerwarteten Widerstand sto\3en. In ihrem Kontrollmodul ist eine hochentwickelte, KI-gest"utzte Firewall installiert, die versucht, das Kontrollmodul des Psychonauten zum Absturz zu bringen. Ein Gehirnscan bei \ml{} ist daher nahezu unm"oglich.

\begin{remarks}
	W"ahrend die Charaktere versuchen, Informationen zu sammeln, spitzt sich die Lage zu. Es ist wichtig, den Spannungsbogen aufrechtzuerhalten und den Spielern nicht zu viel Zeit zu gew"ahren, um zu "uberlegen, wie sie an Informationen "uber die Neuro Intelligence-Mitarbeiter gelangen k"onnen. 
	
	Im Zweifelsfall muss der Trupp versuchen, die Mitarbeiter aus der Anlage zu bringen, um sie f"ur weitere Befragungen verf"ugbar zu haben. 
	
	Alle relevanten Informationen kann aber im Zweifel auch \ml{} bereitstellen, sobald sie aus der Einrichtung heraus in Sicherheit gebracht wurde.
\end{remarks}

\subsection{Die Absicherung} 
W"ahrend der Befragung sichern die unterst"utzenden Truppmitglieder die Anlage ab und verminen die Eing"ange, wobei sie sich mit dem Charakter abstimmen, den sie als Vorgesetzten akzeptieren. Ein von einer KI manipulierter Omega-Soldat wird die Mine an einer der hinteren T"uren, f"ur die er verantwortlich ist, nicht aktivieren. Stattdessen platziert er an mehreren Stellen im Geb"aude fernz"undbare Sprengs"atze.

\subsection[\xl{}s "Uberwachung]{Xiao Longs "Uberwachung}  
Da \xl{} damit rechnet, \ml{}, deren Existenz ihr durch ihren Besuch bei Prof.~Dr.~Sanders bekannt ist, aufgreifen zu m"ussen, "uberwacht sie die Einrichtung kontinuierlich per Video, um jederzeit eingreifen zu k"onnen. Mit dem Plasmabrenner ist es m"oglich aus den Wartungssch"achsten von unten in das Institut einzudringen. F"ur \xl{} hat \ml{} oberste Priorit"at. Nach ihrer Einsch"atzung kann \ml{} alle f"ur sie wichtigen Informationen "uber die \emph{Freien KIs} (beschrieben \cref{sec:freieki}) in ihrem Kopf liefern.

\subsection{Smith-Singer} 
Smith-Singer, der "uber die Vorf"alle im Blackhole und im Ice Club informiert ist, wartet auf den n"achsten Zug der Ermittler. Sobald der Eingreiftrupp die Einrichtung "uber den Raumhafen der Zone betritt, werden die Sicherheitsgardisten von Smith-Singer sofort dar"uber informiert, dass ein Angriff auf Cyberbrain stattfinden wird. Falls das Luna-Syndikat einen tempor"aren Ausfall der Stromversorgung verursacht, versuchen die Sicherheitskr"afte zun"achst, den Ursprung und die Auswirkungen zu ermitteln. In diesem Fall haben die Charaktere etwas mehr Zeit. Fr"uher oder sp"ater werden die Sicherheitsgardisten jedoch das Cyberbrain-Institut umstellen. 

Als Gr"under des Cyberbrain Instituts wird Smith-Singer darauf bestehen, die Sicherheitsgardisten zu begleiten. Beim Eintreffen vor dem Institut wird Smith-Singer darauf dr"angen, die Situation gewaltsam zu kl"aren, um ein Blutbad in der Zone zu provozieren und damit eine sp"atere Besetzung Valhallas durch Kr"afte des Konzernrats zu legitimieren. Sein zweites Ziel ist es, sowohl die Ermittler als auch die Mitarbeiter von Neuro Intelligence auszuschalten.

\subsection{Eintreffen der Sicherheitsgardisten} 
Die TransSec Sicherheitsgardisten treffen fr"uhestens w"ahrend der ersten Befragung der Neuro Intelligence Mitarbeiter ein und umstellen das Geb"aude. Haben die Ermittler oder die unterst"utzenden Soldaten oder S"oldner eine "Uberwachungsdrohne in den G"angen postiert oder Zugriff auf die "Uberwachungsvideos, werden sie bemerken, wenn die Sicherheitsgardisten unter der F"uhrung von Smith-Singer das Geb"aude einkesseln. Alternativ kann \xl{} den Trupp "uber das Netz der Energieversorgung dar"uber informieren, dass sie Besuch bekommen.

Die Sicherheitsgardisten sind wie \cref{sec:Sicherheitsgardisten} ausger"ustet. Vor und w"ahrend der Auseinandersetzung mit dem Eingreiftrupp muss sich der Spielleiter nicht auf die Anzahl der Gardisten festlegen.

\subsection{Der Attent"ater} 
Werden die Charaktere von einem Trupp der Protektorats-Soldaten begleitet, tritt \emph{Thunder}, der von Smith-Singer befehligte Attent"ater, in Aktion. Smith-Singer nutzt das ComNetz, um Thunder den Befehl zu erteilen, alle Personen im Geb"aude zu t"oten.

Thunder z"undet die von ihm platzierten Sprengs"atze und er"offnet sofort das Feuer. Sein prim"ares Ziel ist es zun"achst, seine Kameraden auszuschalten. Wenn m"oglich, wird Thunder als N"achstes die Mitarbeiter von Neuro Intelligence eliminieren, wobei \ml{} zuf"allig verschont bleibt.

\subsection{Das Geb"aude wird gest"urmt} 
Sollte Thunder das Feuer innerhalb des Geb"audes er"offnen, werden die Sicherheitsgardisten sofort st"urmen. Falls kein Attent"ater im Geb"aude ist, kann der Spielleiter zun"achst eine Verhandlung mit den Geiselnehmern initiieren. Smith-Singer wird jedoch darauf dr"angen, m"oglichst schnell das Geb"aude einzunehmen.

In jedem Fall werden die Gardisten fr"uher oder sp"ater beginnen, die T"uren zu entriegeln. Die Sprengfallen an den T"uren werden einige der Gardisten ausschalten. Falls erforderlich, wird \xl{} den Eingreiftrupp unterst"utzen und die TransSec-Mitarbeiter von hinten angreifen. \xl{} wird nur dann aktiv werden, wenn keiner der Ermittler sie sehen kann. Sollten die Charaktere auf Leichen sto\3en, kann dies Fragen aufwerfen.

Um dem Kampfgeschehen eine kommunikative Komponente hinzuzuf"ugen, k"onnen die Gardisten nach einem ersten misslungenen Angriff versuchen, erneut mit den Eindringlingen zu verhandeln. Die Gardisten werden diese Zeit nutzen, um sich neu zu formieren.

Geht der Eingreiftrupp nicht zum Gegenangriff "uber, werden die Gardisten das Geb"aude erneut st"urmen und Schock- sowie Rauchgranaten einsetzen.

\subsection{Die Flucht}
M"ussen sich die Gardisten aus dem Geb"aude zur"uckziehen, werden sie sich in den G"angen verschanzen und auf die n"achste Reaktion der aus ihrer Sicht Terroristen im Geb"aude warten, bevor sie einen erneuten Angriff starten. Das gibt der Gruppe Zeit, sich f"ur einen Fluchtweg zu entscheiden. Die Charaktere haben mehrere M"oglichkeiten zu fliehen:

\begin{description}
	\item [Die Ausg"ange] Der Eingreiftrupp kann durch einen der Eing"ange im Erdgeschoss "uber die Wege rund um das Geb"aude in die 
		Wartungstunnel zur"uckkehren.
	\item [Im ersten Stock] Haben sich die Charaktere im ersten Stock verschanzt, k"onnen sie wie im Erdgeschoss durch das Tor auf der
		R"uckseite "uber die Rampe auf den Weg hinter dem Geb"aude gelangen.
	\item [Durch den Boden] Der Eingreiftrupp kann versuchen, sich im Geb"aude zu verschanzen und sich mit Plasmabrennern durch den Boden in	
		die Wartungssch"achte zu schneiden.
	\item [Nebengeb"aude] "Uber den Bereich hinter dem Foyer ist es m"oglich, sich durch die Wand in das Nebengeb"aude zu sprengen oder zu	
		schwei\3en. Das Gleiche ist auch im ersten Stock m"oglich. Die anderen Seiten des Geb"audes f"uhren auf die umliegenden G"ange.
\end{description}

Fl"uchtet der Trupp in die Wartungstunnel, werden sie dort m"oglicherweise entdeckt. Die Sicherheitsgardisten werden zun"achst versuchen, sie von oben zu beschie\3en, bevor sie ebenfalls in die G"ange eindringen.

\subsection{Das Nebengeb"aude Stimulus Fungi} 
Auf der vom Foyer rechten Seite befindet sich ein weiteres Geb"aude. Dieses anliegende Geb"aude ist das biologische Labor der Stimulus Fungi Forschungseinrichtung. Das Geb"aude ist leer, da die Mitarbeiter w"ahrend der Belagerung des Cyberbrain Instituts durch die Sicherheitskr"afte evakuiert wurden.

Das Geb"aude erstreckt sich mit einer gro\3en Halle "uber beide Ebenen der Zone und ist etwa dreimal so gro\3 wie die Cyberbrain Forschungseinrichtung. Es ist als gro\3es Gew"achshaus ausgelegt. Unter fokussierter Neonbeleuchtung befindet sich ein mit G"angen durchzogener Irrgarten aus Pflanzenbeeten auf vier Etagen. An verschiedenen Stellen werden die Pflanzenbeete durch Arbeitstische unterbrochen. Etwa die H"alfte der untersten Etage wird von einem Wohnkomplex eingenommen, der dem oberen Stockwerk des Cyberbrain Geb"audes "ahnelt. In der obersten Ebene befindet sich etwa in der Mitte des Geb"audes ein Chemielabor.

Das Geb"aude kann auf beiden Ebenen auf der vom Cyberbrain Geb"aude aus gesehen hinteren Seite "uber ein als Schleuse ausgelegtes Rolltor betreten werden. Im vorderen Bereich befinden sich die Mitarbeitereing"ange, ebenfalls als Schleuse ausgelegt. An den dreckig wei\3en W"anden der Einrichtung prangt, wo nicht durch Pflanzen bedeckt, das Firmenlogo.

\subsection{Fluch "uber den Raumhafen}
Eine unerkannte Flucht "uber den Raumhafen ist nicht m"oglich. Auch wenn die Zone "uber den Raumhafen betreten wurde, muss der Eingreiftrupp die Zone "uber die Wartungstunnel zwischen den Ebenen verlassen. Falls ihnen die Tunnel nicht bekannt sind, wird \xl{} sie am Cyberbrain Institut abholen und in die Tunnel f"uhren.

\subsection[\ml{}]{Mailin}
Wird \ml{} nicht von den Ermittlern aus dem Geb"aude gebracht, wird \xl{} sie mit sich nehmen. In jedem Fall muss \ml{} die Flucht aus dem Geb"aude gelingen.
\vfill

\begin{remarks}
	\underline{Gewonnene Information:}
	
	\begin{itemize}
		\item Hannibal und Slingshot wurden im Cyberbrain Institut als erste Probanden mit KIs ausgestattet.
		\item Prof. Dr. Sanders, der Klinikleiter des Rondra Hospitals, hat die medizinischen Eingriffe im Cyberbrain Institut 
			durchgef"uhrt.
		\item Im Rondra Hospital wurden weiteren Personen KIs eingesetzt.
		\item Die sogenannten Neuronalkopplungen stammen von der Firma Neuro Intelligence.
		\item Prof.~Dr.~Naratova ist Gr"underin und Firmenchefin von Neuro Intelligence.
		\item Dr.~Dan~Leitner ist ein enger Vertrauter von Prof.~Dr.~Naratova. Er hat gemeinsam mit ihr die Neuronalkopplungen konzipiert.
		\item \ml{} ist die leitende Programmiererin bei Neuro Intelligence, die die KIs auf das menschliche Gehirn trainiert hat.
	\end{itemize}	 

	\underline{Konfrontation:}

	Die Infiltration des Cyberbrain Instituts ist auf eine bewaffnete Konfrontation ausgelegt, bei der es zu Verletzten und Toten kommt. Die USI nutzt den Vorfall als Beispiel f"ur einen terroristisch-militanten Angriff auf sensibles Konzerneigentum, um eine kritische Sicherheitslage im Jovianischen System zu verargumentieren, die ein Eingreifen der Kampfverb"ande des Konzernrats erforderlich macht.

	Um zu verhindern, dass kritische Mitwisser in die H"ande der Ermittler fallen, wird Smith-Singer versuchen, die Mitarbeiter von Neuro Intelligence im Cyberbrain Institut zu eliminieren. Das Ausschalten der Neuro Intelligence Mitarbeiter bereits in dieser Episode vereinfacht den weiteren Verlauf und erh"oht die Dramatik der Geschichte.
	
	Bis auf \ml{} und \xl{} gibt es keine Personen, die den Angriff auf das Institut zwingend "uberleben m"ussen. \ml{} muss von der Gruppe oder \xl{} aus der Zone extrahiert werden.

	\underline{Ausr"ustung:}

	Neben Feuerwaffen und R"ustung kann der Eingreiftrupp, falls gew"unscht, auf Granaten, ferngesteuerte Sprengs"atze, Plasmabrenner, "Uberwachungsdrohnen, Magschloss-Knacker und EMP-Granaten zur"uckgreifen. Die Ausr"ustung kann mit Genehmigung von Blackeart vom Protektoratsmilit"ar bereitgestellt werden. Solange die Spieler erkl"aren k"onnen, wie sie das Equipment in die Zone transportieren, stehen ihnen alle M"oglichkeiten offen. Ein Einstieg "uber den Raumhafen bietet dabei nur begrenzte Optionen, aber auch der Weg durch die engen Tunnel erlaubt nicht alle Freiheiten.
\end{remarks}
