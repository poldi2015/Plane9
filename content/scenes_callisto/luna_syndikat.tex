%% Copyright 2019 Bernd Haberstumpf
%% License: CC BY-NC
% !TeX spellcheck = de_DE
\newsection[Das Luna-Syndikat schl"agt zu]{Das Luna-Syndikat schl"agt zu}
                                
Das Luna-Syndikat, das f"uhrende Verbrechersyndikat in Valhalla, verfolgt die Ermittler seit dem Verlassen der Oberstadt. Durch einen W"urfelwurf auf Investigation k"onnen die Charaktere feststellen, dass sie w"ahrend ihrer Recherchen beschattet werden. Die Verfolger scheinen immer wieder unterschiedliche Personen zu sein. Man bemerkt Ghetto-Kids, Schl"ager und generell zwielichtige Gestalten, die verschwinden, sobald sie entdeckt werden. Nachdem sich die ersten Nachforschungen als nicht erfolgreich erweisen, kann die Gruppe beharrlich weitersuchen, einen Verfolger aufgreifen, einen der Schieber oder "Arzte in die Mangel nehmen oder direkt den Kontakt zum Luna-Syndikat suchen, wenn sie auf das Syndikat bereits aufmerksam gemacht wurden.

Unabh"angig davon, f"ur welche Option sich die Spieler entscheiden, stehen die Ermittler fr"uher oder sp"ater unerwartet neun Gangstern gegen"uber, angef"uhrt von einer imposanten Asiatin. Die Anf"uhrerin tr"agt einen verzierten leichten Servopanzer und hat ein Katana auf den R"ucken geschnallt. Die Asiatin, die die Gruppe sp"ater als \xls{} kennenlernt, ist h"ubsch mit einem verwegenen L"acheln im Gesicht. Ihr linkes Auge ziert eine Narbe. Trotz ihres jugendlichen Aussehens sticht sie durch die Pr"asenz eines Anf"uhrers hervor. \xl{} und ihre Bande sind im Auftrag von Nemesis, dem ``Duke von Valhalla'', ausgesandt, um die Gruppe zu ihm zu bringen. Zun"achst wird sie jedoch versuchen, m"oglichst viel "uber die Gruppe in Erfahrung zu bringen. Die Gangster, bis auf \xl{}, halten Bolzenpistolen und Railguns in den H"anden. \xl{} baut sich vor der Gruppe auf, ohne eine Waffe zu ziehen, und herrscht die Ermittler an:

\speak{Ihr stellt sehr viele Fragen. Was wollt ihr hier?}

W"ahrenddessen haben die restlichen Ganoven ihre Schusswaffen in Anschlag gebracht, die H"alfte davon auf den Omega-Krieger gerichtet. Reagieren die Ermittler ausweichend oder erz"ahlen keine "uberzeugende Geschichte, erh"oht \xl{} den Druck und wendet sich an ihre Mitstreiter:

\speak{K"onnen wir mit dieser Geschichte etwas anfangen?}

Die Gang verneint. Quicksilver, ihre rechte Hand, fasst es in Worte: \say{Ziemlicher Quatsch.}

\xl{} wartet kurz ab, ob die Gruppe noch weitere Informationen preisgibt. Falls der Omega-Soldat in Aktion tritt, geht sie auf Abstand und richtet das Wort an ihn. Sollten die Charaktere keine weiteren Informationen preisgeben, wendet sie sich ebenfalls an ihn.

\speak{Es gibt eine Vereinbarung mit dem Protektorat. Bei uns hat die Armee keine Befugnisse. Das ist ein Problem, \pinyin{guai4wu4}.}

An ihre Mitstreiter gewandt, ohne den Omega-K"ampfer aus den Augen zu lassen:

\speak{Schaltet ihn aus.}

Jetzt geht alles sehr schnell. F"unf der Gangster, die ihre Waffen auf den Omega-Krieger gerichtet haben, dr"ucken ab. Sie schie\3en mit Schockprojektilen (beschrieben \cref{sec:heavyweapons}). Reagiert der Spieler ohne Z"ogern, kann sein Charakter versuchen, den Projektilen auszuweichen und dabei einen der Angreifer in den Nahkampf zu zwingen. Auch in diesem Fall werden die anderen Gangster, ohne R"ucksicht auf ihren Kameraden, weiter auf den Soldaten schie\3en. \xl{} bringt sich w"ahrend der Auseinandersetzung au\3er Reichweite und zieht ihre kurzl"aufige Multigun. Die Gangster, die den Rest der Gruppe in Schach halten, bleiben auf Abstand. Egal, ob der Omega-K"ampfer rechtzeitig reagiert und m"oglicherweise sogar einen Gangster zu Boden streckt, werden ihn die Schockprojektile ausschalten. Er bricht bewusstlos zuckend zu Boden. F"ur die anderen Ermittler sieht die Lage hoffnungslos aus. Der kurze Schlagabtausch dauert keine 2 Sekunden und hat ihnen keine M"oglichkeit gegeben zu reagieren. \xl{}s Ziehen einer Waffe war bestenfalls als Schemen zu erkennen.

Der Angriff auf den Omega war der zweite Auftrag, den \xl{} vom Gangsterboss erhalten hat. Nemesis ist zwar gezwungen, den Ermittlern zu helfen (siehe n"achstes Kapitel), aber ein Eindringen in sein Reich kann er nicht ohne Vergeltung hinnehmen.

Quicksilver meldet sich zu Wort. 

\speak{Eine Nachricht von Nemessis. Er will, dass wir sie mitnehmen.}

\xl{} gibt sich "uberrascht. 

\speak{Sieh an. Ihr habt eine pers"onliche Audienz beim Herren der Stadt gewonnen.}

\xl{} an die Gangster gewandt: 

\speak{Packt sie ein, aber sch"on vorsichtig. Den \pinyin{xue1ruo4} lassen wir hier.}

Tote Gangster werden zur"uckgelassen. Die Charaktere, mit Ausnahme des niedergestreckten Omega-Kriegers, der die Gruppe nicht begleiten kann, werden gefesselt in gel"andetaugliche Trucks verfrachtet. Dann geht es los. Der Tross f"ahrt durch die mit Stra\3ensperren gut gesicherten Gebiete des Breidablik-Bezirks.

\vfill

\begin{remarks}
	\underline{Spielfluss:}

	In dieser und der n"achsten Szene ist es schwierig, den Spielern Freir"aume zu gew"ahren, ohne den Plot zu gef"ahrden oder die Authentizit"at der Rollen von \xl{} und Nemessis zu besch"adigen. Viel von der Atmosph"are h"angt hier von der Bereitschaft der Spieler ab, in die Dramatik einzusteigen. Der Spielleiter sollte den Spielern nur kurz erlauben, sich auszutauschen, um sich dann an die Gangster zu wenden. Ebbt die Initiative der Spieler ab, sollte \xl{} sofort den Schie\3befehl erteilen. Der Spielleiter sollte dabei den Spieler des Omega-Kriegers nicht nach seiner n"achsten Handlung fragen. Nur wenn der Spieler sofort selbst reagiert und einen Gegenangriff ank"undigt, kann er in das Geschehen eingreifen.

	\underline{\xl{} und das Luna-Syndikat}

	Diese Szene dient in erster Linie dazu, \xl{} und das Luna-Syndikat als skrupellose Gangster einzuf"uhren. \xl{}, die  \cref{sec:xiaolong} im Detail beschrieben wird, spielt im Rest der Geschichte eine zentrale Rolle. Zum einen ersetzt sie Firedon als Begleiter der Ermittler. Zum anderen "ubernimmt sie eine Doppelrolle mit einer eigenen Agenda, die hoffentlich erst ganz am Ende aufgekl"art wird. \xl{} wurde, wie den Attent"atern, eine k"unstliche Intelligenz eingesetzt. In ihrem Fall allerdings ohne Bindung an die USI. Zus"atzlich ist ihr K"orper durch hervorragende Cyberware aufger"ustet und damit ein m"achtiger Charakter, der aber bis zum Ende der Geschichte seine Rolle erf"ullen muss. \xl{} ist eine clevere Draufg"angerin, cool und abgekl"art, mit der Undurchschaubarkeit eines Samurai. Ihr Charakter sollte aber in den folgenden Szenen nicht zu oft in den Vordergrund gezogen werden, damit ihre Doppelrolle nicht auffliegt.

	\xl{} wird beim Angriff auf den Protektorats-Soldaten ihre vollautomatische Multigun (ihre kurzl"aufige Waffe) nicht einsetzen, um ihre "Uberlegenheit zu demonstrieren. Da sie die Waffe jedoch mit "ubernat"urlicher Geschwindigkeit zieht, ist klar, dass sie "uber Reflexe verf"ugt, die denen eines Omega-Kriegers ebenb"urtig sind.
\end{remarks}

\begin{remarks}	
	\underline{\xl{}s Hintergrund:}

	Kommt der Chefermittler von Cynarian oder sein Assistent aus dem Asteroideng"urtel zwischen Mars und Jupiter und war bei der Jagd auf Piraten beteiligt, kann der Spielleiter bei dieser ersten Begegnung mit \xl{} vor dem Beginn des Dialog einen W"urfelwurf auf Knowledge ohne weitere Begr"undung verlangen. Bei vollem Erfolg kann der Charakter \xl{} als die Anf"uhrerin des Piratenbunds ``Roter Drache'' identifizieren, die auf Valhalla festgenommen wurde. Besteht sp"ater die M"oglichkeit, "uber das ComNetz zu recherchieren, l"asst sich herausfinden, dass die Piratin im Gef"angnis verstorben ist. N"aheres ist in den Protokollen nicht vermerkt.

	\underline{Chinesische Begriffe:}

	Folgende chinesische Begriffe werden in den Dialogen genutzt:

	\begin{itemize}
		\item \pinyin{guai4wu4} bedeutet Missgeburt. Bezeichnet den Omega-Krieger.
		\item \pinyin{xue1ruo4} bedeutet Kr"uppel. Bezeichnet den Omega-Krieger.
	\end{itemize}
\end{remarks}


\newsection{Treffen mit Nemessis}

Nach der vergangenen Auseinandersetzung wird die Gruppe f"ur ein Treffen mit Nemessis durch eine gro\3e Maschinenhalle, die das "ortliche Fusionskraftwerk beherbergt, zum erh"ohten Leitstand gef"uhrt. In einem weitr"aumigen B"uro, in dem sich bereits mehrere Capos und gut ger"ustete S"oldner befinden, steht ein hochgewachsener Mann in einem langen schwarzen Mantel in der Mitte des Raumes. Mit dem R"ucken zu den Eintretenden gewandt, steht er vor einem ausladenden Schreibtisch und ist leise in ein Gespr"ach mit einer anderen Person vertieft. Die Charaktere werden aufgefordert, einige Meter vor den Personen am Schreibtisch stehenzubleiben. Nach "uber einer Minute dreht sich der Mann im Mantel zu ihnen um. Vor den Ermittlern steht ein h"uhnenhafter Cyborg. Soweit man erkennen kann, ist ein Gro\3teil des K"orpers des Mannes durch k"unstliche Metallteile ersetzt worden. Die Reste des Gesichts, die noch "ubrig geblieben sind, zeichnen ihn durch Flecken und Runzeln als Slack aus. Mit hypnotisch stechenden Augen und einer blechernen Stimme wie bei ein Roboter wendet er sich an die Gruppe:

\speak{Mein Name ist Nemessis. Sch"on, dass Sie zu mir gefunden haben.}

An \xl{} gewandt, \say{\xl{}, gab es Schwierigkeiten?}. \xl{}: \say{Keine}. Nemessis f"ahrt, an die Ermittler gewandt, fort:

\speak{Meine Zeit ist leider sehr begrenzt. Deshalb gleich zur Sache. Welche Nachforschungen f"uhren Sie in meine Stadt?}

Wenn die Charaktere nicht alles erz"ahlen hakt Nemessis nach:

\speak{Das ist doch so nicht ganz vollst"andig, oder? Versucht bitte etwas pr"aziser zu sein.}

Die Charaktere k"onnen versuchen, Nemessis davon zu "uberzeugen, dass durch die Vorkommnisse die Sicherheit des jovianischen Systems gef"ahrdet ist und m"oglicherweise eine milit"arische Intervention seitens der Protektoratsstreitkr"afte droht. Unabh"angig davon schl"agt Nemessis den Ermittlern daraufhin vor, den Blackhole Club f"ur weitere Nachforschungen zu besuchen, und stellt ihnen \xl{} zur Seite, die sich f"ur einen winzigen Moment ein diabolisches Grinsen nicht verkneifen kann.

\speak{Pass gut auf unsere G"aste auf, \xl{}. Sie stehen nun unter dem Schutz des Luna-Syndikats.}

Damit ist die m"oglicherweise etwas bizarr wirkende Audienz beendet. \xl{} fordert die Ermittler mit einer knappen Geste auf, ihr und ihrem Tross zu folgen. Nach einigen Minuten erreichen sie das nahe gelegene  \emph{Sunshine Hotel}. Auf dem Weg dorthin werden ihnen die Fesseln abgenommen. \xl{} antwortet auf dem Weg bereitwillig auf Fragen der Gruppe, "uberspringt jedoch geflissentlich die Auseinandersetzung bei ihrer ersten Begegnung.

In der Lobby des Hotels befinden sich bereits einige Angestellte des Dukes sowie ein "alterer Mann in einem Arztkittel. \xl{} tauscht leise ein paar Worte mit dem Chinesen im Arztkittel aus. Beide sprechen in einem im Asteroideng"urtel gebr"auchlichen chinesischen Dialekt. Die Diskussion wird hitziger, als die Begriffe Omega und Blackhole Club in den Raum geworfen werden. Freundschaftlich klopft sie dem alten Mann auf die Schulter und beendet das Gespr"ach. Daraufhin wendet sich der Arzt, der sich als Dr.~\pinyin{Li4} \pinyin{Li4} vorstellt, etwaigen Verletzten zu. \xl{} bereitet sich w"ahrenddessen auf den Besuch im Blackhole Club vor.

\begin{remarks}
	\underline{Gewonnene Information:}
	
	\begin{itemize}
		\item Kontaktaufnahme zu Nemessis, dem Duke von Valhalle, Anf"uhrer der Verbrecherorganisation Luna-Syndikat.
		\item \xl{} wird Kontaktfrau der Ermittler.
	\end{itemize}

	\underline{\xl{}:}

	\xl{} ist ab dieser Szene die Begleiterin der Gruppe in ihrem eigenen Interesse. Sie fungiert damit f"ur den Spielleiter als Joker, um den Weg, den die Spieler einschlagen, im Zweifel subtil zu korrigieren. Wie in den Anmerkungen im letzten Kapitel beschrieben, verfolgt sie jedoch auch eine eigene Agenda, bei der die Ermittler lediglich als ungewollte Helfer fungieren. St"uck f"ur St"uck hat sie begonnen, ihr zweites Bewusstsein, die in ihrem Kopf eingesetzte KI, zu erforschen und die Hintergr"unde sowie Akteure aufzusp"uren.  Um nicht als gejadge Attent"aterin ins Visir zu geraten, sieht sie sich gezwungen, alle weiteren Wissenstr"ager auszuschalten, ohne selbst Spuren zu hinterlassen. Gleichzeitig versucht sie dabei selbst an die eingesetzte Technologie zu gelangen.
\end{remarks}

\begin{remarks}
	\underline{\pinyin{Li4} \pinyin{Li4}}

	Der Arzt \pinyin{Li4} \pinyin{Li4} ist ein alter Freund der Familie \pinyin{Wang2}. Er ist um das Wohlergehen seines Sch"utzlings \xl{} besorgt. \xl{} hat sich in den vergangenen Tagen auf mehrere Wettk"ampfe in der Arena des Blackhole Clubs, auch mit Omega-Kriegern, eingelassen und einige Blessuren eingesteckt. \pinyin{Li4} \pinyin{Li4} spielt in der Geschichte an sich keine Rolle, kann aber  \cref{sec:newgoal} die Ermittler in Bezug auf Fragen zu \xl{} eventuell unterst"utzen.
\end{remarks}


\newsection{Blackhearts Intervention}

Minuten nachdem die Wagenkolonne mit den Gangstern abgezogen ist, erwacht der niedergeschossene Omega-Ermittler aus seiner Starre. Unter starken Schmerzen beginnt er, seine Umgebung wahrzunehmen. Um ihn herum hat sich mit geb"uhrendem Abstand eine Gruppe von Slags und anderem heruntergekommenem Gesindel versammelt. Ein Slag, der gerade noch neben dem Soldaten gekauert hatte, verschafft sich panisch Abstand auf allen Vieren und kr"achzt dabei: \say{Ich hab nichts angefasst, ich hab nichts angefasst!} Die kybernetischen Systeme im K"orper des Omega-Kriegers haben sich bereits wieder reaktiviert und sch"utten Schmerzmittel sowie muskelentspannende Drogen aus. Eine kurze Einsch"atzung der Umgebung zeigt keine akute Gefahr.

Mangels weiterer Informationen bietet es sich f"ur den Ermittler an, die urban kaum erschlossenen Gebiete zu verlassen und sich in die Oberstadt zu begeben, um sich mit der F"uhrung in Verbindung zu setzen und um weitere Instruktionen zu bitten. Blackheart, die von Thunderbolt sofort informiert wird, wei\3 umgehend, von wem die Attacke ausging und was sie zu bedeuten hat:

Seit Beginn des Protektorats gibt es eine Vereinbarung zwischen dem Luna-Syndikat und Blackheart, dass das Syndikat Valhalla ohne Eingriffe des Milit"ars kontrollieren darf. Im Gegenzug sorgt das Syndikat f"ur den reibungslosen Betrieb der Stadt. Die Gebiete au\3erhalb der Oberstadt sind f"ur das Milit"ar Tabu.

Blackheart tobt. Durch die vorangegangene Blockade der Ermittler und die Provokation durch den Angriff auf einen ihrer Soldaten ist das Ma\3 voll. Blackheart droht Nemessis mit milit"arischem Eingreifen, sollte sich Nemessis noch irgendetwas erlauben. Sie befiehlt ihm, die Ermittler umfassend zu unterst"utzen und f"ur deren Sicherheit zu sorgen.

Dank Blackhearts Intervention hat der niedergeschlagene Ermittler nun die M"oglichkeit, zur Gruppe zur"uckzukehren. Am Rande von Neu Gr"oning wird er von dem in Bezug auf ihn skeptischen Quicksilver mit einem Buggy abgeholt und im halsbrecherischen Tempo in das Hotel Sunshine gebracht, in dem die anderen Charaktere bereits untergekommen sind.

\begin{remarks}
	In dieser Szene l"asst sich das Temperament von Blackheart voll ausspielen. Blackheart wird zwar nicht auf ihre Abmachung mit dem Duke eingehen, aber der Name Nemessis, \say{Das wird ihm noch leidtun} und \say{Glaubt jetzt jeder, mir auf der Nase herumtanzen zu k"onnen!?} wird in dem Gespr"ach fallen, das Thunderbolt, der Kontaktmann des Ermittlers, an ihn gefiltert weitergibt.
\end{remarks}
