\newsection[Zusammestross mit dem Luna-Syndikat]{Zusammestross mit dem\newline{}Luna-Syndikat}
\newcommand{\xl}{\pinyin{Xiao3} \pinyin{Long2}}
\newcommand{\xlsn}{\pinyin{Xiao3} \pinyin{Long2}}

Das Luna--Syndikat verfolgt die Ermittler seit dem Verlassen des Raumhafens und Rosenfurth genau. Durch einen W"urfelwurf auf Investigation bemerken Charaktere dass die Gruppe immer wieder verfolgt wird. Die Verfolger scheinen immer wieder unterschiedliche Personen zu sein, Ghetto Kids, Schl"ager, d.h.~zwielichte Personen die verschwinden wenn sie das Gef"uhl haben entdeckt worden zu sein. Nachdem sich die ersten Nachforschungen als nicht erfolgreich erweisen k"onnte die Gruppe beharrlich weiter suchen, einen Verfolger aufgreifen, einen der Schieber oder "Arzte in die Mangel nehmen oder den Kontakt zum Syndikat suchen. 

Unabh"angig davon f"ur welche Option sich die Spieler entscheiden stehen den Ermittlern fr"uher oder sp"ater unerwartet 9 Gangster angef"uhrt von \xl{} gegen"uber. \xlsn{} und ihre Bande ist im Auftrag von Nemessis ausgesandt die Gruppe zu ihm zu bringen. Sie allerdings wird zuerst versuchen m"oglichst viel "uber den Wissensstand der Gruppe in Erfahrung zu bringen. Die Gangster bis auf \xlsn{} halten Multiguns in den H"anden. \xlsn{} tritt den Charakteren ohne eine Waffe in der Hand einen Schritt entgegen. Sie herrscht den Charakter der ihr direkt gegen"uber steht an.

\speak{Ihr stellt sehr viele Fragen. Was wollt ihr hier?}

W"ahrenddessen haben die restlichen Ganoven ihre Schusswaffen in Anschlag gebracht. Die H"alfte davon auf den Omega. Reagieren die Ermittler ausweichend wird sie den Druck erh"ohen und wendet sich an ihre Mitstreiter:

\speak{Ist diese Geschichte glaubw"urdig?}

Die Gang verneint. Quicksilver, ihre rechte Hand, fast es in Worte \say{ziemlicher Quatsch.}

\xlsn{} wird zun"achst abwarten welche Informationen die Gruppe bereit ist zu geben. Tritt der Omega in Aktion wird sie sich auf Abstand begeben und sich an diesen wenden. Geben die Charaktere keine weiteren Informationen preis wendet sie sich ebenfalls an den Omega.

\speak{Es gibt eine Vereinbarung mit dem Protektorat. Bei uns hat die Armee keine Befugnisse. Das ist ein Problem.}

An ihre Mitstreiter gewandt ohne den Omega aus den Augen zu lassen:

\speak{Schaltet ihn aus.}

Jetzt geht alles ganz schnell. Die Gangster die Waffen auf den Omega gerichtet feuern ihre Waffen ab. Sie schie\3en allerdings keine penetrierenden Geschosse sondern Schockprojektile. Reagiert der Spieler des Omegas sofort kann er versuchen den Projektile auszuweichen und dabei einen der Angreifer in den Nahkampf zu zwingen. In diesem Fall werden die anderen Gangster ohne R"ucksicht auf ihren Kameraden weiter auf den Omega schie\3en. Der Angriff auf den Omega war wohl geplant und von Nemessis beauftragt. Nemessis ist zwar gezwungen den Ermittlern zu halfen (siehe weiter unten) aber ein Eindringen in sein Reich kann er nicht einfach hinnehmen. 

\xlsn{} bringt sich bei der Auseinandersetzung au\3er Reichweite. Die Gangster den Rest der Gruppe in Schach halten bleiben auf Abstand. Egal wie der Kampf ausgeht wird \xlsn{} die Charaktere "uberzeugen, dass die Gruppe nicht ohne Verluste aus einer weiteren Konfrontation heraus kommen werden. Die Waffen der verbliebenen Angreifer sind mit scharfer Munition geladen, behauptet sie. Wichtig in jedem Fall ist das \xlsn{} am Ende die Kontrolle beh"alt. Wenn der Omega ausgeschaltet wurde und die Gangster nach wie vor in der "Uberzahl sind kann \xlsn{} ein weiteres Mal versuchen mehr Informationen aus den Ermittlern zu pressen.

Quicksilver meldet sich zu Wort. 

\speak{Eine Nachricht von Nemessis. Er will das wir sie mitnehmen.}

\xlsn{} gibt sich "uberrascht. 

\speak{Sieh an. Ihr habt eine pers"onliche Audienz beim Herren der Stadt gewonnen.}

\xlsn{} an die Gangster gewandt: 

\speak{Packt sie ein, aber sch"on vorsichtig. Den Omega lassen wir da.}

Verwundete oder tote Gangster die sich nicht eigenst"andig Fortbewegen k"onnen werden ebenfalls da gelassen. Die Charaktere werden bis auf den Omega, der die Gruppe nicht begleiten kann, gefesselt und in gel"andetaugliche Buggies gesetzt. Dann geht es los.

\begin{remarks}
	In dieser und der n"achsten Szene ist es schwierig den Spielern Freiheitsgrade zu lassen ohne den Plot zu gef"ahrden oder die Authentizit"at der Rollen von \xl{} und Nemessis zu besch"adigen. Viel h"angt hier von der Bereitschaft der Spieler ab in die Dramatik einzusteigen. Der Spielleiter sollte den Spielern erlauben zu kommunizieren aber sie auch Druck aufbauen zu handeln. Ebbt die Initiative der Spieler ab sollte \xl{} die Erschie\3ung des Omegas Befehlen. Der Spielleiter sollte dabei den Spieler des Omegas nach seiner n"achsten Handlung fragen sondern beschreiben wie die Gangster ihr Ziel anvisieren und dann sofort schie\3en lassen.

	Der Auftrag f"ur den "uberfall ist die Charaketere pers"onlich zu Nemessis zu bringen und gleichzeitig dem Protektorat mitzuteilen, dass das Vorgehen der Charaktere im Territorium des Syndikat nicht geduldet werden kann. Wichtig bei der ganzen Auseinandersetzung ist es, dass keiner der Charaktere get"otet wird und \xl{} als Sieger aus der Konfrontation hervor geht. \xl{} mu\3 in dieser Szene ihre Skrupellosigkeit und ihre Rolle als Anf"uhrerin zeigen. Quicksilver ist der Joker um eine Auseinandersetzung zu beenden und im Sinne des Syndikat zu einem Erfolg zu bringen.

	Greift der Omega zur Waffe sollte der Spielleiter ihn die Wahl lassen ob er mit t"odlicher Munition oder auch mit Schockmunition schie\3en m"ochte.
\end{remarks}


\newsection{Blackhearts Intervention}

Parallel zur Audienz hat der Omega die M"oglichkeit zur Garnison zur"uck zu kehren und das geschehene zu Berichten. 

Zwischen dem Luna--Syndikat und Blackheart gibt es seit Beginn des Protektorats die Vereinbarung dass das Syndikat ohne Eingriff des Milit"ars Valhalla kontrollieren darf. Im Gegenzug k"ummert sich das Syndikat um den reibungslosen Betrieb der Stadt. Durch die  vorangegangene Blockade der Ermittler und der Provokation durch den Angriff auf den Omega droht Blackheart Nemessis mit einem milit"arischen Eingreifen seitens des Protektorats, sollte Nemessis die Charaktere nicht umfassend unterst"utzen und f"ur deren Sicherheit zu sorgen. Der Omega hat damit wieder die M"oglichkeit nach der Audienz der Gruppe bei Nemessis die anderen Ermittler zu begleiten.


\newsection{Treffen mit Nemessis}

F"ur ein Treffen mit Nemessis werden die Ermittler durch eine gro\3e Maschinenhalle, die das "ortliche Fusionskraftwerk beherbergt, zum erh"oht angebrachten Leitstand gef"uhrt. In einem weitr"aumigen B"uro, in dem sich bereits mehrere Capos und gut ger"ustete S"oldner befinden, steht ein hochgewachsener Mann in einem langen schwarzen Mantel mit dem R"ucken zu den Anwesenden vor einem ausladenden Schreibtisch an dem er mit einer anderen Person leise spricht. Die Charaktere werden aufgefordert, einige Meter vor ihm stehen zu bleiben. Nach etwa einer Minute dreht sich der Mann, der sich damit als Nemessis zu erkennen gibt, zu den Charakteren um. Er st"utzt sich dabei auf seinen Gehstock.

\speak{Mein Name ist Nemessis. Sch"on dass Sie zu mir gefunden haben.}

An \xl{} gewandt, \say{\xlsn{}, gab es Schwierigkeiten?}. \xl{}:  \say{Keine}. 

Nemessis f"ahrt an die Ermittler gewandt fort. 

\speak{Meine Zeit ist sehr begrenzt. Deshalb gleich zur Sache. Welche Nachforschungen f"uhren Sie in meine Stadt?}

Wenn die Charaktere nicht alles erz"ahlen erkl"art Nemessis:

\speak{Das ist doch so nicht ganz vollst"andig, oder? Versuchen Sie es bitte noch einmal etwas pr"aziser.}

Die Charaktere sollten Nemessis versuchen davon zu "uberzeugen, dass durch die Vorkommnisse die Sicherheit des jovianischen Systems gef"ahrdet ist und m"oglicherweise eine milit"arische Intervention Seitens der Protektoratsstreitkr"afte drohen. Nemessis schl"agt den Ermittlern darauf hin vor den Blackhole Club zu besuchen und stellt \xl{} an ihre Seite.

Nach Beendigung der der Audienz wird die Gruppe durch \xlsn{} in die Lobby des Suushine Hotels gef"uhrt. Auf dem Weg dahin werden ihnen die Fesseln abgenommen. Im Raum befinden sich bereits einige Angestellten des Dukes wie auch ein "alterer Mann in einem Arztkittel. \xlsn{} bedeutet dem Arzt, der sich als Dr.~\pinyin{Li3} \pinyin{Li3} vorstellt, etwaige Wunden zu versorgen.

\begin{remarks}
	Gewonnene Informationen: Kontakt Nemessis und \xl{}.

	\xl{} ist ab dieser Szene die Begleiterin der Gruppe in ihrem eigenen Interesse und ersetzt damit den Adjutanten Firedon. \xl{} strebt an an Naratovas Forschungsergebnisse zu gelangen und alle weiteren Informationen zu den KIs zu vernichten.

	In Begleitung anderer Gangster tritt \xl{} als Anf"uhrer auf und verteilt Aufgaben unterst"utzt aber auch ihre Untergebenen soweit m"oglich und sinnvoll. Bei der Unterst"utzung der Gruppe ist sie aber meist alleine von den Partie und operiert autonom. Durch ihre "uberragenden k"ampferischen F"ahigkeiten kann sie leicht in allen Gefahrensituationen eingesetzt werden.
\end{remarks}
