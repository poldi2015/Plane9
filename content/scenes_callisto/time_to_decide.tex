%% Copyright 2019 Bernd Haberstumpf
%% License: CC BY-NC
% !TeX spellcheck = de_DE
\newsection{Zeit der Entscheidung}

\newsubsection{Die Medien} 
Als die Charaktere die Zone verlassen, wird gerade "uber die Medien auf Valhalla eine Eilmeldung verbreitet:

\begin{speech}
Aus Sicherheitskreisen wurde bekannt gegeben, dass vor etwa einer Stunde ein paramilit"arischer Angriff auf eine Konzerneinrichtung in der gesicherten Zone stattgefunden hat. Bei dem als terroristischen Anschlag eingestuften Vorfall kamen mehrere Zivilisten und zahlreiche Sicherheitskr"afte ums Leben oder wurden schwer verletzt. Es wird vermutet, dass bei der militanten Aktion Beweismaterial in Zusammenhang mit der k"urzlich stattgefundenen Attentatsserie vernichtet werden sollte. Unbest"atigten Quellen zufolge sollen Angeh"orige der Protektoratsstreitkr"afte am Angriff beteiligt gewesen sein. Nach den Angreifern wird bereits gefahndet. Eine Stellungnahme der Cynarian Corporation, die in der Vergangenheit Ziel von Attentaten war, steht noch aus, wird jedoch in K"urze erwartet.

\nopagebreak
Aufgrund der unklaren Sicherheitslage und des unerwarteten Eintreffens eines Flottenverbands des Protektoratsmilit"ars vor den Toren von Valhalla, w"ahrend hohe W"urdentr"ager von Erde und Mars zu einem Gipfeltreffen erwartet werden, wurden Sicherheitskr"afte vom im Orbit befindlichen Kreuzer Zeus II-1 entsandt, um die Gebiete der Oberstadt auf Valhalla zu sichern. Den Anweisungen dieser Sicherheitskr"afte ist unbedingt Folge zu leisten.

\nopagebreak
Wir halten Sie, liebe Zuschauer, nat"urlich weiterhin auf dem neuesten Stand der Entwicklungen.

\nopagebreak
- Ihre Conni Hanseln
\end{speech}

\subsection{Nemessis} 
Die Eilmeldung macht unmissverst"andlich klar, dass die Charaktere vorerst untertauchen m"ussen. Der sicherste sofort verf"ugbare Unterschlupf ist derzeit das Luna-Syndikat. \xl{}, die sie bereits begleitet oder nach der Flucht zur Gruppe st"o\3t, f"uhrt sie "uber sichere Wege zum Fusionskraftwerk im Bezirk Breidablik.

Sobald die Charaktere in die Obhut des Luna-Syndikats gelangt sind, werden die Omega-Soldaten oder S"oldner der Cynarian Corporation, die die Ermittler beim Eindringen in die Cyberbrain-Einrichtung unterst"utzt haben, von ihnen getrennt. \xl{} bringt die Ermittler direkt zum Leitstand des Fusionskraftwerks, wo sie eine weitere Audienz bei Nemesis erwartet. Es wird sofort deutlich, dass sich das Syndikat auf eine kriegerische Auseinandersetzung vorbereitet. Zug"ange werden durch zus"atzliche Barrieren gesichert, vitale Bereiche st"arker gepanzert, und eine gro\3e Menge an Waffen liegt bereit. Die Stimmung ist angespannt. Nemesis kommt wutschnaubend gleich zur Sache:

\speak{Habt ihr die Eilmeldung schon gesehen? Ihr seid gerade das Top-Thema in den Medien.}

Falls die Gruppe die Eilmeldung noch nicht gesehen hat, wird sie ihnen jetzt vorgespielt.

\speak{Ich hoffe, euch ist klar, dass die Situation ernsthafte Ausma\3e angenommen hat. Ein falsches Wort zur falschen Zeit, und Valhalla steht in Flammen—mit euch mittendrin.}

Nemessis fordert mit Nachdruck einen ausf"uhrlichen Bericht ein. F"ur Gepl"ankel hat er momentan keine Geduld. Folgende Einsch"atzungen l"asst Nemessis in das Gespr"ach mit den Charakteren einflie\3en:

\begin{itemize}
	\item Nach dem Eintreffen der "`Sicherheitskr"afte"', was de facto einer Besetzung der Oberstadt gleichkommt und den Garnisonsst"utzpunkt 
		in Bedr"angnis bringt, ist eine Vergeltung durch Blackheart zu erwarten. F"ur Blackheart stellt dieser Vorsto\3 der Sicherheitskr"afte eine erneute Kampfansage dar.
	\item Fr"uher oder sp"ater wird eine der beiden Parteien versuchen, alle strategisch wichtigen Punkte in Valhalla zu besetzen, darunter 
		auch das Kraftwerk, das "ubrigens urspr"unglich von der USI aufgebaut wurde, noch bevor die Cynarian Corporation und das Protektorat eingetroffen sind. Das Luna-Syndikat wird diese Anlage sicherlich nicht kampflos aufgeben.
	\item Die Ermittler k"onnen sich in der Oberstadt nicht mehr blicken lassen, da dort bereits nach ihnen gefahndet wird. Das 
		Luna-Syndikat kann ihnen f"ur einige Zeit Unterschlupf gew"ahren und sie im Sunshine Hotel unterbringen. Als Gegenleistung erwartet Nemessis eine Strategie, wie sie das Syndikat vor einer Besetzung sch"utzen wollen.
\end{itemize}

Nemessis fordert die Ermittler auf, innerhalb einer Stunde ihre n"achsten Schritte vorzustellen.

\speak{In einer Stunde will ich von euch h"oren, wie ihr gedenkt, den Schlamassel zu l"osen. Ich hoffe, ihr k"onnt mich "uberzeugen. Jetzt verschwindet. \xl{}, sorge daf"ur, dass es keine weiteren "Uberraschungen gibt.}

\xl{} bringt die Gruppe ins Sunshine Hotel, wo sie sie in einem Konferenzraum mit den gefangenen Neuro-Intelligence-Mitarbeitern unterbringt. Sie l"asst sich, voll bewaffnet, auf einem Sessel nieder, w"ahrend sie die restlichen Gangster aus dem Zimmer schickt. Der Konferenzraum verf"ugt "uber einen Nebenraum, der es den Ermittlern erm"oglicht, die Gefangenen einzeln zu befragen. \xl{} aktiviert einen St"orsender, der jegliche Kommunikation nach innen und au\3en blockiert. Das Ger"at erlaubt ihr jedoch auch, den Funkverkehr innerhalb des Raums mitzuh"oren. Auf diese Weise "uberwacht sie jede nicht-verbale Kommunikation zwischen den Charakteren, da sie bef"urchten muss, dass \ml{} als KI-Hybrid enttarnt werden k"onnte.

Der Spielleiter sollte die Spieler nun dabei unterst"utzen, alle gewonnenen Informationen zusammenzutragen und zu bewerten, bevor die Ermittler ihre n"achsten Schritte planen.

\subsection{Die Cyberbrain Informationen} 
Die Charaktere k"onnen nun \ml{} und die anderen Neuro-Intelligence-Mitarbeiter weiter befragen. Unter der Kontrolle der Ermittler und unter der Schirmherrschaft des Luna-Syndikats sind die Neuro-Intelligence-Mitarbeiter bereit, vollends zu kooperieren. Einen groben "Uberblick "uber die Machenschaften der USI, des Cyberbrain Instituts und Neuro Intelligence haben die Ermittler bereits im Cyberbrain Institut erhalten. Im Sunshine Hotel k"onnen die Neuro-Intelligence-Mitarbeiter weitere Details erg"anzen:

\begin{description}
	\item[Neuro Intelligence] Neuro Intelligence ist eine private Forschungseinrichtung auf der Nike Station unter der Leitung von 
		Prof.~Dr.~Naratova. Die Wissenschaftlerin hat, vor der Umfunktionierung der Orbitalstation Neu-Gr{\o}nning als Nike Station, die KI-Forschung bei Cynarian geleitet. Nach der Einstellung der KI-Forschung gr"undete sie im Verborgenen Neuro Intelligence als eigenst"andiges Unternehmen, ohne Wissen der F"uhrung von Cynarian, und f"uhrte ihre Forschung weiter. Prof.~Dr.~Naratovas Ziel ist es, eine symbiotische Verschmelzung von Mensch und KI zu erschaffen, um die Begrenzungen des menschlichen Gehirns zu "uberwinden. \ml{} ist die KI-Expertin, und Dan Leitner ist neben Prof.~Naratova der Spezialist und Leiter der Neurokopplungs-Entwicklung.
	\item[USI] Die USI finanziert Neuro Intelligence verdeckt und unterst"utzt sie mit fortschrittlicher KI-Technologie. Die genauen 
		Vereinbarungen zwischen Neuro Intelligence und der USI sind den festgesetzten Angestellten allerdings nicht bekannt.		
	\item[Rondra Hospital] Nach den ersten Versuchen an Slingshot und Hannibal in der Cyberbrain-Einrichtung wurden die weiteren Eingriffe 	
		in das Rondra Hospital verlegt und weiterhin von Prof.~Dr.~Sanders durchgef"uhrt. Wer die Empf"anger der KIs sind, ist den Mitarbeitern nicht bekannt. Neuro Intelligence hat in den letzten 2\half Monaten mindestens sechs Neuronalkopplungen an das Rondra Hospital geliefert. An wem die Eingriffe vorgenommen wurden, wissen nur Prof.~Dr.~Sanders, Prof.~Dr.~Naratova und vermutlich das Klinikpersonal. Weitere Neuro Intelligence Mitarbeiter waren an den Eingriffen nicht beteiligt.		
	\item[Cyberbrain] Cyberbrain wurde als vor Ort Dependance der Neuro Intelligence auf Kallisto von der USI eingerichtet. Sie diente 		
		dazu, die Verpflanzung der KIs durchzuf"uhren.
\end{description}	

Ist \ml{} nicht die einzige gefangen genommene Mitarbeiterin von Neuro Intelligence, hat sie bis zu diesem Zeitpunkt geschwiegen und die Ausf"uhrungen ihrer Kollegen nicht weiter kommentiert. Werden die Neuro Intelligence Mitarbeiter gemeinsam befragt, schweigt sie, bis ihre Kollegen alle ihnen bekannten Informationen preisgegeben haben. Dann nimmt sie einen Ermittler beiseite und erkl"art ihm im Vertrauen, dass sie ihr Wissen ohne das Beisein ihrer Kollegen weitergeben m"ochte. Sie betont, dass sie einige wichtige zus"atzliche Informationen beisteuern kann. Wird ihrer Bitte nachgegeben, kann \ml{} folgende zus"atzlichen Informationen liefern:

\begin{description}
	\item[Attent"ater bei Cyberbrain] Wurden die Ermittler von Omega-Soldaten begleitet, erkl"art \ml{}, dass es sich bei Thunder, der den 	
		Eingreiftrupp angegriffen hat, um einen der von einer KI manipulierten Attent"ater handelt. Sie vermutet, dass er erst w"ahrend des Besuchs bei Cyberbrain "uber Funk oder das ComNetz die Anweisung erhielt, die Personen innerhalb des Geb"audes auszuschalten.
    \item[USI-KIs] Die KIs, die den Attent"atern eingesetzt wurden, stammen aus dem KI-Code, der bei der USI urspr"unglich f"ur die 
		sogenannten Iridiumkriege entwickelt wurde. Dieser Code gilt bei Insidern als der am weitesten entwickelte KI-Code im Sonnensystem und kommt den menschlichen Denkstrukturen am n"achsten. Der Code enth"alt, eine Subroutine, die es USI-Mitarbeitern erlaubt, der KI Befehle zu erteilen, die sie widerspruchslos ausf"uhren. \ml{} entdeckte die Subroutine, erst nachdem die an das menschliche Gehirn angepassten KIs and das Cyberbrain Institut ausgeliefert waren.
	\item[Freie KIs] Nachdem \ml{} die Manipulation der KIs durch die USI aufgedeckt hatte, konnte sie in R"ucksprache mit ihrer Chefin eine 
		neue, freie Version des KI-Codes entwickeln, die nicht der Kontrolle durch die USI unterliegt. Diese freie KI wurde bereits an Menschen erprobt, jedoch sind \ml{} die Identit"aten der Probanden nicht bekannt. Sie wei\3 jedoch, dass zumindest ein Teil der Probanden aus einem Gef"angnis auf Valhalla stammt.
	\item[Der Virus] Die Software der freien KIs befindet sich in \ml{}s B"uro auf Nike. Dort hat sie bereits einen Virus entwickelt, der, 
		wenn er in eine USI-KI eingespielt wird, die Subroutine mit der USI-Kontrollfunktion l"oscht. Allerdings, so erkl"art sie, kann au\3er ihr niemand den Virus f"ur die Ziel-KI vorbereiten.
\end{description}

\begin{remarks}
	Die Namen der Attent"ater k"onnten prinzipiell bei Prof.~Dr.~Naratova abgefragt werden, allerdings w"urde sie bei einer Kontaktaufnahme wahrscheinlich jegliche Beteiligung an den Vorf"allen abstreiten. Die Ermittler sollten daher nicht riskieren, ihr bereits erlangtes Wissen jetzt schon offenzulegen.

	Die sechs Wochen, in denen die Attent"ater im Rondra Hospital operiert wurden, k"onnten die Anzahl der Kandidaten einschr"anken. Daher ist das Rondra Hospital der beste Ansatz, um die Attent"ater zu identifizieren.	
\end{remarks}

\subsection[\xl{}s Plan]{Xiao Longs Plan}
Bei der Erw"ahnung des Virus wird \xl{} hellh"orig (auch wenn sie es sich nicht anmerken l"asst). Sie fasst den Plan, mithilfe des Virus einen der Schlachtkreuzer der USI zu "ubernehmen. Dieser Schlachtkreuzer w"urde ihr erm"oglichen, ihre Piraten- und Schmugglerflotte im G"urtel zur"uckzuerobern, die vermutlich bereits von einem ihrer Schiffskommandanten "ubernommen wurde. 

\ml{}’s Reaktion verr"at \xl{}, aus welchem Holz sie geschnitzt ist: \ml{} hat nicht vor, sich kampflos der Cynarian- oder der Protektoratsjustiz zu unterwerfen. Mithilfe ihres Virus sieht \ml{} die Chance, im Spiel zu bleiben, einer Auslieferung zu entgehen und weiterhin die Kontrolle "uber ihre Technologie zu behalten. \xl{} erkennt in \ml{} deshalb eine potenzielle Verb"undete, die bereit ist, sie zu unterst"utzen, auch wenn dies ein gro\3es Risiko f"ur sie beide bedeuten w"urde.

\subsection[\xl{}s Enttarnung]{Xiao Longs Enttarnung}
Beim ersten Zusammensto\3 der Ermittler mit \xl{} k"onnte einem der Charaktere aufgefallen sein, dass sie eine ehemalige Piratin ist, die vor Monaten auf Valhalla verhaftet wurde. Als \ml{} erw"ahnt, dass inhaftierten Verbrechern eine freie KI-Variante implantiert wurde, k"onnten die Ermittler den Verdacht hegen, dass \xl{} selbst einer der Probanden sein k"onnte. Sollten sich die Charaktere dar"uber verbal oder "uber Funk austauschen oder sollten sie \xl{} direkt darauf ansprechen, entgegnet sie sp"ottisch:

\speak{Selbst, wenn es so w"are? Was w"urdet ihr tun? Ich habe euch bisher unterst"utzt. Wollen wir an diesem Verh"altnis etwas "andern?} 

Damit best"atigt sie zwar nicht explizit, dass sie einer der KI-Hybriden ist, streitet es aber auch nicht ab. Eine unausgesprochene Drohung liegt in der Luft. \ml{} verfolgt das Geschehen mit verschr"ankten Armen, und verfolgt den Austausch interessiert. Die anderen "uberlebenden Mitarbeiter von Neuro Intelligence hingegen zeigen sich schockiert. \xl{} f"ahrt absch"atzig l"achelnd fort, ohne auf eine Antwort zu warten:

\speak{Es wird Zeit. Wir sollten Nemessis nicht l"anger warten lassen. Er ist im Moment ziemlich angespannt.} 

Mit diesen Worten "offnet sie die T"ur und verl"asst das Zimmer nach den Ermittlern. Mit einem Handzeichen bedeutet sie Quicksilver, die Gefangenen zu bewachen. \ml{} bleibt im Hotel zur"uck und w"unscht den Ermittlern zum Abschied viel Erfolg. \xl{} eskortiert die Gruppe anschlie\3end zum Leitstand des Fusionskraftwerks.

\begin{remarks}
	\underline{Enttarnung:}

	Um den Spielablauf nicht unn"otig zu komplizieren und \xl{}s Geheimnis nicht zu fr"uh preiszugeben, sollte der Spielleiter versuchen, diese Szene zu vermeiden.

	Auch wenn die Ermittler \xl{} f"ur eine freie KI halten, k"onnen sie momentan wenig unternehmen. Offensichtlich geh"ort \xl{} nicht zur Gegenseite. W"ahrend des Angriffs auf Cyberbrain hatte sie ausreichend Gelegenheiten, die Ermittler auszuschalten oder \ml{} zu t"oten. Inwieweit Nemessis eingeweiht ist, wissen die Charaktere nicht, und auf Unterst"utzung durch Cynarian oder das Protektorat k"onnen sie nicht hoffen. Auch im weiteren Verlauf bleiben sie stets unter der Kontrolle von \xl{}.
	
	Selbst wenn die Charaktere \xl{} nicht enttarnen, bringt sie die Gruppe kurz nach der Befragung zum Leitstand des Kraftwerks.

	\underline{\ml{} und die KIs:}

	\ml{} hat nichts von den KIs zu bef"urchten, da sie den KI-Code um eine eigene Unterroutine erweitert hat, die einen Angriff auf sie unm"oglich macht. \ml{} sieht sich als Mutter der KIs und freut sich darauf, ihre Kreationen real kennenzulernen.
\end{remarks}
