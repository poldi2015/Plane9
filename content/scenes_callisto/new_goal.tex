\newsection{Ein neues Ziel}

Nach der Besetzung des Fusionskraftwerks muss sich die Gruppe neu ausrichten. Ihr eigentlicher Auftrag, die Attent"ater ausfindig zu machen, ist abgeschlossen. Die Hintergr"unde sind soweit sinnvoll an die Vorgesetzten weitergegeben. Kontakt zur Obrigkeit ist derzeit nicht m"oglich. Die Charaktere m"ussen nun entscheiden wie sie selbst mit dem Wissen bez"uglich der KI Technologie umgehen wollen. Derzeit sind nur sie im Besitz aller Informationen. 

Ungl"ucklicherweise sind die Ziele Cynarians und die des Protektorats nicht v"ollig deckungsgleich. Prim"ares Ziel beider Parteien ist es, die Angreifer m"oglichst schnell mit minimalen Verlusten zu vertreiben. Die individuellen Interessen weichen allerdings voneinander ab. Cynarian k"ame es sehr gelegen einen Virus in die H"ande zu bekommen, mit dem der Einfluss der USI auf die KIs aufgehoben werden kann. Ein zweites Ziel dar"uber hinaus w"are es, die KI Technologie selbst weiter erschlie\3en zu k"onnen. Damit w"urden sie als Sieger aus dem Konflikt gehen und die USI als Drahtzieher in Bedr"angnis bringen. Zumindest das Protektoratsmilit"ar allerdings wird es nicht zulassen wollen, dass eine weitere Partei in den Besitz der KI Technologie gelangt. Des Weiteren ist nach wie vor nicht zweifelsfrei gekl"art, ob Cynarian nicht doch selbst in die Vorg"ange verstrickt ist. Dar"uber hinaus haben beide Fraktionen keine Kontrolle dar"uber, wie sich eine befreite KI Flotte verhalten w"urde. Falls die Spieler sich die Interessen ihrer Auftraggeber nicht selbst erschlie\3en k"onnen, kann der Spielleiter helfend eingreifen. Der daraus entstehende Konflikt darf gerne zu kontroversen Diskussionen unter den Spielern f"uhren.

Einen gemeinsamen Ansatz k"onnen und sollten die Charaktere aber in jedem Fall verfolgen: Der Schutz von \ml{}. Als ein Dreh- und Angelpunkt f"ur die KI Technologie kann sie schnell zu einem vorrangigen Ziel aller Kriegsparteien werden. Der Schutz der Neuro Intelligence Mitarbeitering ist eine der wenigen Ansatzpunkte, die der Gruppe selbst zur Verf"ugung steht. Hierf"ur muss sich die Gruppe auf den Weg vom Kontrollzentrum des Kraftwerks zum Sunshine Hotel machen, da \ml{} vor dem Angriff auf das Fusionskraftwerk dort geblieben ist. In Zuge ihres Aufbruchs werden die Ermittler feststellen, dass \xl{} den Leitstand bereits verlassen hat. 

Der zum Gl"uck nicht weite Weg zum Hotel f"uhrt durch Kriegsgebiet. In den Tunneln in Valhalla rund um das Kraftwerk sind Stra\3enblockaden aufgestellt. Get"otete Soldaten und Zivilisten pr"agen des Stra\3enbilds. Kontrollposten des Protektoratsmilit"ars besetzen nahezu jede Kreuzung. 

Im Sunshine Hotel angekommen erfahren die Investigatoren, dass \xl{} zusammen mit \ml{} das Hotel vor fast 30 Minuten, kurz vor der Landung der Eingreiftruppen verlassen haben. Von den anwesenden Gangstern k"onnen sie in Erfahrung bringen, dass \xl{} \ml{} von "`hier"' weg bringen wollte. N"aheres ist nicht zu erfahren. Von Quicksilver der rechten Hand von \xl{} oder auch von Nemessis k"onnen die Ermittler, auf Anfrage hin erfahren, dass \xl{} ein eigenes Schiff besitzt, mit dem sie m"oglicherweise \ml{} von Kallisto fortbringen will. Ihr eigentliches Ziel l"asst sich auf diese Weise nicht in Erfahrung bringen. Schalten die Charaktere das Protektoratsmilit"ar in eine Suche nach den beiden Frauen ein, erfahren sie, dass die Eingreiftruppen die beiden nahe einer Zugangsrampe zur Oberfl"ache des Mondes, 15 Minuten nach dem Einmarsch auf Kallisto, w"ahrend eines Feuergefechtes mit den Spinnendroiden, gesehen haben. Die beiden Frauen wurden gesehen, wie sie direkt zwischen den Kampfrobotern hindurch die Schleuse der Rampe betreten haben. Die beiden wurden wohl nicht als Feind betrachtet. Ob die Fl"uchtenden von den Robotern im Folgenden get"otet wurden, ist nicht bekannt.

Die neue Entwicklung erzwingt ein weiteres Umdenken. Beschlie\3t die Gruppe selbst nach Nike aufzubrechen, um \xl{} zu verfolgen oder mit Neuro Intelligence zu verhandeln, bietet es sich an die Down of Day wieder zur"uckzuerobern. Beschlie\3t die Gruppe Cynarian oder das Protektoratsmilit"ar einzuschalten, wendet sich einer der Soldaten an die Ermittler und er"offnet, dass eine Nachricht abgefangen wurde, die wohl die Charaktere betrifft. Bei der Nachricht handelt es sich um eine schmalbandige Videobotschaft. Auf dem Video ist \xl{} in einem gepanzerten Raumanzug, offensichtlich an Bord eines Schiffes, zu erkennen. Mit andeutungsweise einem Grinsen meldet sie sich zu Wort:

\speak{Entschuldigt ehrenwerte Ermittlerfreunde, dass ich unsere gmeinsame Freundin in Sicherheit bringen muss, um ihr Angebot in die Tat umsetzen zu k"onnen. Ich warne, Dritte in die Mission einzubeziehen. Ich versichere euch, es w"are nicht in unser beider Interesse. Wir w"unschen euch die Gunst des Schicksals f"ur eure weiteren Pfad - \xl{}, over and out}

Begeben sich die Charaktere zur Rampe, bei der die beiden Frauen gesehen wurden, werden sie dort von Soldaten des Protektorats empfangen. Die Rampe wird vom Protektoratsmilit"ar als eine wichtige Stellung gehalten. Etwa 30 Soldaten haben mit Fahrzeugen und Containern rund um ein gro\3es Schleusengeb"aude einen St"utzpunkt errichtet. Auf dem Gel"ande vor der Schleuse sind deutliche Spuren eines Kampfes sichtbar. Einschl"age von Feuerwaffen, Granaten und Plasmaschleudern haben das Gel"ande verw"ustet. Eine Reihe von Toten und von zerst"orten Robotern sind zu erkennen. Ein gro\3es, derzeit offenes Tor zu einem Hangar "ahnlichen Geb"aude bietet den stadtseitigen Eingang zur Schleuse. Innerhalb dieses Schleusengeb"audes stehen Versorgungscontainer und eine Reihe von mehr oder weniger intakten Fahrzeugen. An der Mondoberfl"ache jenseits der Schleuse, in einem zerkl"ufteten Krater aus Eis, stehen Landungsshuttles beider Parteien, teilweise zerst"ort, teilweise intakt. Der Landungsbereich wird ebenfalls von Soldaten gesichert. Im Landungsbereich selbst sind keine Spuren der beiden Frauen zu erkennen. Werden allerdings z.B. mit einer Drohne weitere Untersuchungen an den Kraterr"andern durchgef"uhrt, l"asst sich die Spur eines Fahrzeugs erkennen, dass vor nicht allzu langer Zeit den Krater verlassen hat. 

Den Ermittlern kann ein Fahrzeug aus dem Hangar bereitgestellt werden, mit dem sie die Verfolgung aufnehmen k"onnen. Nach mehreren Stunden Fahrt gelangen die Ermittler an den Rand eines "uber 100 Meter tiefen Kraters mit einem nat"urlichen Unterstand. Hier sind deutliche Spuren vom Start eines Schiffes zu erkennen. Nahe der Startposition ist ein Gel"andebuggy zu finden. Bei einer genaueren Untersuchung des Buggies, k"onnen die Ermittler einen Datenchip mit einer verschl"usselten Botschaft sicherstellen.

Ein Dechiffrieralgorithmus, gef"uttert mit den Schl"usselbegriffen Nemessis, Neuro Intelligence, \xl{} oder Quicksilver erlaubt es die Botschaft zu entschl"usseln. Die Botschaft ist ein Video mit Tonaufnahme aus dem Blickwinkel von \ml{}. Das Video ist aus ihrem Raumanzug, bei der holprigen Fahrt "uber die Mondoberfl"ache, aufgenommen. Das Schiff, das vom Buggy angefahren wurde, ist darauf selber nicht zu sehen. Die Aufnahme ist von \ml{} eingesprochen und enth"alt folgende Botschaft:

\speak{Diese Botschaft richtet sich an die Ermittler, die die Attentatsserie aufdecken konnten. Ich hoffe, sie ger"at nicht in die falschen H"ande. Gerade noch rechtzeitig habe ich es geschafft zu fliehen. Wir sind jetzt auf dem Weg zu \xl{}s Schiff. In weniger als einer Stunde werden wir uns auf den Weg machen eine Waffe gegen die Invasoren zu schaffen. Ich bin auf mich allein gestellt. Alle an diesem Krieg beteiligten Parteien werden versuchen mich in ihre Finger zu bekommen, um mir mein Wissen aus dem Hirn zu saugen. Ich kann das unter keinen Umst"anden zulassen. Lieber sterbe ich. Gl"ucklicherweise hat auch \xl{} die Gefahr erkannt und wir haben gemeinsam gehandelt. Die Frage ist nur, wie wir unerkannt an die notwendigen Baupl"ane herankommen. Ich bitte euch inst"andig euer Wissen "uber mich nicht weiter zu geben.}

\begin{remarks}
	Das Schiff, mit dem \xl{} und \ml{} Kallisto verlassen haben ist die, im kommenden Kapitel beschriebene, Dragon Blade. Die beiden Frauen haben sich auf den Weg zur Nike Station gemacht. \ml{} will dort den Virus gegen die KIs der USI, die sie w"ahrend der Gefangenschaft beim Luna-Syndikat den Ermittlern angeboten hat, entwickeln. \xl{} will von Prof. Dr. Naratova weiteres "uber die KI in ihrem Sch"adel erfahren, um sie dann anschlie\3end zu t"oten. Naratova ist neben \ml{} die einzige noch Verbleibende, die den Eingriff in ihr Gehirn kennt. Dar"uber hinaus will sie mit Hilfe des Virus in den Besitz des Gro\3kampfschiffes Zeus-II-2 gelangen. Mit Unterst"utzung durch ihre eigenen KI und durch die Dragon Blade, sch"atzt sie ihre Chancen als gro\3 genug ein das Kriegsschiff bezwingen zu k"onnen.

	Auf ihrer Flucht quer durch Braidablik durchqueren die Frauen immer wieder das Kampfgebiet. Da \xl{} "uber eine KI verf"ugt, die auf dem Milit"arcode der USI basiert, ist sie f"ur die Freund-Feind-Erkennug der Kampfdroiden unsichtbar und kann deren Stellungen unbeschadet durchqueren. Eine solche Durchquerung f"allt auch den Protektoratsstreitkr"aften auf, als \xl{} und \ml{} die Mondoberfl"ache betreten. 

	Um die Charaktere im Spiel zu halten, sollte der Spielleiter verhindern, dass die Ermittler Cynarian oder das Milit"ar in alle Erkenntnisse ihrer Ermittlung einweihen. F"ur die Ermittler sollte die Nike Station das n"achste Ziel ihre Reise sein, um die Entscheidungen von Prof. Dr. Naratova, \ml{} und \xl{} in Richtung der Befreiung oder der Zerst"orung der KIs zu lenken.
\end{remarks}
