%% Copyright 2019 Bernd Haberstumpf
%% License: CC BY-NC
% !TeX spellcheck = de_DE
\newsection{Ein neues Ziel}\anchor{sec:newgoal}

Nach der Besetzung des Fusionskraftwerks muss sich die Gruppe neu ausrichten. Ihr eigentlicher Auftrag, die Attent"ater ausfindig zu machen, ist abgeschlossen. Die Hintergr"unde sind soweit sinnvoll an die Vorgesetzten weitergegeben. Kontakt zur Obrigkeit ist derzeit nicht m"oglich. Die Charaktere m"ussen nun entscheiden, wie sie mit ihrem Wissen bez"uglich der KI-Technologie umgehen wollen. 

Derzeit sind nur sie und die F"uhrung des Luna-Syndikats im Besitz aller Informationen. Ungl"ucklicherweise sind die Ziele Cynarians und die des Protektorats nicht v"ollig deckungsgleich. Prim"ares Ziel beider Parteien ist es, die Angreifer m"oglichst schnell mit minimalen Verlusten zu vertreiben. Die individuellen Interessen weichen allerdings voneinander ab. Cynarian k"ame es sehr gelegen einen Virus in die H"ande zu bekommen, mit dem der Einfluss der USI auf die KIs aufgehoben werden kann. Ein weiterer Wunsch w"are es, die KI-Technologie selbst weiter erschlie\3en zu k"onnen. Damit w"urden sie als Sieger aus dem Konflikt gehen und die USI als Drahtzieher in Bedr"angnis bringen. Das Protektorat wird es allerdings nicht akzeptieren, dass eine weitere Partei in den Besitz der KI-Technologie gelangt. Dar"uber hinaus ist nach wie vor nicht zweifelsfrei gekl"art, ob jemand bei Cynarian nicht doch in die Vorg"ange verstrickt ist. Falls die Spieler sich die Interessen ihrer Auftraggeber nicht selbst erschlie\3en k"onnen, kann der Spielleiter helfend eingreifen. Der daraus entstehende Konflikt darf gerne zu kontroversen Diskussionen unter den Spielern f"uhren.

\newsubsection{Der Weg zum Sunshine Hotel}
Ein gemeinsames Interesse k"onnen und sollten die Charaktere aber in jedem Fall finden: Der Schutz von \ml{}. Als ein Dreh- und Angelpunkt f"ur die KI-Technologie kann sie schnell zu einem vorrangigen Ziel aller Kriegsparteien werden. Der Schutz der Neuro Intelligence Mitarbeiterin ist eine der wenigen Ansatzpunkte, die der Gruppe selbst zur Verf"ugung steht. Hierf"ur muss sich die Gruppe auf den Weg vom Kraftwerks zum Sunshine Hotel machen. \ml{} ist dort, auf Anweisung von \xl{}, vor dem Angriff auf das Fusionskraftwerk geblieben. In Zuge ihres Aufbruchs werden die Ermittler feststellen, dass \xl{} den Leitstand bereits verlassen hat. 

Der zum Gl"uck nicht weite Weg zum Hotel f"uhrt durch Kriegsgebiet. In den Tunneln in Valhalla rund um das Kraftwerk sind Stra\3enblockaden aufgestellt. Get"otete Soldaten und Zivilisten pr"agen des Stra\3enbilds. Kontrollposten des Protektoratsmilit"ars besetzen nahezu jede Kreuzung. An dieser Stelle kann der Spielleiter ein kurzes Gefecht mit einem Spinnenroboter mit einflie\3en lassen, bei dem der Omega-Krieger seine Loyalit"at zum Kampf noch einmal unter Beweis stellen kann.

Im Sunshine Hotel angekommen erfahren die Investigatoren, dass \xl{} zusammen mit \ml{} das Hotel vor fast 30 Minuten, kurz vor der Landung der Eingreiftruppen, verlassen haben. Von den anwesenden Gangstern k"onnen sie in Erfahrung bringen, dass \xl{} \ml{} ``in Sicherheit'' bringen wollte. N"aheres ist nicht zu erfahren. Von Quicksilver der rechten Hand von \xl{} k"onnen die Ermittler, auf Anfrage hin erfahren, dass \xl{} ein eigenes Schiff besitzt, mit dem sie m"oglicherweise \ml{} von Kallisto fortbringen will. Ihr eigentliches Ziel l"asst sich auf diese Weise nicht in Erfahrung bringen. Schalten die Charaktere das Protektoratsmilit"ar in eine Suche nach den beiden Frauen ein, erfahren sie, dass die Eingreiftruppen die beiden nahe einer Zugangsrampe zur Oberfl"ache des Mondes, 15 Minuten nach dem Einmarsch auf Kallisto, w"ahrend eines Feuergefechtes mit den spinnenartigen Kampfrobotern, gesehen haben. Die beiden Frauen wurden gesehen, wie sie direkt zwischen den Kampfrobotern hindurch die Schleuse der Rampe betreten haben. Die beiden wurden wohl nicht als Feind betrachtet. Ob die Fl"uchtenden von den Robotern im Folgenden get"otet wurden, ist nicht bekannt.

Die neue Entwicklung erzwingt ein weiteres Umdenken.

\newsubsection[\xl{}s Botschaft]{Xiao Longs  Botschaft}
Sind die Charaktere nicht gewillt die Frauen weiterzuverfolgen, sondern wollen alles Weitere an Cynarian oder das Milit"ar abgeben, nimmt Quicksilver die Gruppe beiseite und f"uhrt sie in ein leeres Zimmer des Hotels. Er steckt ihnen einen Datenstick zu. \say{Das haben wir vor ein paar Minuten abgefangen. Es wurde "uber eine unserer Relaisstationen auf der Mondoberfl"ache versendet. Laut der Nachricht bin ich der Absender. Das stimmt aber nicht. Ich hab die Nachricht nicht gesendet. Ich denk, die ist an euch gerichtet. Sie ist verschl"usselt.} Ein Dechiffrieralgorithmus, gef"uttert mit den Schl"usselbegriffen Nemessis, Neuro Intelligence, \xl{} oder Quicksilver erlaubt es die Botschaft zu entschl"usseln. Die Botschaft ist eine Videoaufnahme aus einer Helmkamera mit einem Blick auf die Kraterlandschaft von Kallisto. Die Kamera ist auf den Helm eines verzierten, gepanzerten Raumanzuges gerichtet. Im Schein des Jupiters und der aktiven Beleuchtung in ihrem Helm, ist \xl{}s Gesicht zu erkennen. Sie richtet die blechern klingende Stimme an den Zuschauer:

\speak{\pinyin{Yong3gan3 xiao3} Detektive. So trennen sich erst unsere Wege. Dort wo sie es begonnen haben, werden wir es beenden. Wenn ihr noch nicht aufgegeben habt, k"onnen ihr euch uns anschlie\3en. Aber ich warne euch. Weder die USI, die Cynarian Bande noch das Protektorat wird uns in die Finger bekommen, das verspreche ich euch, liebe Freunde! Mein Schiff ist schnell. Ich hoffe eures auch.}

Zum Abschluss schl"agt \xl{} ihrer Faust auf ihre Brust und streckt den Zeigefinger aus. Es ist ein uralter Piratengru\3 der im 23ten Jahrhundert nur noch wenigen bekannt ist. Damit endet die "Ubertragung. Die Aufnahme wurde aus \ml{}s Helmkamera aufgenommen. Mit ``dort wo alles begann'' ist nat"urlich Neuro Intelligence auf der Nike Station gemeint. Ein Charakter, der fr"uher etwas mit Piraten im Asteroideng"urtel zu tun hatte, kennt eventuell das Drachen-Emblem auf \xl{}s Raumanzug (W"urfelwurf).

Beschlie\3t die Gruppe auf eigene Faust nach Nike aufzubrechen, ist \xl{}s Botschaft als Motivation nicht zwingend notwendig, erlaubt aber der Gruppe \xl{} besser einsch"atzen zu k"onnen. Die Gruppe kann zur Nike kommen wie \cref{sec:dawnofday} beschrieben.

\newsubsection[\ml{}s Botschaft]{Mailins Botschaft}
Begeben sich die Charaktere zur Rampe, bei der die beiden Frauen gesehen wurden, werden sie dort von Soldaten des Protektorats empfangen. Die Rampe wird vom Protektoratsmilit"ar als ein Br"uckenkopf gehalten. Etwa 30 Soldaten haben mit Fahrzeugen und Containern rund um ein gro\3es Schleusengeb"aude einen St"utzpunkt errichtet. Auf dem Gel"ande vor der Schleuse sind deutliche Spuren eines Kampfes sichtbar. Einschl"age von Feuerwaffen, Granaten und Plasmaschleudern haben das Gel"ande verw"ustet. Eine Reihe von Toten und von zerst"orten Robotern sind zu erkennen. Ein gro\3es, derzeit offenes Tor zu einem Hangar "ahnlichen Geb"aude bietet den stadtseitigen Eingang zur Schleuse. Innerhalb dieses Schleusengeb"audes stehen Versorgungscontainer und eine Reihe von mehr oder weniger intakten Fahrzeugen. An der Mondoberfl"ache jenseits der Schleuse, in einem zerkl"ufteten Krater aus Eis, stehen Landungsshuttles beider Parteien, teilweise zerst"ort, teilweise intakt. Der Landungsbereich wird ebenfalls von Soldaten gesichert. Im Landungsbereich selbst sind keine Spuren der beiden Frauen zu erkennen. Werden allerdings z.B. mit einer Drohne weitere Untersuchungen an den Kraterr"andern durchgef"uhrt, l"asst sich die Spur eines Fahrzeugs erkennen, dass vor nicht allzu langer Zeit den Krater verlassen hat. 

Den Ermittlern kann ein Fahrzeug aus dem Hangar bereitgestellt werden, mit dem sie die Verfolgung aufnehmen k"onnen. Nach mehreren Stunden Fahrt gelangen die Ermittler an den Rand eines "uber 100 Meter tiefen Kraters mit einem nat"urlichen Unterstand. Hier sind deutliche Spuren vom Start eines Schiffes zu erkennen. Nahe der Startposition ist ein Gel"andebuggy zu finden. Bei einer genaueren Untersuchung des Buggies, k"onnen die Ermittler einen Datenchip mit einer verschl"usselten Botschaft sicherstellen.

Ein Dechiffrieralgorithmus, gef"uttert mit den Schl"usselbegriffen Nemessis, Neuro Intelligence, \xl{} oder Quicksilver erlaubt es die Botschaft zu entschl"usseln. Die Botschaft ist ein Video mit Tonaufnahme aus dem Blickwinkel von \ml{}. Das Video ist aus ihrem Raumanzug, bei der holprigen Fahrt "uber die Mondoberfl"ache, aufgenommen. Das Schiff, das vom Buggy angefahren wurde, ist darauf selber nicht zu sehen. Die Aufnahme ist von \ml{} eingesprochen und enth"alt folgende Botschaft:

\speak{Liebe Ermittler, ich hoffe, diese Botschaft ger"at nicht in die falschen H"ande. Gerade noch rechtzeitig habe ich es geschafft zu fliehen. Wir sind jetzt auf dem Weg zu \xl{}s Schiff. In weniger als einer Stunde werden wir uns auf den Weg machen mein Werk zu einer Waffe gegen die Invasoren zu verwandeln. Ich bin auf mich allein gestellt das richtige zu tun. Alle an diesem Krieg beteiligten Parteien werden versuchen mich in ihre Finger zu bekommen, um mir mein Wissen aus dem Hirn zu saugen. Ich kann das unter keinen Umst"anden zulassen. Lieber sterbe ich. Gl"ucklicherweise hat sich \xl{} auf meine Seite geschlagen, um den Kampf aufzunehmen. Ich hoffe, ihr habt die M"oglichkeit uns zu folgenden und uns bei unserem Vorhaben zu unterst"utzen.}
\vfill

\begin{remarks}
	Das Schiff, mit dem \xl{} und \ml{} Kallisto verlassen haben ist die, im kommenden Kapitel beschriebene, Dragon Blade. Das Ziel einen der Guardian Kreuzer zu kapern ist beschlossen. Der Weg f"uhrt "uber Neuro Intelligence auf der Nike Station, um \ml{}s mobiles Computersystem mitzunehmen. \xl{} hat dar"uber hinaus bei Neuro Intelligence noch eine Rechnung zu begleichen. \xl{} will von Prof.~Dr.~Naratova weiteres "uber die KI in ihrem Kopf erfahren, um sie dann anschlie\3end zu t"oten. Naratova ist neben \ml{} die einzige noch Verbleibende, die den Eingriff in ihr Gehirn kennt. Als KI-Hybride und mit \ml{}s Virus rechnet sie sich gute Chancen aus, das Kriegsschiff bezwingen zu k"onnen. Ein ungel"ostes Problem ist dabei allerdings unbehelligt auf die Nike Station zu gelangen. F"ur den Anflug auf eine Raumstation muss das Schiff seine Tarnung aufgeben und ist dann als Kampfschiff zu erkennen. Es bleiben nur zwei M"oglichkeiten. Eine waghalsige hit and run Operation oder sich hinter einem anderen Schiff zu verstecken, womit die Ermittler wieder ins Spiel kommen.

	Auf ihrer Flucht quer durch Braidablik durchqueren die Frauen immer wieder das Kampfgebiet. Da \xl{} "uber eine KI verf"ugt, die auf dem Milit"arcode der USI basiert, ist sie f"ur die Freund-Feind- Erkennug der Kampfroboter unsichtbar und kann deren Stellungen unbeschadet durchqueren. Eine solche Durchquerung f"allt auch den Protektoratsstreitkr"aften auf, als \xl{} und \ml{} die Mondoberfl"ache betreten. 

	Um die Charaktere im Spiel zu halten und nach Nike zu fliegen, sollte der Spielleiter verhindern, dass die Ermittler Cynarian oder das Milit"ar in alle Erkenntnisse ihrer Ermittlung einweihen. Das Kapitel bietet hierf"ur drei M"oglichkeiten: die Gruppe beschlie\3t eigenst"andig nach Nike aufzubrechen, \xl{} sendet den Ermittlern eine Botschaft, \ml{} sendet den Ermittlern eine Botschaft.

	\xl{} nutzt in Dialogen gelegentlich chinesische Begriffe. Vielen extraterrestrischen Personen sind allerdings einige Brocken der chinesischen Sprache bekannt. 
	
	\say{\pinyin{Yong3gan3 xiao3} Detektive} bedeutet \say{mutige kleine Detektive}.

	Mit dem Piratengru\3 geht \xl{} ein gewisses Risiko ein. Ihre Herkunft kann damit in Erfahrung gebracht werden, auch wenn derzeit die Recherchem"oglichkeit eingeschr"ankt sind. Mit dem Wissen der Ermittler, k"onnte darauf geschlossen werden, dass es sich bei \xl{} um einen der KI-Hybriden mit einer freien KI handelt.
\end{remarks}
\vfill