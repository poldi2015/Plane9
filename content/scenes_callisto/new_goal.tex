\newsection{Ein neues Ziel}

Nach der Besetzung des Fusionskraftwerks muss sich die Gruppe neu ausrichten. Ihr eigentlicher Auftrag die Attent"ater ausfindig zu machen ist abgeschlossen. Die Hintergr"unde sind soweit sinnvoll an die Vorgesetzten weiter gegeben. Kontakt zur Obrigkeit ist derzeit nicht m"oglich. Prim"are Ziel beider Parteien, Cynarian und des Protektorats, ist es die Angreifer m"oglichst schnell mit minimalen Verlusten zu besiegen. Die individuellen Interessen weichen allerdings voneinander ab. Cynarian k"ame es sehr gelegen einen Virus in die H"ande zu bekommen, mit dem der USI Einfluss auf die KIs aufgehoben werden und dar"uber hinaus die KI Technologie erschlie\3en zu k"onnen. Damit w"urden sie als Sieger aus dem Konflikt gehen und die USI als Drahtzieher in Bedr"angnis bringen. Zumindest das Protektoratsmilit"ar allerdings wird es nicht zulassen wollen, dass eine weitere Partei in den Besitz der KI Technologie gelangt. Des Weiteren ist nach wie vor nicht klar ersichtlich ob Cynarian nicht doch selbst in die Vorg"ange verstrickt ist. Dar"uber hinaus haben beide Fraktionen keine Kontrolle dar"uber wie sich eine befreite KI Flotte verhalten w"urde.

Ein Ansatz k"onnen die Charaktere aber in jedem Fall gemeinsam Anstreben: der Schutz von \ml{} als Dreh und Angelpunkt f"ur die Beseitigung der angreifenden KI Truppen. Hierf"ur muss sich die Gruppe auf den Weg von Kontrollzentrum des Kraftwerks zum Sunshine Hotel machen. In diesem Zuge werden sie feststellen, dass \xl{} den Leitstand bereits verlassen hat. Der zum Gl"uck nicht weite Weg zum Hotel f"uhrt durch Kriegsgebiet. In den Tunneln durch Valhalla rund um das Kraftwerk sind Stra\3enblockaden aufgestellt. Get"otete Soldaten und Zivilisten pr"agen des Stra\3enbilds. Kontrollposten des Protektoratsmilit"ars besetzen nahezu jede Kreuzung.

Im Sunshine Hotel angekommen erfahren die Investigatoren, dass \xl{} zusammen mit \ml{} das Hotel vor fast 30 Minuten kurz vor der Landung der Eingreiftruppen verlassen haben. Von den anwesenden Gangstern k"onnen sie in Erfahrung bringen, dass \xl{} \ml{} von "`hier"' weg bringen wollte. N"aheres ist nicht bekannt. Von Quicksilver der rechten Hand von \ml{} oder von Nemessis k"onnen die Ermittler auf Anfrage hin erfahren, dass \xl{} ein eigenes Schiff besitzt, mit dem sie m"oglicherweise \ml{} von Kallisto fortbringen w"urde. Ihr eigentliches Ziel l"asst sich auf diese Weise nicht in Erfahrung bringen. Schalten die Charaktere das Protektoratsmilit"ar in eine Suche nach den beiden Frauen ein erfahren sie, dass die Eingreiftruppen die beiden nahe einer Zugangsrampe zur Oberfl"ache des Mondes 15 Minuten nach dem Einmarsch auf Kallisto w"ahrend eines Feuergefechtes mit den Spinnendroiden gesehen haben. Die beiden Frauen wurden gesehen wie sie direkt zwischen den Kampfrobotern hindurch die Schleuse der Rampe betreten haben. Die beiden wurden wohl zun"achst nicht als Feind betrachtet. Ob die Fl"uchtenden von den Androiden sp"ater get"otet wurden ist nicht bekannt.

Die neue Entwicklung erzwingt ein weiteres Umdenken. Beschlie\3t die Gruppe selbst nach Kallisto aufzubrechen, um \xl{} vermeintlich zu verfolgen oder mit Neuro Intelligence zu verhandeln, bietet es sich an die Down of Day wieder zur"uckzuerobern. Beschlie\3t die Gruppe Cynarian oder das Protektoratsmilit"ar einzuschalten wendet sich einer der Soldaten an die Ermittler und er"offnet, dass eine Nachricht abgefangen wurde, die wohl die Charaktere betrifft. Bei der Nachricht handelt es sich um eine schmalbandige Videobotschaft. Auf dem Video ist \xl{} in einem gepanzerten Raumanzug offensichtlich an Bord eines Schiffes zu erkennen. Mit andeutungsweise einem Grinsen meldet sie sich zu Wort:

\speak{Entschuldigt ehrenwerte Ermittler Freunde, dass ich unsere Freundin in Sicherheit bringen muss um ihr Angebot in die Tat umsetzen zu k"onnen. Ich warne Dritte in die Mission einzubeziehen. Ich versichere euch, es w"are nicht in unser beiderseitigem Interesse. Ich w"unsche euch viel Erfolg auf eurem weiteren Weg -- \xl{} over and out}

Begeben sich die Charaktere zur Rampe bei der die beiden Frauen gesehen wurden, werden sie dort von Soldaten des Protektorats empfangen. Die Rampe wird vom Protektoratsmilit"ar als eine wichtige Stellung gehalten. Etwa 30 Soldaten haben mit Fahrzeugen und Containern rund um ein gro\3es Schleusengeb"aude einen St"utzpunkt errichtet. Auf dem Gel"ande vor der Schleuse sind deutliche Spuren eines Kampfes sichtbar. Einschl"age von Feuerwaffen, Granaten und Plasmaschleudern haben das Gel"ande verw"ustet. Eine Reihe von Toten und zerst"orten Robotern sind zu erkennen. Ein gro\3es, derzeit offenes Tor zu einem Hangar "ahnlichen Geb"aude bietet den stadtseitigen Eingang zur Schleuse. Innerhalb dieses Schleusengeb"audes stehen Versorgungscontainer und eine Reihe mehr oder weniger intakte Fahrzeuge. An der Mondoberfl"ache jenseits der Schleuse in einem zerkl"ufteten Krater aus Eis stehen Landungsshuttles beider Parteien, teilweise zerst"ort, teilweise intakt. Der Landungsbereich wird ebenfalls von Soldaten gesichert. Im direkten Landungsbereich sind keine Spuren der beiden Frauen zu erkennen. Werden allerdings, z.B. mit einer Drohne weitere Nachforschungen an den Kraterr"andern durchgef"uhrt l"asst sich die Spur eines Fahrzeugs erkennen, dass vor nicht allzu langer Zeit den Bereich um das Anfluggebiet verlassen hat. Den Ermittlern kann ein Fahrzeug bereitgestellt werden, mit dem sie die Verfolgung aufnehmen k"onnen. Nach mehreren Stunden Fahrt gelangen die Ermittler an den Rand eines "uber 100 Meter tiefen Kraters mit einem nat"urlichen Unterstand. Hier sind deutliche Spuren vom Start eines Schiffes zu erkennen. Nahe der Startposition ist ein zweisitziges Buggy zu finden. Bei einer genaueren Untersuchungen des Buggies k"onnen die Ermittler einen Datenchip mit einer verschl"usselten Botschaft sicherstellen.

Ein Dechiffrieralgorithmus gespeist mit den Schl"usselbegriffen Nemessis, Neuro Intelligence, \xl{} oder Quicksilver erlaubt es die Botschaft zu entschl"usseln. Die Botschaft ist ein Video und Tonaufnahme aus dem Blickwinkel von \ml{} aufgenommen aus ihrem Raumanzug bei der holprigen Fahrt zur Landestelle der Dragon Blade. Das Schiff selber ist darauf nicht zu sehen. Die Aufnahme ist von \ml{} selbst eingesprochen und enth"alt folgende Botschaft:

\speak{Diese Botschaft richtet sich an die Ermittler, die die Attentatsserie aufdecken konnten. Ich hoffe, sie ger"at nicht in die falschen H"ande. Gerade noch rechtzeitig habe ich es geschafft Kallisto zu verlassen. \xl{} und ich sind jetzt auf dem Weg zu ihrem Schiff. In weniger als einer Stunde werden wir uns auf den Weg machen eine Waffe gegen die Invasoren zu entwerfen. Ich bin auf mich allein gestellt. Alle an diesem Krieg beteiligten Parteien werden versuchen mich in ihre Finger zu bekommen und mir mein Wissen aus dem Hirn zu saugen. Ich kann das unter keinen Umst"anden zu lassen. Lieber sterbe ich. Gl"ucklicherweise hat \xl{} unsere Chance und die Gefahr verstanden und wir haben gehandelt. Die Frage ist nur wie wir unerkannt an die notwendige Baupl"ane herankommen werden. Mir bleibt nur auf die Diskretion der Ermittler als Wissenstr"ager zu hoffen.}

\begin{remarks}
	Um die Charaktere im Spiel zu halten, sollte der Spielleiter verhindern, dass die Ermittler Cynarian oder das Milit"ar in alle Details einzuweihen. F"ur die Ermittler sollte die Nike Station das n"achste Ziel ihre Reise sein um die Entscheidungen von Prof.~Dr.~Naratova, \ml{} und \xl{} in Richtung der Befreiung oder der Zerst"orung der KIs zu lenken. Sie sollten versuchen, die Forschungsergebnisse der Neuro Intelligence zu vernichten, um weiteres Unheil zu verhindern.
\end{remarks}
