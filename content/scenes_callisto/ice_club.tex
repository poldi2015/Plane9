%% Copyright 2019 Bernd Haberstumpf
%% License: CC BY-NC
% !TeX spellcheck = de_DE

\pageimage{images/ice_club.jpg}

\newsection{Im Ice Club}\anchor{sec:ice_club}

Der Ice Club ist ein Nachtclub und Bordell, das neben dem Blackhole Club ebenfalls durch das Luna-Syndikat kontrolliert wird, aber weitgehende Autonomie besitzt. Das Bordell geh"ort einer Sonja Ice. 

Durch die Einladung von Fleur Soleil, Codewort Solar Eclipse, wird dem Ermittler, der die Visitenkarte erhalten hat, Einlass gew"ahrt. Weitere Ermittler k"onnen, wenn sie wollen, "uber \xl{} Zugang zum Nachtclub erhalten. Im Club sind keine Waffen erlaubt. F"ur den Eintritt ist Abendgarderobe erforderlich; der Nachtclub kann diese aber auch bereitstellen. Der Eingangsbereich besitzt Zug"ange zu einer ausgedehnten Garderobe, dem Clubraum und in die obere Etage. Die W"ande des Clubraums bestehen aus konserviertem Eis, das reich dekoriert in einer geschwungenen Decke endet. Auf der vom Eingang aus linken Seite befindet sich eine B"uhne, ihr gegen"uber die Bar. Vor der B"uhne sind kleine Tische mit St"uhlen um einen Catwalk aufgereiht. Im hinteren Bereich befinden sich S\'epar\'ees. Versteckt, kunstvoll in das Eis eingelassen, f"uhrt eine Treppe nach oben zu einzelnen Zimmern und nach unten in einen Saunabereich.

\xl{} bringt die Ermittler mit einem Gel"andewagen zum Nachtclub und parkt das Fahrzeug direkt vor dem Eingang, wenn die Gruppe keine Einw"ande erhebt.

Wenn die Charaktere das Bordell betreten, steht Carina als Fleur Soleil gerade auf der B"uhne und erfreut die G"aste mit ihrem Gesang. Bei ihrem Auftritt ist sie in ein hautenges wei\3es Kleid geh"ullt, das au\3er den Paillettenstickereien nahezu durchsichtig ist. Zu dem Kleid tr"agt sie platinblonde Haare mit eingewobenen leuchtenden Kristallen. Die Charaktere haben w"ahrend der Darbietung gen"ugend Zeit, den Clubraum zu inspizieren und ihr weiteres Vorgehen zu beraten.

Am Ende ihres Sets fordert Carina mit einer versteckten Geste den Charakter auf, mit dem sie im Blackhole Club gesprochen hat, sie auf ihr Zimmer zu begleiten. Andere G"aste vertr"ostet sie auf sp"ater oder ein anderes Mal.

Von Carina erfahren die Ermittler, dass sie von zwei M"annern beauftragt wurde, nach Interessenten f"ur Cyberware Ausschau zu halten. Einem dieser M"anner ist der Ermittler bereits im Blackhole Club begegnet, sagt sie, als er sich an ihren Tisch setzte. Er nennt sich Dan Ringdaz. Bevor sie weitere Informationen offenlegt, erkl"art sie, dass sie von den M"annern verfolgt und angegriffen wurde und deshalb um ihr Leben f"urchten muss. Seit ihrem Treffen im Blackhole Club verschanzt sie sich im Ice Club. Aus diesem Grund bittet sie ihren Gast, sie umgehend an einen sicheren Ort zu bringen. Sagt der Charakter ihr zu, wechselt sie in seinem Beisein die Garderobe und macht sich ausgehfertig, die Haare in passender Farbe zu ihrer Kleidung. Sie schaut nach wie vor umwerfend aus. Die "ubergeworfene Kapuze macht sie nur bedingt unauff"allig. Zum gegenw"artigen Zeitpunkt will sie keine weiteren Informationen preisgeben, sagt sie entschlossen. Alles Weitere wird sie offenlegen, wenn sie an einem sicheren Ort untergekommen ist.

Beim Verlassen des Zimmers z"ogert sie, bleibt noch einmal stehen, knabbert an ihrer Unterlippe und schaut dem Ermittler mit einem verunsicherten Blick tief in die Augen.

\speak{Ihr werdet mich doch nicht an die Konzernhaie oder das Protektorat ausliefern, oder? Es war eine ganz normale Kontaktvermittlung. Slingshot und der andere wollten etwas. Die Anzugtr"ager konnten es anbieten. Beide Parteien waren einverstanden. Was dann passiert ist, konnte ich nicht ahnen und verstehe es immer noch nicht \dots Ihr bringt mich in Sicherheit. Ich erz"ahle euch alles, was ich wei\3. Ihr lasst den Kerlen ihre gerechte Strafe zukommen. Ich kann euch auch das Geld, also den "ubrig gebliebenen Teil daf"ur auszahlen. Ich hatte Slingshot gerne. Er war wirklich lustig und irgendwie s"u\3, verstehst du. Was passiert ist, hat er nicht verdient.}

Nach dem Verlassen des Clubs wird Carina in Begleitung der Gruppe angegriffen. Kurz nach dem Losfahren wird \xl{}s Fahrzeug mit einer Granate beschossen. \xl{} kann dem direkten Einschlag des Geschosses gerade noch ausweichen. Die Druckwelle fegt auf dem Boden stehende Gegenst"ande beiseite. Sie bringt den trotz allem besch"adigten Wagen zum Stehen. Niemand ist schwer verletzt, aber das Fahrzeug ist nicht mehr voll fahrtauglich. 

Die Angreifer, in ihrer Deckung, werden angef"uhrt von \emph{Frederic Johnson}, dem zweiten USI-Kontaktmann von Carina. Frederic Johnson ist in Begleitung von zwei S"oldnern, \emph{Lazor} und \emph{Flinn}. Lazor h"alt eine Railgun im Anschlag, w"ahrend Flinn seinen Granatwerfer beiseite wirft und ebenfalls zu einer Railgun greift. Frederic Johnson h"alt eine Bolzenpistole in der Hand. 

\xl{} "offnet eine Kiste mit Waffen -- drei Bolzenpistolen, zwei Railguns und zwei Granaten -- zwischen den Sitzen. Sie st"o\3t die T"uren an der den Angreifern abgewandten Seite auf. \say{Raus!}, ruft sie. Es kommt zu einer Schie\3erei. Durch einen Omega-Krieger und \xl{} ist die Gruppe den Angreifern "uberlegen. Die Agenten werden daher nach einem kurzen Schusswechsel versuchen, zu fl"uchten.

Sind die Angreifer vertreiben oder erledigt, kehrt die Gruppe zun"achst in den nur eine Tunnelbiegung entfernten Ice Club zur"uck und wird von dort aus von anderen Gangstern des Syndikats zusammen mit \xl{} zum Sunshine Hotel gebracht. Von Carina erfahren die Ermittler dort, dass einer der Angreifer, Frederic Johnson, der zweite Mann war, der sie angeheuert und die Deals mit Hannibal und Slingshot abgewickelt hat. Im Blackhole Club hat sie Dan Ringdaz und seinem Partner die Kontakte zu Hannibal und Slingshot vermittelt. Bei einem der Treffen hat sie zuf"allig den Namen der Forschungseinrichtung \emph{Cyberbrain} aufgeschnappt, in der Slingshot und Hannibal mit neuer Cyberware ausgestattet werden sollten. Nach den Treffen zwischen Hannibal beziehungsweise Slingshot und ihren Auftraggebern hat sie die beiden nicht mehr gesehen.

\begin{remarks}
	\underline{Gewonnene Information:}
	
	\begin{itemize}
		\item Das Forschungseinrichtung Cyberbrain in der Zone hat die KIs bei Hannibal und Slingshot implantiert.
		\item Die Namen der USI-Agenten: Dan Ringdaz. Frederic Johnson und gegebenenfalls Smith-Singer
	\end{itemize}

	\underline{Carina und das Luna-Syndikat:}

	Egal, welche Bedeutung die Ermittler Carinas Unschuldsbeteuerungen geben, wird \xl{} daf"ur sorgen, dass die Gruppe nach der Auseinandersetzung vor dem Ice Club wieder zum Luna-Syndikat und damit unter ihre Kontrolle zur"uckkehrt.
\end{remarks}

\begin{remarks}
	\underline{Frederic Johnson:}

	Kann Frederic Johnson gefangen genommen werden, l"asst sich in Erfahrung bringen, dass er im Auftrag eines Dritten arbeitet, dessen Namen er nicht preisgeben will. Zusammen mit Dan Ringdaz hatte er den Auftrag, Interessenten f"ur Cyberware anzusprechen, Konditionen zu kl"aren und ein Treffen am Raumhafen zu veranlassen, von wo aus sie dann von ihrem Auftraggeber zusammen mit anderen Personen abgeholt wurden. Wird Frederic Johnson einem Gehirnscan unterzogen, wird er versuchen, den eingreifenden Psychonauten mit seinen eigenen psychonautischen F"ahigkeiten durch falsche Erinnerungen abzuwehren, ohne seine F"ahigkeiten aufzudecken. Gelingt dem Spieler der psychonautische Angriff, kann er den Namen Smith-Singer als Auftraggeber der Agenten in Erfahrung bringen. Er erf"ahrt auch, dass die Gehirnmanipulation in der Cyberbrain-Einrichtung in der Zone durchgef"uhrt wurde. Au\3erdem l"asst sich in Erfahrung bringen, dass die Agenten im Auftrag der USI arbeiten.	

	\underline{Cyberbrain:}

	Wo die Cyberbrain-Forschungseinrichtung zu finden ist, l"asst sich durch eine Anfrage bei Cynarian herausfinden. Cyberbrain geh"ort zu einer Reihe von kleinen Forschungseinrichtungen in der Zone, deren Aufgaben unter Verschluss stehen und auch Cynarian nicht bekannt sind.
\end{remarks}
