%% Copyright 2019 Bernd Haberstumpf
%% License: CC BY-NC
% !TeX spellcheck = de_DE
\subsection{Die Dragon Blade}

Die Dragon Blade ist die Kaperf"ahre von \xl{}, mit der sie aus dem G"urtel nach Kallisto gekommen ist. Sie ist ein Tarnschiff, in Gr"o\3e, Ausstattung und Bewaffnung einer Korvette nicht un"ahnlich. Im Gegensatz zu einer Korvette besitzt sie einen Bereich f"ur ein Enterkommando, gr"o\3ere Lagerr"aume und Einrichtungen, um leichter an einem anderen Schiff festzumachen und zu entern. Als Bewaffnung besitzt sie Railguns als Kurzstreckenbewaffnung, Tark"orper und einen Torpedoschacht. 

Vor der Flucht von \xl{} und \ml{} zur Nike Station, liegt die Dragon Blade versteckt in einem Mondkrater auf Kallisto und ist erst nach der "Uberquerung des Kraterrandes zu erkennen. \xl{} nutzt die Dragon Blade, wie im letzten Kapitel beschrieben, um zusammen mit \ml{} nach Nike zu gelangen. Nach dem Betreten der F"ahre bringt \xl{} das Schiff mit einem kurzen Schubman"over in eine Flugbahn, die es mittels der schwachen Gravitation des Mondes antriebslos um den Mond f"uhrt. Erst auf der Valhalla abgewendeten Seite startet sie das Haupttriebwerk und bringt die Dragon Blade auf Kurs nach Nike. 

\subsection{Die Dawn of Day}
Die Dawn of Day ist die erste Wahl, wenn die Gruppe die Verfolgung von \xl{} aufnehmen will. Im Zweifel kann der Spielleiter aber auch ein anderes Schiff anbieten wenn, die Charaktere es schaffen Kontakt mit Cynarian oder dem Protektorat aufzunehmen. In den weiteren Beschreibungen wird die Dawn of Day stellvertretend als das von den Ermittlern gew"ahlte Schiff genutzt.

Hat Blackheart noch keine Besetzung Valhallas befohlen und ist es noch nicht zum Attentat auf dem Gipfeltreffen gekommen ist es schwer f"ur die Gruppe die Dawn of Day zu erreichen und zu starten. Die Oberstadt ist durchzogen von Kontrollposten der Zeus II-1. Die Dawn of Day selbst steht unter Beobachtung. Sollten die Spieler zu so einem Zeitpunkt versuchen in ihr Shuttle zu gelangen und unerkannt zu starten m"ussen sie dem Spielleiter einen sehr guten Plan vorlegen.

Ist auf Valhalla allerdings der Stra\3enkampf zwischen dem Protektorate und den Konzerntruppen der Zeus II-1 ausgebrochen kann die Gruppe im allgemeinen Trubel versuchen in ihr Schiff zu gelangen und nach Nike aufzubrechen. Auf dem Gel"ande des Raumhafens gibt es an allen Ecken Feuergefechte zwischen dem Protektorat und den spinnenbeinigen Kampfdroiden des schweren Kreuzers. In diesem Zusammenhang sind die Ermittler als gejagte Terroristen nicht mehr interessant und  m"ussen keine sofortige Festnahme bef"urchten. Stattdessen besteht allerdings die Gefahr in die Kampfhandlungen selbst involviert zu werden. Ab dem Beginn der K"ampfe ist durch St"orsender das gesamte ComNetz lahmgelegt. Der komplette Informationsfluss aus den Medien ist abgebrochen und Kommunikation ist nur noch "uber individuellen Kurzstreckenfunk m"oglich.

Vor dem Betreten des Raumhafens sollten die Charaktere sicherstellen, dass das Shuttle startklar ist. Der beste Kontakt daf"ur ist Sonja Frost. "Uber ein Terminal im Garnisonsst"utzpunkt oder im Raumhafen selbst ist sie erreichbar und ist auch bereit das Shuttle startklar machen zu lassen.

Der Spielleiter sollte die Spieler das Spie\3rutenlaufen zwischen den Kampfparteien in Teilen ausspielen lassen. Auf  dem Weg zum Shuttle k"onnen die Charaktere Commander Lockhead um Unterst"utzung durch Kampftruppen bitten. Die Truppen der Garnison f"uhren in der Oberstadt den Kampf mit den Konzerntruppen. Commander Lockhead wird f"ur die Unterst"utzung einen alten Bekannten, seinen Adjutanten Firedon bereitstellen.

Nach dem Abflug aus dem Raumhafen bietet sich ein erschreckendes Bild. J"ager der Martell liefern sich ein Gefecht mit KI J"agern der Zeus II-1 "uber der Stadt. Nahkampfgesch"utze der gro\3en Schlachtschiffe belegen J"ager und Torpedos mit einem Kugelteppich. Das Shuttle der Ermittler wird kurz nach dem Start von beiden Kampfparteien gescant. Kann sich das 
Shuttle als Zugeh"orig zur Cynarian Corporation ausweisen und ist mit einer unauff"alligen Schiffskennung getarnt k"onnen 
die Ermittler unbehelligt am Kampfgebiet vorbeifliegen. Fliegt die Dawn of Day unter ihrer wahren Kennung, die der USI bereits bekannt ist, wird das Schiff der Ermittler von J"agern der Zeus II-1 angegriffen und muss sich mit Unterst"utzung von J"agern der Martell verteidigen.

\xl{} wird den Ermittlern mit ihrem eigenen Schiff unerkannt folgen.

\subsection{Schiffskennung und Legitimation}
Um mit dem Schiff nicht nach dem Start und beim Anflug der Nike Station unter Beschuss zu geraten bietet es sich an dem Schiff eine unverd"achtige Schiffskennung zu geben. Eine Schiffskennung der Cynarian Corporation vereinfacht vor allem das Andocken an die Nike Station. Eine Schiffskennung kann die Cynarian Corporation "uber ihren Agenten Mr.~Klark bereitstellen. Allerdings m"ussen die Ermittler als Gegenleistung auch einen Grund f"ur den Flug und ein Flugziel nennen. Um den Besuch der Neuro Intelligence auf Nike zu verschleiern, muss eine triftige Ausrede vorgelegt werden. Ein m"oglicher vorgeschobener Grund w"are die sichergestellten Daten aus der Risikozone Valhalla zu bringen. In diesem Zusammenhang k"onnen die Charaktere ihr Darlegung mit einem guten W"urfelwurf auf ihre empathischen F"ahigkeiten untermauern. Eine andere M"oglichkeit w"are es um eine Transporterkennung das Luna-Syndikat zu bitten.
