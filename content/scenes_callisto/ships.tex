%% Copyright 2019 Bernd Haberstumpf
%% License: CC BY-NC
% !TeX spellcheck = de_DE
\newsubsection{Die Dragon Blade}

Die Dragon Blade ist die Kaperf"ahre von \xl{}, mit der sie aus dem Asteroideng"urtel nach Kallisto gekommen ist. Sie ist ein Tarnschiff, das in Ausstattung und Bewaffnung einer Korvette "ahnelt, jedoch kleiner ist. Mit einer Gesamtl"ange von 50 Metern ist sie etwa doppelt so gro\3 wie ein Shuttle. Im Gegensatz zu einer Korvette besitzt sie keinen Shuttlehangar, jedoch einen Raum f"ur ein Enterkommando. Ausfahrbare Krallen erm"oglichen es, sich leichter an einem anderen Schiff festzumachen und es zu entern. Zur Bewaffnung geh"oren Railguns, Tarnk"orper und ein Torpedoschacht.

Vor der Flucht zur Nike Station liegt die Dragon Blade versteckt in einem Mondkrater auf Kallisto und ist erst nach der "Uberquerung des Kraterrandes sichtbar. \xl{} nutzt die Dragon Blade, wie im letzten Kapitel beschrieben, um zusammen mit \ml{} nach Nike zu gelangen.

Nach dem Betreten der F"ahre bringt \xl{} das Schiff mit einem kurzen Schubman"over in eine Flugbahn, die es, durch die schwache Gravitation des Mondes unterst"utzt, antriebslos um den Mond f"uhrt. Erst auf der von Valhalla abgewandten Seite startet sie das Haupttriebwerk und bringt die Dragon Blade auf Kurs nach Nike.

\newsubsection{Die Dawn of Day}\anchor{sec:dawnofday}
Die Dawn of Day ist die erste Wahl, wenn die Gruppe die Verfolgung von \xl{} aufnehmen m"ochte. Im Zweifelsfall kann der Spielleiter jedoch auch ein anderes Schiff anbieten, falls die Charaktere Kontakt mit Cynarian oder dem Protektorat aufnehmen k"onnen. In den folgenden Beschreibungen wird die Dawn of Day stellvertretend als das von den Ermittlern gew"ahlte Schiff verwendet.

Falls Blackheart noch keine Besetzung von Valhalla befohlen hat und das Attentat auf der Konferenz noch nicht stattgefunden hat, wird es f"ur die Gruppe schwierig, die Dawn of Day zu erreichen und zu starten. Die Oberstadt ist von Kontrollposten der Zeus II-1 durchzogen, und die Dawn of Day selbst steht unter Beobachtung. Sollten die Spieler zu diesem Zeitpunkt versuchen, in ihr Shuttle zu gelangen und unerkannt zu starten, m"ussen sie dem Spielleiter einen sehr guten Plan vorlegen.

Hat jedoch der Stra\3enkampf zwischen dem Protektorat und den Konzerntruppen der Zeus II-1 auf Valhalla bereits begonnen, kann die Gruppe im allgemeinen Trubel versuchen, in ihr Schiff zu gelangen und nach Nike aufzubrechen. Auf dem Gel"ande des Raumhafens gibt es an allen Ecken Feuergefechte zwischen dem Protektorat und den spinnenbeinigen Kampfrobotern des schweren Kreuzers. In diesem Chaos sind die Ermittler als gejagte Terroristen nicht mehr von Interesse und m"ussen keine sofortige Festnahme bef"urchten. Stattdessen besteht jedoch die Gefahr, selbst in die Kampfhandlungen verwickelt zu werden.

Ab dem Beginn der K"ampfe ist durch St"orsender das gesamte ComNetz lahmgelegt. Der komplette Informationsfluss aus den Medien ist abgebrochen, und Kommunikation ist nur noch "uber individuellen Kurzstreckenfunk m"oglich.

Vor dem Betreten des Raumhafens sollten die Charaktere sicherstellen, dass das Shuttle startklar ist. Der beste Kontakt daf"ur ist Sonja Frost. "Uber ein Terminal im Garnisonsst"utzpunkt oder im Raumhafen selbst ist sie erreichbar und bereit, das Shuttle startklar machen zu lassen.

Der Spielleiter sollte die Spieler das Spie\3rutenlaufen zwischen den Kampfparteien teilweise ausspielen lassen. Auf dem Weg zum Shuttle k"onnen die Charaktere Commander Lockhead um Unterst"utzung durch Kampftruppen bitten. Die Truppen der Garnison f"uhren in der Oberstadt den Kampf gegen die Konzerntruppen. Commander Lockhead wird f"ur die Unterst"utzung einen alten Bekannten, seinen Adjutanten Firedon, bereitstellen.

Nach dem Abflug aus dem Raumhafen bietet sich ein erschreckendes Bild. J"ager der Martell liefern sich ein Gefecht mit KI-J"agern der Zeus II-1 "uber der Stadt. Nahkampfgesch"utze der gro\3en Schlachtschiffe belegen J"ager und Torpedos mit einem Kugelteppich. Das Shuttle der Ermittler wird kurz nach dem Start von beiden Kampfparteien gescannt. Kann sich das Shuttle als Zugeh"orig zur Cynarian Corporation ausweisen und ist es mit einer unauff"alligen Schiffskennung getarnt, k"onnen die Ermittler unbehelligt am Kampfgebiet vorbeifliegen. Fliegt die Dawn of Day jedoch unter ihrer wahren Kennung, die der USI bereits bekannt ist, wird das Schiff von J"agern der Zeus II-1 angegriffen und muss sich mit Unterst"utzung von J"agern der Martell verteidigen.

\newsubsection{Schiffskennung und Legitimation}
Um zu vermeiden, dass das Schiff nach dem Start und beim Anflug auf die Nike Station unter Beschuss ger"at, empfiehlt es sich, dem Schiff eine unauff"allige Schiffskennung zu geben. Eine Schiffskennung der Cynarian Corporation erleichtert vor allem das Andocken an der Nike Station. Eine solche Schiffskennung kann die Cynarian Corporation "uber ihren Agenten Mr.~Klark bereitstellen. Die Ermittler m"ussen jedoch einen Grund f"ur den Flug und ein Flugziel nennen, um die Schiffskennung zu erhalten.

Um den Besuch der Neuro Intelligence auf der Nike Station zu verschleiern, ist eine triftige Ausrede erforderlich. Ein m"oglicher vorgeschobener Grund k"onnte sein, sichergestellte Daten aus der Risikozone Valhalla zu transportieren. In diesem Zusammenhang k"onnen die Charaktere ihre Darlegung durch einen guten W"urfelwurf auf ihre empathischen F"ahigkeiten untermauern. Alternativ k"onnten sie das Luna-Syndikat um eine Transporterkennung bitten.

\begin{remarks}
    Wurden beim Angriff auf das Cyberbrain Institut neben \ml{} auch weitere Mitarbeiter gefangen genommen, sollten diese beim Flug zur Nike Station nicht mitgenommen werden. Andernfalls w"urden sie unweigerlich in die H"ande der Cynarian Corporation fallen.
\end{remarks}
