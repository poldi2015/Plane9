%% Copyright 2019 Bernd Haberstumpf
%% License: CC BY-NC
% !TeX spellcheck = de_DE
\newsection{Treffen mit Smith-Singer (optional)}

W"ahrend die Ermittler den ersten Hinweisen auf Valhalla nachgehen erh"alt einer der Charaktere, vorzugsweise ein Alpha aber kein Omega eine Nachricht von einem Herrn Smith-Singer bzgl.~einem Informationsaustausch zu den Vorg"angen auf Hellgate. Der Nachrichtensender gibt sich als Beobachter im Auftrag des Transnationalen Konzernrates aus. Bei einer R"uckfragen bei Cynarian oder dem Protektorat werden die Ermittler gebeten den Termin wahr zu nehmen aber keine relevanten Informationen preis zu geben. Ob Smith-Singer wirklich im Auftrag des Konzernrates t"atig ist l"asst sich nicht klar bestimmen ist aber auch nicht auszuschlie\3en. Die Ermittler sollen in Erfahrung bringen was sein Auftrag ist und "uber welche Informationen er verf"ugt. Smith-Singer schl"agt vor im Stadtteil Rosenfurth einen Kaffee trinken zu gehen. Er w"urde sich dabei gerne allein mit dem Ermittler treffen.

Kommt der Ermittler alleine wird er Smith-Singer im "`Au\3enbereich"' des Kaffees nahe dem Eingang antreffen. Smith-Singer ist hoch gewachsen und hat eine athletische kr"aftige Statur. Die Finger sind manik"urt, das L"acheln makellos. Smith-Singer hat einen B"urstenhaarschnitt mit wei\3blonden Haaren. Er tr"agt einen teuren ma\3geschneiderten Anzug. Die holografisches Visitenkarte mit authentischem Konzernrats Logo weist ihn als Mitarbeiter des Konzernrates aus. Smith-Singer erkl"art, dass er als Beobachter im Auftrag des Konzernrates in das Jovianische System entsandt wurde um den Aufbau der HE-3 Produktion mit zu verfolgen und eine faire Zusammenarbeit der hiesigen Konzerne sicherzustellen. In diesem Zuge sind ihm die Attentate und die beunruhigende Erkenntnis einer Manipulation der Attent"ater zu Ohren bekommen. Wer seine Informationsquelle ist gibt Smith-Singer nicht an. Er fragt den Protektoratsermittler nach seiner Einsch"atzung der Bedeutung, dass die Attent"ater aus den Reihen der Mutanten gew"ahlt wurde und wen man als Drahtzieher hinter den Attentaten vermutet. Wenn das Gespr"ach anf"angt abzuebben wird er sich freundlich verabschieden begleicht die Rechnung, w"unscht noch einen guten Tag und verschwindet in der Menge.

Mit dem Treffen legt Smith-Singer die Grundlage f"ur ein Eingreifen des Konzernrates mit der USI als Drahtzieher. Zudem schafft er sich selbst mehr Handlungsspielraum indem er sich als Mitarbeiter des Konzernrates platziert. Desweiteren ist er auch daran interessiert sie Ermittler pers"onlich kennen zu lernen um sie besser einsch"atzen zu k"onnen.

\begin{remarks}
	Gewonnene Informationen: Bekanntschaft mit Smith-Singer.	
\end{remarks}
