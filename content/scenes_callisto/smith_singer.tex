%% Copyright 2019 Bernd Haberstumpf
%% License: CC BY-NC
% !TeX spellcheck = de_DE
\newsection{Treffen mit Smith-Singer (optional)}

W"ahrend die Ermittler den ersten Hinweisen auf Valhalla nachgehen, erh"alt einer der Charaktere, vorzugsweise ein Alpha-Mutant, vor allem aber kein Omega Krieger, eine Nachricht von einem Herrn Smith-Singer bzgl.~eines Informationsaustausches zu den Vorg"angen auf Hellgate. Der Absender gibt sich als Beobachter im Auftrag des Transnationalen Konzernrates aus. Smith-Singer schl"agt vor, im Stadtteil Rosenfurth einen Kaffee trinken zu gehen. Er w"urde sich dabei gerne allein mit dem Ermittler treffen. 

Auf eine R"uckfrage bei Cynarian oder dem Protektorat hin werden die Ermittler gebeten, den Termin wahrzunehmen, aber keine relevanten Informationen preis zu geben. Ob Smith-Singer wirklich im Auftrag des Konzernrates t"atig ist l"asst sich damit nicht auszuschlie\3en. Der Ermittler, mit dem Smith-Singer Kontakt aufgenommen hat, wird gebeten in Erfahrung zu bringen, was Singers Auftrag ist und "uber welche Informationen er verf"ugt. 

Kommt der Ermittler allein, wird er Smith-Singer im "`Au\3enbereich"' des Caf\'es mit dem kreativen Namen "`Caf\'e de l'amour"', nahe dem Eingang antreffen. Smith-Singer ist hochgewachsen und hat eine athletische, kr"aftige Statur. Seine Finger sind manik"urt, sein L"acheln makellos. Smith-Singer hat einen B"urstenhaarschnitt mit wei\3blonden Haaren und tr"agt einen teuren, ma\3geschneiderten Anzug. Die holografische Visitenkarte mit authentischem Konzernratslogo weist ihn als Mitarbeiter aus. Smith-Singer erkl"art, dass er als Beobachter im Auftrag des Konzernrates in das Jovianische System entsandt wurde, um den Aufbau der HE-3-Produktion mitzuverfolgen und eine faire Zusammenarbeit der hiesigen Konzerne sicherzustellen. In diesem Zuge sind ihm die Attentate und die beunruhigende Erkenntnis einer Manipulation der Attent"ater zu Ohren gekommen. Wer seine Informationsquelle ist, gibt Smith-Singer nicht preis. Er fragt den Ermittler nach seiner Einsch"atzung der Bedeutung, dass die Attent"ater aus den Reihen der Mutanten gew"ahlt wurden, und wen man als Drahtzieher hinter den Attentaten vermutet. Wenn das Gespr"ach anf"angt abzuebben, wird er sich freundlich verabschieden, die Rechnung begleichen, einen guten Tag w"unschen und in der Menge verschwinden.

Mit dem Treffen legt Smith-Singer die Grundlage f"ur ein Eingreifen des Konzernrates mit der USI als Drahtzieher, indem er subtil die Instabilit"at des Sektors thematisiert. Zudem schafft er sich selbst mehr Handlungsspielraum, indem er sich als Mitarbeiter des Konzernrates platziert. Des Weiteren ist er auch daran interessiert, die Ermittler pers"onlich kennenzulernen, um sie besser einsch"atzen zu k"onnen.

\begin{remarks}
	Gewonnene Information: Bekanntschaft mit Smith-Singer, einem der M"anner, die sich laut Sonja Frost nach der Dawn of Day erkundigt haben.

	Der Transnationale Konzernrat ist die oberste rechtliche Instanz der transnationalen Konzerne. Er ist nochmals im \cref{sec:institutions} beschrieben.

	Den Konzernen steht als Konsulat für Rechtsfragen eine kleine Niederlassung des Konzernrates im Bezirk Headquarter zur Verfügung. Der Status der Niederlassung entspricht in etwa dem einer Botschaft. Auf die Rückfrage, ob Smith-Singer für den Konzernrat arbeitet, wird mit der Gegenfrage geantwortet, in welcher Angelegenheit man mit ihm sprechen wolle. Es wird das Angebot unterbreitet, eine Nachricht zu hinterlassen. Auch Cynarian hat in diesem Zusammenhang keine besseren Möglichkeiten.
	
	Durch Cynarians Übernahme der HE-3-Produktion im jovianischen System hat die USI ihre Vormachtstellung in diesem Gebiet eingebüßt und sich weitgehend zurückgezogen. Aus diesem Grund besitzt die USI keine offiziellen Niederlassungen im jovianischen System mehr. Das Konsulat des Konzernrates ist hier allerdings eine Ausnahme. Es wurde vor der Aufgabe des Systems von der USI finanziert und geführt, ist entsprechend mit dem Konzern verbunden und bietet dadurch auch den USI-Agenten die Tarnidentität als Konzernratsmitarbeiter.
\end{remarks}
