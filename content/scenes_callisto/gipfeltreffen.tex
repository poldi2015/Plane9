%% Copyright 2019 Bernd Haberstumpf
%% License: CC BY-NC
% !TeX spellcheck = de_DE
\pageimage{images/cmyk/planetarium_cmyk.jpg}
\newsection{Die Konferenz}

\newsubsection[Das ``Planetarium'']{Das ''Planetarium''} 
Wenn die Ermittler die Zone verlassen, hat bereits die erste Zusammenkunft der Repr"asentanten von Erde, Mars und Jupiter im imposanten Planetarium stattgefunden. Die Vertreter der Delegationen haben sich anschlie\3end zu Einzelgespr"achen in verschiedene R"aumlichkeiten rund um das Planetarium zur"uckgezogen. Das Planetarium, das Teil des Raumhafens ist, besteht aus einem runden Saal, Radius 16 Meter, mit einer spektakul"aren Glaskuppel, die sich auf H"ohe der Oberfl"ache Kallistos befindet und einen beeindruckenden Blick auf den Jupiter aus nahezu n"achster N"ahe bietet.

Um den Saal herum f"uhrt eine 2 Meter breite Galerie mit T"uren zu den logistischen Bereichen des Saals. Ein 6 Meter breiter Gang rund um die Gallerie f"uhrt in den Backstagebereich, Technikr"aume, Lagerr"aume, Arbeitsbereiche f"ur das Personal und in eine K"uche. "Uber die Galerie gelangt man "uber zwei Treppenh"auser mit Aufz"ugen in die tiefer gelegenen R"aume des Geb"audes, wo sich weitere Konferenzr"aume und ein Hotel befinden. Die Treppenh"auser und Aufz"uge f"uhren weiter hinab zu dem noch tiefer gelegenen Garnisonsgel"ande auf H"ohe der Oberstadt Valhallas.

Eine breite Halbr"ohre verbindet den Eingangsbereich des Planetariums mit den Terminals des Raumhafens. Im Eingangsbereich des Saals, auf H"ohe des umlaufenden Ganges, sind Exponate aus der Anfangszeit der Raumfahrt in Vitrinen ausgestellt. Der Hauptteil des Planetariums ist abgesenkt und von f"unf halbkreisf"ormig angeordneten Ebenen mit Sitzgelegenheiten wie in einem Auditorium umgeben. Im Zentrum des Planetariums stehen kleine Stehtische und zwei Rednerpulte auf einer B"uhne.

Im Falle eines Druckluftverlustes wird das Planetarium "uber Druckschotts vom Rest des Geb"audes und dem Raumhafen abgetrennt. "Uber normalerweise nicht zug"angliche, durch Luftschleusen angebundene Fluchtwege im logistischen Bereich, gelangt man in die darunterliegenden Bereiche. Die Fluchtwege befinden sich zwischen den dem Saal umlaufenden logistischen R"aumen in einem Abstand von im Schnitt 8 Metern 8 Fluchtwege.

\newsubsection{Zeus II-1} 
Eine erste Unterbrechung der Verhandlungen ergibt sich durch das Eintreffen der Sicherheitskr"afte der Zeus II-1. Die Delegierten treffen sich zu einer kurzen Abstimmung und beschlie\3en, den Konzerntruppen keinen Zugang zu den R"aumen des Planetariums zu gew"ahren. Der Tunnel zum Raumhafen wird daraufhin abgeriegelt.

\newsubsection{Personen im Geb"aude} 
Die Sicherung der Veranstaltung wird von Soldaten des Garnisonsst"utzpunktes und Sicherheitskr"aften der Cynarian Corporation "ubernommen. Die Garnison stellt 15 Soldaten als Sicherheitskr"afte bereit, w"ahrend Cynarian weitere 20 Sicherheitskr"afte stellt. Thunderbolt f"uhrt die Sicherheitskr"afte des Protektorats pers"onlich an. Die Einheiten der Cynarian Corporation unterstehen Colonel Scholz. Weitere Bedienstete sorgen f"ur den reibungslosen Ablauf des Treffens.


\newsubsection{Der Plan der Attent"ater} 
Die mit Implantaten von Neuro Intelligence ausgestatteten Mutanten (beschrieben \cref{sec:attentaeter}) planen w"ahrend der Abschlussveranstaltung der Konferenz ein Attentat. Zu diesem Zeitpunkt befinden sich 30 Personen im Planetarium: 20 Konferenzteilnehmer, darunter alle W"urdentr"ager, sowie Bedienstete im Saal. Weitere 10 Bedienstete halten sich im umliegenden Gang und in den R"aumen rund um den Saal auf.

Der geplante Angriff erfolgt, sobald Avenger das Rednerpult zur Abschlussrede betritt. Der Attent"ater Caldron z"undet mehrere Sprengs"atze im Terminalbereich des Raumhafens, was zu Verletzten und Toten f"uhrt. Einige Bereiche der Terminals werden dem Vakuum ausgesetzt. Auch durch den Tunnel, der das Planetarium mit dem Raumhafen verbindet, entweicht Luft. Durch den ausgel"osten Alarm schlie\3en sich die Druckschotts, und weitere Zug"ange zum Raumhafen und zur Garnison werden abgesperrt. Panzerplatten schlie\3en sich "uber dem Kuppeldach des Planetariums.

Im Planetarium selbst befinden sich vier Attent"ater: Artisan und drei Omega-Soldaten. Falls Thunder nicht Teil der Eingreiftruppe zur St"urmung des Cyberbrain-Instituts war, ist er ebenfalls unter den Attent"atern im Raum. Artisan befindet sich direkt im Saal in der N"ahe der Rednerpulte, w"ahrend die drei Soldaten von der Galerie aus die Konferenzteilnehmer angreifen. Sprengs"atze werden im Saal zun"achst nicht gez"undet, da es f"ur die USI von entscheidender Bedeutung ist, das Protektorat als die Schuldigen des Attentats darzustellen. Daher sind klare Aufnahmen aus dem Konferenzraum unerl"asslich. Es ist m"oglich, dass die Attent"ater Sprengs"atze platziert haben, die sie im Falle eines Fehlschlags ihres Angriffs z"unden k"onnten.

Das erste Angriffsziel der Attent"ater sind die Vertreter des Shigano-Kombinats. In der allgemeinen Aufregung wird Artisan parallel dazu zun"achst versuchen, Vandermool zu t"oten und sich dann Avenger zuzuwenden. 

Den Attent"atern kommt zugute, dass sich unter den Teilnehmern Der Konferenz auch der USI-Agent Smith-Singer befindet, der zusammen mit Artisan die M"oglichkeit hatte, Waffen und Munition unbemerkt im Geb"aude zu deponieren. Die Waffen befinden sich je nach Sicherheitskonzept der Veranstaltung entweder in Sch"achten des Bel"uftungssystems, unterhalb der Sitzgelegenheiten im Saal oder werden von den Omega-Soldaten regul"ar bei sich getragen.

\newsubsection{Das Attentat}
Da die Ermittler zum Zeitpunkt des Attentats als gesuchte Terroristen gelten, die Oberstadt von Einheiten der Zeus II-1 patrouilliert wird und die Ereigniskette vor dem Attentat in einem engen Zeitrahmen abl"auft, ist es f"ur die Ermittler unm"oglich, direkt in das Attentat einzugreifen. F"ur die Ermittler ist das Attentat auf der Konferenz daher ein Nebenkriegsschauplatz, den sie nur aus der zweiten Reihe miterleben k"onnen. Auch f"ur die Geschichte selbst hat der Ausgang des Attentats keine wesentliche Auswirkung.

Um die Spieler trotzdem hautnah am Attentat teilnehmen zu lassen, schl"upfen sie "ubergangsweise in die Rollen prominenter Pers"onlichkeiten unter den Konferenzteilnehmern:

\begin{description}
	\item[Colonel Scholz] als Befehlshaber des Cynarian-Sicherheitsteams.
	\item[Thunderbolt] als Befehlshaber der Sicherheitstruppen des Protektorats.
	\item[Hato] als Leibw"achter von Avenger.
	\item[Avenger] der Protektor, der zusammen mit Hato ein Team bildet.
\end{description}

Das Vorgehen der Ermittler kurz vor dem Attentat beeinflusst die weiteren Ereignisse auf der Konferenz. Sollten die Ermittler die Attent"ater identifizieren, wovon auszugehen ist, werden Commander Lockhead, Thunderbolt und Scholz erst kurz vor der Abschlussrede von Avenger informiert. Sie m"ussen dann schnell handeln, um Schlimmeres zu verhindern.

Bevor das Attentat stattfindet, sollte der Spielleiter den Spielern die M"oglichkeit geben, ein Sicherheitskonzept f"ur die Abschlussveranstaltung auszuarbeiten.

\begin{description}
	\item[Commander Lockhead kennt keine Namen] Wird Commander Lockhead "uber Attent"ater informiert, ohne konkrete Namen zu nennen, wird er 	
		in Absprache mit Thunderbolt und Colonel Scholz die Omega-Soldaten des Sicherheitsteams, die im kritischen Zeitraum im Rondra Hospital operiert wurden, ohne Angabe von Gr"unden aus dem Planetarium abziehen lassen. Die Attent"ater starten daraufhin sofort ihr Attentat, noch vor der abschlie\3enden Rede. Colonel Scholz und Thunderbolt m"ussen entscheiden, ob sie Avenger und Hato informieren, bevor die Namen der Attent"ater bekannt sind, und ob Commander Lockhead Soldaten auf Verdacht abziehen soll.
	\item[Lockhead sind die Attent"ater bekannt] Sind Commander Lockhead die Attent"ater bekannt, wird er Thunderbolt und Colonel Scholz 
		informieren und "uber Thunderbolt den anderen Soldaten im Tagungsgel"ande die Anweisung geben, die Attent"ater sofort zu eliminieren. Es kommt umgehend zum Attentatsversuch.
	\item[Vandermool sind die Attent"ater bekannt] Wird Vandermool "uber die Identit"at von Artisan als Attent"ater informiert, wird er 
		versuchen, ihn mit einer Waffe, die er sich von Colonel Scholz hat zustecken lassen, zu t"oten.
	\item[Avenger und Hato] Avenger und Hato k"onnen nur pers"onlich kontaktiert werden, da es Artisan gelungen ist, die gesamte 
		Kommunikation mit ihnen zu unterbinden. Der Protektor und Hato k"onnen pers"onlich am leichtesten von Colonel Scholz, Thunderbolt und Vandermool "uber die drohende Gefahr unterrichtet werden.
	\item[Blackheart besetzt Valhalla] Blackhearts Truppen st"urmen das Konferenzgeb"aude, aber der Angriff erfolgt erst nach dem Attentat.
	\item[Besatzer der Zeus II-1] Die Besatzer der Zeus II-1 wird das Tagungsgeb"aude nicht angreifen.
\end{description}


\begin{remarks}	
	Die Aufgabe der Spieler, die die Rollen von Schl"usselpersonen auf der Konferenz eingenommen haben, ist es, das Geb"aude bestm"oglich abzusichern und nach der Identifikation der Attent"ater m"oglichst clever zu reagieren. Das Kampfgeschehen wird im Zweifelsfall nur sehr kurz sein. Aufgrund der gro\3en Anzahl an Sicherheitskr"aften haben die Attent"ater nur wenig Zeit, um Schaden anzurichten, bevor sie selbst ausgeschaltet werden. 
	
	Die Angreifer haben zwei Asse im "Armel: Der Attent"ater Caldron, der im Raumhafen die Sprengs"atze gez"undet hat, kann unerkannt in das Konferenzgel"ande eindringen. Smith-Singer, der zun"achst nicht selbst am Angriff beteiligt ist, kann im sp"ateren Durcheinander versuchen, unerkannt Konferenzteilnehmer auszuschalten.

	Informieren die Ermittler Commander Lockhead, Blackheart oder den \cref{sec:mrklark} vorgestellten Mr.~Klark "uber die Rolle von Smith-Singer, kann dieser auf der Konferenz festgesetzt werden.
	
	Ob Vandermool oder Avenger oder andere Konferenzteilnehmer zu Tode kommen, ist f"ur den Abschluss der Geschichte unerheblich. Wenn die Spieler die Kampagne bis zum Ende durchspielen m"ochten, werden ihre Charaktere im Folgenden unabh"angig von ihren Auftraggebern agieren m"ussen.
\end{remarks}
