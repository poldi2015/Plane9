%% Copyright 2019 Bernd Haberstumpf
%% License: CC BY-NC
% !TeX spellcheck = de_DE
\newsection{Im Blackhole Club}\anchor{sec:blackholeclub}

Der Blackhole Club gewährt nur Mitgliedern bzw. geladenen Gästen Einlass. Nur ausgewählte Personen werden jemals den Club von innen kennenlernen. Der Club befindet sich etwas versteckt im Herzen von Paradise City. In einschlägigen Kreisen steht der Club in dem Ruf, dass man über die Besucher an alles herankommt, außer an die heißen Girls, die im Club verkehren. Diese stehen bereits auf der Lohnliste des Clubs. Da der Club unter dem Schutz des Luna-Syndikats steht, ist ein Zugang in der Begleitung von \xl{} kein Problem. Vor dem durch den Türsteher \emph{Steelhammer} bewachten Zugang steht eine wilde Mischung aus Geschäftsleuten und halbseidenen Gangstern in Anzügen. In einer zweiten Schlange, stehen ihre Begleiterinnern, die deutlich schneller Einlasse in den Club bekommen. Waffen müssen am Eingang abgegeben werden. \xl{}, in einer figurbetonten Gefechtsuniform, die ihren kräftigen Oberarmen Ausdruck verleiht, ist im Club bereits bekannt und setzt sich nach dem Betreten des Clubs zu einer Gruppe von offensichtlichen Verehrern.

Im zentralen Raum des Clubs, gleich am Eingang, dominiert eine nach unten versetzte, große, gut gefüllte Tanzfläche. An den Ecken der Tanzfläche heizen spärlich bekleidete Go-go-Girls die Clubbesucher zu hämmernden Industrial-Klängen ein. Gegenüber dem Eingang füllt eine gewaltige Bar die ganze Breite des Raums. An der rechten Seite des Raums schließt sich ein abgegrenzter Bereich mit Séparées, aufgebaut wie ein Irrgarten, an. Ebenfalls auf der rechten Seite gelangt man in einen weiteren Raum mit einem großen Käfig, in dem von Zeit zu Zeit "illegale" Zweikämpfe stattfinden. Trotz des Verbots für Soldaten des Protektorats in den äußeren Bezirken von Valhalla hat sich hier eine Schar von Omega-Kriegern, umringt von aufreizend gekleideten Groupies, einen Stammplatz reserviert. Der Boden, die Wände und die Decke aller Räume sind tief schwarz. Eine indirekte Beleuchtung, durchzuckt von Neonblitzen, gibt einen vagen Blick auf die Gäste frei. Die einzigen hell beleuchteten Teile des Clubs sind die Bar und die Kampfarena.

\newsubsection{Carina}

Wenn sich die Ermittler an die Bar setzen, hier ist die Musik lediglich als Hintergrundbeschallung wahrnehmbar, werden sie als Erstes vom Barmann \emph{Rosen} angesprochen, der ihre Getränkebestellung aufnimmt. Fast beiläufig fragt er, wonach sie suchen. Sind es Better-than-Life-Holos, Körpermodifikation oder Waffen? Oder haben sie vielleicht selbst etwas anzubieten? Eventuell kennt er ja einen Gast, der ihnen weiterhelfen kann. Wenn die Ermittler nach Slingshot fragen, lässt sich der Barmann den Gesuchten genauer beschreiben, blickt dann kurz abwesend in die Menge, als hätte er jemanden erspäht, schüttelt kurz den Kopf und wendet sich dann anderen Gästen zu. Einige Zeit später, bevorzugt wenn die Gruppe sich getrennt hat, setzt sich eine hübsche junge Frau wie zufällig neben die Gruppe an die Bar. Die elegante Frau in einem glitzernden Kleid blickt versonnen auf die Auslagen hinter dem Tresen. Ihre langen, blauen, aufwendig frisierten Haare glitzern im Licht des Raums. Nach einer Minute des Schweigens bittet sie den Ermittler, der neben ihr sitzt, mit einem Blick über ihre Schulter, ihr einen Drink zu spendieren. Kommt er ihrer Bitte nach, wendet sie sich wieder von ihm ab und fragt beiläufig:

\speak{Ihr sucht nach einem Slingshot? Vielleicht hab ich von so jemandem schon etwas gehört. Was hat er denn angestellt?}

Wenn sie erfährt, dass Slingshot getötet wurde und dass etwas mit seiner Hardware im Kopf nicht in Ordnung war oder dass er an einem Attentat beteiligt war, reagiert sie betroffen. Schnell hat sie sich wieder unter Kontrolle. Sie reibt sich mit der Hand über den Mund und streicht eine Strähne, die sich gerade eben in ihr Gesicht verirrt hat, beiseite. Ihre Haarfarbe scheint kurzzeitig von Blau nach Giftgrün zu wechseln. Sie überlegt. Dann hat sie eine Entscheidung getroffen. Sie bittet den Charakter, mit dem sie gesprochen hat, ihr zu folgen, um ungestört plaudern zu können.

Bei der Frau, die sich zu der Gruppe gesellt hat, handelt es sich um \emph{Carina}, die Frau, die Lenny Kilkenny als Begleiterin von Slingshot genannt hat. Sie ist auch die Vermittlerin, die Slingshot und Hannibal mit den USI-Agenten rund um Smith-Singer in Verbindung brachte.

Carina führt den Ermittler zielsicher durch die Menge in den Bereich mit den Séparées, um in Ruhe sprechen zu können. Bevor der Ermittler jedoch Genaueres erzählen kann, kommt ein anderer Gast und setzt sich unaufgefordert neben Carina an den Tisch. Er bestellt sich ein Getränk, flirtet mit der Kellnerin und gibt Carina zu verstehen, dass er mit ihr sprechen muss. Carina versucht währenddessen, das Gespräch mit dem Ermittler in belanglose Themen abdriften zu lassen. Wie zufällig schmiegt sie sich dabei näher an den Ermittler heran und legt ihm eine Hand auf den Oberschenkel. Dabei schiebt sie ihm eine kleine Karte zu. Als der andere Gast geht, verabschiedet sie sich ebenfalls und bedankt sich für das nette Gespräch. Leider, so sagt sie, müsse sie mit dem anderen noch etwas besprechen.

Bei dem anderen Gast, der seinen Namen nicht nennt, handelt es sich um den USI-Agenten \emph{Dan Ringdaz}. Er ist von Smith-Singer beauftragt, das Arbeitsverhältnis mit Carina zu beenden. Er hat dabei auch Sorge dafür zu tragen, dass Carina keinerlei Informationen über ihren Auftrag an andere weitergibt. Dan Ringdaz, zusammen mit einem anderen Agenten, \emph{Frederic Johnson}, hatte über Carina den Kontakt zu Slingshot und Hannibal hergestellt. Beide arbeiten nur auf Zuruf für Smith-Singer und kennen die eigentlichen Hintergründe der Operation P9 nicht.

\newsubsection{Die Visitekarte}

Eine Untersuchung der Visitenkarte sollte möglichst außerhalb des Blackhole Clubs erfolgen. Auf der Visitenkarte ist auf den ersten Blick nur ein holografisches Bild von Carina in lasziven Bewegungen zu sehen. Unterschrieben ist das Hologramm mit dem Namen Fleur Soleil. Wird das Bild länger mit einem Finger berührt, taucht eine ComLink-Nummer auf.

Wird eine Nachricht an die Nummer ohne das Unterdrücken der eigenen Nummer verschickt, kommt als Rückantwort: "`Ice Club in zwei Stunden. Komm allein. "`Code Solar Eclipse."'.

\begin{remarks}
	Gewonnene Information: Kontakt Carina, alias Fleur Soleil. Carinas Visitenkarte.

	Der Blackhole Club ist eine gute Gelegenheit, ein bisschen freies Rollenspiel jenseits des Plots einzuflechten. Je nachdem, wie sich die Charaktere gegenüber dem Barmann Rosen und gegenüber anderen Gästen verhalten, kann die Kontaktaufnahme mit Carina völlig anders verlaufen als beschrieben. Der Club ist mit Gangstern, Schiebern, Hehlern und Konzernleuten gefüllt, die nach illegalen Waren oder Dienstleistungen suchen. Dementsprechend vorsichtig sollte die Gruppe mit ihrem Wissen und ihren Fragen umgehen.

	Die Omega-Krieger, die sich in der Nähe der Arena niedergelassen haben, erwarten, dass sich ein Omega, der den Club betritt, zu ihnen gesellt. Im Zweifel nimmt den entsprechenden Ermittler einer der Groupies an die Hand. Nicht alle Omegas im Club sind Soldaten des Protektorats. Ein großer Teil der Anwesenden verdient sich als Söldner oder als Kämpfer in der Arena. Die anwesenden Soldaten des Protektorats sind höchst inoffiziell im Club. Über ihren Clubbesuch darf außerhalb des Clubs nicht gesprochen werden.
	
	Carinas auffällige Haarpracht ist so modifiziert, dass Carina je nach Laune die Farbe ändern kann.
\end{remarks}


\newsection{Bewegungen im Hintergrund}

Beim Verlassen des Blackhole Clubs trennt sich \xl{} von den Ermittlern und stellt ihnen Quicksilver für eine sichere Rückkehr in das Sunshine bereit. \xl{} verfolgt Carina, die sie bereits kennt, beim Verlassen des Clubs. Auf dem Heimweg versucht Dan Ringdaz zusammen mit zwei Straßenschläger Carina in seine Gewalt zu bringen. \xl{} kommt ihr zu Hilfe und tötet alle drei Angreifer. Danach fragt sie Carina aus und erfährt dadurch, dass den Attentätern in der Forschungseinrichtung Cyberbrain die KIs eingesetzt wurden. Sie erfährt auch dass Prof.~Dr.~Sanders die medizinischen Eingriffe durchgeführt hatte. Da sie selbst keine eigene Möglichkeit besitzt Cyberbrain zu infiltrieren schlägt sie Carina vor mit den Ermittlern ein Treffen zu vereinbaren bei dem sie um Schutz durch das Luna Syndikats bittet. Mit ihren Fähigkeiten als Psychonaut löscht sie Carina das Treffen mit ihr und die Mitwirkung Prof.~Dr.~Sanders bei P9 aus dem Gehirn. Prof.~Dr.~Sanders will sie selbst vor den Charakteren verhören.

Beim Eintreffen im Sunshine Hotel werden die Charaktere direkt zu einem Treffen mit Nemessis in seiner Suite im Hotel gebracht. Auch die Suite im Hotel ist eher minimalistisch gehalten. Unnötige Möbel sind entfernt. Technische Geräte füllen den Raum. Nemessis eröffnet den Charakteren, dass Blackheart den Flottenträger Donar unter dem Kommando von Lord Commander Steeler mit Begleitschiffen in den Orbit über Valhalla entsandt und den Kreuzer Pendragon nach Armageddon befehligt hat. Nemessis befürchtet ein Invasion Blackhearts auf Valhalla.

Versuchen die Charaktere einen Kontakt mit Ihren Befehlshabern herzustellen kann das Luna Syndikat aushelfen und einer Verbindung herstellen. Initieren die Charaktere selbst keinen Kontakt werden sie vom Quicksilver darüber informiert, dass sowohl Cynarian wie auch Blackheart die Charaktere kontaktieren möchten.

Bevor die Charaktere sich auf den Weg zum Ice Club machen wird der Cynarian Ermittler von Colonel Scholz kontaktiert wenn er nicht selbst einen Report abliefert. Colonel Scholz informiert den Ermittler, dass sich Vandermool besorgt in Bezug auf die Flottenaktivitäten des Protektorats zeigt. Colonel Scholz eröffnet, dass am nächsten Tag eine Delegation von Erde und Mars, auf Rückfrage der Europäischen Förderation vertreten durch Luc Duval und dem Shigano-Kombinat vertreten durch Sarana auf Valhalla erwartet wird. Bei dem Besuch will man über die Aufnahme von Handelsbeziehungen sprechen. Die Abgeordneten des Protektorats werden sicher auch über auf der Erde zurück gebliebene Mutanten verhandeln wollen. Das Treffen dieser Art ist das erste seit der Gründung des Protektorats und deshalb von größter Wichtigkeit. Ein militärisches Eingreifen des Protektorats könnte schwerwiegende Folgen haben genauso wie weitere Attentate. Die Flottenbewegungen des Protektorats schätzt er deshalb als fragwürdig ein. Colonel Scholz fordert einen Ausführlichen Bericht und zeigt sich überrascht in Bezug auf die Zusammenarbeit mit dem Luna Syndikat. Er eröffnet, dass zwischen dem Protektorat und dem Luna Syndikat offensichtlich eine Vereinbarung über die Administration von Valhalla besteht.

Versuchen die Ermittler Artisan, Avenger oder Thunderbolt zu kontaktieren schlägt dies fehl. Beide sind bereits auf dem Weg nach Valhalla. Ein Kontakt mit Blackheart kommt allerdings zustande. Im Zweifelsfall kontaktiert sie den Ermittler des Protektoratsmilitärs selbst. Auf der Videoaufnahme mit schlechter Qualität ist zu erkennen, dass sie in voller Kampfmontur auf dem Kommandantensessel eines Kriegsschiffes sitzt. Sie wirkt aufgewühlt. Blackheart erwartet einen militärisch knappen Bericht über den Vortschritt und flucht als sie erfährt, dass noch keine wesentlichen neuen Erkenntnisse gewonnen wurden. Sie spricht den Ermittler auf das Gipfeltreffen an und zeigt sich erstaunt, dass die Ermittler noch nicht von Avenger informiert worden sind. Danach erklärt sie ihrem Ermittler, dass zusammen mit der Delegation von Jupiter und Erde auch zwei weitere Kriegsschiffe kurz vor dem Einflug in das jovianische System stehen. Sie erklärt:

\begin{speech}
	Bei den beiden Schiffen handelt es sich, was aus den Triebwerkssignaturen verrät, um zwei \emph{Schlachtkreuzer der Guardian Klasse}. Die Aufbauten sind anders gestaltet wie während der Kämpfe um die Aegis Station aber die Plasmafakel der Triebwerke ist unverkennbar." Ich hoffe ihr wisst was das heißt? Ein Eintreffen von zwei dieser KI Schiffe kommt einer Kriegserklärung gleich! 
\end{speech}

erklärt Blackheart mit tonloser Stimme. 

\begin{speech}
	Kurz vor dem Eintreffen auf Kallisto hat sich eines der beiden Schlachtkreuzer von den Schiffen der Delegation getrennt. Sein neuer Zielort ist noch nicht klar auszumachen. Haltet euch bedeckt und klärt verdammt nochmal auf wie das alles mit den Attentaten zusammen spielt. Von Avenger besteht ein Verbot einen Notstand auszurufen um die Verhandlungen nicht zu gefährden. Egal wie Avenger das sieht gilt höchste Alarmbereitschaft bei den Streitkräften und wir sind auf einen Krieg vorbereitet. Das ist die letzte Ruhe vor der Sturm. Haltet euch vom Raumhafen und der Garnison fern sonst bekommt euch noch die Konzernpolizei in die Finger. Gebt mir sofort einen Bericht wenn es etwas neues gibt! Wegtreten!
\end{speech}

Während die Charaktere diese neue Flut an Informationen überrollt macht sich \xl{} auf den Weg zu Prof.~Dr.~Sanders in seinem privaten Domizil. Sie überfällt seine Unterkunft schlägt ihn nieder und verbindet sich mit seiner Datenbuchse. Dabei erfährt sie von dem Standort der Cyberbrain Einrichtung in der Zone, den Eingriffen im Rondra Hospital und von \emph{\pinyin{Mailin2}} die die Bindung der KIs an die USI entfernt hat. Bei dem durchgeführten Tiefenscan stirbt Prof.~Dr.~Sanders. Über die aktuellen Entwicklungen im Orbit und über das politische Treffen mit Erde und Mars erfährt sie nach ihrer Rückkehr.

\begin{remarks}
	Gewonnene Information: Politisches Treffen mit der Europäischen Förderation und dem Shigano-Kombinat am übernächsten Tag. Eintreffen von KI gesteuerten Guardian Schlachtkreuzern.
\end{remarks}
