%% Copyright 2019 Bernd Haberstumpf
%% License: CC BY-NC
% !TeX spellcheck = de_DE
\newsection{Im Blackhole Club}\anchor{sec:blackholeclub}

Der Blackhole Club gew"ahrt nur Mitgliedern bzw. geladenen G"asten Einlass. Nur ausgew"ahlte Personen werden jemals den Club von innen kennenlernen. Der Club befindet sich etwas versteckt im Herzen von Paradise City. In einschl"agigen Kreisen steht der Club in dem Ruf, dass man "uber die Besucher an alles herankommt, au\3er an die hei\3en Girls, die im Club verkehren. Diese stehen bereits auf der Lohnliste des Clubs. Da der Club unter dem Schutz des Luna-Syndikats steht, ist ein Zugang in der Begleitung von \xl{} kein Problem. Vor dem durch den T"ursteher \emph{Steelhammer} bewachten Zugang steht eine wilde Mischung aus Gesch"aftsleuten und halbseidenen Gangstern in Anz"ugen. In einer zweiten Schlange stehen ihre Begleiterinnen, die deutlich schneller Einlass in den Club bekommen. Waffen m"ussen am Eingang abgegeben werden. \xl{}, in einer figurbetonten Gefechtsuniform, die ihren kr"aftigen Oberarmen Ausdruck verleiht, ist im Club bereits bekannt und setzt sich nach dem Betreten des Clubs zu einer Gruppe von offensichtlichen Verehrern.

Im zentralen Raum des Clubs, gleich am Eingang, dominiert eine nach unten versetzte, gro\3e, gut gef"ullte Tanzfl"ache. An den Ecken der Tanzfl"ache heizen sp"arlich bekleidete Go-go-Girls den Clubbesuchern zu h"ammernden Industrial-Kl"angen ein. Gegen"uber dem Eingang f"ullt eine gewaltige Bar die ganze Breite des Raums. An der rechten Seite des Raums schlie\3t sich ein abgegrenzter Bereich mit S\'epar\'ees, aufgebaut wie ein Irrgarten, an. Ebenfalls auf der rechten Seite gelangt man in einen weiteren Raum der ``Arena'' mit einem gro\3en K"afig, in dem von Zeit zu Zeit ``illegale'' Zweik"ampfe stattfinden. Trotz des Verbots f"ur Soldaten des Protektorats in den "au\3eren Bezirken von Valhalla hat sich hier eine Schar von Omega-Kriegern, umringt von aufreizend gekleideten Groupies, einen Stammplatz reserviert. Der Boden, die W"ande und die Decke aller R"aume sind tief schwarz. Eine indirekte Beleuchtung, durchzogen von Neonblitzen, gibt einen vagen Blick auf die G"aste frei. Die einzigen hell beleuchteten Teile des Clubs sind die Bar und die Kampfarena.

\pageimage{images/blackhole_club.jpg}

\newsubsection{Carina}

Wenn sich die Ermittler an die Bar setzen, hier ist die Musik lediglich als Hintergrundbeschallung wahrnehmbar, werden sie als Erstes vom Barmann \emph{Rosen} angesprochen, der ihre Getr"ankebestellung aufnimmt. Fast beil"aufig fragt er, wonach sie suchen. Sind es Better-than-Life-Holos, K"orpermodifikationen oder Waffen? Oder haben sie vielleicht selbst etwas anzubieten? Eventuell kennt er ja einen Gast, der ihnen weiterhelfen kann. Wenn die Ermittler nach Slingshot fragen, l"asst sich der Barmann den Gesuchten genauer beschreiben, blickt dann kurz abwesend in die Menge, als h"atte er jemanden ersp"aht, sch"uttelt kurz den Kopf und wendet sich dann anderen G"asten zu. Einige Zeit sp"ater, bevorzugt wenn die Gruppe sich aufgeteilt hat, setzt sich eine h"ubsche junge Frau wie zuf"allig neben die Gruppe an die Bar. Die elegante Dame in einem glitzernden Kleid blickt versonnen auf die Auslagen hinter dem Tresen. Ihre langen, blauen, aufwendig frisierten Haare glitzern im Wettstreit mit ihrem Kleid im Neonlicht des Clubs. Nach einer Minute des Schweigens bittet sie den Ermittler, der neben ihr sitzt, mit einem Blick "uber ihre Schulter, ihr einen Drink zu spendieren. Kommt er ihrer Bitte nach, wendet sie sich wieder von ihm ab und fragt beil"aufig:

\speak{Ihr sucht nach einem Slingshot? Vielleicht hab ich von so jemandem schon etwas geh"ort. Was hat er denn angestellt?}

Wenn sie erf"ahrt, dass Slingshot get"otet wurde und dass etwas mit seiner Hardware im Kopf nicht in Ordnung oder dass er an einem Attentat beteiligt war, reagiert sie betroffen. Schnell hat sie sich wieder unter Kontrolle. Sie reibt sich mit der Hand "uber den Mund und streicht eine Str"ahne, die sich gerade eben in ihr Gesicht verirrt hat, beiseite. Ihre Haarfarbe scheint kurzzeitig von Blau nach Giftgr"un zu wechseln. Sie "uberlegt. Dann hat sie eine Entscheidung getroffen. Sie bittet den Charakter, mit dem sie gesprochen hat, ihr zu folgen, um ungest"ort plaudern zu k"onnen.

Bei der Frau, die sich zu der Gruppe gesellt hat, handelt es sich um \emph{Carina}, die Frau, die Lenny Kilkenny als Begleiterin von Slingshot genannt hat. Sie ist die Vermittlerin, die Slingshot und Hannibal mit den USI-Agenten rund um Smith-Singer in Verbindung brachte.

Carina f"uhrt den Ermittler zielsicher durch die Menge in den Bereich mit den S\'epar\'ees, um in Ruhe sprechen zu k"onnen. Bevor der Ermittler jedoch Genaueres erz"ahlen kann, kommt ein anderer Gast und setzt sich unaufgefordert neben Carina an den Tisch. Er bestellt sich ein Getr"ank, flirtet mit der Kellnerin und gibt Carina zu verstehen, dass er mit ihr sprechen muss. Carina versucht w"ahrenddessen, das Gespr"ach mit dem Ermittler in belanglose Themen abdriften zu lassen. Wie zuf"allig schmiegt sie sich dabei n"aher an den Ermittler heran und legt ihm eine Hand auf den Oberschenkel. Dabei schiebt sie ihm eine kleine Karte zu. Als der andere Gast geht, verabschiedet sie sich ebenfalls und bedankt sich f"ur das nette Gespr"ach. Leider, so sagt sie, m"usse sie mit dem anderen noch etwas besprechen.

Bei dem anderen Gast, der seinen Namen nicht nennt, handelt es sich um den USI-Agenten \emph{Dan Ringdaz}. Er ist von Smith-Singer beauftragt, das Arbeitsverh"altnis mit Carina zu beenden. Er hat dabei auch Sorge daf"ur zu tragen, dass Carina keinerlei Informationen "uber ihren Auftrag an andere weitergibt. Dan Ringdaz, zusammen mit einem anderen Agenten, \emph{Frederic Johnson}, hatte "uber Carina den Kontakt zu Slingshot und Hannibal hergestellt. Beide arbeiten nur auf Zuruf f"ur Smith-Singer und kennen die eigentlichen Hintergr"unde der Operation P9 nicht.

\newsubsection{Die Visitekarte}

Eine Untersuchung der Visitenkarte sollte m"oglichst au\3erhalb des Blackhole Clubs erfolgen. Auf der Visitenkarte ist auf den ersten Blick nur ein holografisches Bild von Carina in lasziven Bewegungen zu sehen. Unterschrieben ist das Hologramm mit dem Namen Fleur Soleil. Wird das Bild l"anger mit einem Finger ber"uhrt, taucht eine ComLink-Nummer auf.

Wird eine Nachricht an die Nummer ohne das Unterdr"ucken der eigenen Nummer verschickt, kommt als R"uckantwort: \say{Ice Club in zwei Stunden. Komm allein. Code Solar Eclipse.}

\begin{remarks}
	\underline{Gewonnene Information:}

	\begin{itemize}
		\item Kontakt Carina, alias Fleur Soleil.
		\item Carinas Visitenkarte f"uhrt zu einem Treffen im Ice Club.
	\end{itemize}

	\underline{Freies Rollenspiel:}

	Der Blackhole Club ist eine gute Gelegenheit, ein bisschen freies Rollenspiel jenseits des Plots einzuflechten. Je nachdem, wie sich die Charaktere gegen"uber dem Barmann Rosen und gegen"uber anderen G"asten verhalten, kann die Kontaktaufnahme mit Carina v"ollig anders verlaufen als beschrieben. Der Club ist mit Gangstern, Schiebern, Hehlern und Konzernleuten gef"ullt, die nach illegalen Waren oder Dienstleistungen suchen. Dementsprechend vorsichtig sollte die Gruppe mit ihrem Wissen und ihren Fragen umgehen.

	\underline{Die Arena:}

	Die Omega-Krieger, die sich in der N"ahe der Arena niedergelassen haben, erwarten, dass sich ein Omega, der den Club betritt, zu ihnen gesellt. Im Zweifel nimmt den entsprechenden Ermittler einer der Groupies an die Hand. Nicht alle Omegas im Club sind Soldaten des Protektorats. Ein gro\3er Teil der Anwesenden verdient sich als S"oldner oder als K"ampfer in der Arena. Die anwesenden Soldaten des Protektorats sind h"ochst inoffiziell im Club. "Uber ihren Clubbesuch darf au\3erhalb des Clubs nicht gesprochen werden.
\end{remarks}
\begin{remarks}
	\underline{\xl{}}:

	Wie im letzten Kapitel beschrieben, nimmt \xl{} gelegentlich an Zweik"ampfen in der Arena teil und hat sich dabei eine kleine Fangemeinde aufgebaut. Mit den K"ampfen bietet sie der KI im Kopf die Gelegenheit, sich perfekt mit den F"ahigkeiten ihres K"orpers und ihrer Cyberware vertraut zu machen. Viel schneller und pr"aziser als jeder Mensch berechnet die KI Angriff und Verteidigung und "ubersendet passende Befehle an ihre Cyberware.

	Den Besuchern der Arena ist \xl{}s Zugeh"origkeit zum Luna-Syndikat bekannt. Au\3er, dass sie fast die einzige ist, die den Omegas Kontra geben kann, ist sonst niemandem n"aher zu ihr bekannt. Ach ja, angeblich bevorzugt sie Frauen im Bett. Offensichtlich glauben aber nicht alle diesem Ger"ucht.

	\underline{Carina:}
	
	Carinas auff"allige Haarpracht ist so modifiziert, dass Carina je nach Laune die Farbe "andern kann.
\end{remarks}


\newsection{Bewegungen im Hintergrund}

\newsubsection[\xl{} verfolgt Carina]{Xiao Long verfolgt Carina}

Beim Verlassen des Blackhole Clubs trennt sich \xl{} von den Ermittlern und stellt ihnen Quicksilver f"ur eine sichere R"uckkehr in das Sunshine Hotel beiseite.

\xl{} verfolgt Carina, die ihr bereits bekannt ist, beim Verlassen des Clubs. Auf dem Heimweg versucht Dan Ringdaz zusammen mit zwei Stra\3enschl"agern, Carina in seine Gewalt zu bringen. \xl{} kommt ihr zu Hilfe und t"otet alle drei Angreifer. Danach fragt sie Carina aus und erf"ahrt dadurch, dass den Attent"atern in der Forschungseinrichtung \emph{Cyberbrain} die KIs eingesetzt wurden. Sie erf"ahrt auch, dass Prof.~Dr.~Sanders die medizinischen Eingriffe durchgef"uhrt hat. Da sie selbst keine M"oglichkeit besitzt, Cyberbrain zu infiltrieren, schl"agt sie Carina vor, mit den Ermittlern ein Treffen zu vereinbaren, bei dem sie um Schutz durch das Luna-Syndikat bittet. Mit ihren F"ahigkeiten als Psychonaut l"oscht sie Carina das Treffen mit ihr und die Mitwirkung von Prof.~Dr.~Sanders bei Operation P9 aus dem Ged"achtnis. Prof.~Dr.~Sanders will sie selbst vor den Charakteren verh"oren.

\newsubsection{Treffen mit Nemessis}

Beim Eintreffen im Sunshine Hotel werden die Charaktere direkt zu einem Treffen mit Nemessis in seine Suite im Hotel gebracht. Die Suite ist eher minimalistisch gehalten. Unn"otige Einrichtungsgegenst"ande wurden entfernt, und technische Ger"ate f"ullen den Raum. Nemessis er"offnet den Charakteren, dass Blackheart den Flottentr"ager \emph{Martell} unter ihrem Kommando mit Begleitschiffen in den Orbit "uber Valhalla entsandt und den Kreuzer Pendragon nach Armageddon beordert hat. Nemessis bef"urchtet, dass Blackheart eine Invasion auf Valhalla plant.

Versuchen die Charaktere, einen Kontakt mit ihren Befehlshabern herzustellen, kann das Luna-Syndikat aushelfen und eine Verbindung herstellen. Initiieren die Charaktere selbst keinen Kontakt, werden sie von Quicksilver dar"uber informiert, dass sowohl Cynarian als auch Blackheart die Charaktere kontaktieren m"ochten.

\newsubsection{Colonel Scholz}

Bevor die Charaktere sich auf den Weg zum Ice Club machen, wird der Cynarian-Ermittler von Colonel Scholz kontaktiert, falls er nicht selbst einen Bericht abliefert. Colonel Scholz informiert den Ermittler, dass Vandermool besorgt bez"uglich der Flottenaktivit"aten des Protektorats ist. Er teilt mit, dass am n"achsten Tag eine Delegation von Erde und Mars auf Valhalla erwartet wird. Bei dem Besuch soll "uber die Aufnahme von Handelsbeziehungen gesprochen werden. Die Abgeordneten des Protektorats werden sicherlich auch die auf der Erde verbliebenen Mutanten thematisieren wollen. Dieses Treffen ist das erste seiner Art seit der Gr"undung des Protektorats und daher von gr"o\3ter Wichtigkeit. Ein milit"arisches Eingreifen des Protektorats k"onnte schwerwiegende Folgen haben, ebenso wie weitere Attentate. Colonel Scholz sch"atzt die Flottenbewegungen des Protektorats als fragw"urdig ein und fordert einen ausf"uhrlichen Bericht. Er zeigt sich "uberrascht "uber die Zusammenarbeit mit dem Luna-Syndikat und fragt an, ob die Zusammenarbeit aus einer angeblichen Vereinbarung zwischen Blackheart und dem Luna-Syndikat hervorgegangen ist.

\newsubsection{Blackheart}

Versuchen die Ermittler, Artisan, Avenger oder Thunderbolt zu kontaktieren, schl"agt dies fehl. Alle drei sind bereits auf dem Weg nach Valhalla. Ein Kontakt mit Blackheart kommt jedoch zustande. Im Zweifelsfall kontaktiert sie den Ermittler des Protektoratsmilit"ars selbst. Auf der verrauschten Video"ubertragung ist zu erkennen, dass sie in voller Kampfmontur auf dem Kommandosessel eines Kriegsschiffes sitzt. Sie wirkt aufgew"uhlt. Blackheart erwartet einen milit"arisch knappen Bericht "uber den Fortschritt und flucht, als sie erf"ahrt, dass noch keine wesentlichen neuen Erkenntnisse gewonnen wurden. Sie spricht den Ermittler auf die bevorstehende Konferenz an und zeigt sich erstaunt, dass die Ermittler noch nicht von Avenger informiert wurden. Danach erkl"art sie dem Ermittler, dass zusammen mit der Delegation von Mars und Erde auch zwei weitere Kriegsschiffe kurz vor dem Eintritt in das jovianische System stehen.

\begin{speech}
	Bei den beiden Schiffen handelt es sich, wie uns die Triebwerkssignaturen verraten, um zwei Schlachtkreuzer der Guardian-Klasse. Die Aufbauten sind anders gestaltet als w"ahrend unserer K"ampfe um die Aegis-Station, aber die Plasmafackeln der Triebwerke sind unverkennbar. Ich hoffe, ihr wisst, was das bedeutet? Das Eintreffen von zwei dieser KI-Schiffe kommt einer Kriegserkl"arung gleich!

	Ich hoffe, ihr liefert bald, sehr bald Resultate.
\end{speech},

erkl"art Blackheart mit tonloser Stimme. 

\begin{speech}
	Kurz vor dem Eintreffen auf Kallisto hat sich einer der beiden Schlachtkreuzer von den Schiffen der Delegation getrennt. Sein neuer Zielort ist noch nicht klar auszumachen. Haltet euch bedeckt und kl"art verdammt nochmal auf, wie das alles mit den Attentaten zusammenpasst. Von Avenger besteht ein Verbot, einen Notstand auszurufen, um die Verhandlungen nicht zu gef"ahrden. Egal, wie Avenger das sieht. Es gilt h"ochste Alarmbereitschaft bei den Streitkr"aften, und wir sind auf einen Krieg vorbereitet. Das ist die Ruhe vor dem Sturm. Haltet euch vom Raumhafen und der Garnison fern, sonst bekommt ihr noch die Konzerngardisten in die Finger. Gebt mir sofort einen Bericht, wenn es etwas Neues gibt! Over.
\end{speech}

\begin{remarks}
	\underline{Guardian Schlechtkreuzer:}

	Bei den beiden Schlachtkreuzern handelt es sich um die \emph{Zeus II-1} und \emph{Zeus II-2}, genauer beschrieben \cref{sec:guardian}. Diese Schlachtkreuzer sind KI-gesteuerte Kriegsschiffe, die von der USI eingesetzt werden. Die Mutanten hatten bereits bei ihrem Kampf um die erdnahe Orbitalfestung Aegis Bekanntschaft mit diesen Schiffen gemacht. Dort wurden sie gnadenlos von den spinnenartigen Kampfrobotern, die von den Schiffen ausgesandt wurden, niedergemetzelt.

	\underline{Die Martell:}

	Nach der Identifizierung der beiden Begleitschiffe hat sich Blackheart umgehend auf die Martell, einen der Flottentr"ager des Protektorats, begeben und ist mit Geleitschutz nach Kallisto aufgebrochen. Die Kommunikation mit den Ermittlern f"uhrt sie von der Br"ucke des Kriegsschiffs aus.
\end{remarks}

\newsubsection{Professor Sanders Ende}

W"ahrend die Charaktere von der neuen Flut an Informationen "uberrollt werden, macht sich \xl{} auf den Weg zu Prof.~Dr.~Sanders in seinem privaten Domizil. Sie "uberf"allt sein Anwesen, schl"agt ihn nieder und verbindet sich mit seiner Datenbuchse. Dabei erf"ahrt sie den Standort der Cyberbrain-Einrichtung in der Zone, die Eingriffe im Rondra-Hospital und von \ml{}, die die KI in ihrem Kopf entwickelt hat. Bei dem durchgef"uhrten Tiefenscan stirbt Prof.~Dr.~Sanders. "Uber die aktuellen Entwicklungen im Orbit und das politische Treffen mit Erde und Mars erf"ahrt sie nach ihrer R"uckkehr.

\begin{remarks}
	\underline{Gewonnene Information:}
	
	\begin{itemize}
		\item Politische Konferenz mit Vertretern mit Erde und Mars am "ubern"achsten Tag.
		\item Eintreffen von KI-gesteuerten Guardian-Schlachtkreuzern.
	\end{itemize}

	Die Vertreter von Erde und Mars werden \cref{sec:delegates} beschrieben. Das Cyberbrain-Institut ist Thema \cref{sec:cyberbrain}. Die Ermittler erfahren von dem Institut bei einem folgenden Treffen mit Carina im Ice Club \cref{sec:ice_club}.

	\ml{} ist eine Mitarbeiterin bei Neuro Intelligence. Sie hat alle KIs auf humanoide K"orper angepasst. \ml{} wird \cref{sec:mailin} n"aher beschrieben.
\end{remarks}
