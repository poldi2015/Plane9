%% Copyright 2019 Bernd Haberstumpf
%% License: CC BY-NC
% !TeX spellcheck = de_DE
\pageimage{images/battle_of_callisto.jpg}
\newsection{Kampf um Kallisto}\anchor{sec:fightforcallisto}

Erf"ahrt Blackheart, dass sich Attent"ater aus den Reihen des Protektoratsmilit"ars unter den Verd"achtigen befinden oder dass Artisan als Attent"ater identifiziert wurde, wird sie das Protektoratsmilit"ar in den Kriegszustand versetzen. Dasselbe gilt, wenn sie erf"ahrt, dass die USI hinter der Initiative des Konzernrates steht. Ist Artisan als Attent"ater identifiziert, wird sie Thunderbolt beauftragen, ihn auszuschalten, und Vandermool informieren, da Avenger nicht kontaktiert werden kann. Sollte bekannt werden, dass Omega-Soldaten als Attent"ater infrage kommen, erh"alt Commander Lockhead den Befehl, umgehend alle seine Soldaten vom Gipfeltreffen abzuziehen. Danach befiehlt sie den Soldaten auf der Donar, Valhalla sofort zu besetzen. Wird Blackheart von den Ermittlern nicht informiert, wird sie die Besetzung Valhallas nach dem Attentat auf dem Gipfeltreffen befehlen. Gleichzeitig mit der Besetzung Valhallas befiehlt sie der Donar, den im Orbit von Valhalla befindlichen Guardian-Klasse-Schlachtkreuzer Zeus II-1 anzugreifen. Der zweite Flottentr"ager des Protektorats, die Martell, erh"alt den Befehl, die Zeus II-2 abzufangen. "Uber die Angriffe setzt sie nur ihre Schlachtschiffe in Kenntnis.

Befiehlt Blackheart die Mobilisierung, setzt die Zeus II-1 "uber Valhalla alle ihre verbliebenen Landungsshuttles in Bewegung und beginnt, sich ohne menschliche Besatzung der Donar entgegenzustellen.

Die Kommunikation von und nach Kallisto wird w"ahrend des Angriffs durch St"orsender der Protektoratstruppen unterbrochen. Die Fregatte Isamu des Kombinats wird gewarnt, sich nicht an den Kampfhandlungen zu beteiligen und den Orbit um Valhalla nicht zu verlassen. Mittels Landungsshuttles wird eine Besetzung Valhallas eingeleitet. Die Protektoratstruppen besetzen den Raumhafen und beginnen zusammen mit der Garnison, Knotenpunkte auf Valhalla einzunehmen. Hierf"ur werden Landungstruppen bei mehreren Zug"angen nach Valhalla abgesetzt. Die Landungstruppen des Guardian-Klasse-Schlachtkreuzers handeln entsprechend. Die Kommunikation auf Valhalla bricht durch ein St"orsendergewitter beider Parteien weitestgehend zusammen. 

Ein zentraler Punkt des Angriffs ist das Fusionskraftwerk, das derzeit von Nemessis gehalten wird. Die Charaktere, die sich zu diesem Zeitpunkt im Leitstand des Fusionskraftwerks in Breidablik aufhalten, werden zwangsl"aufig in das Kampfgeschehen einbezogen. Blackheart ben"otigt etwa 30 Minuten, um Kampfverb"ande auf Kallisto abzusetzen, und weitere 15 bis 30 Minuten, um Breidablik einzunehmen. Die Konzernkr"afte erreichen das Kampfgebiet etwa 10 Minuten sp"ater. Blackheart f"uhrt in Breidablik potentiell einen Zweifrontenkrieg: Auf der einen Seite stehen ihr das Luna-Syndikat entgegen, auf der anderen Seite kommen ihr die Truppen der Zeus II-1 hinterher. Ihr Vorteil liegt darin, dass sie sich bereits lange zuvor auf eine Besetzung Valhallas vorbereitet hat, das Gel"ande gut kennt und M"oglichkeiten geschaffen hat, durch das ewige Eis in wichtige Teile des Bezirks einzudringen. Ein weiterer Vorteil ist ihre Verbindung zum Luna-Syndikat. Das Syndikat wird sich ihr am Ende nicht entgegenstellen. Eventuell haben die Charaktere bereits ein gemeinsames Vorgehen abgestimmt.

Nemessis und seine Verb"undeten erfahren fast ohne Verz"ogerung durch ihre eigene "Uberwachung des Orbits "uber Valhalla von dem Angriff. Sie machen sich so gut wie m"oglich auf die bevorstehenden K"ampfe gefasst. Die Protektoratstruppen st"urmen das Kontrollzentrums des Kraftwerks, was zu einem \cref{sec:attackfusionplant} beschriebenen Schusswechsel f"uhrt. Das Sunshine Hotel wird parallel dazu von Protektoratsstreitkr"aften besetzt. Die sich dort aufhaltenden Gangster ergeben sich sofort. Das Protektorat beginnt, den Bezirk rund um das Kraftwerk abzusichern. Das Gebiet ist nun von Barrikaden und K"ampfen mit den spinnenbeinigen Kampfrobotern der USI durchzogen.

\begin{remarks}
	Die K"ampfe um die Stadt sind Hintergrundgeschichte. Der Spielleiter sollte die Spieler nicht in gr"o\3erem Ma\3e an den K"ampfen teilnehmen lassen, um nicht unn"otig Zeit zu verlieren. Dennoch sollten die Spieler die Zuspitzung des Geschehens "`hautnah"' miterleben, um die Dramatik ihrer Handlungen zu verstehen.
 \end{remarks}
