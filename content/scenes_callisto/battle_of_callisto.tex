%% Copyright 2019 Bernd Haberstumpf
%% License: CC BY-NC
% !TeX spellcheck = de_DE
\pageimage{images/battle_of_callisto.jpg}
\newsection{Kampf um Kallisto}

Erf"ahrt Blackheart von den Attent"atern aus den Reihen des Protektoratsmilit"ars oder dass es sich bei einem der Attent"ater um Artisan handelt wird sie das Protektoratsmilit"ar in den Kriegszustand versetzen. Gleiches gilt f"ur den Fall, dass sie erf"ahrt, dass die USI hinter der Initiative des Konzernrates steht. Ist ihr bekannt, dass es sich bei einem der Attent"ater um Artisan handelt wird sie Thunderbolt beauftragen ihn auszuschalten und Vandermool informieren da Avenger nicht kontaktiert werden kann. Ist bekannt, dass Omegas als Attent"ater in Frage kommen erh"alt Commander Lockhead den Befehl umgehend alle seine Soldaten vom Gipfeltreffen abzuziehen. Danach befielt sie den Soldaten auf der Martell eine sofortige Besetzung Valhallas. Wird Blackheart von den Charakteren nicht informiert wird sie eine Besetzung von Valhalla nach dem Attentat auf dem Gipfeltreffen befehlen. Gleichzeitig mit der Besetzung von Valhalla befehligt sie die Martell den im Orbit von Valhalla befindlichen Guardian Kreuzer anzugreifen. Die Donar erh"alt den Befehl die Zeus II-2 abzufangen. "Uber die Angriffe setzt sie nur ihre Schlachtschiffe in Kenntnis.

Befiehlt Blackheart ihren Angriff setzt auch der Guardian Kreuzer "uber Valhalla alle seine verbliebenen Landungsschiffe in Richtung Valhalla in Bewegen und beginnt sich ohne menschliche Besatzung der Martell entgegenzustellen.

Die Kommunikation von und nach Kallisto wird durch St"orsender der Protektoratstruppen unterbrochen, Der Zerst"orer des Kombinats wird gewarnt sich nicht an den Kampfhandlungen zu beteiligt und auch nicht den Orbit um Valhalla zu verlassen. Mittels Landungspods wird eine Besetzung Valhalls eingeleitet. Mit den Truppen des Protektorats trifft auch Blackheart auf Kallisto ein. Die Protektoratstruppen besetzen den Orbitalhafen, und beginnen zusammen mit der Garnison  Knotenpunkte auf Valhalla einzunehmen. Hierf"ur werden Landungstruppen bei mehrere Zug"angen nach Valhalla abgesetzt. Die Landungstruppen des Guardian Kreuzers machen es ihnen gleich. Die Kommunikation auf Valhalla bricht durch ein St"orsendergewitter weitestgehend zusammen. 

Ein zentraler Punkt des Angriffs ist das Fusionskraftwerk das derzeit von Nemessis gehalten wird. Befinden sich die Charaktere zu diesem Zeitpunkt noch in Breidablik sind sie zum Zeitpunkt des Angriffs im Leitstand des Fussionskraftwerks und werden zwangsl"aufig in das Kampfgeschehen einbezogen. Blackheart ben"otigt in etwa 30 Minuten um Kampfverb"ande auf Kallisto abzusetzen und weitere 15 bis 30 Minuten um mit der Besetzung von Breidablik zu beginnen. Die Konzernkr"afte erreichen das Kampfgebiet etwa 10 Minuten sp"ater. Blackheart f"uhrt in Breidablik einen zwei Frontenkrieg. Auf der einen Seite stehen ihr das Luna-Syndikat entgegen, auf der anderen Seite kommen ihr die Truppen der Zeus II-1 hinterher. Ihr Vorteil ist es, dass sie sich bereits lange vorher auf eine Besetzung Valhallas vorbereitet hat, das Gel"ande gut kennt und M"oglichkeiten geschaffen hat durch das ewige Eiss in wichtige Teile des Bezirks einzudringen. Ein weiterer Vorteil ist ihre Verbindung zum Luna Syndikat. Das Syndikat wird sich ihr nicht entgegenstellen. Eventuell haben die Charaktere ein gemeinsames Vorgehen bereits abgestimmt.

Nemessis und seine Verb"undeten erfahren von dem Angriff fast ohne Verz"ogerung durch ihre eigene "Uberwachung des Orbits "uber Valhalla. Sie machen sich so weit es geht auf K"ampfe gefasst. Die Protektoratstruppen durchbrechen die Deckenanlage des Kontrollzentrums der Kraftwerks. Es kommt zu einem kurzen Schu\3wechsel den Nemessis beendet. Das Sunshine Hotel wird parallel dazu von Omegas gest"urmt. Hier treffen sie auf keine Gegenwehr. Die Gangster ergeben sich sofort. Das Syndikat und das Protektorat beginnen den Bezirk rund um das Kraftwerk abzusichern. Das Gebiet ist im folgenden von Barrikaden und K"ampfen mit den Spinnenbeinigen Androiden der USI durchzogen.

\begin{remarks}
	Die K"ampfe um die Stadt sind Hintergrundgeschichte. Der Spielleiter sollte die Spieler nicht selbst an K"ampfen teilnehmen lassen, um nicht unn"otig Zeit zu verschwenden. Allerdings sollten die Spieler die Zuspitzung des Geschehens "`hautnah"' miterleben um die Dramatik ihrer Handlungen zu verstehen.
\end{remarks}
