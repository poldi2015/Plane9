%% Copyright 2019 Bernd Haberstumpf
%% License: CC BY-NC
\newchapter{Rollenspiel}

\newsection{Kommunikation}

Das jovianische System besitzt erst nach der Besiedelung durch das Protektorat eine nennenswerte Infrastruktur. Durch den rasanten Aufbau hat jedoch ein Gro\3teil der Einrichtungen einen provisorischen Charakter. Diese sind zudem auf das Notwendigste beschr"ankt. Das gilt auch f"ur die Kommunikations-- und Informationssysteme. Wo auf Erde und Mars ein gutes ComNetz jegliche Information an jedem Punkt im System bereit stellt, steht ein voll ausgebautes ComNetz hier nur in Konzernsektoren und beim Milit"ar bereit.

Im jovianischen System  betreiben die einzelnen Monde und Stationen meist ein autonomes Kommunikationssystem, das mit den anderen Siedelungen und Anlagen nur in Teilen integriert ist. Ein ComNetz ist nur zwischen Monden und Stationen der Cynarian Corporation eingerichtet. Alle weitere Kommunikation zwischen Monden und Stationen erfolgt über Einzelverbindungen mittels der Sateliten rund um den Jupiter. Diese Verbindungen erlauben nur Kommunikation und Nachrichten zwischen Personen. Da die Stationen und Monde oft mehrere Millionen Kilometer voneinander entfernt sind, muss mit einer Kommunikationsverz"ogerung von mehreren Sekunden gerechnet werden.

Durch den steten unkontrollierten Zustrom von Mutantenfl"uchtlingen, Gl"ucksrittern und neuen Firmen stehen kaum Informationen zu Einzelpersonen und ans"assigen Institutionen und Unternehmen bereit.

\newsection{Fortbewegung und Reisezeiten}

Gro\3e Unternehmen wie Cynarian, das Protektoratsmilit"ar und die Protektoratsadministration betreiben eigene Shuttleflotten innerhalb des Systems soweit n"otig. Einige wenige Personen besitzen ebenfalls Shuttles. Der Rest der Fl"uge zwischen den Jupitertrabanten wird durch Transportunternehmen und Schiffseignern von Shuttles bereitgestellt. Einen Flug bekommt man am einfachsten im Raumhafen der jeweiligen Station. Da die Entfernungen im Jupterystem oft enorm sind sind Reisezeiten von 1 bis 2 Wochen bei zivilen Schiffen "ublich.

\newsection{Regelwerk}

Der Kampagne stellt ein eigenes sehr einfach gehaltenes Rollenspielsystem bereit. Alternativ bieten sich aber auch Systeme wie\emph{Shadowrun} an. Menschen und Mutantenrassen lassen sich leicht auf die Rassen und Archetypen von Shadowrun abbilden. 

Waffen und Cyberware finden ebenfalls ihre Pendants bei Shadowrun. Die Matrixregeln lassen sich f"ur ComNetz und Gehirnscans anwenden. Mehr dazu im Anhang.