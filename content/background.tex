%% Copyright 2019 Bernd Haberstumpf
%% License: CC BY-NC
% !TeX spellcheck = de_DE
\newchapter{Einf"uhrung}

\newsection{Vorgeschichte}

Durch ein menschenverachtendes Vorgehen von k"unstlichen Intelligenzen beim Kampf gegen die Mutantenrebellion sind KIs zum wiederholten Mal in den Augen der Menschheit in Ungnade gefallen. Da die Cynarian Corporation f"ur den Aufbau einer Industrie auf dem Jupiter enorme Ressourcen ben"otigt, nimmt das Unternehmen die aktuelle Stimmung zum Anlass, um unter anderem die vielversprechende aber teure KI-Forschungsabteilung unter der Leitung von Prof.~Dr.~Naratova auf der Mars Orbitalstation Neu-Gr"oning zu streichen.

Im Zuge der Besiedelung des Jupiter wird Neu-Gr"oning daraufhin zum Jupiter geschleppt und als Nike Station zum Verwaltungs- und F\&E St"utzpunkt der Cynarian Corporation im Jovianischen System eingerichtet. Neben Forschungslaboren unter der Schirmherrschaft der Cynarian Corporation werden auch Einrichtungen von Zulieferfirmen auf Nike zugelassen. F"uhrende Mitarbeiter der ehem.~KI-Abteilung gr"unden daraufhin unter der Leitung von Prof.~Dr.~Naratovas den unabh"angigen Headware Zulieferer \emph{Neuro Intelligence} unter anderem auch mit Geldern aus Tarnfirmen der United Space Industries, kurz USI. Die Labore und Produktionsst"atten der Neuro Intelligence werden in alten R"aumlichkeiten der ehemaligen KI-Abteilung auf Ebene 9 der Nike Station eingerichtet. Neuro Intelligence beliefert im Jovianischen System die Kliniken auf Kallisto mit neuronaler Soft- und Hardware. Da die Implantate, die die Neuro Intelligence im Auftrag der Kliniken fertigt, Ma\3anfertigungen sind, erh"alt Neuro Intelligence weitreichende Informationen "uber die zuk"unftigen Tr"ager der Implantate.

Seit der Gr"undung entwickelt Neuro Intelligence im geheimen Auftrag mit Geldern und Technologie des USI-Geheimdienstes KI-Implantate unter dem Decknamen \textbf{Operation P9}. Dabei gelingt es, die KI symbiotisch mit dem menschliche Gehirn zu verbinden und damit der KI je nach Auspr"agung die "Ubernahme des Tr"agers zu erlauben. Nicht jedes Gehirn ist f"ur die "Ubernahme geeignet. Inwieweit das Gehirn und die KI eine Symbiose eingehen, ist nicht bekannt. In der ersten Erprobung werden nur die Gehirne von Mutanten manipuliert. Die Einbringung der KI erfolgt durch ein modifiziertes \emph{Kontrollmodul}, in das die KI eingebettet ist. Die KI selbst ist ein Verbund aus autonomen Nanobots, die sich im Gehirn automatisch verbreiten und sich mit den Synapsen des Gehirns verbinden.

\newsection{Chronologie der Attentate}

Mit Hilfe von durch KI manipulierte Mutanten startet die USI eine Anschlagsreihe auf das Protektorat, um die neue Technologie zu erproben und die Stellung des Potektorats und die der Cynarian Corporation zu diskreditieren.

Bevor die Charaktere in das Spielgeschehen eingebunden werden, hat es bereits mehrere Vorf"alle im Jovianischen System gegeben, die teils auf Attentate zur"uckzuf"uhren sind:

\begin{description}
\item [Vor 10 Wochen] Shuttleabsturz durch Pilotenfehlverhalten auf dem Hangardeck der Minenkolonie \emph{Hellgate}. Vom Shuttle Absturz 
      erfahren die Ermittler erst durch Nachforschungen auf Hellgate. Bei dem Shuttleabsturz handelt es sich um kein Attentat.
\item [Vor 9 Wochen] Fehlfunktion eines Minenschleppers der Hellgate Station durch fehlerhaft programmierte Steuerungskomponenten. Die 
      Reparatur dauert noch an. Der verantwortliche Ingenieur kann nicht festgestellt werden. "Uber die Schlepperfehlfunktion wird der Cynarian Ermittler beim ersten Briefing aufgekl"art. Das Attentat wurde durch den Alpha-Mutanten \emph{Hannibal} durchgef"uhrt der allerdings bereits zwei Wochen vor dem Schlepperunfall auf die Mine HeM03 versetzt wird.      
\item [Vor 7 Wochen] Die \emph{Mine HeM03} auf dem Jupiter wird zerst"ort. Ein Teil der Besatzung kann mit dem  Rettungsschuttle fliehen. 
      Als mutma"sliche Attent"aterin gilt die Alpha-Mutantin Sent die bei dem Attentat get"otet wird. Beim eigentlichen Attent"ater handelt es sich um den Alpha-Mutant Hannibal der mehrere Minenarbeiter sowie Sent t"otet. "Uber den Minenunfall werden beide Chefermittler w"ahrend ihres ersten Briefings informiert.
\item [Vor zwei Wochen] Explosion beim Anbau von neuen Habitaten an den "au\3eren Ring des Armageddon Raumkomplexes. Zug"ange und ein
      ehemaliger Frachter, der als Habitat dienen sollte, werden stark besch"adigt.  Es gibt mehrere Tote. Ursache ist eine Fehlfunktion einer Drohne. Der Attent"ater \emph{Slingshot} wird als solcher nicht erkannt. "Uber den Unfall auf Armageddon wird der Chefermittler des Protektorats bei der Besprechung mit Avenger aufgekl"art.
\item [Vor drei Tagen] Havarie der Mine HeM05. Die mutma\3liche Attent"aterin eine Mutantin mit dem Namen Pitch hat das
      Rettungsshuttle sabotiert und 3 von 5 Tr"agerballons zerst"ort. Danach st"urzt sie von der Galerie der Mine in den Abgrund. Die Mine kann durch den heldenhaften Einsatz von J"agern der Hellgate Station und mit Hilfe eines Minenschleppers gerettet und zur Hellgate Station gebracht werden. W"ahrend der Einsatzbesprechung der Ermittler sitzen die geretteten Minenarbeiter noch in einer Dekompressionskammer. "Uber die Sabotage werden beide Chefermittler w"ahrend ihres ersten Briefings informiert. Der eigentliche
      T"ater ist wie bei Mine HeM03 der Alpha-Mutant Hannibal, der zusammen mit den anderen "Uberlebenden gerettet wird.
\end{description}

\newsection{Die Charaktere}

Die Charaktere "ubernehmen die Rolle der Ermittler, die die Attentatsreihe aufkl"aren sollen. Die Kampagne sieht zwei Ermittler aus den Reihen der Cynarian Corporation und zwei Ermittler aus den Reihen des Protektorats vor. Beide Gruppen sind jeweils ihren eigenen Organisationen verpflichtet und sollten entsprechend handeln (ggf.~durch entsprechende W"urfelw"urfe einfordern).

Cynarian stellt einen Chefermittler und einen Assistenten bereit. Der Protektor stellt einen Chefermittler und einen Angeh"origen des Protektoratsmilit"ars bereit. Wird die Kampagne nur mit drei Spielern gespielt, entf"allt die Rolle des Cynarian Unterst"utzers und der Chefermittler "ubernimmt seine Rolle. Die F"uhrung der gesamten Operation "ubernimmt dann der Chefermittler des Protektorats. 

\newsubsection{Chefermittler Cynarian}
Beim Chefermittler der Cynarian Corporation sollte es sich um einen Norm, einen Mensch mit F"uhrungsqualit"aten und Ermittlererfahrung handeln. Denkbar ist zum Beispiel der Kommandant eines Raumschiffs vom Typ \emph{Corvette} aus dem Asteroideng"urtel zwischen Mars und Jupiter auf der Jagd nach Piraten. Der Chefermittler der Cynarian Corporation sollte als Milit"arangeh"origer klar den Anweisungen der Corporation folgen, auch wenn das bedeutet der Gruppe Informationen vorzuenthalten oder sonstig im Sinne des Konzerns zu handeln.

\newsubsection{Assistierender Ermittler Cynarian}
Der Assistent ist ein \emph{Psychonaut}, ein Mensch mit spezieller Ausbildung, der in die Gedanken einer anderen Person "uber eine Kopf zu Kopf Datenverbindung eintauchen kann. Er sollte als Konzern- oder auch als freier Agent auftreten. Der Assistent sollte versuchen seine spezielle F"ahigkeit vor der Gruppe geheim zu halten und nur die Konzernspitze "uber die detaillierten Ergebnisse seiner Untersuchungen informieren. Der Assistent sollte ein Norm sein.

\newsubsection{Chefermittler des Protektorats}
Der Chefermittler des Protektorats sollte ein Vertrauter des Protektors Avenger und dessen Stabs sein. Er k"onnte sich z.B.~um den Betrieb Armageddons und die Aufnahme von neuen Bewohnern, Fl"uchtlingen von der Erde, k"ummern. Er hat Ortskenntnisse und weitreichende Kontakte im Mutantenteil des jovianischen Systems. Der Chefermittler mu\3 deshalb ein Alpha-Mutant sein. Der Chefermittler des Protektorats hat das Wohlsein der Mutanten als pers"onliche Mission im Blut. Auch sollte er dem Herrscheranspruch des Protektoratsmilit"ars kritisch gegen"uber stehen.

\newsubsection{Assistierender Ermittler des Protektorats}
Der Assistent wird durch das Protektoratsmilit"ar gestellt und hat somit einen milit"arischen Hintergrund. Er ist ein \emph{Omega}, ein Kampfmutant. Sollte er sich von den Cynarian Mitarbeitern befehligt oder hintergangen f"uhlen w"urde er darauf reagieren. Direkte Anweisungen durch Konzernmitarbeitern wird er nicht akzeptieren, wenn es eine Konfliktsituation nicht unbedingt erforderlich macht. Der Omega f"uhlen sich dem Rest der Welt in Sachen St"arke und Willenskraft "uberlegen. Er misstraut Norms und den Konzernen und h"alt die zivilen Einrichtungen des Protektorats f"ur zu nachgiebig.

Inwieweit die Gruppe die vorprogrammierten Spannungen zwischen den Charakteren ausspielt, bleibt der Gruppe vorbehalten. Die Aufgabe des Spielleiters ist es, die Balance zu halten, um den Plot nicht zu gef"ahrden und stattdessen vorantreiben zu k"onnen. 