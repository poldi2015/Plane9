%% Copyright 2019 Bernd Haberstumpf
%% License: CC BY-NC
% !TeX spellcheck = de_DE
\newchapter{Einf"uhrung}

\newsection{Vorgeschichte}

Durch ein menschenverachtendes Vorgehen von k"unstlichen Intelligenzen beim Kampf gegen die Mutantenrebellion sind KIs zum wiederholten Mal in den Augen der Menschheit in Ungnade gefallen. Da die Cynarian Corporation f"ur den Aufbau einer Industrie auf dem Jupiter enorme Ressourcen ben"otigt, nimmt das Unternehmen die aktuelle Stimmung zum Anlass, um unter anderem die vielversprechende aber teure Neuro KI-Forschungsabteilung unter der Leitung von Prof.~Dr.~Naratova auf der Mars Orbitalstation Neu--Gr"oning zu streichen.

Im Zuge der Besiedelung des Jupter wird Neu--Gr"oning daraufhin zum Jupiter geschleppt und als Nike Station zum Verwaltungs- und F\&E St"utzpunkt der Cynarian Corporation im Jovianischen System eingerichtet. Neben Forschungslaboren von Cynarian werden auch Einrichtungen von Zulieferfirmen auf Nike zugelassen. F"uhrende Mitarbeiter der ehem.~KI-Abteilung gr"unden darauf hin unter der Leitung Prof.~Dr.~Naratovas den unabh"angigen Headware Zulieferer Neuro Inteligence unter anderem auch mit Geldern aus Tarnfirmen der United Space Industries, kurz USI. Die Labore und Produktionsst"atten der Neuro Inteligence werden in alten R"aumlichkeiten der ehemaligen KI-Abteilung auf Ebene 9 der Nike Station eingerichtet. Neuro Intelligence beliefert im Jovianischen System die Kliniken auf Kallisto mit neuronaler Soft- und Hardware. Da die Implantate, die die Nero Intelligence im Auftrag der Kliniken fertigt, Ma\3anfertigungen sind, erh"alt Neuro Intelligence weitreichende Informationen "uber die zuk"unftigen Tr"ager der Implantate.

Seit der Gr"undung entwickelt Neuro Intelligence im geheimen Auftrag und mit Geldern und Technologie des USI-Geheimdienstes KI Implantate unter dem Decknamen Operation P9. Dabei gelingt es, die KI symbiotisch mit dem menschliche Gehirn zu verbinden und damit der KI je nach Auspr"agung die "Ubernahme des Tr"agers zu erlauben. Nicht jedes Gehirn ist f"ur die "Ubernahme geeignet. In wie weit das Gehirn und die KI eine Symbiose eingehen ist nicht bekannt. In der ersten Erprobung werden nur die Gehirne von Mutanten manipuliert. Die Einbringung der KI erfolgt durch ein modifiziertes \emph{Command Modul} in dem die KI eingebettet ist. Die KI selbst ist ein Verbund aus autonomen Nanobots die sich im Gehirn automatisch verbreiten und sich mit dem Synapsen des Gehirns verbinden.

\newsection{Chronologie der Attentate}

Mit Hilfe von durch KIs manipulierte Mutanten startet die USI eine Anschlagsreihe auf das Protektorat, um die neue Technologie zu erproben und die Stellung des Potektorats und die der Cynarian Corporation zu diskreditieren.

Bevor die Charaktere mit Nachforschungen beauftragt werden, hat es bereits mehrere Vorf"alle im Jovianischen System gegeben, die teils auf Attentate zur"uckzuf"uhren sind:

\begin{description}
\item [Vor 10 Wochen] Shuttleabsturz durch Pilotenfehlverhalten auf dem Hangardeck von Hellgate. Vom Schuttleabsturz erfahren die 
      Ermittler erst durch Nachforschungen auf Hellgate. Bei dem Shuttleabsturz handelt es sich nicht um ein Attentat.
\item [Vor 9 Wochen] Schlepperfehlfunktion auf Hellgate durch fehlerhaft programmierte Steuerungskomponenten. Reparatur dauert
      mehrere Tage. Verantwortlicher Ingenieur kann nicht festgestellt werden. "Uber die Schlepperfehlfunktion wird der Cynarian Ermittler beim ersten Briefing aufgekl"art. Das Attentat wurde durch den Alpha Mutanten Hanibal durchgeführt der allerdings bereits zwei Wochen vor dem Schlepperunfall auf HeM03 versetzt wurde.      
\item [Vor 7 Wochen] HeM03 Mine auf dem Jupiter wird zerst"ort. Ein Teil der Besatzung kann mit dem  Rettungsschuttle fliehen. Als 
      mutma"sliche Attent"aterin gilt die Alpha Mutantin Sent die bei dem Attentat get"otet wird. Beim eigentlichen Attent"ater handelt es sich um den Alpha Hanibal der mehrere Minenarbeiter sowie Sent t"otet. "Uber den Minenunfall werden beide Chefermittler w"ahrend ihres ersten Briefings informiert.
\item [Vor zwei Wochen] Explosion beim Anbau von neuen Habitaten an dem Armageddon Ring. Zug"ange und ein
      ehemaliger Frachter, der als Habitat dienen sollte, werden stark besch"adingt.  Mehrere Tote. Ursache ist eine Fehlfunktion einer Drohne. Der Attent"ater Slingshot wird als solcher nicht erkannt. "Uber den Unfall auf Armageddon wird der Chefermittler des Protektorats bei der Besprechung mit Avenger aufgekl"art.
\item [Vor drei Tagen] Havarie der Mine HeM05. Die mutma"sliche Attent"aterin eine Mutantin mit dem Namen Pitch hat das
      Rettungsshuttle sabotiert und 3 von 5 Tr"agerballons zerst"ort. Danach st"urzt sie von der Gallerie der Mine in den Abgrund. Die Mine kann durch den heldenhaften Einsatz von J"agern der Hellgate Station und mit Hilfe eines Minenschleppers gerettet und zur Hellgate Station gebracht werden. W"ahrend der Einsatzbesprechung der Ermittler sitzen die geretteten Minenarbeiter noch in einer Dekompressionskammer. "Uber die Sabotage werden beide Chefermittler w"ahrend ihres ersten Briefings informiert. Der eigentliche
      Täter ist wie bei Mine HeM03 der Alpha Mutant Manibal der zusammen mit den anderen Überlebenden gerettet wird.
\end{description}

\newsection{Die Charaktere}

Die Charaktere "ubernehmen die Rolle der Ermittler, die die Attentatsreihe aufkl"aren sollen. Die Kampagne sieht zwei Ermittler aus den Reihen der Cynarian Corporation und zwei Ermittler aus den Reihen des Protektorats vor. Beide Gruppen sind jeweils ihren eigenen Organisationen verpflichtet und sollten entsprechend handeln (ggf.~durch entsprechende W"urfelw"urfe einfordern).

Cynarian stellt einen Chefermittler und eine Assistenten bereit. Der Protektor stellt einen Chefermittler und einen Angeh"oringen des Protektoratsmilit"ars bereit. Wird die Kampagne nur mit drei Spielern gespielt entf"allt die Rolle des Cynarian Unterst"utzer und der Chefermittler "ubernimmt seine Rolle. Die F"uhrung der gesamten Operation "ubernimmt dann der Chefermittler des Protektorats. 

Beim Chefermittler der Cynarian Corporation sollte es sich um einen Norm mit F"uhrungsqualit"aten und Ermittlererfahrung handeln. Denkbar ist zum Beispiel der Kommandant einer Corvette aus dem Asteroideng"urtel auf der Jagd nach Piraten. Der Cynarian Assisten sollte ein Norm mit Ortskenntnis im jovianischen System sein. Der Assistent ist ein Psychonaut und sollte als Agent oder Detektiv auftreten. Der Chefermittler der Cynarian Corporation sollte als Milit"arangeh"origer klar den Anweisungen der Corporation folgen auch wenn das bedeutet der Gruppe Informationen vorzuenthalten oder im Sinne des Konzerns zu handeln. Der Assistent sollte versuchen seine spezielle F"ahigkeit vor der Gruppe geheim zu halten und nur die Konzernspitze "uber die detailierten Ergebnisse seiner Untersuchungen zu informieren.

Der Chefermittler des Protektorats sollte ein Vertrauter des Protektors Avenger und dessen Stabs sein. Er k"onnte sich z.B.~um den Betrieb Armageddons und die Aufnahme von neuen Bewohnern, Fl"uchtlingen von der Erde, k"ummern. Er hat Kontakte und Ortskenntnisse speziell im Mutantenteil des jovianischen Systems. Der Chefermittler sollte deshalb ein Alpha Mutant sein. Der Assistent wird durch das Protektoratsmilit"ar gestellt und hat somit einen milit"arischen Hintergrund. Der Assistent ist ein Omega. Der Chefermittler des Protektorats hat das Wohlsein der Mutanten als per"onliche Mission im Blut. Sollten er sich von den Cynarion Mittarbeitern hintergangen f"uhlen sollte er darauf reagieren. Auc sollte er dem Vorherrschungsanspruch des Protektoratsmilit"ars kritisch gegen"uber stehen. Die Omega f"uhlen sich dem Rest der Welt in Sachen st"arke und Willenskraft "uberlegen. Sie misstrauen zu einem gewissen Grad den Norms und den Konzernen und halten die zivielen Einrichtungen des Protektorats f"ur zu nachgiebig.

In wie weit die Gruppe die vorprogrammierten Spannungen zwischen den Charakteren ausspielt bleibt der Gruppe vorehalten. Die Aufgabe des Spielleiters ist es die Balance zu halten um den Plot nicht zu gef"ahrden und voran zu treiben. 
