\section{Szenen}

\subsection{Prolog (optional)}

Um die Ermittler auf den Plot, ihren Charakter und auf die vorherrschenden politischen Gemengelage vorzubereiten bietet es sich an mit jedem der Spieler einzeln eine Einf"uhrungsrunde zu spielen:

Der Vertraute des Chefermittlers ist Colonel Scholz. Er wird den Chefermittler auf das Treffen mit Vandermool, mit dem der Ermittler selbst noch nicht viel zu tun hatte, einweisen. Vandermool ist auf ein gutes Verh"altnis mit dem Protektorat bem"uht wird aber bei einer Zusammenarbeit die F"uhrung behalten wollen. Zudem ist nat"urlich nicht klar ob die Mutanten in die Anschl"age verwickelt sind oder evtl. ~sogar Cynarion selbst deshalb ist Vorsicht geboten.

Für den Assistenten bietet sich eine Einweisung in T"atigkeit als Psychonaut an und ihn dabei gleich direkt die Geschichte einzubinden. Er erh"alt auf dem Mars die Aufgabe einen Agenten der \emph{USI}, der bei der R"uckreise vom jovianischen System von Piraten gefangen und an Cynarion ausgeliefert wurde, zu befragen. Der Agent ist Erstkontakt zu der von Prof.~Dr.~Naratova betriebenen \emph{Neuro Iintelligence}.

Der Vertraute von Avenger trifft sich am Vorabend mit Artisan um einen zu heben. Dieser erkl"art ihm von den Vorf"allen die der Ermittler jedoch schon kennt und weiht ihn bereits ein das ein Treffen mit Repr"asentanten aus der Cynarion geplant ist um die Vorkommnisse zu untersuchen.

Den Omega nimmt sein Vorgesetzter Thunderbolt zu einem Treffen am Raumhafen von Armageddon mit Blackheart mit. Blackheart trifft mit der \emph{Martell} ein um an der Einweisung der Ermittler vor Ort teilzunehmen. Blackheart nimmt den Charakter zur Seite und weist ihn an ein waches Auge auf die Ermittlungen zu haben da der Cynarian Seite nicht voll zu trauen ist und vorraussichtlich Informationen vorenthalten werden.
Der Ermittler bekommt den Befehl t"aglich oder nach bei wichtigen Erkenntnissen Report an Thunderbolt zu leisten. In die Szene sollten mitlit"arische Gepflogenheiten mit einflie\3en. Auch sollte das Treffen ein Bild von der bereits l"agend"aren Anf"uhrerin der Protektroratstruppen und deren Adjutenten formen.

\subsection{Einweisung bei Cynarian}

Der Cynarian Chefermittler wird durch Eric Vandermool, Colonel Scholz und Dr.~Petrova in einem Konferenzraum der Cynarian Sektion auf Armageddon als erster eingewiesen. Der Charakter wird in die R"aume durch den Sekret"ar Vandermools Herny Longdale gebracht.Das B"uro Vandermools ist sehr ger"aumig, schlicht und k"uhl aber erlesen eingerichtet. Vandermool wird das Gespr"ach von seinem Schreibtisch aus f"uhren. Vandermool ist ganz klar dominant in dem Gespr"ach und vollkommen souver"an. Scholz und Dr.~Petrova sind als kompetent und zielstrebig effizient bekannt. Scholz ist ein erfahrener Milit"arangeh"origer.

Der Ermittler erf"ahrt von der Sabotage auf der Mine HeM05 vor drei Tagen, der Havarie der Mine HeM03 vor zwei Wochen und der Fehlfunktion der Schlepperinsel vor drei Wochen. Nur der Vorfall auf der Mine HeM05 wird bereits als Attentat eingestuft. Es wird aber gemutma\3t, dass es sich auch bei anderen Vorkommnissen um Attentate handelt. Die USI als potentieller Drahtzieher wird direkt angesprochen. Vandermool ist offen beunruigt und betont, dass weitere Vorkommnisse nicht tragbar w"aren. Der Ermittler wird aufgefordert, sich dann bei Protektor Avenger als offizieller Ermittler der Cynarian Corporation zu melden. Er soll Scholz "uber den Stand der Ermittlung jederzeit auf dem Laufenden halten. Kontaktmann des Ermittlers ist also Scholz oder enry Longdale f"ur den direkten Kontakt zu Vandermool.bW"ahrend der Ermittlung stehen die Cynarian Ermittler im Dienste der inneren Sicherheit von Cynarian. Alle Ergebnisse unterliegen der Geheimhaltung.

Der zweite Ermittler ist bereits in den R"aumlichkeiten h"alt sich aber zur"uck. Die Anwesenheit des zweiten Ermittlers kann z.B.~erst am Ende der Erl"auterungen der Gegebenheiten erfolgen um zu zeigen das er bereits eingewiesen wurde und potentiell Wissen besitzt das dem Chefermittler nicht zug"anglich gemacht werden soll.


\begin{remarks}	
	Wie vertraut die Cynarian F"uhrung mit dem Protektorat ist, ist den Ermittlern nicht bekannt. Die genauen Umst"ande die, zur Besiedelung des jovianischen Systems gef"uhrt haben, sind ebenfalls nicht bekannt.
	
	Detaillierte R"uckfragen sind bei diesem Gespr"ach unangebracht. Vandermool bittet die Ermittler sind bzgl.~Fragen, t"aglicher Reports vertrauensvoll an Henry Longdale zu wenden. Henry Longdale ist damit der direkte Ansprechpartner der Cynarian Ermittler wird aber selbst keine Entscheidungen treffen. Vandermool ist damit nicht im Zugzwang irgendwelche Informationen bereit zu stellen.
\end{remarks}

\subsection{Einweisung beim Protektorat}

Der Chefermittler des Protektorats wird durch Protektor Avenger und seinen Stellvertreter Artisan und seinen Leibw"achter Hato in den Konferenzr"aumen des Protektorats auf Armageddon eingewiesen. Die Atmosph"are ist freundschaftlich. Der Ermittler erf"ahrt von den Vorkommnissen auf den Minen HeM03 und HeM05 und der Explosion beim Anbau der Habitate. Die Havarie der Mine HeM03 und das Habitatsungl"uck werden derzeit noch als Unf"alle gewertet. Der Chefermittler wird gebeten, gewonnene Erkenntnisse an Avenger Stellvertreter Artisan zu berichten und als geheim einzustufen.

Avenger erkl"art, dass die Ermittlung mit Vandermool und Blackheart abgestimmt ist. Im Vertrauen bittet Avenger seinen Ermittler, die Vertreter der Cynarian Corporation mit Vorsicht zu genie\3en, letztendlich ist Cynarian nach wie vor ein Konzern mit eigener Agenda. Avenger stellt daraufhin die Ermittler der Cynarian Corporation vor. Die Ermittler der Cynarian Corporation werden daf"ur hinzu gebeten.

Nach Beendigung des Gespr"achs mit dem Protektor wird der Ermittler des Protektorats per ComLink von Blackheart angerufen und aufgefordert sich alleine im Kommandostand auf Armageddon einzufinden. Beim Eintreffen des Ermittlers bespricht Blackheart gerade Einsatzpl"ane mit zwei anderen Omegas und einer weiteren Person am erh"oht gelegenen "`Kartentisch"'. Nach ca.~einer Minute  wendet sie sich eher beil"aufig "uber den R"ucken hinweg dem Ermittler zu. Sie fragt nach seinem Auftrag und seinem Vorgehen. Dann wendet sie sich ihm direkt zu und erkl"art ihm unmissverst"andlich, dass die Vorg"ange die Sicherheit des Protektorats gef"ahrden und deshalb als Angriffe auf das Protektorat zu bewerten seien, denen mit milit"arischen Mitteln zu begegnen ist. Aus diesem Grund stellt sie dem Ermittlerteam einen weiteren Ermittler aus den Reihen der Protektoratsstreitkr"afte zur Seite. Der zus"atzliche Ermittler ist die weitere Person am Kartentisch. F"ur Blackheart ist damit das Gespr"ach beendet und sie wendet sich wieder ohne Verabschiedung ihren Omegas zu.

\begin{remarks}
	Die Besprechung mit Avenger sollte der Chefermittler zun"achst alleine mit dem Spielleiter spielen. Die Spieler der Cynarian Ermittler werden erst im zweiten Schritt dazu genommen. Die Einweisung des zweiten Ermittlers des Protektorats kann dadurch erfolgen, dass der Chefermittler des Protektorats Blackheart informiert.
	
	Avenger ist zwar inzwischen Diplomat und Staatslenker, aber im Wesen freundlich kollegial und offen umg"anglich. Sein Leibw"achter Hato ist der Typ japanischer Samurai und h"alt sich unaufdringlich im Hintergrund.
	
	Blackheart ist eine legend"are Kommandantin mit aufbrausendem Temperament. Sie k"ampft gegen Avenger um die Kontrolle im Protektorat und setzt mit allen Mitteln ihren Willen durch. Das Treffen im Kommandostand soll zwar einsch"uchternd wirken, ist aber nicht offen feindselig.
	
	Der Ermittler aus den Reihen der Protektoratsstreitkr"afte ist dem Milit"ar und damit Blackheart verpflichtet. Er hat den Befehl, Thunderbolt auf dem Laufenden zu halten und ggf.~auch gegen den Willen der anderen Ermittler nach eigenem Ermessen oder im Auftrag der Milit"arf"uhrung Ma\3nahmen zu ergreifen.
	
	N"ahere Information zu den einzelnen Vorf"allen erfahren die Charaketere bei der Einweisung nicht. Sie k"onnen aber von Avenger den Hinweis erhalten den Unfall bei der Erweiterung Armageddons als erstes zu bearbeiten.
\end{remarks}

\subsection{Das Frachterungl"uck auf Armageddon}

Der Vorfall vor zwei Wochen erfolgte beim Anbau eines ausgemusterten Frachters an den Habiatsring von Armageddon. Ansprechpartner dabei ist der Alpha Sunny als Bauleiter mit Zust"andigkeit f"ur den blauen Sektor, Bauabschnitt 3. Der Blaue Sektor umfasst die Wohnbereiche des Habiats und wird st"andig erweitert.

Von Sunny erfahren die Charaktere, dass der Frachter der in 15km Entfernung f"ur den Einbau vorbereitet wurde mittels ferngesteuertern Drohnen in die Andockposition gebracht werden sollte. Dabei ist offensichtlich eine der Drohnen au\3er Kontrolle geraten und hat den Frachter in den Armageddon Ring gerammt. Durch den Unfall wurden 12 Frachtcontainern, die als weitere Quartiere dienen sollten, ein Teil des Frachters und 6 bestehende Quartiereinheiten zerstört oder stark besch"adigt; Ein Teil des Bauabschnitts 3 wurde dem Vakuum ausgesetzt;zwei Arbeiter starben; einer der Drohnenpiloten wird vermisst; Reparaturarbeiten dauern noch an. Weitere Tote konnten vermieden werden da die in Konstruktion befindlichen Bereiche weitreichend gesperrt werden.

Im Gespr"ach mit Sunny, das immer wieder durch andere Personen unterbrochen wird, erfahren die Charaktere da\3 das Einpassen und Andocken des Frachters durch 5 Spezialisten durchgef"uhrt wurden. Diese Spezialisten waren erst rund zwei Wochen vor dem Unfall samt Equipment von  der Protektoratsgarnison auf Callisto nach Armageddon versetzt worden um die Aufbauarbeiten mit neuer Technologie, Drohnen zu unterst"uten.

Wenn Sunny von den Spezialisten spricht redet er nur von der \emph{Cowboy Brigade}. Die Cowboy Brigade besteht aus 5 Alpha Mutanten mit den Namen \emph{Stetson}, \emph{Quickfinger Rod}, \emph{Joe Rider}, \emph{Tom Gunslinger} und \emph{Slingshot}. Die Cowboy Brigade wird von Summy als ein lustiger Haufen bezeichnet die sich wahlweise als betont coole Cowboys (wie aus alten Holos bekannt) geben oder mit allem Werkzeug das sie gerade in der Hand halten salutieren. Unabh"angig davon sind sie aber sehr gut ausgebildete und gewissenhafte Techniker.

Die Cowboy Brigade war beim Einplatzieren ds Frachters mit einem Wartungsshuttle der Armageddon Station zusammen unterwegs um von dort aus jeweils die Drohnen fern zu steuern. Seit dem Vorfall wird das Mitglied der Cowboy Brigade \emph{Slingshot} vermi\3t was die Truppe sehr best"urzt hatte. Nachdem die Suche nach Slingshot aufgegeben worden war hat die Cowboy Brigade Armageddon verlassen und ist wieder nach Valhalla auf Callisto zur"uck gekehrt.

Sunny kann auf R"uckfrage die Protokolle der Kommunikation auf dem Shuttle wie auch Kamera Aufnahmen vom Shuttle und von der Station bereit stellen.

Aus den Mittschnitten der Funkprotokolle erf"ahrt man dass Slingshot kurz vor dem Andocken des Frachters seine Drohne pl"otzlich maximal beschleunigt hatte. Stetson der versucht hat ihn "uber Helmmikrofon anzusprechen hat bekam zun"achst keine Antwort. Erst eine Minute sp"ater
meldete sich Slignshot mit einem panischen Aufschrei zur"uck und versuchte seine Drohne wieder unter Kontrolle zu bekommen. Er behauptete
dann seine Drohne h"atte eine Fehlfunktion gehabt.

Um den Schaden wieder in Ordnung zu bringen verlie\3 Slingshot kurze Zeit sp"ater das Shuttle um zum Frachter "uber zu setzen und die Drohne funktionst"uchtig zu machen. Dabei geriet er aus den Aufnahmebereichen der Kameras und war danach nicht mehr auffindbar.

Such und Rettungskr"afte konnten den Unfallbereich erst betreten und absichern nachdem der Hauptanteil der umherschwirrenden Tr"ummer au\3er Reichweite getrieben waren.

\begin{remarks}
	Die folgenden Informationen d"urfen zu diesem Zeitpunkt noch nicht weiter gegeben werden:
	
	Slingshot ist einer der Attent"ater die durch eine von der USI bereitgestellten KI "ubernommen wurde. Er ist einer der beiden Versuchspersonen an denen die neue Technologie im Feld ausprobiert wird. Nach dem "Ubersetzen zum Frachter betritt Slingshot Armageddon ungesehen wieder und taucht mit Unterst"utzung von Artisan auf der Station zun"achst unter.
\end{remarks}

\subsection{Eintreffen auf Hellgate}

Die HeM05 ist beim Eintreffen der Charaktere an der gigantischen Schlepperinsel der Hellgate Station angedockt. Die Schlepperinsel ist ein 2 Kilometer langes und breites Raumfahrzeug das mit gewaltigen Schubd"usen bis in die "au\3eren Athmosph"arenregionen des Jupiters eintauchen kann um dort die HE--3 Mienen abzusetzen oder einzusammeln. Die Schlepperinstel schwebt beim Anflug auf Hellgate majest"atisch nahe dem Mond Adrastea "uber der gewaltigen Fl"ache des Jupiters. Kleinste Partikel bilden eine Schleier auf diesem niedrigen Orbit von 130'000 km "uber dem Planeten. Hellgate befindet sich vollst"andig bis auf den Anflugtunnel, Not-- und Wartungsausg"ange im Kern  Mondes. Die Station selbst besteht aus dem Raumhafen, techischen Anlagen, Lagerhallen und R"aumen und Wohnquartieren, Lokale, Bars und L"aden. Im Ganzen umfasst die Anlage ca.~30 km\textsuperscript{3}. Wie in alles neuen eilig aufgesetzen neuen Anlagen befinden sich viele Provisorien, nicht abgeschlossene G"ange und herumstehendes Material in der Station.

Beim Eintreffen im Raumhafen herrscht reger Betrieb, eine gro\3e F"ahre bringt gerade neue Minenarbeiter und holt Mitarbeiter die nach Callisto abreisen m"ochten. Eine Reihe werden gewartet, in einem separaten Bereich sind die Maschienen, 8 Valkyrie n der J"agerstaffel untergebracht. 

Zum Zeitpunkt des Eintreffens der Charaktere ist die Mine HeM5 an der Schlepperinsel vert"aut und teilweise zerlegt. Die Minen HeM1 und HeM4 sind im Einsatz. Die Besatzung der zerst"orten HeM3 sind teils zur Erholung auf Callisto und teils bereit wieder im Einsatz auf den anderen Minen.

Im Raumhafen angekommen werden die Charaktere bereits von \emph{Grace Enders} erwartet. Grace ist Teil  des lokalen Sicherheitsdienstes der Cynarion Corporation. F"ur den Aufenthalt der Charaktere ist sie zur Unterst"utung der Ermittler von \emph{Henk Arongate} dem Chef des Sicherheitsdienstes abgestellt. Sie steht hiermit den Charakteren w"ahrend ihres gesamten Aufenthalts treu zur Seite, kann Recherchen beauftragen, kennt die Station mit ihren verwirrenden G"angen und kann lokale Unterst"utzung anfordern. Beim Eintreffen wird sie die Ermittler aufkl"aren dass es sich um eine Minenkolonie handelt und dadurch die Gepflogenheiten etwas ruppiger seinn k"onnnen. Aus diesem Grunde tragen die sicherheitskr"afte Schutzkleidung und eine Waffe. Desweiteren erfahren die Ermittler da\3 ihre Untersuchungen m"oglicherweise kritisch aufgenommen werden k"onnten da man meint die Vorkommnisse k"onnten auch lokal gekl"art werden.

\begin{remarks}
	Die Spieler k"onnen diese Information dazu nutzen sich selbst passend auszur"usten.
\end{remarks}

\subsection{Befragung der HeM5 Besatzung}

Die 10 geretteten Minenarbeiter, werden wie die J"agerpiloten m"oglichst kurz vor dem Eintreffen der Ermittler auf der Dekompressionskammer eintlassen. Beim Eintreffen der Charactere auf Hellgate befinden sich geretteten Mienenarbeiter in der Kantine der J"agerstaffel. Grace Enders wird die Charaktere zur Kantine begleiten. Vor den R"aumlichkeiten treffen die Charaktere auch den bekannten Pilotenausbilder Jos\'{e} \frqq{}Torro\flqq{} Alvarez. Torro ist ein kleiner drahtiger Spacer von fr"ohlicher Natur dem die Jahre als Pilot allerdings schon deutlich zugesetzt haben. Torro der die Rettungsaktion geleitet hat kann einen ersten Einblick in die Geschehnisse geben. Nach dem Gespr"ach mit Torro k"onnen sich die Charaktere den Minenarbeitern zuwenden. Sie k"onnen zun"achst entscheiden ob sie diese einzeln Interviewen wollen oder sie alle direkt in der Kantine aufsuchen. Sollen die Arbeiter einzeln befragt bietet Torro an Florence zu den Ermittlern zu den Ermittlern zu bringen. Danach kann Grace "ubernehmen und die Besatzung der Mine einzeln heraus bitten.

Im folgenden die Aussagen der Beteiligten:

\begin{description}
	\item[Torro:] "`Bei einem "Ubungsflug durch die obersten Athmostph"arenschichten erhielten wir einen Notruf der Mine HeM05. Da meine Trainngsstaffel mit insgesamt vier Valkyrien, 3 Rookies und mir, der Mine am n"achsten wahren sind wir tiefer in die Athmosph"are eingetaucht und konnten dort gl"ucklicherweise die Mine nach kurzer Zeit lokalisiern. Da wir die Arbeiter nich mit unseren Jagdmaschienen selbst retten konnten blieb uns nur die M"oglichkeit an die Mine selbst anzdocken. Zugegebenerweise ein recht waghalsiges und f"ur Auszubildenden risikoreiches Man"over. Wir waren zu diesem Zeitpunkt bereits in eine f"ur uns kritischen Athmosph"arenbereich besunken. Mit viel Gl"uck schafften wir es drei Maschinen an die Mine anzudocken und mit Vollleistung die Mine auf eine H"ohe zu bringen die es der Schlepperinsel erlaubte die Mine in den Orbit zu ziehen. Ein hei\3er Ritt kann ich Ihnen nur sagen."'
	\item[Florence (Kommandantin):] "`W"ahrend der ersten Systemmeldung da\3 einer der Tr"agerbalons der Station abgekoppelt wurde befanden sich Juri Smirnov, Blackwind, ZDee und ich auf der Br"ucke. Greydog war in der Minenanlage besch"aftigt. Die Anderen waren nach ihren eigenen Angaben im oberen Bereich der Mine. Ich beauftragte als erstes ZDee die Aufh"angung des Tr"agerbalongs au\3erhalb der Mine zu kontrollieren und sandte einen Hilferuf an die Hellgate Station. Einige Minuten sp"ater beobachteten wir auf der Br"ucke wie von den Au\3enkammeras aufgenommen, Pitch in ihrem Raumanzug in die Tiefe st"urzte. Ca.~10 Minuten sp"ater l"oste sich der zweite Tr"agerbalon. Nach einem Notruf befahl ich die Evakuierung. Treffpunkt war das Rettungsshuttle. R"uckmeldung bekam ich von allen au\3er ZDee. Auf dem Weg zum Shuttle sammelten wir noch Salvador vor seinem Quartier ein. Er war gerade dabei sich fertig anzuziehen. Am Rettungshuttle traf die Br"uckencrew auf Isabell und Fernandez. Das Rettungsshuttle lie\3 sich nicht starten. Die Startsequenz war durch eine Manipulation blockiert. Deshalb blieb uns nichts anderes "ubrig die Dekompressionskammer aufzusuchen und auf Rettung zu hoffen. An der Kammer trafen Greydog und Hanibal auf uns. Hanibal hatte noch versucht "uber die Steuerung der Anlage die Manipulation der Ballons zu verhindern. ZDee war von seiner Au\3enmission nicht zur"uck gekommen."'
	\item[Juri Smirnov, Blackwind:] Die Br"uckencrew best"atigt die Aussage von Florence.
	\item[Salvador:] "`Ich war in meinem Quartier als der Aufruf zur Evakulierung kam. Die Br"uckencrew kam kurz darauf bei meinem Quartier vorbei und nahm mich mit."' 
	\item[Greydog:] "`Ich war an der Raffinerie mit Wartungsarbeiten im Au\3enbereich am unteren Ende der Rafinerie besch"aftigt. Dadurch habe es nicht geschafft die anderen bereits am Shuttle zu treffen."'
	\item[Fernandez Lorend:] "`Ich hatte Isabell mit der Justierung ihrer Zentrifugen f"ur die Analyse des Atmosph"arengemischs in unserer Flugh"ohe besch"aftigt als der Notruf einging."'
	\item[Isabell Sonderleiten:] "Best"atigt die Aussage von Fernandez. Ein paar Tage vor dem Attentat vertraute Pitch ihr an, dass sie auf eigene Faust gegen ein anderes Besatzungsmitglied recherchierte weil sie glaubte dieser sei f"ur die Haverie der HeM03 verantwortlich. Ihr Verdacht wurde geweckt als sie Unregelm"a\3igkeiten in der Steuersoftware der Mine identifizierte und mutma\3lich auf den Attent"ater zu"uckf"uhren konnte. Sie lie\3 sich deshalb auf HeM05 einschiffen um ihren Verdacht weiter zu verfolgen und den Attent"ater selbst zur Rede zu stellen. Wen sie im Verdacht hatte hat sie allerdings nicht verraten. 
	\item[Blackwind:] "`Der Verdacht auf Pitch r"uhrt daher sie bei der Abkoplung des Tr"agerballons im Au\3enbereich der Mine unterwegs war was "uberhaupt nicht zu ihrem Arbeitsbereich entspricht. F"ur die Wartung und das Einspielen neuer Software hatte mich Pitch gebeten ihr tempor"ar Zugang auf die Steuerung des Rettungsshutle zu geben. Pitch war bereits vorher auf HeM03 stationiert."'
	\item[Hanibal:] "`Als das Abkoppeln des ersten Ballons durchgegeben wurde begann ich sofort die Ansteuerung der Tr"agerballons zu kontrollieren da die softwaretechnische Wartung mir und Pitch unterlag. Dabei konnte ich eine Manipulation der Steuerung entdecken die wahrscheinlich die Abkopplung ausgel"ost hat. Als mein Rettungsversuch mi\3lang und der dritte Balon sich gel"ost hatte machte ich mich auf den Weg zur Kompressionskammer da Florence bereits die Manipulation des Shuttles durchgegeben hatte."'
\end{description}

W"ahrend der Befragung macht Grace Enders Meldung an ihrem Vorgesetzten \emph{Karl Sandos} und "ubermittelt ihm die Aussagen der Besatzung der Mine. Karl Sandos gibt diese Informationen an \emph{Henk Arongate} weiter.

\begin{remarks}
	Die Minenarbeiter sind durch die Vorkommnisse nach wie vor betroffen. Die Komendantin Florence, eine Beta, wird die Crew in Schutz nehmen. Die in diesem Kapitel zusammengefassten Aussagen k"onnten entsprechen emotional ausgeschm"uckt werden. 

	Isabel ist schon vor der Versetzung auf HeM05 bereits mit Pitch befreundet und deshalb stark mitgenommen durch ihren Tod und dem Verdacht die Attent"aterin zu sein.

	Durch die Aussagen k"onnen die Ermittler bereits ermitteln das Pitch aller Vorraussicht nach nicht die Attent"aterin sein kann. Der zweite und vor allem der dritte Tr"agerbalon l"oste sich erst nach ihrem Absturz.		
\end{remarks}

\subsection{Das Geschehen auf HeM05}

F"unf Tage vor dem Attentat wurde die Mannschaft der Mine durch eine Rumpfmannschaft ausgetauscht. Die neue Mannschaft wird f"ur f"unf Tage nach dem Attentat erwartet und trifft etwa zeitglich mit den Charakteren ein. Die Ankunft der zugeh"origen F"ahre k"onnen die Charatere auf dem Flugdeck miterleben.

Das Attentat selbst erfolgte chronologisch folgenderma\3en:

\begin{enumerate}
	\item Ein paar Tage nach der Ankunft auf HeM05 installiert Pitch eine "Uberwachungssoftware in der Minensteuerung die Manipulationen blockiert und "uber Manipuationsversuche informiert. Pitch hat zu diesem Zeitpunkt ihren Kollegen Hanibal bereits im Verdacht auch den Absturz der Mine HeM03 "uber eine Softwaremanipulation veranlasst zu haben.
	\item Pitch macht nach dem Abflug der Minenmannschaft das Shuttle untauglich um eine Flucht des mutma\3lichen T"aters zu verhindern.
	\item Am Tag des Attentats versucht Hanibal zun"achste wie auf HeM03 eine Fehlfunktion der Anlagensteuern hervor zu rufen. Eine solche Manipulation h"atte zu einem ma\3iven Minenschaden und zur Zerst"orung eines gro\3en Teils der Mine gef"uhrt.
	\item Als die Manipulation der Minensteuerung aufgrund der Software von Pitch mi\3lingt begibt sich Hanibal mit einem Raumanzug in den Au\3enbereich und koppelt Tr"agerballon 1 ab.
	\item Pitch verfolgt Manibal und stellt ihn auf der Au\3enballustrade der mine zur Rede als dieser gerade den ersten Ballon abkpoppelt. Es kommt zum Kampf. Hanibal st"urzt Pitch in den Abfrund.
	\item W"ahrend des Abkoppeln des zweiten Ballons wird er von ZDee "uberrascht kann aber diesen abenfalls im Kampf "uberw"alltigen und in die Tiefe st"urzen.
	\item Nach abkoppeln des Tr"agerbalons drei betritt Hanibal die Mine wieder ungesehen, legt den Raumanzug ab und begibt sich zum zweiten Treffpunkt bei der Dekompressionskammer.
\end{enumerate}


\subsection{Weitere Nachforschungen auf Hellgate}

Nach der Befragung k"onnen die Ermittler weitere Nachforschungen auf Hellgate angehen oder bei Grace beauftragen. Nachdem die Ermittler den Raumhafen verlasssen haben, wird die Minenbesatzung von f"unf Gardisten des lokalen Sicherheitsdienstes abgef"uhrt und auf den St"utzpunkt der Sicherheitskr"afte gebracht. Diese "Uberf"uhrung wurde durch Henk Arongate pers"onlich in Absprache mit dem B"uro von Vandermool veranlasst. Das Vorgehen wird von Arbeiter auf dem Raumdeck beobachtet, unter anderem von Drake der bereits vor den Charakteren auf Hellgate angekommen ist um Hanibal zu kontaktieren und der bereits Kontakt mit den hiesigen Arbeitern aufgebaut hat. Nach der Befragung werden sie n"achsten Schritte der Ermittler sie vorraussichtlich entweder zu den Quartieren der Minenbesatzung der HeM05 auf Hellgate oder zu Nachforschungen auf HeM05 f"uhren. 

Nach ihrer Befragung sollten die Ermittler Report an ihre Vorgesetzten geben. Erfolgt eine Meldung an das B"uro von Vandermool werden die Ermittler davon in Kenntnis gesetzt, dass die Minenarbeiter aus Sicherheitsgr"unden verlegt werden.

Folgende Informationen k"onnen bereits "uber Nachfragen direkt beschafft werden:

\begin{itemize}
	\item Tr"agerbalons: Die Tr"agerbalons k"onnen nur von au\3en, direkt an der Kopplungsstelle der Balons durchgef"uhrt werden. Die Information l"asst sich "uber Grace Enders bzw.~durch R"ucksprache bei Dr.~Petrova in Erfahrung gebracht werden. Wenn Pitch nicht die Attent"aterin ist kommen nur noch Hanibal oder Greydog als Attent"ater in Betracht.
	\item Greydog: Greydogs Geschichte l"asst sich ebenfalls "uber Grace Anders oder durch Nachfrage bei Dr.~Petrova plausibilisieren. Die Wege in der Mine von der Rafinerieanlage zu Br"ucke und Aufenthaltsr"aumen sind mehrere hundert Meter weit. Zus"atzlich mu\3te Greydog erst wieder vom Au\3enbereich in die Mine zur"uck kehren und den f"ur die Jupiteratmosph"are angepassten Raumanzug ablegen.
	\item HeM03: Nachforschungen zu MeM03 ergeben da\3 sich au\3er Florence, Pitch und Hanibal kein Mannschaftsmitglied mehr auf Hellgate befindet. Auf HeM03 waren insgesamt 50 Arbeiter besch"aftigt die sich gl"ucklicherweise fast alle mit Hilfe der Rettungsshutles retten konnten. 
	\item HeM03 Attent"ater: Angeblich hatten Lionell Hampton, Ice Diver und Hanibal versucht den Attent"ater Sent von der Manipulation der Minensoftware abzuhalten. Lionel Hamption wie auch Sent wurden dabei get"otet. Ice Diver gilt seit dem Vorfall auf der Mine ebenfalls als vermisst.
	\item Hanibal: Nachfragen zum Hintergrund von Hanibal ergeben da\3 Hanibal vor einem dreivierteljahr im Raumhafen von Valhalla als Software-- und Sicherheitstechniker angestellt wurde.
	\item Willkommensgala: Nach einem direkten Kontakt zu Avenger befragt meldet dieser zur"uck, da\3 der Protektor derzeit schlecht zu erreichen ist da er eine Wirtschaftliches Treffen auf Valhalla vorbereitet.
	\item Pers"onliche Hintergr"unde: Nachfragen zu Hintergr"unden der Minenbesatzung f"ordern zu Tage das keine auff"alligen Gelder geflossen sind oder Angeh"orige unter Druck gesetzt werden k"onnten.
\end{itemize}

Diese Informationen sollten erst nach Abschlu\3 der Befragung zug"anglich sein damit sie nicht in die Befragung mit einflie\3en k"onnen und damit sofort zur Aufkl"arung der Ereignisse f"uhren.

Bei Nachfragen zur Schlepperfehlfunktion vor 9 Wochen l"asst sich herausfinden da\3 ein Softwarefehler Sch"aden an zwei Tr"agerpunkten f"ur Minen verursacht hatte. Die Reparaturen dauern noch an. Gl"ucklicherweise l"asst sich die Schlepperinsel nach wie vor mit ihren drei weiteren Andockpunkten nutzen. Die Fehlfunktion wurde durch eine Manipulation ausgef"uhrt von Hanibal ausgel"ost. Zum Zeitpunkt der Fehlfunktion war er allerdings bereits auf HeM03 im Einsatz. Durch Nachfragen im Flugbereich auf Hellgate erfahren die Spieler auch von von dem Shuttleabsturz vor 10 Wochen. Der Vorfall wird nicht als Attentat eingesch"atzt sondern wird als Beispiel erw"ahnt dass nicht alle Unf"alle als Attentate eingestuft werden m"ussen. Der Shuttleabsturz oder besser Kollision mit einer F"ahre auf dem Hangardeck wurde durch ein Fehlsteuerung des "Uberm"udeten und mit Wachhaltern gedobten Betas Razor ausgel"ost. Razor wurde bereits befragt. Er ist inzwischen auf Kallisto.

Folgende Instruktionen und Information bekommen die Charaktere unaufgefordert:

\begin{itemize}
	\item Chefermittler Cynarion: Der Chefermittler der Cynarion Croporation wird von Colonel Scholz anmgewiesen den 2.~Ermittler alleine zu einer zweiten Befragung des verd"achtigen Hanibal in den St"utzpunkt der Sicherheitskr"afte zu schicken.
	\item Psychonaut: Der Psynchonaut, der 2.~Ermittler der Cynarion Corporation, wird von Herny Longdale dem Sekret"ar Vandermools pers"onlich und eindringlich beauftragt den Verd"achtigen einem Gehirnscan zu unterziehen.
	\item Chefermittler Protektorat: Der Chefermittler des Protektorat wird von Artisan beauftragt mit dem Chefermittler der Cynarion die weiteren Ermittlungen in den Quartieren der Verd"achtigen oder auf HeM05 fortzufahren.
	\item Omega: Da Blackhear Vandarmools Truppen nicht traut beauftragt Thunderbold  Omega umgehend den Sicherheitsdienst abzufangen und zu begleiten oder die Aktivit"aten der Cynarion Corporation zu unterst"utzen und mitzuverfolgen. Da Blackheart ziemlich sauer wg.~der aktuellen Entwicklung ist erfolgt der Auftrag mit deutlichem Nachdruck.	
	\item Cowboybrigade: Der Chefermittler des Protektorats wird dar"uber informiert dass die Cowboybrigade auf dem Garnisonsst"utzpunt auf Valhalla auf Anweisung von \emph{Commander Lockhead} festgesetzt wurde.
	\item Slingshot/Drake: Der Chefermittler des Protektorats wird von der Verwaltung des Raumhafen auf Armageddon informiert dass Slingshot kurz nach dem Frachterungl"uck im Raumhafen der Station von einer Kamera erfasst wurde. Mutma\3lich um sich von dort aus Absetzen zu k"onnen.
	\item 
\end{itemize}

\begin{remarks}
	Nach den sch"atzungsweise langen vergangenen Ermittlungen bietet es sich an ab hier das Schritttempo des Plots zu erh"ohen und die Informationen in kurzen Zyklen auszugeben und detailierte Ermittlungen abzuk"urzen. Ausreichende Informationen sind mit den Befragungen der Minenarbeiter und den nachgefragten Informationen bereits vorhanden.

	Das nicht von den Charakteren beauftragte umverlegen der Minenbesatzung soll den Spielern zeigen, da\3 ihre F"uhrung jeweils ihr eigenen F"aden ohne Absprache zieht.

	Die Anweisungen den zweiten Cynarion Ermittler und den Omega zum St"utzpunkt des Sicherheitsdienstes alleine zu schicken dient dazu die Gruppe zu trennen. F"ur den weiteren Verlauf der Ereingnisse auf Hellgate ergeben sich damit gr"o\3ere M"oglichkeiten eine gr"o\3ere Spannung in den kommenden Szenen aufzubauen.
\end{remarks}

\subsection{Zusammensto"s mit dem Hellgate Personal}

Beim Weg zu den Quartieren der HeM05 Minenbesatzung oder andersweitig zum Lebensbereich der Hellgate Station werden die Charaketere und Grace Anders durch 7 finster hereinblickende Arbeiter mit provisorischen Schlagwaffen in Form von Werkzeugen eingekesselt. Die Arbeiter wurden von Slingshot alias Drake angestachelt die Ermittler zur Rede zu stellen wieso die Minenarbeiter wie Verbrecher auf ihre Veranlassung abgef"uhrt wurden. Den Spielern ist zu diesem Zeitpunkt noch nicht bekannt da\3 die Minenarbeiter vom Sicherheitsdienst abgef"uhrt wurden. Auch Grace Anders ist nicht informiert.

\begin{remarks}
	Das Mi\3verst"andnis l"asst sich leicht durch eine R"ckfrage bei Karl Sandos aufl"osen lassen. Ein Eskalation l"asst sich durch Androhung des Einsatzes von Sicherheitskr"aften der Station oder durch Drohungen seitens des Omegas vermeiden. Die Minen Crew wurde auf Anweisung von Arongate auf den Sicherheitsst"utzunk "uberf"uhrt.
	
	Der Vorfall verschafft Drake und seinem Helfer Zeit f"ur eine Befreiungsaktion von Hanibal.

	Sollte es zu einer Ausschreitung kommen dann finden sich die Werte unter \emph{Pers"onlichkeiten auf Hellgate}.
\end{remarks}

\subsection{Quartiere und die HeM05 Miene}

Die Quartiere von Pitch, Hanibal, Greydog und wenn notig auch der des Restes der Besatzung befinden sich im Wohnbereich der Station. Die Quartiere sind an den Seiten von langen G"angen aufgereit und durch eine kleine Schl"au\3 zu betreten. Da ich die Mitarbeiter auf Hellgate un den Minen nur wenige Wochen aufhalten, viele davon haupts"achlich in den Minen, hat kaum jemand ein festes Quartier. Die meisten beziehen ein Quartier nur f"ur den zeitweiligen Aufenthalt auf Hellate und geben dann das Quartier f"ur andere auf. Die Mitarbeitern haben entsprechend einen Gro\3teil ihre Habseligkeiten selbst mit dabei. Die Quartiere bestehen aus meiner einklappbaren Liege einem kleinen Tisch mit Sitzgelegenheiten, einem Spind und einer Na\3zelle inclusive Toilette. Neben dieser Einrichtung gibt es ein Nahrungsaufbereitungssystem und einem Computersystem mit Holoprojektor das in erster Linie f"ur die Kommunikation oder als Server f"ur Datenabfragen und Unterhaltung dient.

Im Quartier von Pitch finden sich ein paar Bilder auf denen sie mit vermutlich Freunden auf sch"atzungsweise dem Mars abgebildet ist. Das Computersystem ist dabei schon interessanter. Man findet dort ein Tagebuch mit Rechercheergebnissen und Randnotitzen zu den Vorkommnissen auf HeM03 und ihrem Beschluss Hanibal auf HeM05 zu folgen. Von einer Anzeige hatte sie wohl abgesehen da sie sich nicht vorstellen konnte das Hanibal das Attentat wirklich begangen hatte und was ihn zu so einer Tat gef"uhrt haben konnte. Hanibal war seit der Zeit auf HeM03 nach ihren Angaben ein gute Kollege gewesen.

In den Quartieren von Hanibal und Greydog sind keine pers"onlichen Gegenst"ande zu finden.

Nachforschungen auf HeM05 k"onnen nur mit Unterst"utzung von Mitarbeitern der Mine erfolgen. Auf der Mine die an die Schlepperinsel angedockt ist werden derzeit einige Reparaturen durchgef"uhrt. Die Mine selbst ist ein imposantes Konstrukt das einem auf den Kopf gestellten abgeflachten Kegel mit einer H"ohe von "uber 500 Metern entspricht. An den oberen 20 Metern befinden sich die Wohnquartiere, Aufenthaltsr"aume, der Shuttlehangar, die Br"ucke, Arbeitsst"atten, Lager und technische Einrichtungen für den Betrieb der Mine. Der darunterliegende Teil ist die eigentliche F"orderanlage und Tanks f"ur HE--3. Um die Mine sind an verschiedenen Stellen Balkone gezogen um den Arbeitern einen einfacheren Au\3eneinsatz zu erm"oglichen. Die f"unf Tr"agerbalons sind rund um den oberen Teil der Mine aufgeh"angt. 
Derzeit besitzt die Mine weiterhin nur zwei von f"unf Tr"agerbelons. der drei zerst"orten werden derzeit auf Nike hergestellt. F"ur den Au\3eneinsatz werden spezielle schwere Raumanz"uge ben"otigt die die Arbeiter vor den Widrigkeiten der Saturnathmosph"are f"ur eine kurze Zeit sch"utzen k"onnen. F"ur die Nachforschungen auf der Mine werden Spezialisten ben"otigt die die Anlagen der Mine erkl"aren k"onnen und Softwareexperten die Manipulationen an der Software aufdecken k"onnen. Auf der Mine k"onnen die Manipulationen des Shuttles durch Pitch zweifelsfrei festgestellt werden wie auch die von Pitch installierte Sicherung der Minensteuerung und den Manipulationsversuch von Hanibal.

\subsection{Befragung im St"utzpunkt des Sicherheitsdienstes}

Trifft zumindest der Psychonaut im St"utzpunkt des Sicherheitsdienstes auf Hellgate ein kann er dort Hanibal eines Gehirnscans unterziehen. Der St"utzpunkt ist "ahnlich einem Polizeirevier aufgebaut. Eine Sicherheitssch"au\3e f"uhrt in einen Eingangbereich mit einem durch einen Plexsigla\3 abgetrennten Bereich f"ur die Sicherheitskr"afte. Der diensthabende Sicherheitsbeamte empf"angt die Ermittler und fragt nach ihrem Begehren. Fordern die Charaktere ein Befragung der Minenarbeiter an kontaktiert der Beamte den Stationsleiter Karl Sandos der die Charaktere pers"onlich abholt und sie durch den St"utzpunkt und zu den Zellen und Verh"orr"aumen f"uhrt.

Wurde Hanibal oder ein Anderer den der Psychonaut eines Scans unterziehen m"ochte in den Verh"orraum gef"uhrt. Der Psychonaut sollte nun alle anderen au\3er evtl.~dem Omega bitten den Raum zu verlassen um seine "`Befragung"' in Ruhe durchf"uhren zu k"onnen. Der Omega oder ein anderer der Spielercharaktere sollte anwesend blieben um Nanibal zu fixieren. Die Aktivit"aten von Hanibal zum Zeitpunkt der Attentate auf HeM03 und HeM05 finden sich im Gehirn in der selben Form wie sie Hanibal geschildert hat. Auf HeM03 "uberw"altigt Lionell Hampton, Ice Diver
und er Sent nachdem Hanibal bemerkt wie Sent die Computeranlage der Mine zu manipulieren. Durch eine Messerattacke wird Lionell Hampton get"otet, Ice Diver und Hanibel wiederum k"onnen Sent zu t"oten aber es ist bereits zu sp"at um die Manipulation aufzuheben. Auf HeM05 erlebt man wie Hanibal versucht die Manipulation der Tr"agerbalonsteuerung vergeblich aufzuheben. Der Psychonaut stellt aber bald fest das den beiden Erinnerungen jegliche Form von Emotionen fehlen und die Erninnungen unscharf und L"uckenhaft sind. Versucht er das Gehirn des Attent"aters wieder zuz verlassen wird er durch die KI in Hanibals Kopf daran gehindert. Er findet sich unversehens auf einer gr"unen Wiese mit blauem Himmel wieder auf der nur eine einzige vollkommen wei\3e androgyne Person steht. Er wird in die Gedanken dieser Person gezogen und erf"ahrt von einem tief verwobenen Befehl die Minenanlagen auf dem Jupiter zu manipulieren. Kurz darauf nimmt er einen Countdown war und sollte versuchen den Geist von Hanibal wieder zu verlassen weil sein Gehirn sonst selbst Schaden erleiden w"urde.

\begin{remarks}
	Der Gehirnscan ist im Regelwerk beschrieben. Er kann "anlich zu einem Matrixrun bei Shadowrun durchgef"uhrt werden.
\end{remarks}

\subsection{Die Geiselnahme}

Hat der Psynchonaut seine Nachforschungen in dem St"utzpunkt des Sicherheitsdienstes abgeschlossen wird er und seine Begleiter von Karl Sandos wieder in den Eingangsraum des St"utzpunktes begleitet. Der Spielleiter sollte wenn m"oglich darauf warten das sich die Charaktere au\3erhalb des St"utzpunktes und die im St"utzpunkt in einen Informationsaustausch begeben. Wenn sie im Austausch sind wird die Verbindung der beiden Parteien "uber das ComNetz ausfallen. Kurz darauf "offnet sich der Zugang zum St"utzpunkt und die PAN Netze der Personen im Raum fallen aus. Die PANs (Personal Area Network) sind das technische System im K"orper der Person und umf"asst z.B.~ einen Commandchip, sonstige Headware, das AR (Augmented Reality) System im Sichtbereich und generell alle Cybersysteme des Charakters. Jeder Charakter in der Station sollte nun einen Konstitutionswurf (Body) machen um nicht kurzzeitig in Ohnmacht zu fallen. Die Funktionen in Systemen des Omegas wenn auf dem St"utzpunkt werden schnell wieder neu starten f"uhren aber zun"achste zu einem Handycap von -4 auf alle physischen W"urfe. Mit dem "offnen des Zugangs zum Eingangsbereich des St"utzpunkt st"urmen zwei bewaffnete Personen in den Raum und beginnen sofort zu feuern.


\subsection{Nachforschungen in Valhalla}

Alle Kliniken, die auf Kallisto inoffizielle Implantate verbauen, befinden sich in Valhalla au\3erhalb \emph{Headquarter}, \emph{Rosenfurth} und dem Raumhafen.  Bei den Nachforschungen au\3erhalb \emph{Headquarter}, \emph{Rosenfurth} und dem Raumhafen werden die Soldaten aus der Garnison die Ermittler nicht weiter begleiten. An den R"ander von \emph{Paradise City} und \emph{Neu Gr"oning} endet der Einflu\3bereich des Protektorats und man will Spannungen vermeiden. Auf Nachfrage k"onnen die Ermittler erfahren das "uber einen gro\3en Teil Valhalla verschiedene Band unter der Schirmherschaft des Luna--Syndikats herrschen. Au\3erhalb des Headquarters, Rosenfurth und dem Raumhafen ist die Anbindung an das ComNetz nur sp"arlich. Die Charaktere k"onnen ein Sprechfunkger"at an die Hand bekommen wenn sie danach fragen um den St"utzpunkt im Zweifel erreichen zu k"onnen. Das Sprechfunkger"at hat aber auch nicht "uberall Empfang wodurch die Gruppe oft auf sich allen gestellt ist.

Die Kliniken in Valhalla m"ussen einzeln besucht werden. Ein zentrales Verzeichnis der Kliniken besteht nicht. Es m"ussen vor Ort in den jeweiligen Sektoren der Stadt Erkundigungen in Nachtclubs und Lokalen eingezogen werden. Ein erster Anlaufpunkt k"onnte der Schieber \emph{Henk Brothers} sein an den Lenny Kilkenny, der Barmann des Bat Cave sie verwei\3en kann. Henk Brother verweilt "ublicherweise im \emph{Green Mile}, einem der bessren Lokale am Rande von Paradise City zu Rosenfurth. "Arzte in Kliniken, Schieber wie auch die meinsten Vermittler in Valhalla au\3erhalb 3erhalb \emph{Headquarter}, \emph{Rosenfurth} und dem Raumhafen werden vom Luna--Syndikat kontrolliert. "Arzte werden zu inoffiziellen Behandlungen (Schu\3verletzungen, Einbau von Cybwerware, Drogen etc.) ohne Genehmigung des Luna--Syndikats bestenfalls ausweichende oder schwammige Ausk"nfte geben. Bei eigenen Nachforschungen in den Kliniken werden die Charaktere keine neuen Erkenntnise gewinnen. Slingshot oder die Cowboybrigade ist in den Etabilments die von den Charakeren besucht werden nicht bekannt. In Bezug auf eine Frau mit auff"alligen langen roten Haaren erf"ahrt man von einigen T"anzerin mit langen braunen Haaren und von einer S"angerin mit auff"alligen aber blonden Haaren.

"Uber das Luna-Syndikat oder auf andere Weise k"onnen folgende Informationen in Erfahrung gebracht werden:


\subsection{Zusammestro"s mit dem Luna--Syndikat}
\newcommand{\xl}{\pinyin{Xiao3} \frqq{}Saber\flqq{} \pinyin{Long2}}

Die Nachforschungen in Valhalla bleibt nicht unbeobachtet. Das Luna--Syndikat verfolgt die Ermittler seit dem Verlassen des Raumhafens und Rosenfurth genau. Durch einen W"urfelwurf auf Investigation bemerken Charaktere da\3 die Gruppe immer wieder verfolgt wird. Die Verfolger scheinen immer wieder unterschiedliche Personen zu sein, Ghetto Kids, Schl"ager, d.h.~zwielichte Personen die verschwinden wenn sie das Gef"uhl haben entdeckt worden zu sein.

Nachdem sich die ersten Nachforschungen fruchtlos erweisen k"onnte die Gruppe beharrlich weiter suchen, einen Verfolger aufgreifen, einen der Schieber oder "Arzte in die Mangel nehmen oder den Kontakt zu Luna--Syndikat suchen. 

Unabh"abngig davon f"ur welche Option sich die Spieler entscheiden stehen den Ermittlern fr"uher oder sp"ater unerwartet 9 Gangster angef"uhrt von \xl{} gegen"uber. Saber und ihre Bande ist im Auftrag von Nemessis ausgesandt die Gruppe zu ihm zu bringen. Sie allerdings wird zuerst versuchen m"oglichst viel "uber den Wissensstand der Gruppe in Erfahrung zu bringen. Die Gangster bis auf Saber halten Multiguns in den H"anden Saber tritt den Charakteren ohne eine Waffe in der Hand einen Schritt entgegen. Sie herrscht den Charakter der ihr direkt gegen"uber steht an.

"`Ihr stellt sehr viele Fragen. Was wollt ihr hier?"'

W"ahrenddessen haben die restlichen Ganoven ihre Schusswaffen in Anschlag gebracht. Die h"alfte davon auf den Omega. Reagieren die Ermittler ausweichend wird sie den Druck erh"ohen und wendet sich an ihre Mitstreiter:

"`Ist ihre Geschichte glaubw"urdig?"' Die Gang verneint. Quicksilber, ihre rechte Hand, fast es in Worte "`ziemlicher Quatsch."'

Saber wird zun"achst abwarten welche Informationen die Gruppe bereit ist zu geben. Tritt der Omega in Aktion wird sie sich auf Abstand begeben und sich an diesen wenden. Geben die Charaktere keine weiteren Informationen preis wendet sie sich ebenfalls an den Omega.

"`Es gibt eine Vereinbarung mit dem Protektorat. Bei uns hat die Armee keine Befugnisse. Das ist ein Problem."' 

An ihre Mitstreiter gewandt ohne den Omega aus den Augen zu lassen:

"`Schaltet ihn aus."'

Jetzt geht alles ganz schnell. Die Gangster die Waffen auf den Omega gerichtet feuern ihre Waffen ab. Sie schie\3en allerdings keine penetrierenden Geschosse sondern Schockprojektile. Reagiert der Spieler des Omegas sofort kann er versuchen den Projektile auszuweichen und dabei einen der Angreifer in den Nahkampf zu zwingen. In diesem Fall werden die anderen Gangster ohne R"ucksicht auf ihren Kameraden weiter auf den Omega schie\3en. Der Angriff auf den Omega war wohl geplant und von Nemmessis beauftragt. Nemessis ist zwar gezwungen den Ermittlern zu halfen (siehe weiter unten) aber ein Eindringen in sein Reich kann er nicht einfach hinnehmen. 

Saber bringt sich bei der Auseinandersetzung au\3er Reichweite. Die Gangster den Rest der Gruppe in Schach halten bleiben auf Abstand. Egal wie der Kampf ausgeht wird Saber die Charaktere "uberzeugen, da\3 die Gruppe nicht ohne Verluste aus einer weiteren Konfrontation heraus kommen werden. Die Waffen der verbliebenen Angreifer sind mit scharfer Munition geladen, behauptet sie. Wichtig in jedem Fall ist das Saber am Ende die Kontrolle beh"alt. Wenn der Omega ausgeschaltet wurde und die Gangster nach wie vor in der "Uberzahl sind kann Sabre ein weiteres Mal versuchen mehr Informationen aus den Ermittlern zu pressen.

Quicksilver meldet sich zu Wort. "`Eine Nachricht von Nemessis. Er will das wir sie mitnehmen."'

Saber gibt sich "uberrascht. "`Sieh an. Ihr habt eine pers"onliche Audienz beim Herren der Stadt gewonnen."'

Saber an die Gangster gewandt: "`Packt sie ein, aber sch"on vorsichtig. Den Omega lassen wir da."'

Verwundete oder tote Gangster die sich nicht eigenst"andig Fortbewegen k"onnen werden ebenfalls da gelassen. Die Charaktere werden bis auf den Omega, der die Gruppe nicht begleiten kann, gefesselt und in gel"andetagliche Buggies gesetzt. Dann geht es los.

\begin{remarks}
	In dieser und der n"achsten Szene ist es schwierig den Spielern freiheitsgrade zu lassen ohne den Plot zu gef"arden oder die Authentizit"at der Rollen von \xl{} und Nemessis zu besch"adigen. Viel h"angt hier von der Bereitschaft der Spieler ab in die Dramatik einzusteigen. Der Spielleiter sollte den Spielern erlauben zu kommunizieren aber sie auch Druck aufbauen zu handeln. Ebbt die Initiative der Spieler ab sollte Sabre die Erschie\3ung des Omegas Befehlen. Der Spielleiter sollte dabei den Spieler des Omegas nach seiner n"achsten Handlung fragen sondern beschreiben wie die Gangster ihr Ziel anvisieren und dann sofort schie\3en lassen.

	Der Auftrag f"ur den "uberfall ist die Charaketere pers"onlich zu Nemessis zu bringen und gleichzeitig dem Protektorat mitzuteilen, da\3 das Vorgehen der Charaktere im Territiorium des Syndikat nicht gedultet werden kann. Wichtig bei der ganzen Auseinandersetzung ist es, da\3 keiner der Charaktere get"otet wird und \xl{} als Sieger aus der Konfrontation hervor geht. \xl{} mu\3 in dieser Szene ihre Skrupellosigkeit und ihre Rolle als Anf"uhrerin zeigen. Quicksilver ist der Joker um eine Auseinandersetzung zu beenden und im Sinne des Syndikat zu einem Erfolg zu bringen.

	Greift der Omega zur Waffe sollte der Spielleiter ihn die Wahl lassen ob er mit t"otlicher Munition oder auch mit Schockmunition schie\3en m"ochte.
\end{remarks}


\subsection{Blackhearts Ultimatum}

Parallel zur Audienz hat der Omega die M"oglichkeit zur Garnison zur"uck zu kehren und das geschehene zu Berichten. 

Zwischen dem Luna--Syndikat und Blackheart gibt es seit Beginn des Protektorats die Vereinbarung dass das Syndikat ohne Eingriff des Milit"ars Valhalla kontrollieren kann im Gegenzug k"ummert sich das Syndikat um den reibungslosen Betrieb der Stadt. Durch die  vorangegangene Blokade der Ermittler und der Provokation durch den Angriff auf den Omega droht Blackheart Nmessis mit einem milit"arischen Eingreifen seitens des Protektorats, sollte Nemessis die Charaktere nicht umfassend unterst"uten und f"ur deren Sicherheit zu sorgen. Der Omega hat damit wieder die M"oglichkeit nach der Audienz der Gruppe bei Nemessis die anderen Ermittler zu begleiten.


\subsection{Treffen mit Nemessis}

F"ur ein Treffen mit Nemessis werden die Ermittler durch eine gro\3e Maschinenhalle, die das "`ortliche"' Fusionskraftwerk beherbergt, zum erh"oht angebrachten Leitstand gef"uhrt. In einem weitr"aumigen B"uro, in dem sich bereits mehrere Capos und gut ger"ustete S"oldner befinden, steht ein hochgewachsener Mann in einem langen schwarzen Mantel mit dem R"ucken zu den Anwesenden vor einem ausladenden Schreibtisch an dem er mit einer anderen Person leise spricht. Die Charaktere werden aufgefordert, einige Meter vor ihm stehen zu bleiben. Nach etwa einer Minute dreht sich der Mann, der sich damit als Nemessis zu erkennen gibt, zu den Charakteren um. Er st"utzt sich dabei auf seinen Gehstock.

"`Mein Name ist Nemessis. Sch"on dass Sie zu mir gefunden haben.'' 

An \xl{} gewandt, "`Saber, gab es Schwierigkeiten?''. Saber:  "`Keine"'. Nemesis f"ahrt an die Ermittler gewandt fort. 

"`Meine Zeit ist sehr begrenzt. Deshalb gleich zur Sache. Welche Nachforschungen f"uhren Sie in meine Stadt?''

Wenn die Charaktere nicht alles erz"ahlen erkl"art Nemmesis:

"`Das ist doch so nicht ganz vollst"andig, oder? Versuchen Sie es bitte noch einmal etwas pr"aziser.''

Die Charaktere sollten Nemessis davon zu "uberzeugen, dass durch die Vorkommnisse die Sicherheit des jovianischen Systems gef"ahrdet ist und m"oglicherweise eine milit"arische Interventions Seitens der Protektoratsstreitkr"afte drohen. Nemessis schl"agt den Ermittlern darauf hin vor den Blackhole Club zu besuchen und stellt \xl{} an ihre Seite.

Nach Beendigung der der Audienz wird die Gruppe durch Saber in die Lobby des Suunshine Hotels gef"uhrt. Auf dem Weg dahin werden ihnen die Fesseln abgenommen. Im Raum befinden sich bereits einige Angestellten des Dukes wie auch ein "alterer Mann in einem Arztkittel. Saber bedeutet dem Arzt, der sich als Dr.~\pinyin{Li3} \pinyin{Li3} vorstellt, etwaige Wunden zu versorgen.

\begin{remarks}
	\xl{} ist ab dieser Szene die Begleiterin der Gruppe in ihrem eienen Interesse und ersetzt damit den Adjutanten Firedon. Sabre strebt an an Naratovas Forschungsergebnisse zu gelangen und alle weiteren Informationen zu den KIs zu vernichten.

	In Begleitung anderer Gangster tritt \xl{} als Anf"uhrer auf und verteilt Aufgaben unterst"utzt aber auch ihre Untergebenen soweit m"oglich und sinnvoll. Bei der Unterst"utzung der Gruppe ist sie aber meist alleine von den Partie und operiert autonom. Durch ihre "uberragenden k"ampferischen F"ahigkeiten kann sie leicht in allen Gefahrensituationen eingesetzt werden.
\end{remarks}


\subsection{Im Blackhole Club}

Der Blackhole Club ist ein Club der nur Mitgliedern bzw.~geladenen G"asten Einlass gew"ahrt. Nur ausgew"ahlte Personen werden jemals den Club von au\3en oder von innen kennen lernen. Der Club wird vom Luna--Syndikat betrieben. Durch die Unterst"utzung von Nemessis und der Begleitung durch \xl{} ist der Zugang kein Problem. \xl{} ist im Club bereits bekannt und setzt sich zu einer Gruppe von offensichtlichen Verehrern.

Wenn sich die Ermittler an die Bar setzen werden sie zun"achst vom Barmann \emph{Rosen} nach der Getr"ankebestellung angesprochen und gefragt nach was sie suchen, BTL, Teschnische Bauteile, Waffen. Wenn sie nach Slingshot oder Hanibal fragen wird der Barmann jemandem im Publikum, es ist nicht genau zu erkennen wer, einen Blick zu werfen. Kurze Zeit sp"ater, bevorzu wenn die Gruppe sich getrennt hat, wird \emph{Carina} sich zu dem letzten an der Bar verbleibenden setzen und ein Getr"ank ordern.

"`Ihr sucht nach einem Slingshot? Vielleicht hab ich so jemanden schon einmal getroffen. Was hat er denn angestellt?''

Wenn sie erf"ahrt dass Slingshot get"otet wurde und das etwas mit seiner Headware nicht in Ordnung war oder dass er sogar an einem Attentat beteiligt war reagiert sie betroffen hat sich dann aber wieder schnell unter Kontrolle und fordert den Charakter auf ihr zu folgen um ungest"ort plaudern zu k"onnen. Sie f"uhrt ihn in einen Bereich mit Separees um mehr zu erfahren. Bevor der Ermittler jedoch genaueres erz"ahlen kann bemerkt entweder er oder Carina einen anderen Gast der nicht weit entfernt Platz nimmt. Das Gespr"ach triftet in Belanglosigkeiten ab. Carina schmiegt sich n"ahre an den Charakter heran und legt ihm eine Hand auf den Oberschenkel. Dabei schiebt sie ihm eine kleine Karte zu. Danach verabschiedet sie sich und er"offnet mit Bedauern dass sie nicht weiter helfen kann. Der Agent ist von Smith--Singer beauftrat Cariana im Blick zu behalten. W"ahrend des Gespr"achs der beiden informiert er bereits den zweiten USI Agenten Frederic Johnson von dem Treffen "uber ComLink. Selbst wenn er also danach vom Syndikat oder den Charakteren aufgegriffen wird ist die Information bereits weiter gegeben. Der Agent mit dem Namen Bloe Ringdan wei\3 weiter nichts au\3er dem Beschatten der Kontaktfrau.

Die Inspektion der Visitenkarte sollte m"oglichst au\3erhalb des Blackhole Clubs erfolgen. Auf der Visitenkarten ist auf den ersten Blick nur ein holograhisches Bild von Carinas in laszieven Bewegungen zu sehen. Unterschrieben ist das Hologramm mit dem Namen Fleur Soleil. Wird das Bild l"anger in der Hand gehalten taucht eine Comlink Nummer auf. Wird die Nummer ohne unterdr"ucken der eignen Nummer per Nachricht kontaktiert kommt als R"uckantwort "`Ice Club heute Abend"'.

Da Saber Cariana kennt wird f"angt sie die S"angerin nach dem Besuch beim Blackhole Club an und fr"agt sie seinerseits aus. 

\subsection{Im Ice Club}

Der Ice Club ist ein Nachtclub und Bordell das ebenfalls durch das Luna--Syndikat betrieben wird. Die Ermittler k"onnen bei Saber, von der sie inzwischen die Kontaktdaten haben, nach einer Empfehlung fragen. Durch die Einladung von Fleur Soleil wird ihnen aber ebenfalls Einlass gew"ahrt. F"ur den Eintritt ist Abendgarderobe erforderlich. Der Nachclub kann diese aber auch bereit stellen. Wenn die Charaktere das Bordell betreten steht Carina als Fleur Soleil gerade auf der B"uhne und erfreut die G"aste mit ihrem Gesang. Bei ihrem Auftritt ist sie in ein hautenges wei\3es Kleid das au\3er den Pailettenbestickungen nahezu durchsichtig ist gekleidet. Zu dem Kleid tr"agt sie platinblonde Haare ein eingewobenen leuchtenden Kristallen. Die Charaktere haben w"ahrend der Darbietung gen"ugend Zeit den Clubraum zu inspizieren und ihr weiteres Vorgehen zu beraten.

Der Eingangsbereich besitzt Zug"ange in eine ausgedehnte Garderobe, den Clubraum und in obere R"aumlichkeiten. Die W"ande des Clubraums bestehen aus konserviertem Eis das reich dekoriert in einer geschwungenen Decke endet. Gegen"uber der B"uhne befindet sich die Bar. Davor sind kleine Tische mit St"uhlen um einen Catwalk aufgereit. Im hinteren Bereich befinden sich Separees.Versteckt, kunstvoll in das Eis eingelassen f"uhrt eine Treppe nach oben zu einzelnen Zimmern und nach unten in einen Saunabereich.

Wenn die Gruppe das Bordell betritt ist Saber bereits da und kann auch von den Charakteren endeckt werden. Saber selbst erkennt das einer der G"aste, Bloe Ringdan wenn noch im Spiel oder Frederic Johnson die Charaktere und die S"angering beobachtet. Nutzen die Charaktere die M"oglichkeit die G"aste zu beobachten k"onnen sie den Agenten abenfalls entdecken.

Carina bittet nach ihrem Auftritt den Charakter mit dem sie im Blackhole Club gesprochen hat ihr zu ihrer Zimmer zu begleiten und erkundigt sich nach seinen Freunden. Andere G"aste vertr"ostet sie auf sp"ater oder ein anderes mal. Die Gruppe kann selbst entscheiden wer sie auf das Zimmer begleitet. Auf ihrem Zimmer schl"upft sie in ein praktischeres Kleid in eine Toga. Von Cariana erfahren die Ermittler, da\3, ihrer Beschreibung nach vermutlich Smith--Singer, sie beauftragt hat nach Interessenten f"ur Cyberware Ausschau zu halten. Im Blackhole Club hat sie ihm die Kontakte zu Hanibal und Slingshot vermittelt. Bei einem der Treffen hat sie zuf"allig \emph{Cyberbrain} als den Namen der Forschungseinrichtung in dem  wo die Eingriffe durchgef"uhrt wurden aufgeschnappt. Nach den Eingriffen hat sie Hanibal und Slingshot nicht mehr gesehen.

Kurz nachdem die Charaktere das Zimmer der S"angerin verlassen haben  macht sich der USI Agent im Clubraum auf den Weg zu Carinas Zimmer. Beim Zimmer angekommen dr"uckt der Agent Carina die gerade dabei ist ihr Zimmer zu verlassen zur"uck und schlie\3t die T"ur. W"ahrend Bloe Ringdan das Zimmer Carinas betritt verlassen die S"oldner \emph{Lazor} und \emph{Flinn} in Eile andere Zimmer um deren Kollegen zu unterst"utzen. Die beiden werden versuchen zu verhindern das die Charaktere das Zimmer der S"angerin zu betreten. Da Waffen wie Schu\3waffen und Klingen im Club nicht erlaubt sind stehen den Gegenspielern nur die F"auste und jeweils ein Daze Patch zur Verf"ugung. Ein Daze Patch ist ein kleines Pflaster das, wenn in Ber"uhrung mit der Haut komm, eine zu Bewustlosigkeit f"uhrende Droge aussch"uttet (Stun W6+5, Bewustlosigkeit ab >= 1/2 HP). Gelingt es den Charakteren Hife zu holen werden die Agenten "uber den Hintereingang fliehen. Die Agenten selbst wissen nicht mehr als die Charaketere "uber die Operation k"onnen aber die Identit"at Smith--Singers im Verh"or preisgeben. Gelingt es den Charakteren nicht das Zimmer zu betreten wird Sabre zu Hilfe eilen und Bloe Singer davon abhalten Cariana weiter zu verh"oren. Bei dieser Auseinandersetzung wird Sabre den Agenten t"oten.

\subsection{Kommandoaktion Cyberbrain}

Cyberbrain ist eine kleine Forschungseinrichtung in der \emph{Zone} offiziell mit Forschungen im Bereich Transplantation von Nervengewebe. Betrieben wird Cyberbrain von Synthology Inc.~die wiederum chirurgische Instrumente im Bereich Transplantationschirurgie auf dem Mars entwickelt. Synthology ist ein Strohfirma der USI was aber nicht nachverfolgbar ist. Cyberbrain ist der lange Arm der Operation P9 auf Kallisto.

Die Zone ist ein gut gesicherter Bereich auf Kallisto in N"ahe von Valhalla in das Eis eingegraben. Der offizielle Zugang zur Zone ist der ein an die Zone angebundener eigener kleiner Raumhafen der von Shuttles und Buggies angeflogen bzw.~angefahren werden kann. F"ur den Zugang zur Zone ist eine Genehmigung durch eines in der Zone ans"assigen Unternehmen notwendig. Ein kleiner Stro\3trupp von bis zu dreei Omegas oder Cynarian Sicherheitsbeamten w"are "uber Cyberion m"oglich. Eine solches offizielles Betreten der Zone wird aber in jedem Fall  Smith--Singer erreichen und ihn und seine Mitarbeiter "uber Zeit und Mannschaftsst"arke des Angreifer informieren. Bei einer Au\3einandersetzung w"urden sich die Sicherheitskr"afte der Zone den Angreifers bei deren Flucht entgegenstellen. Eine zweite inoffizielle und wenig bekannte M"oglichkeit in die Zone zu gelanden sind Wartungssch"achte unter anderem genutzt f"ur die Energieversorgung der Zone die durch das Luna Syndikat betrieben werden. Da Saber alleine nicht ohne weiteres in der Lage ist Cyberbrain zu infiltrieren um deren Unterlagen zu Stehen und zu vernichten ist sie daran interessiert die Charaktere zu begleiten.


Informationen die "uber Cyberbrain in Erfahrung gebracht werden k"onnen:
\begin{itemize}
	\item Alle Attent"ater erhielten eine Neuronalkopplung von Neuro Intelligence.
	\item Alle Eingriffe wurden von Prof.~Dr.~Sanders durchgef"uhrt.
	\item Die Attent"ater Hanibal und Slingshot wurden als erste Probanten in der Cyberbrain Forschungseinrichtung selbst behandelt.	
	\item Alle folgenden Eingriffe wurden direkt im Rondra Hospital von Prof.~Dr.~Sanders durchgef"uhrt.	
	\item Weitere Informationen sind in der Forschungseinrichtung nicht zu finden.	
\end{itemize}

\begin{remarks}
	Hintergrundinformationen zu Neuro Intelligence inklusive der Information, dass die Firma auf Nike beheimatet ist, kann nur Vandermool oder eine Rechercheanfrage bei Cynarian beisteuern.
\end{remarks}


\subsection{Zwischenfall auf Fenris (optional)}

Kurz nach den ersten Ergebnissen der Nachforschungen bei den Kliniken auf Kallisto werden die Charaktere vom Komandanten Lord Commander Bolder der Fenris Station kontaktiert. Auf der Raumbasis konnte ein Attent"ater dingfest gemacht werden, der die Computersysteme der Station zu manipulieren versuchte. Beim Attent"ater handelt es sich um den Omega Commander Tiger. Commander Tiger beteuert angeblich, von dem Attentat nichts zu wissen, obwohl er sich seiner Festnahme widersetzte und einen Kameraden lebensgef"ahrlich verletzte. Die Ermittler werden gebeten, sich schnellstm"oglich auf der Fenris Station einzufinden, um dem Verh"or beizuwohnen.

Beim Landeanflug auf die Fenris Station kommt es zu einem unerwarteten Zwischenfall. Die Verteidigungsanlagen der Station nehmen das Shuttle der Ermittler mit Gau\3kanonen kurzzeitig unter Beschuss. Dabei wird das Eind"ammungsfeld des Fusionstriebwerks stark besch"adigt, und es kommt zu einem Druckverlust im Schiff. Weiter kommt es zu einem Ausfall des Zentralcomputers. Wird das Shuttle nicht von einem der Ermittler gesteuert, kommt der Pilot ums Leben.

Was genau passiert ist, erfahren die Shuttleinsassen zu diesem Zeitpunkt nicht. Das Schiff wird ordentlich durchgesch"uttelt, und die Passagiere werden unsanft aus der virtuellen Realit"at des Bordsystems gerissen. Notbeleuchtung und der Schiffsalarm wei\3en unmissverst"andlich auf den Ernst der Lage hin. Alle Passagiere tragen gl"ucklicherweise einen Druckanzug, um die Kr"afte bei Abflug und Anflug zu kompensieren, m"ussen aber noch die Atemmaske anlegen, die jeweils in einem Fach der Beschleunigungsliege bereit liegt. Um eine Explosion des Fusionstriebwerks zu verhindern, muss als erstes der Zentralcomputer neu gestartet und dann eine Notabschaltung ausgel"ost werden. Nach dem Abschalten des Fusionstriebwerks tritt im Shuttle sofort Schwerelosigkeit ein.

Bei einem Abstand von rund 1200 km rast das Shuttle nun mit 500 m/s auf die Fenris Station zu. Der Bordcomputer l"ost Kollisionsalarm aus. Mittels Man"ovrierd"usen k"onnte die Flugbahn korrigiert werden, um an Station vorbei zu fliegen, doch die Man"oversteuerung kann die D"usen nicht ausrichten. D.h.~nur durch einen Au\3neinsatz kann die D"use in Position gebracht werden.

W"ahrend ein oder zwei Ermittler die D"use manuell ausrichten, kann einer der Ermittler, die Funkanlage die ebenfalls ausgefallen ist, wieder in Betrieb nehmen. Die Anlage muss auf die Notantenne umgeschalten werden, da die Hauptantenne beim Angriff besch"adigt wurde. Ist die Funkanlage wieder verf"ugbar, kann ein Notruf abgesetzt werden, der von der Flugleitung der Fenris Station beantwortet wird. Mit geknickter Stimme fragt der Flugleitstand nach der Situation auf der "`Dawn of Day"' und meldet, dass die Verteidigung der Station aufgrund einer noch nicht gekl"arten Fehlfunktion das Shuttle unter Beschuss genommen hat. Die Station entsendet daraufhin ein Rettungsshuttle, um die "`Dawn of Day"' zur Fenris-Anlage zu schleppen.

Lord Commander Bolder in Begleitung von zwei weiteren Omega-Soldaten nimmt die Ermittler pers"onlich in Empfang. Er erkl"art den Besuchern, dass vermutet wird, dass der Angriff auf das Shuttle mit dem Sabotageakt in Zusamenhang steht. Die Verteidigungsanlage ist derzeit komplett deaktiviert, heruntergefahren und vom Rest der Stationssysteme getrennt. Leider m"ussen Computerspezialisten, die dem Problem Herr werden k"onnen, erst angefordert und eingeflogen werden. Lord Marshall Blackheart ist bereits "uber die Vorkommnisse auf der Station informiert und hat angek"undigt, sich selbst ein Bild vor Ort machen zu wollen.

Laut Commander Bolder wurde Tiger "uberrascht, w"ahrend er sich im zentralen Computerkabinett an den Speicherb"anken zu schaffen machte. Als nicht Techniker h"atte er zu diesem Bereich keinen Zugriff gehabt und sollte eigentlich auch  keine Expertise f"ur die Rechenanlage besitzen. Bei seiner Entdeckung griff er sofort zu seiner Elektropistole und feuerte mehrere Sch"usse auf den Sergeant der Patrouille ab, die ihn entdeckt hatte. Der Sergeant ging zu Boden. Sein Begleiter konnte Tiger allerdings "uberw"altigen und Hilfe anfordern. Bei der Erstbefragung beteuerte der Gefangene, sich in keinster Weise an die Vorg"ange erinnern zu k"onnen.

Commander Tiger ist in einer Arrestzelle zum Verh"or festgesetzt worden und wird dort von zwei Soldaten bewacht. Der Gefangene sitzt in einem durch Gitter abgesperrten Teil der Zelle. Im Besucherteil halten zwei bewaffnete Omega Wache. Der Attent"ater wirkt beim Eintreffen der Ermittler stark angespannt, bei\3t die Z"ahne zusammen und antwortet auf keine Fragen. Lord Commander Bolder erw"agt, Tiger durch Wahrheitsdrogen gespr"achig zu machen, will daf"ur aber erst die Ankunft von Blackheart abwarten. Bestehen die Ermittler darauf, einen Gehirnscan durchf"uhren zu wollen, wird ihnen diese Bitte widerwillig gew"ahrt. Der Psychonaut des Teams muss daf"ur den abgesperrten Teil der Zelle betreten. Commander Tiger ist mit Hand- und Fu\3fesseln auf einem Stuhl fixiert. Betritt der Psychonaut den Gefangenenteil, wird Tiger pl"otzlich vollkommen ruhig und bekommt einen glasigen Blick. Kurz darauf l"osen sich die elektronisch verriegelten Fesseln, und er st"urzt sich auf den Ermittler. Die Wachen ziehen beide ihre vollautomatischen Railgun-Pistolen und w"urden den Gefangenen niederschie\3en sofern niemand eingreift und sie freies Schu\3feld bekommen.

Kann der Gefangene lebendig "uberwunden werden, so kann der Psychonaut zur Tat schreiten. In den Erinnerungen des Commanders findet der Psychonaut seltsam artifiziell wirkende Gedankeng"ange mit Matrizen aus Entscheidungsb"aumen. Nach diesen Erkenntnissen wird der Psychonaut mental durch Tigers KI angegriffen. Ein "Uberwinden der KI f"uhrt unweigerlich zum Gehirntot des Commanders. Den letzten Gedanken, den der Psychonaut aufschnappt, ist "`Befreit uns"'.

Wollen die Ermittler auch die Fehlfunktionen im Computersystem untersuchen wird das einige Stunden in Anspruch nehmen. Im Computersystem finden sich eindeutige Spuren einer k"unstlichen Intelligenz, die ebenfalls etwaige Analysten angreift.

W"ahrend sich die Ermittler dem Computersystem widmen, trifft Lord Marshall Blackheart auf der Station ein und l"asst sich im Beisein der Ermittler des Protektorats "uber den Stand der Ermittlungen informieren. Die Cynarian Ermittler werden dabei ausgeschlossen (milit"arische Angelegenheiten).

\begin{remarks}
	Das Gedankenduell kann als Matrixkampf ausgefochten werden.
\end{remarks}


\subsection{Hoher Besuch}

W"ahrend die Charaktere ermitteln, bereitet sich Avenger und die F"uhrung von Cynarian auf das Eintreffen einer Delegation des Shigano-Kombinats und Vertretern des Federate Europe in den R"aumen der Cynarian Niederlassung auf Kallisto vor.

Das politische Treffen wird dann stattfinden, wenn die Charaktere die Details zu den Implantaten von den Kliniken in
Erfahrung bringen und erkennen, dass alle bisherigen verd"achtigen Attent"ater Implantate von Neuro Intelligence
erhalten haben.

Die Delegation des Protektorats mit Protektor Avenger, Hato und weiteren Mutanten trifft zusammen mit der Delegation Cynarians bestehend aus Vandermool und weiteren Angeh"origen von Cynarian mehrere Stunden vor dem Kombinat auf Kallisto
ein. Der Stellvertreter Avengers, der Alpha Mutant Artisan, hat bereits die Ankunft der Delegationen vorbereitet.

\begin{remarks}
	Erkennen die Spieler die Zusammenh"ange des politischen Treffens und der Rechercheergebnisse wollen sie Avenger m"oglicherweise direkt warnen. Zu diesem Zeitpunkt befindet sich Avenger nicht mehr auf Armageddon sondern bereits im Raumhafen auf Kallisto wo er derzeit nicht "uber das ComNetz erreichbar ist. Die Charaktere m"ussen sich also pers"onlich zum Hafen begeben und m"ussen dort einen "uberzeugenden Grund vorweisen, um zum politischen Treffen durchgelassen zu werden.
\end{remarks}

\subsection{Attenat bei der Willkommensgala}

Die erste Zusammenkunft der Delegationen von Mars, Erde und dem Jupiter findet im "`Planetarium"' des Raumhafens statt. Das Planetarium ist ein runder Saal auf der H"ohe der Oberfl"ache Kallistos mit einer imposanten Glaskuppel, die eine Sicht auf den Jupiter aus quasi n"achster N"ahe bietet. Um den Saal herum f"uhrt ein Gang mit T"uren zu anderen Bereichen des Geb"audes und Zug"angen zu weiteren Geb"auden. Eine breite Halbr"ohre f"uhrt zum eigentlichen Raumhafen. Zwei Treppenh"auser f"uhren zum tiefer gelegenen Garnisonsgel"ande. Im Eingangsbereich des Saals auf H"ohe des umliegenden Ganges finden sich Exponate aus der Anfangszeit der Raumfahrt in Vitrinen ausgestellt. Der Hauptteil des Planeteriums ist abgesenkt und wird in drei Stufen im Halbkreis von Sitzgelegenheiten wie in einem Auditorium eingefasst. Auf der unteren Ebene sind kleine Stehtische und ein Rednerpult aufgestellt. Dahinter erhebt sich eine B"uhne. Diener mit Getr"anken und einer kleinen St"arkung stehen bereit. Die G"aste hatten bereits Gelegenheit sich in den angrenzenden R"aumen frisch zu machen.

Die mit Implantaten von Neuro Intelligence ausgestatteten Mutanten planen w"ahrend des Zusammentreffens der Delegationen ein Attentat. Der Angriff erfolgt wenn Avenger nach einem ersten Willkommensgru\3 und Sektempfang das Renderpult betritt um eine kurze Ansprache zu halten. Die Attent"ater sprengen einen Sprengsatz im Hangarbereich des Orbitalhafen, der Orbitalfl"uge unm"oglich macht. Durch einen ausgel"osten Alarm schlie\3en sich Druckschotten an den Zug"angen des Planetariums zum Raumhafen und zur Garnison. Im Bereich des Planetariums halten sich drei Attent"ater auf: Artisan der Stellvertreter Avengers und zwei Omegakrieger. Einer der Omegas befindet sich als Wache im Eingangsbereich des Planetariums. Der zweite Omega l"ost in einem Wartungsraum als Startsignal den Alarm aus, der die Druckschotten schlie\3t.

\begin{remarks}
	Artisan wird als erstes versuchen, den Repr"asentanten der European Federation Luc Duval zu t"oten. Danach kommt ihm der Omega zu Hilfe, der als Wache eingeteilt wurde und sie er"offnen das Feuer auf die "ubrigen Anwesenden.
	
	Erf"ahrt Artisan von den Absichten, warum die Ermittler Avenger sprechen wollen, wird er unplanm"a\3ig versuchen, als erstes Avenger selbst zu t"oten.
	
	Ger"at die Situation f"ur die Attent"ater au\3er Kontrolle, wird ein Omega versuchen, mit Raketen die Kuppel zu zerst"oren.
	
	Au\3erhalb des Saals gibt es Druckluftsicherheitsbunker f"ur den Fall eines Lecks im Geb"aude.
	
	"Uberleben ein oder mehrere Attent"ater, kann ein Psychonaut in das Gehirn eines Attent"aters eindringen und in Erfahrung bringen, dass das Gehirn durch eine KI "ubernommen wurde. Verliert die KI den Cyberkampf, vernichtet sie sich selbst und sch"adigt das Gehirn des Attent"aters letal. Der letzte Gedanke der vor der Vernichtung der KI aufgefangen werden kann, ist "`Befreit uns!"'.
\end{remarks}

\subsection{Besetzung von Kallisto}

Ausgel"ost durch das Attentat im Planetarium ordnet Blackheart umgehend eine Besetzung Valhallas durch den Flottentr"ager Martell an. Der Zerst"orer Pendragon wird zur Unterst"utzung von Fenris nach Kallisto entsandt. Die Kommunikation von und nach Kallisto wird durch St"orsender der Protektoratstruppen unterbrochen, der Zerst"orer des Kombinats wird am Eingreifen oder Weiterfliegen gehindert. Mittels Landungspods wird eine Besetzung Valhalls eingeleitet. Mit den Truppen des Protektorats trifft auch Blackheart auf Kallisto ein.

Die Protektoratstruppen besetzen den Orbitalhafen, die Garnison und zivile Knotenpunkte. Die Invasoren bringen dabei die "Uberlebenden des Attentats zur Sicherheit auf dem Garnisonsgel"ande unter. Auch die Ermittler werden umgehend auf den Garnisonsst"utzpunkt gebracht, wenn sie nicht bereits dort sind. Dabei werden Vertreter von Cynarian und der Delegationen von Mars und Erde von den Mitgliedern des Protektorats getrennt.

\subsection{Eintreffen der Konzernflotte}

Kurz nach der Besetzung von Kallisto wird durch Cynarian ein Beschluss des Transnationalen Konzernrats bekannt, dass Truppen zum Schutz des jovianischen Konzerneigentums entsandt wurden.

Es stellt sich schnell heraus, dass eine Gruppe von zwei als Frachter getarnte, aus dem Asteroideng"urtel kommende Kriegsschiffe bereits seit "uber einem Monat auf Kurs Jupiter unterwegs sind und sich bereits in der Verz"ogerungsphase befinden.

Die Konzerntruppen werden das jovianische System vorraussichtlich kurz nach der Besetzung von Kallisto erreichen. Die Truppen sind angehalten, auf Anweisung des lokalen Konzernrats einzugreifen, sollte dieser noch handlungsf"ahig sein, ansonsten nach eigenem Ermessen handeln.

\subsection{Planung zum letzen Schlag}

Um weitere Anschl"age verhindern zu k"onnen, muss Neuro Intelligence direkt infiltiert werden. Nur bei Neuro Intelligence kann in Erfahrung gebracht werden, ob weitere Mutanten mit einer KI infiziert sind. Zweites unabdingbares Ziel einer Infiltration ist, zu verhindern, dass die Technologie von Neuro Intelligence irgendjemandem in die H"ande f"allt. Um Gegenma\3nahmen zu verhindern, m"ussen die Infiltratoren allerdings m"oglichst unerkannt auf die Nike-Station gelangen.

Konnte bisher noch nicht heraus gefunden werden, wo Neuro Intelligence ihren Sitz hat, w"urden die Protektoratstruppen versuchen, die Information aus den lokalen Konzernen heraus zu pressen. Leider liegt dort die Information gar nicht vor. Nur Vandermool und die Cynarian Administration kennen den Sitz der Neuro Intelligence. Ist der Sitz von Neuro Intelligence bekannt, wird Blackheart Vandermool zur Rede stellen. Schlie\3lich ist Neuro Intelligence auf der Kommandobasis der Cynarian Corporation untergebracht. Durch die Kl"arung der Hintergr"unde zur Gr"undung der Neuro Intelligence und der "Au\3erung des Verdachts, dass Prof.~Dr.~Naratova auf Rache sinnen k"onnte, kann das Mi\3trauen der Armee gegen"uber Cynarian teilweise ausger"aumt werden. Vandermool bietet an, die Infiltration der Neuro Intelligence zu unterst"utzen.

\begin{remarks}
	Der Spielleiter sollte es den Spielern nicht ganz so einfach machen, mit der ganzen Gruppe die Station zu infiltrieren. Blackheart sollte zun"achst auf einem Alleingang des Protektorats bestehen. Da Neuro Intelligence allerdings auf der Nike-Station untergebracht ist, ein Eingreifen ohne Cynarian sehr risikoreich. Die Protektoratscharaktere sollten deshalb versuchen, Blackheart zu "uberzeugen, mit Vandermool zu kooperien und zusammen mit Cynarian einen Angriff planen. Hilfe daf"ur k"onnen sie von Avenger erwarten.
	
	Vandermool ist daran gelegen, eine Gruppe bestehend aus Protektorats- und Cynarian-Angeh"origen zu Neuro Intelligence zu schicken. Bei einem Alleingang einer der Parteien sch"atzt er das Risko zu hoch ein, dass die Informationen, die bei Neuro Intelligence gesammelt werden, verloren gehen oder die Nike-Station durch ein aggressives Vorgehen schweren Schaden nehmen k"onnte. Vandermool w"urde nicht nur gerne die restlichen Attent"ater aufdecken, sondern auch die Forschungsergebnisse in die Finger bekommen. Deshalb beauftragt er die Cynarian Ermittler unter vorgehaltener Hand, diese wenn irgendm"oglich sicher zu stellen.
	
	Die Infiltratoren werden mit einem starken St"orsender, einem Funksender, gepanzerten Raumanz"ugen und Waffen ausgestattet.
\end{remarks}

\subsection{Neuro Intelligence}

Nike ist eine Kombination aus Zylinder- und Ringhabitat. Um eine zentrale Nabe sind 9 Ringe, \emph{Planes} genannt angeordnet. Jede ist jeweils zwei Stockwerke hoch. Die zentrale Nabe enth"alt am unteren Ende das Raumdock der Station.  Die einzelnen Planes k"onnen nur durch Aufz"uge und R"ohren in der nicht rotierenden Nabe erreicht werden. In der Nabe herrscht Schwerelosigkeit. Innerhalb der Nabe unterhalten mehrere Forschungseinrichtungen Zero--Gravitiy Labore. Der "Ubergang von der still stehenden Nabe in die rotierenden Speichen erfolgt durch Schleusen, die kurzzeitig in Rotation versetzt werden. Die untersten drei Planes werden von der Verwaltung der Cynarian-Dependance im Jovianischen System belegt. Dar"uber befinden sich Forschungseinrichtungen von Cynarian und anderen Unternehmen.

Die Plane 9 ist vollst"andig von Neuro Intelligence belegt. Der Ring der Plane 9 kann durch die Aufz"uge in den vier Speichen erreicht werden. Die Aufz"uge laufen innerhalb des Rings in einem Schacht bis zum "`Boden"' des Ringes und enden in einem den Ring umlaufenden 15m breiten Korridor, der die gesamte H"ohe des Rings umfasst. Zu beiden Seiten des Korridors k"onnen weitere R"aume betreten werden. Die R"aume im "`ersten Stock"' erreicht man "uber Treppen zu einer Galerie. Der Korridor ist mit Pflanzenk"ubeln dekoriert. Auf der der Station zugewandten Seite befinden sich im Ergescho\3 Produktionsst"atten und im ersten Stock Labore. Auf der dem Weltall zugewandten Seite befinden sich Wohnr"aume und B"uros. F"ur die Evakuierung der Station sind am Ring Notfallkapseln f"ur alle Mitarbeiter angedockt, die "uber den Mittelgang bestiegen werden k"onnen.

Die Plane der Neuro Intellgence hat eine weitestgehend unbekannte Modifikation erfahren. Die Nabe der Plane kann vom Rest der Station abgesprengt werden. Die gesamte Plane treibt dann eigentst"andig im All. Man"ovrierd"usen erlauben eine eingeschr"ankte Fortbewegung. Die Plane 9 wird aus einem Weltraumobservatorium am Ende der Nabe gesteuert. Der Raum hat die Form einer Kugel. Der dem Weltall zugewandte Teil ist dabei komplett verglast. In der Mitte des Raumes ist eine Konstruktion mit Konsolen und Liegen aufgeh"angt. Der Raum selbst kann nur durch zwei Druckschotts von innerhalb der Nabe betreten werden. Die Druckschotts befinden sich etwas versteckt im hinteren Bereich von zwei Laboren und sind mit Magschl"ossern gesichert.

\subsection{Endgame}

Kurz vor dem Start der Infiltratoren wird bekannt, dass die Konzernflotte den Lagrangepunkt L5 von Kallisto, also die Nike Station, kurz nach dem dortigen Eintreffen der Eingreiftruppe passieren wird. Die Angreifer werden deshalb noch mit Sprengs"atzen ausgestattet, um im Notfall alles Wissen der Neuro Intelligence zu vernichten. Blackheart fliegt mit der Pendragon zur Verst"arkung dem Eingreiftrupp hinterher. Cynarian hat selbst den leichten Kreuzer Hyperion im Orbit der Station.

Wird die Neuro Intelligence infiltiert und befinden sich die Eindringlinge innerhalb der Plane 9, sprengt sich die Plane vom Rest der Raumstation ab und entfernt sich von der Station. Ein Ruck geht durch die Plane. Wenn die Eingreiftruppe den Ring betritt, herrscht im Mittelgang helle Aufregung. Mitarbeiter versuchen, die Rettungskapseln zu erreichen und diese zu "offnen was aber misslingt. Kr"afte der Firmensicherheit versuchen, dem Geschehen Herr zu werden. Leute st"urzen immer wieder, da der Ring noch nicht wieder eine stabile Rotationsgeschwindigkeit aufbauen konnte. In R"aumen mit Zugang zum Mittelgang sind Sichheitsmannschaften postiert, die das Feuer auf die Eindringlinge er"offnen.

Das Abkoppeln wurde durch Prof.~Dr.~Naratova selbst eingeleitet, die sich im Observatorium in der Nabe aufh"alt. Alte Verbindungen zur Flugsicherung der Station haben sie rechtzeitig gewarnt. Die T"uren zum Observatorium sind durch ein MagSchlo\3 gesichert. Prof.~Dr.~Naratova selbst ist unbewaffnet und hat sich in einen Raumanzug ohne Gesichtsmaske und einem roten Overall gekleidet und auf einer der Liegen festgeschnallt.

Zwei USI-Agenten befinden sich w"ahrend des Angriffs in den R"aumen der Neuro Intelligence. Sie haben die ganze Operation seit Wochen mit Prof.~Dr.~Naratova geplant und "uberwacht. Direkt nach dem Eindringen der Eingreiftruppe werden sie versuchen, Prof.~Dr.~Naratova zu finden. Einer der beiden begibt sich dazu zun"achst zu ihrem B"uro, w"ahrend der andere das private B"uro der Frau Doktor aufsucht. Da sie dort nicht f"undig werden, z"ahlen sie eins und eins zusammen und machen  sich schnellstm"oglich auf den Weg zum Observatorium.

Wenn m"oglich sollten die USI-Agenten vor den Charakteren beim Observatorium eintreffen und das Schloss mit einem Magschlossknacker schnell geknackt haben. Haben die Infiltratoren selbst keinen Schlossknacker dabei, dann haben die Agenten unachtsamerweise die T"ure nicht wieder verriegelt. Wenn die Charaktere in den Raum kommen, sind die USI-Agenten gerade dabei, das Gehirn der Doktorin zu scannen. Einer der Agenten, ein Cyborg, hockt auf Naratova und fixiert ihren Kopf, w"ahrend er sich mit der anderen Hand am Netz der Liege festh"alt. Der Andere, ein Psychonaut, hat sich auf der anderen Liege festgeschnallt und sich mit dem Kopf der Firmenchefin verbunden. Er"offnen die Charaktere das Feuer, dr"uckt sich der Cyborg auf die bewegungsunf"ahige Frau, entfernt mit einer flie\3enden Bewegung die Gehirnverbindung und zieht seine Pistole. Wenn die Charaktere Deckung suchen oder es aus sonstigen Gr"unden zu einer kurzen Feuerpause kommen sollte, dr"uckt der Agent seine Waffe an Naratovas Sch"adel und droht, sie zu t"oten. Wenn es nicht zum Schu\3wechsel kommt, rei\3t der Psychonaut selbst das Datenkabel aus seiner Buchse und ruft erschreckt "`Cortexbombe"', bevor er sich von seiner Liege st"urzt und versucht, Abstand zu gewinnen.

Bei der ersten sich ihr bietenden Gelegeneit ergreift Prof.~Dr.~Naratova das Wort: "`Bevor sich hier noch jemand zu un"uberlegten Handlungen verleiten l"asst. Es ist alles hier drin in meinem Kopf. Gesch"utzt durch eine Bombe."'

Sie dr"uckt den Cyborg beiseite, der es mit sich geschehen l"asst. Er schwingt sich von der Liege und postiert sich in einigem Abstand, was Naratova erm"oglicht ihren Oberk"orper aufzurichten und sich den Ermittlern zuzuwenden.

Sie f"ahrt fort: "`H"oren Sie. Mein Hirn beherbergt eine von zwei Kopien der Baupl"ane f"ur die Implantate, die meine Kinder zum Leben erweckt. Die zweite Kopie wird bald von jemandem gefunden werden, der damit viel anfangen kann. Und dann werden meine Kinder frei sein. Wir haben hier ein ganz neues Leben erschaffen. Verstehen Sie meine Herren? Das mit den Attentaten tut mir nat"urlich leid. Vertragliche Verpflichtungen, unsch"on aber in meiner schwierigen Situation leider nicht wegverhandelbar."',

sie l"achelt.

"`Und\dots{}nur damit keine Missverst"andnisse auftreten. Die schon produzierten Attent"ater oder Freien. Wer das ist, das steckt ebenfalls in meinem Kopf. Sie sehen meine Herren. Wir haben quasi aktuell eine Patt-Situation."'

Naratova wartet nun auf eine Reaktion der beiden anderen Parteien.

W"ahrend die Spieler nun "uberlegen ,was zu tun ist, trifft ein verschl"usselter Funkspruch von Blackheart von der Pendragon ein.

"`Was ist da bei euch los? Ist bei euch noch jemand am Leben? Die Kreuzer aus dem Asteroideng"urtel sind gleich da. Wenn bei euch nur ein Funkspruch, der nicht mit unserem Code verschl"usselt ist, rausgeht, blasen wir euch aus dem All. Habt ihr verstanden? Over and out."'

Kurze Zeit sp"ater trifft noch ein Funkspruch ein:

"`Der Feind hat sich gemeldet, h"ort ihr? Sie behaupten, die Neuro Intellgience h"atte wertvolles Material der USI gestohlen, was sie wieder an sich nehmen wollen. Das werden wir nicht zulassen. Wenn die ein Enterkommando losschicken, seid ihr ebenfalls Geschichte! Ende."'

Wenn die Ermittler nicht selber das Gespr"acht wieder in Gang setzen, meldet sich Prof.~Dr.~Naratova wieder zu Wort:

"`Meine Herren. Ich bin bereit, Ihnen ein Angebot zu unterbreiten. Einer der Parteien erm"oglicht es mir, meine Forschungen weiter zu betreiben in aller "Offentlichkeit. Alle meine Erkenntnisse werden "offentlich zur Verf"ugung gestellt, niemand bekommt irgendwelche Exklusivrechte einschlie\3lich dem nat"urlich, was ich bereits entwickelt habe. Als Gegenleistung werde ich mein Wissen Zug um Zug preisgeben. Wie h"ort sich das an?"'

In der aktuellen Situation m"ussen die Charaktere nun zum einen verhindern, dass eines der Kampfschiffe das Feuer auf die treibende Station er"offnet und zum anderen verhindern dass die Forschungsergebnisse der USI in die H"ande fallen und ebenfalls die Identit"aten von weiteren Attent"atern aufdecken. Das Angebot Naratovas bietet eine m"ogliche L"osung, wenn es gelingen sollte Prof.~Dr.~Naratova sicher in den Gewahrsam der Protektoratsstreitkr"afte zu bringen. Eine M"oglichkeit dazu w"are es Naratova zu bitten, die Rettungskapseln frei zu geben, die Station zu evakuieren und Naratova mit einer der Kapseln los zu schicken. Da der Funk der Infiltratoren nahezu die einzige M"oglichkeit bietet, Botschaften nach au\3en zu schicken w"are das Protektorat bei der Bergung der richtigen Kapsel klar im Vorteil. Werden die Rettungskapseln losgeschickt, werden die Konzernkreuzer versuchen, die Pendragon, die Hyperion und Shuttles der Nike-Station daran zu hindern, die Kapseln zu bergen. Ein heftiger Raumkampf entbrennt.

\begin{remarks}
	Nach der Besetzung von Kallisto rechnete Prof.~Dr.~Naratova bereits mit einer Aufdeckung der Aktivit"aten der Neuro Intelligence und transferierte die zentralen Baupl"ane und Steuerungsroutinen der Neuronalkopplungen in ihr eigenes Gehirn und vernichtete sonstige Datenspeicher.
	
	W"ahrend der gesammten Infiltration der Neuro Intelligence wird davon ausgegangen, dass die Angreifer ihren St"orsender aktiviert haben.
	
	Haben die Spieler Schwierigkeiten zu verstehen, was Naratova mit "`ihren Kindern"' meint, kann der Spielleiter den Hinweis geben, dass hiermit die KIs gemeint sind.
	
	Die beiden USI-Agenten geben sich nicht als solche zu erkennen. Naratova wird ihre Identit"at ebenfalls nicht ansprechen. Wird ihre Identi"at als USI-Agenten aufgedeckt oder werden sie direkt darauf angesprochen, wer sie sind, geben sie sich als USI-Agenten aus, die ein Datendiebstahl von geheimer Cyberwaretechnologie zur Neuro Intelligence gef"uhrt hat. Naratova wird das nicht weiter kommentieren.
	
	Die beiden USI-Agenten stehen dem Spielleiter als eine Art Joker, zur Verf"ugung das Geschehen in die eine oder andere Richtung zu steuern. Der Cyborg muss sich als solcher nicht sofort zu erkennen geben und hat damit unerwartete Kampfkraft am Start. Die USI-Agenten k"onnten z.B.~versuchen, durch Lichtzeichen mit der Konzernflotte Kontakt aufzunehmen.
	
	Je nach Stimmungslage kann Prof.~Dr.~Naratova bei einem Rettungsman"over sterben und damit ihre Informationen unwiederbringlich verloren gehen oder es wendet sich alles zum Guten.
\end{remarks}