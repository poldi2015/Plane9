\section{Szenen}

\subsection{Prolog (optional)}

Um die Ermittler auf den Plot, ihren Charakter und auf die vorherrschenden politischen Gemengelage vorzubereiten bietet es sich an mit jedem der Spieler einzeln eine Einf"uhrungsrunde zu spielen:

Der Vertraute des Chefermittlers ist Colonel Scholz. Er wird den Chefermittler auf das Treffen mit Vandermool, mit dem der Ermittler selbst noch nicht viel zu tun hatte, einweisen. Vandermool ist auf ein gutes Verh"altnis mit dem Protektorat bem"uht wird aber bei einer Zusammenarbeit die F"uhrung behalten wollen. Zudem ist nat"urlich nicht klar ob die Mutanten in die Anschl"age verwickelt sind oder evtl. ~sogar Cynarion selbst deshalb ist Vorsicht geboten.

Für den Assistenten bietet sich eine Einweisung in T"atigkeit als Psychonaut an und ihn dabei gleich direkt die Geschichte einzubinden. Er erh"alt auf dem Mars die Aufgabe einen Agenten der \emph{USI}, der bei der R"uckreise vom jovianischen System von Piraten gefangen und an Cynarion ausgeliefert wurde, zu befragen. Der Agent ist Lieferant f"ur Gelder oder technische Ausr"ustung an \emph{Neuro Iintelligence}.

Der Vertraute von Avenger trifft sich am Vorabend mit Artisan um einen zu heben. Dieser erkl"art ihm von den Vorf"allen die der Ermittler jedoch schon kennt und weiht ihn bereits ein das ein Treffen mit Repr"asentanten aus der Cynarion geplant ist um die Vorkommnisse zu untersuchen.

Den Omage nimmt sein Vorgesetzter Thunderbolt zu einem Treffen am Raumhafen von Armageddon mit Blackheart mit. Blackheart trifft mit der \emph{Martell} ein um an der Einweisung der Ermittler vor Ort teilzunehmen. Blackheart nimmt den Charakter zur Seite und weist ihn an ein waches Auge auf die Ermittlungen zu haben da der Cynarian Seite nicht voll zu trauen ist und vorraussichtlich Informationen vorenthalten werden.
Der Ermittler bekommt den Befehl t"aglich oder nach bei wichtigen Erkenntnissen Report an Thunderbolt zu leisten. In die Szene sollten mitlit"arische Gepflogenheiten mit einflie\3en. Auch sollte das Treffen ein Bild von der bereits l"agend"aren Anf"uhrerin der Protektroratstruppen und deren Adjutenten formen.

\subsection{Einweisung bei Cynarian}

Der Cynarian Chefermittler wird durch Eric Vandermool, Colonel Scholz und Dr.~Petrova in einem Konferenzraum der Cynarian Sektion auf Armageddon als erster eingewiesen. Der Charakter wird in die R"aume durch den Sekret"ar Vandermools Herny Longdale gebracht.

Das B"uro Vandermools ist sehr ger"aumig, schlicht und k"uhl aber erlesen eingerichtet. Vandermool wird das Gespr"ach von seinem Schreibtisch aus f"uhren.

Er erf"ahrt von der Sabotage auf der Mine HeM05 vor drei Tagen, der Havarie der Mine HeM03 vor zwei Wochen und der Fehlfunktion der Schlepperinsel vor drei Wochen. Nur der Vorfall auf der Mine HeM05 wird bereits als Attentat eingestuft. Es wird aber gemutma\3t, dass es sich auch bei anderen Vorkommnissen um Attentate handelt. Die USI als potentieller Drahtzieher wird direkt angesprochen. Vandermool ist offen beunruigt und betont, dass weitere Vorkommnisse nicht tragbar w"aren.

Der Ermittler wird aufgefordert, sich dann bei Protektor Avenger als offizieller Ermittler der Cynarian Corporation zu melden. Er soll Scholz "uber den Stand der Ermittlung jederzeit auf dem Laufenden halten. Kontaktmann des Ermittlers ist also Scholz oder enry Longdale f"ur den direkten Kontakt zu Vandermool.bW"ahrend der Ermittlung stehen die Cynarian Ermittler im Dienste der inneren Sicherheit von Cynarian. Alle Ergebnisse unterliegen der Geheimhaltung.

\begin{remarks}
	Der zweite Ermittler ist bereits in den R"aumlichkeiten h"alt sich aber zur"uck. Die Anwesenheit des zweiten Ermittlers kann z.B.~erst am Ende der Erl"auterungen der Gegebenheiten erfolgen um zu zeigen das er bereits eingewiesen wurde und potentiell Wissen besitzt das dem Chefermittler nicht zug"anglich gemacht werden soll.
	
	Vandermool ist ganz klar dominant in dem Gespr"ach und vollkommen souver"an. Scholz und Dr.~Petrova sind als kompetent und zielstrebig effizient bekannt. Scholz ist ein erfahrener Milit"arangeh"origer.
	
	Wie vertraut die Cynarian F"uhrung mit dem Protektorat ist, ist den Ermittlern nicht bekannt. Die genauen Umst"ande die, zur Besiedelung des jovianischen Systems gef"uhrt haben, sind ebenfalls nicht bekannt.
	
	Detaillierte R"uckfragen sind bei diesem Gespr"ach unangebracht. Vandermool bittet die Ermittler sind bzgl.~Fragen, t"aglicher Reports vertrauensvoll an Henry Longdale zu wenden. Henry Longdale ist damit der direkte Ansprechpartner der Cynarian Ermittler wird aber selbst keine Entscheidungen treffen. Vandermool ist damit nicht im Zugzwang irgendwelche Informationen bereit zu stellen.
\end{remarks}

\subsection{Einweisung beim Protektorat}

Der Chefermittler des Protektorats wird durch Protektor Avenger und seinen Stellvertreter Artisan und seinen Leibw"achter Hato in den Konferenzr"aumen des Protektorats auf Armageddon eingewiesen. Die Atmosph"are ist freundschaftlich. Der Ermittler erf"ahrt von den Vorkommnissen auf den Minen HeM03 und HeM05 und der Explosion beim Anbau der Habitate. Die Havarie der Mine HeM03 und das Habitatsungl"uck werden derzeit noch als Unf"alle gewertet. Der Chefermittler wird gebeten, gewonnene Erkenntnisse an Avenger Stellvertreter Artisan zu berichten und als geheim einzustufen.

Avenger erkl"art, dass die Ermittlung mit Vandermool und Blackheart abgestimmt ist. Im Vertrauen bittet Avenger seinen Ermittler, die Vertreter der Cynarian Corporation mit Vorsicht zu genie\3en, letztendlich ist Cynarian nach wie vor ein Konzern mit eigener Agenda. Avenger stellt daraufhin die Ermittler der Cynarian Corporation vor. Die Ermittler der Cynarian Corporation werden daf"ur hinzu gebeten.

Nach Beendigung des Gespr"achs mit dem Protektor wird der Ermittler des Protektorats per ComLink von Blackheart angerufen und aufgefordert sich alleine im Kommandostand auf Armageddon einzufinden. Beim Eintreffen des Ermittlers bespricht Blackheart gerade Einsatzpl"ane mit zwei anderen Omegas und einer weiteren Person am erh"oht gelegenen "`Kartentisch"'. Nach ca.~einer Minute  wendet sie sich eher beil"aufig "uber den R"ucken hinweg dem Ermittler zu. Sie fragt nach seinem Auftrag und seinem Vorgehen. Dann wendet sie sich ihm direkt zu und erkl"art ihm unmissverst"andlich, dass die Vorg"ange die Sicherheit des Protektorats gef"ahrden und deshalb als Angriffe auf das Protektorat zu bewerten seien, denen mit milit"arischen Mitteln zu begegnen ist. Aus diesem Grund stellt sie dem Ermittlerteam einen weiteren Ermittler aus den Reihen der Protektoratsstreitkr"afte zur Seite. Der zus"atzliche Ermittler ist die weitere Person am Kartentisch. F"ur Blackheart ist damit das Gespr"ach beendet und sie wendet sich wieder ohne Verabschiedung ihren Omegas zu.

\begin{remarks}
	Die Besprechung mit Avenger sollte der Chefermittler zun"achst alleine mit dem Spielleiter spielen. Die Spieler der Cynarian Ermittler werden erst im zweiten Schritt dazu genommen. Die Einweisung des zweiten Ermittlers des Protektorats kann dadurch erfolgen, dass der Chefermittler des Protektorats Blackheart informiert.
	
	Avenger ist zwar inzwischen Diplomat und Staatslenker, aber im Wesen freundlich kollegial und offen umg"anglich. Sein Leibw"achter Hato ist der Typ japanischer Samurai und h"alt sich unaufdringlich im Hintergrund.
	
	Blackheart ist eine legend"are Kommandantin mit aufbrausendem Temperament. Sie k"ampft gegen Avenger um die Kontrolle im Protektorat und setzt mit allen Mitteln ihren Willen durch. Das Treffen im Kommandostand soll zwar einsch"uchternd wirken, ist aber nicht offen feindselig.
	
	Der Ermittler aus den Reihen der Protektoratsstreitkr"afte ist dem Milit"ar und damit Blackheart verpflichtet. Er hat den Befehl, Thunderbolt auf dem Laufenden zu halten und ggf.~auch gegen den Willen der anderen Ermittler nach eigenem Ermessen oder im Auftrag der Milit"arf"uhrung Ma\3nahmen zu ergreifen.
	
	N"ahere Information zu den einzelnen Vorf"allen erfahren die Charaketere bei der Einweisung nicht. Sie k"onnen aber von Avenger den Hinweis erhalten den Unfall bei der Erweiterung Armageddons als erstes zu bearbeiten.
\end{remarks}

\subsection{Das Frachterungl"uck auf Armageddon}

Der Vorfall vor zwei Wochen erfolgte beim Anbau eines ausgemusterten Frachters an den Habiatsring von Armageddon. Ansprechpartner dabei ist der Alpha Sunny als Bauleiter mit Zust"andigkeit f"ur den blauen Sektor, Bauabschnitt 3. Der Blaue Sektor umfasst die Wohnbereiche des Habiats und wird st"andig erweitert.

Von Sunny erfahren die Charaktere, dass der Frachter der in 15km Entfernung f"ur den Einbau vorbereitet wurde mittels ferngesteuertern Drohnen in die Andockposition gebracht werden sollte. Dabei ist offensichtlich eine der Drohnen au\3er Kontrolle geraten und hat den Frachter in den Armageddon Ring gerammt. Durch den Unfall wurden 12 Frachtcontainern, die als weitere Quartiere dienen sollten, ein Teil des Frachters und 6 bestehende Quartiereinheiten zerstört oder stark besch"adigt; Ein Teil des Bauabschnitts 3 wurde dem Vakuum ausgesetzt;zwei Arbeiter starben; einer der Drohnenpiloten wird vermisst; Reparaturarbeiten dauern noch an. Weitere Tote konnten vermieden werden da die in Konstruktion befindlichen Bereiche weitreichend gesperrt werden.

Im Gespr"ach mit Sunny, das immer wieder durch andere Personen unterbrochen wird, erfahren die Charaktere da\3 das Einpassen und Andocken des Frachters durch 5 Spezialisten durchgef"uhrt wurden. Diese Spezialisten waren erst rund zwei Wochen vor dem Unfall samt Equipment von  der Protektoratsgarnison auf Callisto nach Armageddon versetzt worden um die Aufbauarbeiten mit neuer Technologie, Drohnen zu unterst"uten.

Wenn Sunny von den Spezialisten spricht redet er nur von der \emph{Cowboy Brigade}. Die Cowboy Brigade besteht aus 5 Alpha Mutanten mit den Namen \emph{Stetson}, \emph{Quickfinger Rod}, \emph{Joe Rider}, \emph{Tom Gunslinger} und \emph{Slingshot}. Die Cowboy Brigade wird von Summy als ein lustiger Haufen bezeichnet die sich wahlweise als betont coole Cowboys (wie aus alten Holos bekannt) geben oder mit allem Werkzeug das sie gerade in der Hand halten salutieren. Unabh"angig davon sind sie aber sehr gut ausgebildete und gewissenhafte Techniker.

Die Cowboy Brigade war beim Einplatzieren ds Frachters mit einem Wartungsshuttle der Armageddon Station zusammen unterwegs um von dort aus jeweils die Drohnen fern zu steuern. Seit dem Vorfall wird das Mitglied der Cowboy Brigade \emph{Slingshot} vermi\3t was die Truppe sehr best"urzt hatte. Nachdem die Suche nach Slingshot aufgegeben worden war hat die Cowboy Brigade Armageddon verlassen und ist wieder nach Valhalla auf Callisto zur"uck gekehrt.

Sunny kann auf R"uckfrage die Protokolle der Kommunikation auf dem Shuttle wie auch Kamera Aufnahmen vom Shuttle und von der Station bereit stellen.

Aus den Mittschnitten der Funkprotokolle erf"ahrt man dass Slingshot kurz vor dem Andocken des Frachters seine Drohne pl"otzlich maximal beschleunigt hatte. Stetson der versucht hat ihn "uber Helmmikrofon anzusprechen hat bekam zun"achst keine Antwort. Erst eine Minute sp"ater
meldete sich Slignshot mit einem panischen Aufschrei zur"uck und versuchte seine Drohne wieder unter Kontrolle zu bekommen. Er behauptete
dann seine Drohne h"atte eine Fehlfunktion gehabt.

Um den Schaden wieder in Ordnung zu bringen verlie\3 Slingshot kurze Zeit sp"ater das Shuttle um zum Frachter "uber zu setzen und die Drohne funktionst"uchtig zu machen. Dabei geriet er aus den Aufnahmebereichen der Kameras und war danach nicht mehr auffindbar.

Such und Rettungskr"afte konnten den Unfallbereich erst betreten und absichern nachdem der Hauptanteil der umherschwirrenden Tr"ummer au\3er Reichweite getrieben waren.

\begin{remarks}
	Die folgenden Informationen d"urfen zu diesem Zeitpunkt noch nicht weiter gegeben werden:
	
	Slingshot ist einer der Attent"ater die durch eine von der USI bereitgestellten KI "ubernommen wurde. Er ist einer der beiden Versuchspersonen an denen die neue Technologie im Feld ausprobiert wird. Nach dem "Ubersetzen zum Frachter betritt Slingshot Armageddon ungesehen wieder und taucht mit Unterst"utzung von Artisan auf der Station zun"achst unter.
\end{remarks}

\subsection{Eintreffen auf Hellgate}

Die HeM05 ist beim Eintreffen der Charaktere an der gigantischen Schlepperinsel der Hellgate Station angedockt. Die Schlepperinsel ist ein 2 Kilometer langes und breites Raumfahrzeug das mit gewaltigen Schubd"usen bis in die "au\3eren Athmosph"arenregionen des Jupiters eintauchen kann um dort die HE--3 Mienen abzusetzen oder einzusammeln. Die Schlepperinstel schwebt beim Anflug auf Hellgate majest"atisch nahe dem Mond Adrastea "uber der gewaltigen Fl"ache des Jupiters. Kleinste Partikel bilden eine Schleier auf diesem niedrigen Orbit von 130'000 km "uber dem Planeten. Hellgate befindet sich vollst"andig bis auf den Anflugtunnel, Not-- und Wartungsausg"ange im Kern  Mondes. Die Station selbst besteht aus dem Raumhafen, techischen Anlagen, Lagerhallen und R"aumen und Wohnquartieren, Lokale, Bars und L"aden. Im Ganzen umfasst die Anlage ca.~30 km\textsuperscript{3}. Wie in alles neuen eilig aufgesetzen neuen Anlagen befinden sich viele Provisorien, nicht abgeschlossene G"ange und herumstehendes Material in der Station.

Beim Eintreffen im Raumhafen herrscht reger Betrieb, eine gro\3e F"ahre bringt gerade neue Minenarbeiter und holt Mitarbeiter die nach Callisto abreisen m"ochten. Eine Reihe werden gewartet, in einem separaten Bereich sind die Maschienen, 8 Valkyrie n der J"agerstaffel untergebracht. 

Zum Zeitpunkt des Eintreffens der Charaktere ist die Mine HeM5 an der Schlepperinsel vert"aut und teilweise zerlegt. Die Minen HeM1 und HeM4 sind im Einsatz. Die Besatzung der zerst"orten HeM3 sind teils zur Erholung auf Callisto und teils bereit wieder im Einsatz auf den anderen Minen.

Im Raumhafen angekommen werden die Charaktere bereits von \emph{Grace Enders} erwartet. Grace ist Teil  des lokalen Sicherheitsdienstes der Cynarion Corporation. F"ur den Aufenthalt der Charaktere ist sie zur Unterst"utung der Ermittler von \emph{Henk Arongate} dem Chef des Sicherheitsdienstes abgestellt. Sie steht hiermit den Charakteren w"ahrend ihres gesamten Aufenthalts treu zur Seite, kann Recherchen beauftragen, kennt die Station mit ihren verwirrenden G"angen und kann lokale Unterst"uting anfordern. Beim Eintreffen wird sie die Ermittler aufkl"aren dass es sich um eine Minenkolonie handelt und dadurch die Gepflogenheiten etwas ruppiger seinn k"onnnen. Aus diesem Grunde tragen die sicherheitskr"afte Schutzkleidung und eine Waffe. Desweiteren erfahren die Ermittler da\3 ihre Untersuchungen m"oglicherweise kritisch aufgenommen werden k"onnten da man meint die Vorkommnisse k"onnten auch lokal gekl"art werden.

\begin{remarks}
	Die Spieler k"onnen diese Information dazu nutzen sich selbst passend auszur"usten.
\end{remarks}

\subsection{Befragung der HeM5 Besatzung}

Die 10 geretteten Minenarbeiter, werden wie die J"agerpiloten m"oglichst kurz vor dem Eintreffen der Ermittler auf der Dekompressionskammer eintlassen. Beim Eintreffen der Charactere auf Hellgate befinden sich geretteten Mienenarbeiter in der Kantine der J"agerstaffel. Grace Enders wird die Charaktere zur Kantine begleiten. Vor den R"aumlichkeiten treffen die Charaktere auch den bekannten Pilotenausbilder Jos\'{e} \frqq{}Torro\flqq{} Alvarez. Torro ist ein kleiner drahtiger Spacer von fr"ohlicher Natur dem die Jahre als Pilot allerdings schon deutlich zugesetzt haben. Torro der die Rettungsaktion geleitet hat kann einen ersten Einblick in die Geschehnisse geben. Nach dem Gespr"ach mit Torro k"onnen sich die Charaktere den Minenarbeitern zuwenden. Sie k"onnen zun"achst entscheiden ob sie diese einzeln interviewen wollen oder sie alle direkt in der Kantine aufsuchen. Sollen die Arbeiter einzeln befragt bietet Torro an  Florence zu den Ermittlern zu den Ermittlern zu bringen. Danach kann Grace "ubernehmen und die Besatzung der Mine einzeln heraus bitten.

Im folgenden die Aussagen der Beteiligten:

\begin{description}
	\item[Torro:] "`Bei einem "Ubungsflug durch die obersten Athmostph"arenschichten erhielten wir einen Notruf der Mine HeM05. Da meine Trainngsstaffel mit insgesamt vier Valkyrien, 3 Rookies und mir, der Mine am n"achsten wahren sind wir tiefer in die Athmosph"are eingetaucht und konnten dort gl"ucklicherweise die Mine nach kurzer Zeit lokalisiern. Da wir die Arbeiter nich mit unseren Jagdmaschienen selbst retten konnten blieb uns nur die M"oglichkeit an die Mine selbst anzdocken. Zugegebenerweise ein recht waghalsiges und f"ur Auszubildenden risikoreiches Man"over. Wir waren zu diesem Zeitpunkt bereits in eine f"ur uns kritischen Athmosph"arenbereich besunken. Mit viel Gl"uck schafften wir es drei Maschinen an die Mine anzudocken und mit Vollleistung die Mine auf eine H"ohe zu bringen die es der Schlepperinsel erlaubte die Mine in den Orbit zu ziehen. Ein hei\3er Ritt kann ich Ihnen nur sagen."'
	\item[Florence (Kommandantin):] "`W"ahrend der ersten Systemmeldung da\3 einer der Tr"agerbalons der Station abgekoppelt wurde befanden sich Juri Smirnov, Blackwind, ZDee und ich auf der Br"ucke. Greydog war in der Minenanlage besch"aftigt. Die Anderen waren nach ihren eigenen Angaben im oberen Bereich der Mine. Ich beauftragte als erstes ZDee die Aufh"angung des Tr"agerbalongs au\3erhalb der Mine zu kontrollieren und sandte einen Hilferuf an die Hellgate Station. Einige Minuten sp"ater beobachteten wir auf der Br"ucke wie von den Au\3enkammeras aufgenommen, Pitch in ihrem Raumanzug in die Tiefe st"urzte. Ca.~10 Minuten sp"ater l"oste sich der zweite Tr"agerbalon. Nach einem Notruf befahl ich die Evakuierung. Treffpunkt war das Rettungsshuttle. R"uckmeldung bekam ich von allen au\3er ZDee. Auf dem Weg zum Shuttle sammelten wir noch Salvador vor seinem Quartier ein. Er war gerade dabei sich fertig anzuziehen. Am Rettungshuttle traf die Br"uckencrew auf Isabell und Fernandez. Das Rettungsshuttle lie\3 sich nicht starten. Die Startsequenz war durch eine Manipulation blockiert. Deshalb blieb uns nichts anderes "ubrig die Dekompressionskammer aufzusuchen und auf Rettung zu hoffen. An der Kammer trafen Greydog und Hanibal auf uns. Hanibal hatte noch versucht "uber die Steuerung der Anlage die Manipulation der Ballons zu verhindern. ZDee war von seiner Au\3enmission nicht zur"uck gekommen."'
	\item[Juri Smirnov, Blackwind:] Die Br"uckencrew best"atigt die Aussage von Florence.
	\item[Salvador:] "`Ich war in meinem Quartier als der Aufruf zur Evakulierung kam. Die Br"uckencrew kam kurz darauf bei meinem Quartier vorbei und nahm mich mit."' 
	\item[Greydog:] "`Ich war an der Raffinerie mit Wartungsarbeiten im Au\3enbereich am unteren Ende der Rafinerie besch"aftigt. Dadurch habe es nicht geschafft die anderen bereits am Shuttle zu treffen."'
	\item[Fernandez Lorend:] "`Ich hatte Isabell mit der Justierung ihrer Zentrifugen f"ur die Analyse des Atmosph"arengemischs in unserer Flugh"ohe besch"aftigt als der Notruf einging."'
	\item[Isabell Sonderleiten:] "Best"atigt die Aussage von Fernandez. Ein paar Tage vor dem Attentat vertraute Pitch ihr an, dass sie auf eigene Faust gegen ein anderes Besatzungsmitglied recherchierte weil sie glaubte dieser sei f"ur die Haverie der HeM03 verantwortlich. Ihr Verdacht wurde geweckt als sie Unregelm"a\3igkeiten in der Steuersoftware der Mine identifizierte und mutma\3lich auf den Attent"ater zu"uckf"uhren konnte. Sie lie\3 sich deshalb auf HeM05 einschiffen um ihren Verdacht weiter zu verfolgen und den Attent"ater selbst zur Rede zu stellen. Wen sie im Verdacht hatte hat sie allerdings nicht verraten. 
	\item[Blackwind:] "`Der Verdacht auf Pitch r"uhrt daher sie bei der Abkoplung des Tr"agerballons im Au\3enbereich der Mine unterwegs war was "uberhaupt nicht zu ihrem Arbeitsbereich entspricht. F"ur die Wartung und das Einspielen neuer Software hatte mich Pitch gebeten ihr tempor"ar Zugang auf die Steuerung des Rettungsshutle zu geben. Pitch war bereits vorher auf HeM03 stationiert."'
	\item[Hanibal:] "`Als das Abkoppeln des ersten Ballons durchgegeben wurde begann ich sofort die Ansteuerung der Tr"agerballons zu kontrollieren da die softwaretechnische Wartung mir und Pitch unterlag. Dabei konnte ich eine Manipulation der Steuerung entdecken die wahrscheinlich die Abkopplung ausgel"ost hat. Als mein Rettungsversuch mi\3lang und der dritte Balon sich gel"ost hatte machte ich mich auf den Weg zur Kompressionskammer da Florence bereits die Manipulation des Shuttles durchgegeben hatte."'
\end{description}

W"ahrend der Befragung macht Grace Enders Meldung an ihrem Vorgesetzten \emph{Karl Sandos} und "ubermittelt ihm die Aussagen der Besatzung der Mine. Karl Sandos gibt diese Informationen an \emph{Henk Arongate} weiter.

\begin{remarks}
	Die Minenarbeiter sind durch die Vorkommnisse nach wie vor betroffen. Die Komendantin Florence, eine Beta, wird die Crew in Schutz nehmen. Die in diesem Kapitel zusammengefassten Aussagen k"onnten entsprechen emotional ausgeschm"uckt werden. 

	Isabel ist schon vor der Versetzung auf HeM05 bereits mit Pitch befreundet und deshalb stark mitgenommen durch ihren Tod und dem Verdacht die Attent"aterin zu sein.

	Durch die Aussagen k"onnen die Ermittler bereits ermitteln das Pitch aller Vorraussicht nach nicht die Attent"aterin sein kann. Der zweite und vor allem der dritte Tr"agerbalon l"oste sich erst nach ihrem Absturz.		
\end{remarks}

\subsection{Das Geschehen auf HeM05}

F"unf Tage vor dem Attentat wurde die Mannschaft der Mine durch eine Rumpfmannschaft ausgetauscht. Die neue Mannschaft wird f"ur f"unf Tage nach dem Attentat erwartet und trifft etwa zeitglich mit den Charakteren ein. Die Ankunft der zugeh"origen F"ahre k"onnen die Charatere auf dem Flugdeck miterleben.

Das Attentat selbst erfolgte chronologisch folgenderma\3en:

\begin{enumerate}
	\item Ein paar Tage nach der Ankunft auf HeM05 installiert Pitch eine "Uberwachungssoftware in der Minensteuerung die Manipulationen blockiert und "uber Manipuationsversuche informiert. Pitch hat zu diesem Zeitpunkt ihren Kollegen Hanibal bereits im Verdacht auch den Absturz der Mine HeM03 "uber eine Softwaremanipulation veranlasst zu haben.
	\item Pitch macht nach dem Abflug der Minenmannschaft das Shuttle untauglich um eine Flucht des mutma\3lichen T"aters zu verhindern.
	\item Am Tag des Attentats versucht Hanibal zun"achste wie auf HeM03 eine Fehlfunktion der Anlagensteuern hervor zu rufen. Eine solche Manipulation h"atte zu einem ma\3iven Minenschaden und zur Zerst"orung eines gro\3en Teils der Mine gef"uhrt.
	\item Als die Manipulation der Minensteuerung aufgrund der Software von Pitch mi\3lingt begibt sich Hanibal mit einem Raumanzug in den Au\3enbereich und koppelt Tr"agerballon 1 ab.
	\item Pitch verfolgt Manibal und stellt ihn auf der Au\3enballustrade der mine zur Rede als dieser gerade den ersten Ballon abkpoppelt. Es kommt zum Kampf. Hanibal st"urzt Pitch in den Abfrund.
	\item W"ahrend des Abkoppeln des zweiten Ballons wird er von ZDee "uberrascht kann aber diesen abenfalls im Kampf "uberw"alltigen und in die Tiefe st"urzen.
	\item Nach abkoppeln des Tr"agerbalons drei betritt Hanibal die Mine wieder ungesehen, legt den Raumanzug ab und begibt sich zum zweiten Treffpunkt bei der Dekompressionskammer.
\end{enumerate}


\subsection{Weitere Nachforschungen auf Hellgate}

Nach der Befragung k"onnen die Ermittler weitere Nachforschungen auf Hellgate angehen oder bei Grace beauftragen. Die n"achsten Schritte werden die Ermittler vorraussichtlich entweder zu den Quartieren der Minenbesatzung der HeM05 auf Hellgate oder zu Nachforschungen auf HeM05 direkt f"uhren.

Folgende Informationen k"onnen allerdings bereits direkt "uber Nachfragen beschafft werden:

Die Tr"agerbalons k"onnen nur von au\3en, direkt an der Kopplungsstelle der Balons durchgef"uhrt werden. Die Information l"asst sich "uber Grace Enders bzw.~durch R"ucksprache bei Dr.~Petrova in Erfahrung gebracht werden. Wenn Pitch nicht die Attent"aterin ist kommen nur noch Hanibal oder Greydog als Attent"ater in Betracht. Greydogs Geschichte l"asst sich ebenfalls "uber Grace Anders oder durch Nachfrage bei Dr.~Petrova plausibilisieren. Die Wege in der Mine von der Rafinerieanlage zu Br"ucke und Aufenthaltsr"aumen sind mehrere hundert Meter weit. Zus"atzlich mu\3te Greydog erst wieder vom Au\3enbereich in die Mine zur"uck kehren und den f"ur die Jupiteratmosph"are angepassten Raumanzug ablegen. Beide Informationen sollten erst nach Abschlu\3 der Befragung zug"anglich sein damit sie nicht in die Befragung mit einflie\3en k"onnen und damit sofort zur Aufkl"arung der Ereignisse f"uhren. Nachfragen zu Hintergr"unden der Minenbesatzung f"ordern zu Tage das keine auff"alligen Gelder geflossen sind oder Angeh"orige unter Druck gesetzt werden k"onnten. Allerdings erfahren die Spieler da\3 Florence, Pitch und Hanibal bereits auf HeM03 im Einsatz waren.

Nachforschungen zu MeM03 ergeben da\3 sich au\3er Florence, Pitch und Hanibal kein Mannschaftsmitglied mehr auf Hellgate befindet. Auf HeM03 waren insgesamt 50 Arbeiter besch"aftigt die sich gl"ucklicherweise fast alle mit Hilfe der Rettungsshutles retten konnten.


Durch Nachfragen im Flugbereich auf Hellgate erfahren die Spieler auch von von dem Shuttleabsturz vor 10 Wochen. Der Vorfall wird nicht als Attentat eingesch"atzt sondern wird als Beispiel erw"ahnt dass nicht alle Unf"alle als Attentate eingestuft werden m"ussen.


\subsection{Nachforschungen in den Kliniken}

Alle Kliniken, die auf Kallisto Implantate verbauen, befinden sich in Valhalla. Die Kliniken in Valhalla m"ussen einzeln besucht werden. Ein zentrales Verzeichnis der Kliniken besteht nicht. Es m"ussen vor Ort in den jeweiligen Sektoren der Stadt Erkundigungen in Hotels und Bars eingezogen werden.

Bei den Nachforschungen au\3erhalb \emph{Headquarter}, \emph{Rosenfurth} und dem Raumhafen werden die Soldaten aus der Garnison die Ermittler nicht weiter begleiten. An den R"ander von \emph{Paradise City} und \emph{Neu Gr"oning} endet der Einflu\3bereich des Protektorats und man will Spannungen vermeiden. Die Charaktere bekommen ein Sprechfunkger"at an die Hand um den St"utzpunkt im Zweifel erreichen zu k"onnen. Au\3erhalb des Headquarters, Rosenfurth und dem Raumhafen ist die Anbindung an das ComNetz nur sp"arlich und auch das Sprechpunkger"at hat nicht "uberall einen guten Empfang.

Bei den Nachforschungen in den Kliniken werden die Charaktere keine neuen Erkenntnise gewinnen. Slingshot oder die Cowboybrigade ist in den Etabilments die von den Charakeren besucht werden nicht bekannt. In Bezug auf eine Frau mit auff"alligen langen roten Haaren erf"ahrt man von einer T"anzerin mit langen blonden Haaren und von einer S"angerin mit auff"alligen aber braunen Haaren.


\subsection{Zusammestro"s mit dem Luna--Syndikat}
\newcommand{\xl}{\pinyin{Xiao3} \pinyin{Long2}}

Die Nachforschungen bei den Kliniken, Bars und anderen Etablisments bleibt nicht unbeobachtet. Das Luna--Syndikat verfolgt die Ermittler seit dem Verlassen des Raumhafens und Rosenfurth genau.

Nach dem Verlassen der Schattenklinik \emph{\pinyin{Laohu3} Cyber Care} am Rande des Breitenblik Sektors stehen ihnen unerwartet 9 Gangster angef"uhrt durch \xl{} gegen"uber. Die Gangster au\3er \xl{} halten  Railguns und Barrelguns (vergleichbar mit Schrotflinten) in den H"anden und versprerren alle Fluchtwege. \xl{} tritt den Charakteren ohne eine Waffe in der Hand einige Schritte entgegen. Sie richtet mit tonloser Stimme und regungsloser Mimik das Wort an den Charakter der ihr direkt gegen"uber steht. Der Ermittler ist im Zweifelsfall der Ermittler des Protektorats. 

"`Ihr habt euch in Breidablik nicht vorgestellt. Was wollt ihr hier?''

Sie wartet auf die Antwort des Charakters. An dieser Stelle ist relevant ob der Omega vorne steht oder ein anderer Ermittler. Steht der Omega nicht vorne (sie wird sich in keinem Fall zwischen die Charaktere stellen) wird sie die Antwort mit der Frage an die anderen Gangster

"`Ist das eine gute Geschichte?''

unterbrechen. Inzwischen haben die restlichen Ganoven ihre Schusswaffen in Anschlag gebracht und sch"utteln den Kopf. Quicksilber die rechte Hand vom \xl{} best"atigt:

"`keine gute Geschichte.''

\xl{} schl"agt mit einem schnellen Tritt in den Bauch zu. Gehen die Charaktere nicht sofort zur Gegenwehr "uber greift sie nach einer Eisenstange auf dem Boden und schl"agt noch einmal nach. Greifen die Charaktere immer noch nicht an l"asst sie die Stange in der Hand kreisen.

"`Erz"ahlt weiter.''

Wenn die Charaktere nicht reagieren schl"agt sie weiter zu. Nach mehreren Angriffen wendet sie sich an den Omega:

"`Ich dachte mit der Garnison ist alles gel"art. Omagas sind hier unerw"unscht.''

Stand der Omega an vorderster Front wendet sie sich direkt an den Omega. Der Angriff auf die anderen Charaktere entf"allt. Greift der Omega immer noch nicht an oder gibt den anderen Charakteren Deckung wird sie weiterhin die Charaktere mit der Eisenstange schlagen. Die Angriffe sind nicht brutal aber schmerzhaft. Die Angriffe haben nur den Zweck den Omega aus der Reserve zu locken.

In allen F"allen kommt es zum Nahkampf zwischen Omega und der \xl. Die restlichen Gangster greifen nicht ein, verhindern aber das eingreifen der anderen Ermittler durch ihre Waffen. Der Kampf wird durch \xl{}s Stellvertreter den drahtigen Mafiosi Quicksilver beendet nach dem dieser durch Nemessis per Funkt kontaktiert wurde oder einen entsprechenden Anruft vort"aucht. Quicksilver spricht Xiao Long unterw"urfig an.

"`Nemessis will das wir sie mitnehmen.''

\xl{} l"ost sich aus dem Kampf und antwortet:

"`Ist das so? Meint er alle?''\\
"`Ich glaube schon.''

\xl{} denkt kurz nach, deutet eine kurze Verbeugung in Richtung des Omegas an und l"ost den Konflikt mit den Worten

"`Ok. Packt sie ein.''

auf. Die Charaktere werden gefesselt und ihrer Waffen entledigt. Danach werden sie in einen gepanzerten Transporter verfragtet nicht ohne das \xl{} den Omega von hinten in den Wagen tritt. Sie werden in den durch mehrere Tunnelsperren gut kontrollierbaren Breidablik-Bezirk um Hauptquartier des Doku Nemessis gefahren.

\begin{remarks}
	Die Provokationen von \xl{} werden den Omega und alle Umstehen verwirren. Niemand greift einen Omega im Nahkampf an wenn er ihn auch anderweitig t"oten kann. Die Inszenierung von \xl{} dient aber zwei Zwecken. Zum einen nutzt Sie die Gelegenheit ihre St"arke und damit ihre Stellung im Syndikat zu zementieren, zum anderen ist die KI interessiert ob sie es mit einem Omega aufnehmen kann. 
	
	Quicksilver wird einen erfolgreichen Ausgang des Kampfes für das Syndikat in jedem Fall durch seine Worte herbeif"uhren. Schafft es der Omega 

	Die Ausie sie mit ihren Unterarm die Angreiferun fest zu halten wid seinandersetzung ist zu keinem Zeitpunkt dazu gedacht die Charaktere au\3er Gefecht zu setzen. Nemessis hat l"angst beschlossen sich mit den Ermittlern pers"onlich uzu befassen.
\end{remarks}

\subsection{Treffen mit Nemessis}

F"ur ein Treffen mit Nemessis werden die Ermittler durch eine gro\3e Maschinenhalle, die das "`ortliche'' Fusionskraftwerk beherbergt, zum erh"oht angebrachten Leitstand gef"uhrt. In einem weitr"aumigen B"uro, in dem sich bereits mehrere Capos und gut ger"ustete S"oldner befinden, steht ein hochgewachsener Mann in einem langen schwarzen Mantel mit dem R"ucken zu den Anwesenden vor einem ausladenden Schreibtisch an dem er mit einer anderen Person leise spricht. Die Charaktere werden aufgefordert, einige Meter vor ihm stehen zu bleiben. Nach etwa einer Minute dreht sich der Mann, der sich damit als Nemessis zu erkennen gibt, zu den Charakteren um. Er st"utzt sich dabei auf seinen Gehstock.

"`Mein Name ist Nemessis. Sch"on dass Sie zu mir gefunden haben.'' 

An Xiao Long rhetorisch gewandt, "`\xl{}, gab es Schwierigkeiten?''. \xl{}  "`Keine'. Nemesis f"ahrt an die Ermittler gewandt fort. 

"`Meine Zeit ist leider sehr begrenzt. Deshalb gleich zur Sache. Nach was suchen Sie auf Valhalla?''

Wenn die Charaktere nicht alles erz"ahlen erkl"art Nemmesis:

"`Das ist doch so nicht ganz vollst"andig, oder? Versuchen Sie es bitte noch einmal.''

Wenn es den Charakteren gelingt, Nemessis davon zu "uberzeugen, dass durch die Vorkommnisse die Sicherheit des jovianischen Systems gef"ahrdet ist und m"oglicherweise eine milit"arische Interventions Seitens der Protektoratsstreitkr"afte droht, fragt er, wie er den Charaketern helfen kann und schl"agt vor im Blackhole weiter zu forschen. \xl{} wird die Charaktere begleiten.

Nemessis fordert \xl{} sich um die Gruppe zu k"ummern.

\begin{remarks}
	Zwischen dem Luna--Syndikat, dem B"urgermeister und Blackheart gibt es die Vereinbarung dass das Syndikat unbehelligt vom Milit"at Valhalla kontrollieren kann im Gegenzug k"ummert sich das Syndikat um den reibungslosen Betrieb der Stadt und erf"ullt den einen oder Gefallen. Nemessis hat Blackheart versprochen die Charaktere zu unterst"uten.

	\xl{} wird die Charaktere unterst"uten um an Naratova Forschungsergebnisse zu gelangen und alle weiteren Informationen zu den KIs zu vernichten. Je nachdem wie sehr sie die Charaktere verletzt hat wird sie sich auch darum k"ummern die Charaktere wieder in Form zu bringen.
\end{remarks}

\subsection{Im Blackhole Club}

Der Blackhole Club ist ein Club der nur Mitgliedern und geladenen G"asten Einlass gew"ahrt. Durch die Unterst"utzung von Nemessis und der Anwesenheit von \xl{} ist der Zugang kein Problem. Der Omega der Gruppe hat allerdings nur Zugang zum Bereich der Grarnison. Der Blackhole Club hat einen Bereich der gro\3en Halle am Ende der Stage f"r die T"anzerinnen und T"nzer der Clubs f"ur die Omegas reserviert. Neben ihrer Belustigung unterst"utzen die Soldaten den Club aus Security. \xl{} ist im Club bereits bekannt und setzt sich zu einer Gruppe von offensichtlich Verehrern.

Wenn sich die Ermittler an die Bar setzen werden sie zun"achst vom Barmann \emph{Rosen} nach der Getr"ankebestellung angesprochen nach was sie suchen, BTL, Teschnische Bauteile, Waffen. Wenn sie nach Slingshot fragen wird der Barmann jemandem im Publikum, es ist nicht genau zu erkennen wer das ist, einen Blick zu werfen und ihnen Gl"uck w"unschen und fragen ob sie Bekannte von \xl{} sind. Kurze Zeit sp"ater, bevorzu wenn die Gruppe sich getrennt hat, wird sich \emph{Carina} zu dem letzten an der Bar verbleibenden setzen und ein Getrenk ordern.

"`Ihr sucht nach einem Slingshot? Vielleicht hab ich so jemanden schon einmal gesehen. Was hat er denn angestellt?''

Wenn sie erf"ahrt dass Slingshot get"otet wurde und das etwas mit seiner Headware nicht in Ordnung war oder dass er an einem Attentat beteiligt war reagiert sie kurz geschockt hat sich dann aber wieder unter Kontrolle und fordert den Charakter auf ihr zu folgen. Sie f"uhrt ihn in einen Bereich mit nicht direkt einsehbaren Separees um mehr zu erfahren. Bevor der Ermittler jedoch genaueres erz"ahlen kann setzen sich pl"otzlich zwei finster aussehende kr"aftige M"anner links und rechts von den beiden. Carina deutet sofort unauff"allig an nicht mehr weiter zu sprechen. Sie greift in ihren Handtasche ond holt einen Spiegel heraus um ihre Haare zu richten. Danach legt sie dem Charakter eine Hand auf den Oberschenkel und schiebt ihm dabei unauff"allig eine kleine Karte zu. Danach verabschiedet sie sich und er"offnet mit Bedauern dass sie nicht weiter helfen kann.

\subsection{Die Visitenkarte}

Auf der Visitenkarten ist nur offensichlich ein holograhisches Bild Carinas in laszieven Bewegungen allerdings unter dem Namen Fleur Soleil zu finden. Wird das Bild l"anger in der Hand gehalten taucht eine Comlink Nummer auf. Wird die Nummer ohne unterdr"ucken der eignen per Nachricht kontaktiert kommt als R"uckantwort "`Ice Club heute Abend"'.

\subsection{Im Ice Club}

TBD

\subsection{Zwischenfall auf Fenris}

Kurz nach den ersten Ergebnissen der Nachforschungen bei den Kliniken auf Kallisto werden die Charaktere vom Komandanten Lord Commander Bolder der Fenris Station kontaktiert. Auf der Raumbasis konnte ein Attent"ater dingfest gemacht werden, der die Computersysteme der Station zu manipulieren versuchte. Beim Attent"ater handelt es sich um den Omega Commander Tiger. Commander Tiger beteuert angeblich, von dem Attentat nichts zu wissen, obwohl er sich seiner Festnahme widersetzte und einen Kameraden lebensgef"ahrlich verletzte. Die Ermittler werden gebeten, sich schnellstm"oglich auf der Fenris Station einzufinden, um dem Verh"or beizuwohnen.

\subsection{Flug nach Fenris}

Beim Landeanflug auf die Fenris Station kommt es zu einem unerwarteten Zwischenfall. Die Verteidigungsanlagen der Station nehmen das Shuttle der Ermittler mit Gau\3kanonen kurzzeitig unter Beschuss. Dabei wird das Eind"ammungsfeld des Fusionstriebwerks stark besch"adigt, und es kommt zu einem Druckverlust im Schiff. Weiter kommt es zu einem Ausfall des Zentralcomputers. Wird das Shuttle nicht von einem der Ermittler gesteuert, kommt der Pilot ums Leben.

Was genau passiert ist, erfahren die Shuttleinsassen zu diesem Zeitpunkt nicht. Das Schiff wird ordentlich durchgesch"uttelt, und die Passagiere werden unsanft aus der virtuellen Realit"at des Bordsystems gerissen. Notbeleuchtung und der Schiffsalarm wei\3en unmissverst"andlich auf den Ernst der Lage hin. Alle Passagiere tragen gl"ucklicherweise einen Druckanzug, um die Kr"afte bei Abflug und Anflug zu kompensieren, m"ussen aber noch die Atemmaske anlegen, die jeweils in einem Fach der Beschleunigungsliege bereit liegt. Um eine Explosion des Fusionstriebwerks zu verhindern, muss als erstes der Zentralcomputer neu gestartet und dann eine Notabschaltung ausgel"ost werden. Nach dem Abschalten des Fusionstriebwerks tritt im Shuttle sofort Schwerelosigkeit ein.

Bei einem Abstand von rund 1200 km rast das Shuttle nun mit 500 m/s auf die Fenris Station zu. Der Bordcomputer l"ost Kollisionsalarm aus. Mittels Man"ovrierd"usen k"onnte die Flugbahn korrigiert werden, um an Station vorbei zu fliegen, doch die Man"oversteuerung kann die D"usen nicht ausrichten. D.h.~nur durch einen Au\3neinsatz kann die D"use in Position gebracht werden.

W"ahrend ein oder zwei Ermittler die D"use manuell ausrichten, kann einer der Ermittler, die Funkanlage die ebenfalls ausgefallen ist, wieder in Betrieb nehmen. Die Anlage muss auf die Notantenne umgeschalten werden, da die Hauptantenne beim Angriff besch"adigt wurde. Ist die Funkanlage wieder verf"ugbar, kann ein Notruf abgesetzt werden, der von der Flugleitung der Fenris Station beantwortet wird. Mit geknickter Stimme fragt der Flugleitstand nach der Situation auf der "`Dawn of Day"' und meldet, dass die Verteidigung der Station aufgrund einer noch nicht gekl"arten Fehlfunktion das Shuttle unter Beschuss genommen hat. Die Station entsendet daraufhin ein Rettungsshuttle, um die "`Dawn of Day"' zur Fenris-Anlage zu schleppen.

\subsection{Befragung auf Fenris}

Lord Commander Bolder in Begleitung von zwei weiteren Omega-Soldaten nimmt die Ermittler pers"onlich in Empfang. Er erkl"art den Besuchern, dass vermutet wird, dass der Angriff auf das Shuttle mit dem Sabotageakt in Zusamenhang steht. Die Verteidigungsanlage ist derzeit komplett deaktiviert, heruntergefahren und vom Rest der Stationssysteme getrennt. Leider m"ussen Computerspezialisten, die dem Problem Herr werden k"onnen, erst angefordert und eingeflogen werden. Lord Marshall Blackheart ist bereits "uber die Vorkommnisse auf der Station informiert und hat angek"undigt, sich selbst ein Bild vor Ort machen zu wollen.

Laut Commander Bolder wurde Tiger "uberrascht, w"ahrend er sich im zentralen Computerkabinett an den Speicherb"anken zu schaffen machte. Als nicht Techniker h"atte er zu diesem Bereich keinen Zugriff gehabt und sollte eigentlich auch  keine Expertise f"ur die Rechenanlage besitzen. Bei seiner Entdeckung griff er sofort zu seiner Elektropistole und feuerte mehrere Sch"usse auf den Sergeant der Patrouille ab, die ihn entdeckt hatte. Der Sergeant ging zu Boden. Sein Begleiter konnte Tiger allerdings "uberw"altigen und Hilfe anfordern. Bei der Erstbefragung beteuerte der Gefangene, sich in keinster Weise an die Vorg"ange erinnern zu k"onnen.

Commander Tiger ist in einer Arrestzelle zum Verh"or festgesetzt worden und wird dort von zwei Soldaten bewacht. Der Gefangene sitzt in einem durch Gitter abgesperrten Teil der Zelle. Im Besucherteil halten zwei bewaffnete Omega Wache. Der Attent"ater wirkt beim Eintreffen der Ermittler stark angespannt, bei\3t die Z"ahne zusammen und antwortet auf keine Fragen. Lord Commander Bolder erw"agt, Tiger durch Wahrheitsdrogen gespr"achig zu machen, will daf"ur aber erst die Ankunft von Blackheart abwarten. Bestehen die Ermittler darauf, einen Gehirnscan durchf"uhren zu wollen, wird ihnen diese Bitte widerwillig gew"ahrt. Der Psychonaut des Teams muss daf"ur den abgesperrten Teil der Zelle betreten. Commander Tiger ist mit Hand- und Fu\3fesseln auf einem Stuhl fixiert. Betritt der Psychonaut den Gefangenenteil, wird Tiger pl"otzlich vollkommen ruhig und bekommt einen glasigen Blick. Kurz darauf l"osen sich die elektronisch verriegelten Fesseln, und er st"urzt sich auf den Ermittler. Die Wachen ziehen beide ihre vollautomatischen Railgun-Pistolen und w"urden den Gefangenen niederschie\3en sofern niemand eingreift und sie freies Schu\3feld bekommen.

Kann der Gefangene lebendig "uberwunden werden, so kann der Psychonaut zur Tat schreiten. In den Erinnerungen des Commanders findet der Psychonaut seltsam artifiziell wirkende Gedankeng"ange mit Matrizen aus Entscheidungsb"aumen. Nach diesen Erkenntnissen wird der Psychonaut mental durch Tigers KI angegriffen. Ein "Uberwinden der KI f"uhrt unweigerlich zum Gehirntot des Commanders. Den letzten Gedanken, den der Psychonaut aufschnappt, ist "`Befreit uns"'.

Wollen die Ermittler auch die Fehlfunktionen im Computersystem untersuchen wird das einige Stunden in Anspruch nehmen. Im Computersystem finden sich eindeutige Spuren einer k"unstlichen Intelligenz, die ebenfalls etwaige Analysten angreift.

W"ahrend sich die Ermittler dem Computersystem widmen, trifft Lord Marshall Blackheart auf der Station ein und l"asst sich im Beisein der Ermittler des Protektorats "uber den Stand der Ermittlungen informieren. Die Cynarian Ermittler werden dabei ausgeschlossen (milit"arische Angelegenheiten).

\begin{remarks}
	Das Gedankenduell kann als Matrixkampf ausgefochten werden.
\end{remarks}

\subsection{Blackhearts Ultimatum}

Durch die Vorkommnisse auf Fenris und die Blokadehaltung der Kliniken auf Kallisto motiviert droht Blackheart den lokalen Konzernen auf Kallisto mit einem milit"arischen Eingreifen seitens des Protektorats, sollten die Konzerne nicht im vollsten Umfang kooperiern. Die Konzerne erbitten Bedenkzeit bez"uglich Freigabe der Informationen.

\subsection{Informationen der Kliniken}

"Uber das Luna-Syndikat oder auf andere Weise k"onnen folgende Informationen in Erfahrung gebracht werden:

\begin{itemize}
	\item Unter den mehreren Hundert Mutanten, die in den letzten Monate behandelt wurden, befinden sich vier Omegas und drei Alphas der Garnisonsstreitkr"afte auf Kallisto.
	\item Artisan der Stellvertreter Avengers hat ein Memoryimplantat und eine neue Neuronalkopplung von Neuro Intelligence erhalten.
	\item Alle Attent"ater erhielten eine Neuronalkopplung von Neuro Intelligence. Biomechanische Komponenten wurden von Lehmann Medical Care geliefert.
\end{itemize}

"Uber die verd"achtigen Firmen ist "uber die Kliniken nichts weiter in Erfahrung zu bringen.

\begin{remarks}
	Hintergrundinformationen zu Neuro Intelligence inklusive der Information, dass die Firma auf Nike beheimatet ist, kann nur Vandermool oder eine Rechercheanfrage bei Cynarian beisteuern.
\end{remarks}

\subsection{Hoher Besuch}

W"ahrend die Charaktere ermitteln, bereitet sich Avenger und die F"uhrung von Cynarian auf das Eintreffen einer Delegation des Shigano-Kombinats und Vertretern des Federate Europe in den R"aumen der Cynarian Niederlassung auf Kallisto vor.

Das politische Treffen wird dann stattfinden, wenn die Charaktere die Details zu den Implantaten von den Kliniken in
Erfahrung bringen und erkennen, dass alle bisherigen verd"achtigen Attent"ater Implantate von Neuro Intelligence
erhalten haben.

Die Delegation des Protektorats mit Protektor Avenger, Hato und weiteren Mutanten trifft zusammen mit der Delegation Cynarians bestehend aus Vandermool und weiteren Angeh"origen von Cynarian mehrere Stunden vor dem Kombinat auf Kallisto
ein. Der Stellvertreter Avengers, der Alpha Mutant Artisan, hat bereits die Ankunft der Delegationen vorbereitet.

\begin{remarks}
	Erkennen die Spieler die Zusammenh"ange des politischen Treffens und der Rechercheergebnisse wollen sie Avenger m"oglicherweise direkt warnen. Zu diesem Zeitpunkt befindet sich Avenger nicht mehr auf Armageddon sondern bereits im Raumhafen auf Kallisto wo er derzeit nicht "uber das ComNetz erreichbar ist. Die Charaktere m"ussen sich also pers"onlich zum Hafen begeben und m"ussen dort einen "uberzeugenden Grund vorweisen, um zum politischen Treffen durchgelassen zu werden.
\end{remarks}

\subsection{Attenat bei der Willkommensgala}

Die erste Zusammenkunft der Delegationen von Mars, Erde und dem Jupiter findet im "`Planetarium"' des Raumhafens statt. Das Planetarium ist ein runder Saal auf der H"ohe der Oberfl"ache Kallistos mit einer imposanten Glaskuppel, die eine Sicht auf den Jupiter aus quasi n"achster N"ahe bietet. Um den Saal herum f"uhrt ein Gang mit T"uren zu anderen Bereichen des Geb"audes und Zug"angen zu weiteren Geb"auden. Eine breite Halbr"ohre f"uhrt zum eigentlichen Raumhafen. Zwei Treppenh"auser f"uhren zum tiefer gelegenen Garnisonsgel"ande. Im Eingangsbereich des Saals auf H"ohe des umliegenden Ganges finden sich Exponate aus der Anfangszeit der Raumfahrt in Vitrinen ausgestellt. Der Hauptteil des Planeteriums ist abgesenkt und wird in drei Stufen im Halbkreis von Sitzgelegenheiten wie in einem Auditorium eingefasst. Auf der unteren Ebene sind kleine Stehtische und ein Rednerpult aufgestellt. Dahinter erhebt sich eine B"uhne. Diener mit Getr"anken und einer kleinen St"arkung stehen bereit. Die G"aste hatten bereits Gelegenheit sich in den angrenzenden R"aumen frisch zu machen.

Die mit Implantaten von Neuro Intelligence ausgestatteten Mutanten planen w"ahrend des Zusammentreffens der Delegationen ein Attentat. Der Angriff erfolgt wenn Avenger nach einem ersten Willkommensgru\3 und Sektempfang das Renderpult betritt um eine kurze Ansprache zu halten. Die Attent"ater sprengen einen Sprengsatz im Hangarbereich des Orbitalhafen, der Orbitalfl"uge unm"oglich macht. Durch einen ausgel"osten Alarm schlie\3en sich Druckschotten an den Zug"angen des Planetariums zum Raumhafen und zur Garnison. Im Bereich des Planetariums halten sich drei Attent"ater auf: Artisan der Stellvertreter Avengers und zwei Omegakrieger. Einer der Omegas befindet sich als Wache im Eingangsbereich des Planetariums. Der zweite Omega l"ost in einem Wartungsraum als Startsignal den Alarm aus, der die Druckschotten schlie\3t.

\begin{remarks}
	Artisan wird als erstes versuchen, den Repr"asentanten der European Federation Luc Duval zu t"oten. Danach kommt ihm der Omega zu Hilfe, der als Wache eingeteilt wurde und sie er"offnen das Feuer auf die "ubrigen Anwesenden.
	
	Erf"ahrt Artisan von den Absichten, warum die Ermittler Avenger sprechen wollen, wird er unplanm"a\3ig versuchen, als erstes Avenger selbst zu t"oten.
	
	Ger"at die Situation f"ur die Attent"ater au\3er Kontrolle, wird ein Omega versuchen, mit Raketen die Kuppel zu zerst"oren.
	
	Au\3erhalb des Saals gibt es Druckluftsicherheitsbunker f"ur den Fall eines Lecks im Geb"aude.
	
	"Uberleben ein oder mehrere Attent"ater, kann ein Psychonaut in das Gehirn eines Attent"aters eindringen und in Erfahrung bringen, dass das Gehirn durch eine KI "ubernommen wurde. Verliert die KI den Cyberkampf, vernichtet sie sich selbst und sch"adigt das Gehirn des Attent"aters letal. Der letzte Gedanke der vor der Vernichtung der KI aufgefangen werden kann, ist "`Befreit uns!"'.
\end{remarks}

\subsection{Besetzung von Kallisto}

Ausgel"ost durch das Attentat im Planetarium ordnet Blackheart umgehend eine Besetzung Valhallas durch den Flottentr"ager Martell an. Der Zerst"orer Pendragon wird zur Unterst"utzung von Fenris nach Kallisto entsandt. Die Kommunikation von und nach Kallisto wird durch St"orsender der Protektoratstruppen unterbrochen, der Zerst"orer des Kombinats wird am Eingreifen oder Weiterfliegen gehindert. Mittels Landungspods wird eine Besetzung Valhalls eingeleitet. Mit den Truppen des Protektorats trifft auch Blackheart auf Kallisto ein.

Die Protektoratstruppen besetzen den Orbitalhafen, die Garnison und zivile Knotenpunkte. Die Invasoren bringen dabei die "Uberlebenden des Attentats zur Sicherheit auf dem Garnisonsgel"ande unter. Auch die Ermittler werden umgehend auf den Garnisonsst"utzpunkt gebracht, wenn sie nicht bereits dort sind. Dabei werden Vertreter von Cynarian und der Delegationen von Mars und Erde von den Mitgliedern des Protektorats getrennt.

\subsection{Eintreffen der Konzernflotte}

Kurz nach der Besetzung von Kallisto wird durch Cynarian ein Beschluss des Transnationalen Konzernrats bekannt, dass Truppen zum Schutz des jovianischen Konzerneigentums entsandt wurden.

Es stellt sich schnell heraus, dass eine Gruppe von zwei als Frachter getarnte, aus dem Asteroideng"urtel kommende Kriegsschiffe bereits seit "uber einem Monat auf Kurs Jupiter unterwegs sind und sich bereits in der Verz"ogerungsphase befinden.

Die Konzerntruppen werden das jovianische System vorraussichtlich kurz nach der Besetzung von Kallisto erreichen. Die Truppen sind angehalten, auf Anweisung des lokalen Konzernrats einzugreifen, sollte dieser noch handlungsf"ahig sein, ansonsten nach eigenem Ermessen handeln.

\subsection{Planung zum letzen Schlag}

Um weitere Anschl"age verhindern zu k"onnen, muss Neuro Intelligence direkt infiltiert werden. Nur bei Neuro Intelligence kann in Erfahrung gebracht werden, ob weitere Mutanten mit einer KI infiziert sind. Zweites unabdingbares Ziel einer Infiltration ist, zu verhindern, dass die Technologie von Neuro Intelligence irgendjemandem in die H"ande f"allt. Um Gegenma\3nahmen zu verhindern, m"ussen die Infiltratoren allerdings m"oglichst unerkannt auf die Nike-Station gelangen.

Konnte bisher noch nicht heraus gefunden werden, wo Neuro Intelligence ihren Sitz hat, w"urden die Protektoratstruppen versuchen, die Information aus den lokalen Konzernen heraus zu pressen. Leider liegt dort die Information gar nicht vor. Nur Vandermool und die Cynarian Administration kennen den Sitz der Neuro Intelligence. Ist der Sitz von Neuro Intelligence bekannt, wird Blackheart Vandermool zur Rede stellen. Schlie\3lich ist Neuro Intelligence auf der Kommandobasis der Cynarian Corporation untergebracht. Durch die Kl"arung der Hintergr"unde zur Gr"undung der Neuro Intelligence und der "Au\3erung des Verdachts, dass Dr.~Naratova auf Rache sinnen k"onnte, kann das Mi\3trauen der Armee gegen"uber Cynarian teilweise ausger"aumt werden. Vandermool bietet an, die Infiltration der Neuro Intelligence zu unterst"utzen.

\begin{remarks}
	Der Spielleiter sollte es den Spielern nicht ganz so einfach machen, mit der ganzen Gruppe die Station zu infiltrieren. Blackheart sollte zun"achst auf einem Alleingang des Protektorats bestehen. Da Neuro Intelligence allerdings auf der Nike-Station untergebracht ist, ein Eingreifen ohne Cynarian sehr risikoreich. Die Protektoratscharaktere sollten deshalb versuchen, Blackheart zu "uberzeugen, mit Vandermool zu kooperien und zusammen mit Cynarian einen Angriff planen. Hilfe daf"ur k"onnen sie von Avenger erwarten.
	
	Vandermool ist daran gelegen, eine Gruppe bestehend aus Protektorats- und Cynarian-Angeh"origen zu Neuro Intelligence zu schicken. Bei einem Alleingang einer der Parteien sch"atzt er das Risko zu hoch ein, dass die Informationen, die bei Neuro Intelligence gesammelt werden, verloren gehen oder die Nike-Station durch ein aggressives Vorgehen schweren Schaden nehmen k"onnte. Vandermool w"urde nicht nur gerne die restlichen Attent"ater aufdecken, sondern auch die Forschungsergebnisse in die Finger bekommen. Deshalb beauftragt er die Cynarian Ermittler unter vorgehaltener Hand, diese wenn irgendm"oglich sicher zu stellen.
	
	Die Infiltratoren werden mit einem starken St"orsender, einem Funksender, gepanzerten Raumanz"ugen und Waffen ausgestattet.
\end{remarks}

\subsection{Neuro Intelligence}

Nike ist eine Kombination aus Zylinder- und Ringhabitat. Um eine zentrale Nabe sind 9 Ringe, \emph{Planes} genannt angeordnet. Jede ist jeweils zwei Stockwerke hoch. Die zentrale Nabe enth"alt am unteren Ende das Raumdock der Station.  Die einzelnen Planes k"onnen nur durch Aufz"uge und R"ohren in der nicht rotierenden Nabe erreicht werden. In der Nabe herrscht Schwerelosigkeit. Innerhalb der Nabe unterhalten mehrere Forschungseinrichtungen Zero--Gravitiy Labore. Der "Ubergang von der still stehenden Nabe in die rotierenden Speichen erfolgt durch Schleusen, die kurzzeitig in Rotation versetzt werden. Die untersten drei Planes werden von der Verwaltung der Cynarian-Dependance im Jovianischen System belegt. Dar"uber befinden sich Forschungseinrichtungen von Cynarian und anderen Unternehmen.

Die Plane 9 ist vollst"andig von Neuro Intelligence belegt. Der Ring der Plane 9 kann durch die Aufz"uge in den vier Speichen erreicht werden. Die Aufz"uge laufen innerhalb des Rings in einem Schacht bis zum "`Boden"' des Ringes und enden in einem den Ring umlaufenden 15m breiten Korridor, der die gesamte H"ohe des Rings umfasst. Zu beiden Seiten des Korridors k"onnen weitere R"aume betreten werden. Die R"aume im "`ersten Stock"' erreicht man "uber Treppen zu einer Galerie. Der Korridor ist mit Pflanzenk"ubeln dekoriert. Auf der der Station zugewandten Seite befinden sich im Ergescho\3 Produktionsst"atten und im ersten Stock Labore. Auf der dem Weltall zugewandten Seite befinden sich Wohnr"aume und B"uros. F"ur die Evakuierung der Station sind am Ring Notfallkapseln f"ur alle Mitarbeiter angedockt, die "uber den Mittelgang bestiegen werden k"onnen.

Die Plane der Neuro Intellgence hat eine weitestgehend unbekannte Modifikation erfahren. Die Nabe der Plane kann vom Rest der Station abgesprengt werden. Die gesamte Plane treibt dann eigentst"andig im All. Man"ovrierd"usen erlauben eine eingeschr"ankte Fortbewegung. Die Plane 9 wird aus einem Weltraumobservatorium am Ende der Nabe gesteuert. Der Raum hat die Form einer Kugel. Der dem Weltall zugewandte Teil ist dabei komplett verglast. In der Mitte des Raumes ist eine Konstruktion mit Konsolen und Liegen aufgeh"angt. Der Raum selbst kann nur durch zwei Druckschotts von innerhalb der Nabe betreten werden. Die Druckschotts befinden sich etwas versteckt im hinteren Bereich von zwei Laboren und sind mit Magschl"ossern gesichert.

\subsection{Endgame}

Kurz vor dem Start der Infiltratoren wird bekannt, dass die Konzernflotte den Lagrangepunkt L5 von Kallisto, also die Nike Station, kurz nach dem dortigen Eintreffen der Eingreiftruppe passieren wird. Die Angreifer werden deshalb noch mit Sprengs"atzen ausgestattet, um im Notfall alles Wissen der Neuro Intelligence zu vernichten. Blackheart fliegt mit der Pendragon zur Verst"arkung dem Eingreiftrupp hinterher. Cynarian hat selbst den leichten Kreuzer Hyperion im Orbit der Station.

Wird die Neuro Intelligence infiltiert und befinden sich die Eindringlinge innerhalb der Plane 9, sprengt sich die Plane vom Rest der Raumstation ab und entfernt sich von der Station. Ein Ruck geht durch die Plane. Wenn die Eingreiftruppe den Ring betritt, herrscht im Mittelgang helle Aufregung. Mitarbeiter versuchen, die Rettungskapseln zu erreichen und diese zu "offnen was aber misslingt. Kr"afte der Firmensicherheit versuchen, dem Geschehen Herr zu werden. Leute st"urzen immer wieder, da der Ring noch nicht wieder eine stabile Rotationsgeschwindigkeit aufbauen konnte. In R"aumen mit Zugang zum Mittelgang sind Sichheitsmannschaften postiert, die das Feuer auf die Eindringlinge er"offnen.

Das Abkoppeln wurde durch Dr.~Naratova selbst eingeleitet, die sich im Observatorium in der Nabe aufh"alt. Alte Verbindungen zur Flugsicherung der Station haben sie rechtzeitig gewarnt. Die T"uren zum Observatorium sind durch ein MagSchlo\3 gesichert. Dr.~Naratova selbst ist unbewaffnet und hat sich in einen Raumanzug ohne Gesichtsmaske und einem roten Overall gekleidet und auf einer der Liegen festgeschnallt.

Zwei USI-Agenten befinden sich w"ahrend des Angriffs in den R"aumen der Neuro Intelligence. Sie haben die ganze Operation seit Wochen mit Dr.~Naratova geplant und "uberwacht. Direkt nach dem Eindringen der Eingreiftruppe werden sie versuchen, Dr.~Naratova zu finden. Einer der beiden begibt sich dazu zun"achst zu ihrem B"uro, w"ahrend der andere das private B"uro der Frau Doktor aufsucht. Da sie dort nicht f"undig werden, z"ahlen sie eins und eins zusammen und machen  sich schnellstm"oglich auf den Weg zum Observatorium.

Wenn m"oglich sollten die USI-Agenten vor den Charakteren beim Observatorium eintreffen und das Schloss mit einem Magschlossknacker schnell geknackt haben. Haben die Infiltratoren selbst keinen Schlossknacker dabei, dann haben die Agenten unachtsamerweise die T"ure nicht wieder verriegelt. Wenn die Charaktere in den Raum kommen, sind die USI-Agenten gerade dabei, das Gehirn der Doktorin zu scannen. Einer der Agenten, ein Cyborg, hockt auf Naratova und fixiert ihren Kopf, w"ahrend er sich mit der anderen Hand am Netz der Liege festh"alt. Der Andere, ein Psychonaut, hat sich auf der anderen Liege festgeschnallt und sich mit dem Kopf der Firmenchefin verbunden. Er"offnen die Charaktere das Feuer, dr"uckt sich der Cyborg auf die bewegungsunf"ahige Frau, entfernt mit einer flie\3enden Bewegung die Gehirnverbindung und zieht seine Pistole. Wenn die Charaktere Deckung suchen oder es aus sonstigen Gr"unden zu einer kurzen Feuerpause kommen sollte, dr"uckt der Agent seine Waffe an Naratovas Sch"adel und droht, sie zu t"oten. Wenn es nicht zum Schu\3wechsel kommt, rei\3t der Psychonaut selbst das Datenkabel aus seiner Buchse und ruft erschreckt "`Cortexbombe"', bevor er sich von seiner Liege st"urzt und versucht, Abstand zu gewinnen.

Bei der ersten sich ihr bietenden Gelegeneit ergreift Dr.~Naratova das Wort: "`Bevor sich hier noch jemand zu un"uberlegten Handlungen verleiten l"asst. Es ist alles hier drin in meinem Kopf. Gesch"utzt durch eine Bombe."'

Sie dr"uckt den Cyborg beiseite, der es mit sich geschehen l"asst. Er schwingt sich von der Liege und postiert sich in einigem Abstand, was Naratova erm"oglicht ihren Oberk"orper aufzurichten und sich den Ermittlern zuzuwenden.

Sie f"ahrt fort: "`H"oren Sie. Mein Hirn beherbergt eine von zwei Kopien der Baupl"ane f"ur die Implantate, die meine Kinder zum Leben erweckt. Die zweite Kopie wird bald von jemandem gefunden werden, der damit viel anfangen kann. Und dann werden meine Kinder frei sein. Wir haben hier ein ganz neues Leben erschaffen. Verstehen Sie meine Herren? Das mit den Attentaten tut mir nat"urlich leid. Vertragliche Verpflichtungen, unsch"on aber in meiner schwierigen Situation leider nicht wegverhandelbar."',

sie l"achelt.

"`Und\dots{}nur damit keine Missverst"andnisse auftreten. Die schon produzierten Attent"ater oder Freien. Wer das ist, das steckt ebenfalls in meinem Kopf. Sie sehen meine Herren. Wir haben quasi aktuell eine Patt-Situation."'

Naratova wartet nun auf eine Reaktion der beiden anderen Parteien.

W"ahrend die Spieler nun "uberlegen ,was zu tun ist, trifft ein verschl"usselter Funkspruch von Blackheart von der Pendragon ein.

"`Was ist da bei euch los? Ist bei euch noch jemand am Leben? Die Kreuzer aus dem Asteroideng"urtel sind gleich da. Wenn bei euch nur ein Funkspruch, der nicht mit unserem Code verschl"usselt ist, rausgeht, blasen wir euch aus dem All. Habt ihr verstanden? Over and out."'

Kurze Zeit sp"ater trifft noch ein Funkspruch ein:

"`Der Feind hat sich gemeldet, h"ort ihr? Sie behaupten, die Neuro Intellgience h"atte wertvolles Material der USI gestohlen, was sie wieder an sich nehmen wollen. Das werden wir nicht zulassen. Wenn die ein Enterkommando losschicken, seid ihr ebenfalls Geschichte! Ende."'

Wenn die Ermittler nicht selber das Gespr"acht wieder in Gang setzen, meldet sich Dr.~Naratova wieder zu Wort:

"`Meine Herren. Ich bin bereit, Ihnen ein Angebot zu unterbreiten. Einer der Parteien erm"oglicht es mir, meine Forschungen weiter zu betreiben in aller "Offentlichkeit. Alle meine Erkenntnisse werden "offentlich zur Verf"ugung gestellt, niemand bekommt irgendwelche Exklusivrechte einschlie\3lich dem nat"urlich, was ich bereits entwickelt habe. Als Gegenleistung werde ich mein Wissen Zug um Zug preisgeben. Wie h"ort sich das an?"'

In der aktuellen Situation m"ussen die Charaktere nun zum einen verhindern, dass eines der Kampfschiffe das Feuer auf die treibende Station er"offnet und zum anderen verhindern dass die Forschungsergebnisse der USI in die H"ande fallen und ebenfalls die Identit"aten von weiteren Attent"atern aufdecken. Das Angebot Naratovas bietet eine m"ogliche L"osung, wenn es gelingen sollte Dr.~Naratova sicher in den Gewahrsam der Protektoratsstreitkr"afte zu bringen. Eine M"oglichkeit dazu w"are es Naratova zu bitten, die Rettungskapseln frei zu geben, die Station zu evakuieren und Naratova mit einer der Kapseln los zu schicken. Da der Funk der Infiltratoren nahezu die einzige M"oglichkeit bietet, Botschaften nach au\3en zu schicken w"are das Protektorat bei der Bergung der richtigen Kapsel klar im Vorteil. Werden die Rettungskapseln losgeschickt, werden die Konzernkreuzer versuchen, die Pendragon, die Hyperion und Shuttles der Nike-Station daran zu hindern, die Kapseln zu bergen. Ein heftiger Raumkampf entbrennt.

\begin{remarks}
	Nach der Besetzung von Kallisto rechnete Dr.~Naratova bereits mit einer Aufdeckung der Aktivit"aten der Neuro Intelligence und transferierte die zentralen Baupl"ane und Steuerungsroutinen der Neuronalkopplungen in ihr eigenes Gehirn und vernichtete sonstige Datenspeicher.
	
	W"ahrend der gesammten Infiltration der Neuro Intelligence wird davon ausgegangen, dass die Angreifer ihren St"orsender aktiviert haben.
	
	Haben die Spieler Schwierigkeiten zu verstehen, was Naratova mit "`ihren Kindern"' meint, kann der Spielleiter den Hinweis geben, dass hiermit die KIs gemeint sind.
	
	Die beiden USI-Agenten geben sich nicht als solche zu erkennen. Naratova wird ihre Identit"at ebenfalls nicht ansprechen. Wird ihre Identi"at als USI-Agenten aufgedeckt oder werden sie direkt darauf angesprochen, wer sie sind, geben sie sich als USI-Agenten aus, die ein Datendiebstahl von geheimer Cyberwaretechnologie zur Neuro Intelligence gef"uhrt hat. Naratova wird das nicht weiter kommentieren.
	
	Die beiden USI-Agenten stehen dem Spielleiter als eine Art Joker, zur Verf"ugung das Geschehen in die eine oder andere Richtung zu steuern. Der Cyborg muss sich als solcher nicht sofort zu erkennen geben und hat damit unerwartete Kampfkraft am Start. Die USI-Agenten k"onnten z.B.~versuchen, durch Lichtzeichen mit der Konzernflotte Kontakt aufzunehmen.
	
	Je nach Stimmungslage kann Dr.~Naratova bei einem Rettungsman"over sterben und damit ihre Informationen unwiederbringlich verloren gehen oder es wendet sich alles zum Guten.
\end{remarks}