\section{Hintergrund}

\subsection{Vorgeschichte}

Durch ein menschenverachtendes Vorgehen von zum Einsatz gebrachten K"unstlichen Intelligenzen beim Kampf gegen die Mutantenrebellion vor der Gr"undung des Protektorats im jovianischen System sind KIs zum wiederholten Mal in den Augen der Menschheit in Ungnade gefallen. Da die Cynarian Corporation f"ur den Aufbau einer Industrie auf dem Jupiter enorme Ressourcen ben"otigt, nahm das Unternehmen die aktuelle Stimmung zum Anlass, um unter anderem die vielversprechende aber teure KI-Forschungsabteilung unter der Leitung von Dr.~Naratova auf der Mars Orbitalstation Neu--Gr"oning zu streichen.

Im Zuge der Besiedelung des Jupiters wurde Neu--Gr"oning daraufhin zum Jupiter geschleppt und als Nike Station zum Verwaltungs- und F\&E St"utzpunkt der Cynarian Corporation im Jovianischen System eingerichtet. Neben Forschungslaboren von Cynarian wurden auch Einrichtungen von Zulieferfirmen auf Nike zugelassen. F"uhrende Mitarbeiter der ehem.~KI-Abteilung gr"undeten darauf hin unter der Leitung Dr.~Naratovas den unabh"angigen Headware Zulieferer Neuro Inteligence unter anderem auch mit Geldern aus Tarnfirmen der USI. Die Labore und Produktionsst"atten der Neuro Inteligence wurden in alten R"aumlichkeiten der ehemaligen KI-Abteilung auf Ebene 9 der Nike Station eingerichtet. Neuro Intelligence beliefert im Jovianischen System die Kliniken auf Kallisto mit neuronaler Soft- und Hardware. Da die Implantate, die die Nero Intelligence im Auftrag der Kliniken fertigt, Ma\3anfertigungen sind, erh"alt die Neuro Intelligence weitreichende Informationen "uber die zuk"unftigen Tr"ager der Implantate.

Seit der Gr"undung entwickelt die Neuro Intelligence im geheimen Auftrag und mit Informationen des USI-Geheimdienstes ihre KI Software weiter. Dabei gelingt es, den Zugriff auf das menschliche Gehirn f"ur die weiter entwickelten KIs zu "offnen und der KI die "Ubernahme des Tr"agers zu erlauben. Nicht jedes Gehirn ist f"ur die "Ubernahme geeignet. In der ersten Ausbaustufe k"onnen nur die Gehirne von Mutanten manipuliert werden.

\subsection{Chronologie der Attentate}

Mit Hilfe von durch KIs infizierter Mutanten startet die USI eine Anschlagsreihe auf das Protektorat, um die neue Technologie im Extremfall zu erproben und die Stellung des Potektorats und die der Cynarian Corporation zu diskreditieren.

Bevor die Charaktere mit Nachforschungen beauftragt werden, hat es bereits ein bekanntes Attentat auf eine F"orderstation auf dem Jupiter und mehrere Unf"alle gegeben, die sich als Attentate heraus stellen werden:

\begin{description}
\item [Vor 8 Wochen:] Shuttleabsturz durch Pilotenfehlverhalten auf dem Hangardeck von Hellgate. 3 Shuttles
      werden zust"ort. Der Pilot war ein Eta mit dem Namen Razor. Vom Schuttleabsturz erfahren die Ermittler erst durch Nachforschungen auf Hellgate.
\item [Vor 6 Wochen:] Schlepperfehlfunktion durch fehlerhaft verbaute Steuerungskomponenten. Reparatur dauert
      mehrere Tage. Verantwortlicher Ingenieur kann nicht festgestellt werden. "Uber die Schlepperfehlfunktion wird der Cynarian Ermittler beim ersten Briefing aufgekl"art.
\item [Vor 5 Wochen:] HeM03 Mine auf dem Jupiter wird zerst"ort. Ein Teil der Besatzung kann mit dem
      Rettungsschuttle fliehen. Die Attent"aterin Sent t"otet mehrere Minenarbeiter, bevor sie sich selbst richtet. "Uber den Minenunfall werden beide Chefermittler w"ahrend ihres ersten Briefings informiert.
\item [Vor zwei Wochen.] Explosion beim Anbau von neuen Habitaten an dem Armageddon Ring. Zug"ange und ein
      ehemaliger Frachter, der als Habitat dienen sollte, werden weitestgehend zerst"ort.  Mehrere Tote. Ursache kann nicht ermittelt werden. "Uber den Unfall auf Armageddon wird der Chefermittler des Protektorats bei der Besprechung mit Avenger aufgekl"art.
\item [Vor einer Woche.] Munitionsexplosion auf dem Flottentr"ager Donar. Von der Munitionsexplosion erf"ahrt der
      von Blackheart eingesetzte Ermittler, wenn er den Komandostab "uber aktuelle Ergebnisse der Nachforschungen auf Kallisto unterrichtet.
\item [Vor drei Tagen:] Havarie der Mine HeM05. Der Attent"ater ein Mutant mit dem Namen Pitch hat das
      Rettungsshuttle sabotiert und 3 von 5 Tr"agerballons zerst"ort. Danach hat er sich von der Gallerie der Mine in den Abgrund gest"urzt. Die Mine kann durch heldenhaften Einsatz von J"agern der Hellgate Station und mit Hilfe eines Minenschleppers gerettet und zur Hellgate Station gebracht werden. W"ahrend der Einsatzbesprechung der Ermittler sitzen die geretteten Minenarbeiter noch in der Dekompressionskammer. "Uber die Sabotage werden beide Chefermittler w"ahrend ihres ersten Briefings informiert.
\end{description}

\subsection{Informationsbereitstellung und Kommunikation}

Das jovianische System besitzt erst nach der Besiedelung durch das Protektorat eine nennenswerte Infrastruktur. Durch den rasanten Aufbau hat jedoch ein Gro\3teil der Einrichtungen einen provisorischen Charakter. Diese sind zudem auf das Notwendigste beschr"ankt. Das gilt auch f"ur die Kommunikations-- und Informationssysteme. Wo auf Erde und Mars ein gutes ComNetz jegliche Information an jedem Punkt im System bereit stellt, steht ein voll ausgebautes ComNetz hier nur in Konzernsektoren und beim Milit"ar bereit.

Im jovianischen System  betreiben die einzelnen Monde und Stationen meist ein autonomes Kommunikationssystem, das mit den anderen Siedelungen und Anlagen nicht integriert ist. Kommunikation zwischen den Siedlungen ist "uber Funk nur "uber einen jeweils dedizierten Anruf m"oglich. Daten und audiovisuellen Anrufen steht nur eine begrenzte Bandbreite zur Verf"ugung, sind also entsprechend "`rationiert"'. Da die Stationen und Monde oft mehrere Millionen Kilometer voneinander entfernt sind, muss mit einer Kommunikationsverz"ogerung von mehreren Sekunden gerechnet werden.

Durch den steten unkontrollierten Zustrom von Mutantenfl"uchtlingen, Gl"ucksrittern und neuen Firmen stehen kaum Informationen zu Einzelpersonen und ans"assigen Institutionen und Unternehmen bereit.

\subsection{Fortbewegung und Reisezeiten}

Gro\3e Unternehmen wie Cynarian, das Protektoratsmilit"ar und die Protektoratsadministration betreiben eigene Shuttleflotten innerhalb des Systems soweit n"otig. Der Rest der Fl"uge zwischen den Jupitertrabanten wird durch Transportunternehmen und Schiffseignern von Shuttles bereitgestellt. Einen Flug bekommt man am einfachsten im Raumhafen der jeweiligen Station. Da die Entfernungen im Jupitersystem oft enorm sind sind Reisezeiten von 1 bis 2 Tagen bei zivilen Schiffen "ublich.

Den Ermittlern selbst steht w"ahrend ihrer Ermittlungen das Shuttle "`Dawn of Day"' der Cynarian Corporation zur Verf"ugung.

\section{Rollenspiel}

\subsection{Die Charaktere}

Die Charaktere "ubernehmen die Rolle der Ermittler, die die Attentatsreihe aufkl"aren sollen. Der Plot sieht zwei Ermittler aus den Reihen der Cynarian Corporation und zwei Ermittler aus den Reihen des Protektorats vor. Beide Gruppen sind jeweils ihren eigenen Organisationen verpflichtet und sollten entsprechend handeln (ggf.~durch entsprechende W"urfelw"urfe einfordern).

Jede der beiden Gruppen hat einen Chefermittler und einen Assistenten. Der Chefermittler wird als erstes durch die F"uhrung der Cynarian Corporation bzw.~des Protektorats eingewiesen und muss dann den Assistenten einweisen.

Beim Chefermittler der Cynarian Corporation sollte es sich um einen Psychonauten mit F"uhrungsqualit"aten und Ermittlererfahrung handeln. Denkbar ist zum Beispiel der Komandant einer Corvette aus dem Asteroideng"urtel mit dem Auftrag, Piraterie zu unterbinden. Der Cynarian Assisten sollte jemand mit Ortskenntnis im jovianischen System sein der mit Konzernkontakten aushelfen kann. Beide Cynarian Ermittler sind Norms.

Der Chefermittler des Protektorats sollte ein Vertrauter des Protektors Avenger sein. Er hat Kontakte und Ortskenntnisse im Mutantenteil des jovianischen Systems. Der Chefermittler sollte deshalb ein Alpha Mutant sein. Der Assistent wird durch das Protektoratsmilit"ar gestellt und hat somit einen milit"arischen Hintergrund. Der Assistent kann ein Omega oder Eta sein.

\subsection{Regelwerk}

Als Regelwerk f"ur den Plot bietet sich das Rollenspielsystem \emph{Shadowrun} an. Menschen und Mutantenrassen lassen sich leicht auf die Rassen und Archetypen von Shadowrun abbilden. Waffen und Cyberware finden ebenfalls ihre Pendants bei Shadowrun. Die Matrixregeln lassen sich f"ur ComNetz und Gehirnscans anwenden. Mehr dazu im Anhang.
